
\usepackage{amsmath,amssymb,amsthm}
\usepackage{mathtools}

%\usepackage{fouriernc}
\usepackage{unicode-math}
\setmathfont{STIX2Math.otf}
\setmainfont[
         BoldFont={STIX2Text-Bold.otf}, 
         ItalicFont={STIX2Text-Italic.otf},
         BoldItalicFont={STIX2Text-BoldItalic.otf}
         ]{STIX2Text-Regular.otf}
%\setmainfont{STIXTwoText-Regular.otf}
%% \setmainfont[
%%          BoldFont={STIXTwoText-Bold.otf}, 
%%          ItalicFont={STIXTwoText-Italic.otf},
%%          BoldItalicFont={STIXTwoText-BoldItalic.otf}
%%          ]{STIXTwoText-Regular.otf}

%\setmainfont{Times New Roman}
%\setmathfont{Erewhon-Math.otf}

%\setmainfont{TeX Gyre Schola}[Scale=0.90] % Text
% Math: mix of Erewhon and Schola
%\setmathfont{Erewhon-Math.otf}
% \setmathfont{TeX Gyre Schola Math}[Scale=0.93,
%%     range={up/{latin,Latin,num}, it/{latin,Latin,num},
%%            bfup/{latin,Latin,num}, bfit{latin,Latin,num}}]
\usepackage{fontspec}

\usepackage{booktabs,threeparttable}
\usepackage{multirow}
\usepackage{nicematrix}

\usepackage{dsfont}
\usepackage[algo2e, ruled]{algorithm2e}

\usepackage{proof-at-the-end}
\newtheorem{remark}{\textbf{Remark}}
\newtheorem{lemma}{\textbf{Lemma}}
\newtheorem{assumption}{\textbf{A}}
\newtheorem{theorem}{\textbf{Theorem}}
\newtheorem{proposition}{\textbf{Proposition}}
\newtheorem{definition}{\textbf{Definition}}

\usepackage{mdframed}
\newmdtheoremenv{framedtheorem}{\textbf{Theorem}}
\newmdtheoremenv{framedproposition}{\textbf{Proposition}}
\newmdtheoremenv{framedlemma}{\textbf{Lemma}}

\usepackage{url}
\usepackage{hyperref}
\usepackage[noabbrev, capitalise, nameinlink]{cleveref}

\crefname{framedtheorem}{Theorem}{Theorems}
\crefname{framedproposition}{Proposition}{Propositions}
\crefname{framedlemma}{Lemma}{Lemmas}


%% \newcommand*{\figref}[2][]{%
%%   \hyperref[{#2}]{%
%%     Figure~\ref*{#2}%
%%     \ifx\\#1\\%
%%     \else
%%       \,#1%
%%     \fi
%%   }%
%% }

\pgfkeys{/prAtEnd/global custom defaults/.style={
    %proof at the end,
    end,
    %normal,
    restate,
    %text link={\textit{Proof.} The proof is in the \textit{supplementary material}.
    text link={\textit{Proof.} See the \hyperref[proof:prAtEnd\pratendcountercurrent]{\textit{full proof}} in~\cref{section:proofs}.
    }
  }
% Fix link later for camera ready version.
}

\def\code#1{\texttt{#1}}
\DeclareMathOperator*{\minimize}{minimize}
\DeclareMathOperator*{\maximize}{maximize}
%\DeclareMathOperator*{\argmax}{arg\,max}
%\DeclareMathOperator*{\argmin}{arg\,min} 

\newcommand*\xbar[1]{%
  \hbox{%
    \vbox{%
      \hrule height 0.6pt % The actual bar
      \kern0.33ex%         % Distance between bar and symbol
      \hbox{%
        \kern-0.1em%      % Shortening on the left side
        \ensuremath{#1}%
        \kern-0.1em%      % Shortening on the right side
      }%
    }%
  }%
} 

\newcommand{\E}[1]{\mathbb{E}\left[\,#1\,\right]}
\newcommand{\Esub}[2]{\mathbb{E}_{#1}\left[\,#2\,\right]}
\newcommand{\V}[1]{\mathbb{V}\left[\,#1\,\right]}
\newcommand{\Vsub}[2]{\mathbb{V}_{#1}\left[\,#2\,\right]}
\newcommand{\Cov}[1]{\mathrm{Cov}\left(\,#1\,\right)}
\newcommand{\Covsub}[2]{\mathrm{Cov}_{#1}\left(\,#2\,\right)}
\newcommand{\Corr}[1]{\mathrm{Corr}\left(\,#1\,\right)}

\newcommand{\Df}[2]{D_{f}(#1\parallel#2)}
\newcommand{\DKL}[2]{D_{\mathrm{KL}}(#1\parallel#2)}
\newcommand{\DChi}[2]{D_{\chi^2}(#1\paallel#2)}
\newcommand{\norm}[1]{{\left\lVert\,#1\,\right\rVert}}
\newcommand{\DTV}[2]{{\left\lVert\,#1 - #2\,\right\rVert}_{\mathrm{TV}}}

%\newcommand{\symbfup}[1]{\mathbf{#1}}
%\newcommand{\symbf}[1]{\mathbf{#1}}

\newcommand{\vX}{\symbfup{X}}
\newcommand{\vY}{\symbfup{Y}}
\newcommand{\vZ}{\symbfup{Z}}

\newcommand{\va}{\symbfup{a}}
\newcommand{\vb}{\symbfup{b}}
\newcommand{\vc}{\symbfup{c}}
\newcommand{\vd}{\symbfup{d}}
\newcommand{\ve}{\symbfup{e}}
\newcommand{\vf}{\symbfup{f}}
\newcommand{\vg}{\symbfup{g}}
\newcommand{\vh}{\symbfup{h}}
\newcommand{\vi}{\symbfup{i}}
\newcommand{\vj}{\symbfup{j}}
\newcommand{\vk}{\symbfup{k}}
\newcommand{\vl}{\symbfup{l}}
\newcommand{\vm}{\symbfup{m}}
\newcommand{\vn}{\symbfup{n}}
\newcommand{\vo}{\symbfup{o}}
\newcommand{\vp}{\symbfup{p}}
\newcommand{\vq}{\symbfup{q}}
\newcommand{\vr}{\symbfup{r}}
\newcommand{\vs}{\symbfup{s}}
\newcommand{\vt}{\symbfup{t}}
\newcommand{\vu}{\symbfup{u}}
\newcommand{\vv}{\symbfup{v}}
\newcommand{\vw}{\symbfup{w}}
\newcommand{\vx}{\symbfup{x}}
\newcommand{\vy}{\symbfup{y}}
\newcommand{\vz}{\symbfup{z}}
\newcommand{\valpha}{\symbfup{\alpha}}
\newcommand{\vmu}{\symbfup{\mu}}
\newcommand{\vtheta}{\symbfup{\theta}}
\newcommand{\vlambda}{\symbfup{\lambda}}

\newcommand{\mA}{\symbfup{A}}
\newcommand{\mB}{\symbfup{B}}
\newcommand{\mC}{\symbfup{C}}
\newcommand{\mD}{\symbfup{D}}
\newcommand{\mE}{\symbfup{E}}
\newcommand{\mF}{\symbfup{F}}
\newcommand{\mG}{\symbfup{G}}
\newcommand{\mH}{\symbfup{H}}
\newcommand{\mI}{\symbfup{I}}
\newcommand{\mJ}{\symbfup{J}}
\newcommand{\mK}{\symbfup{K}}
\newcommand{\mL}{\symbfup{L}}
\newcommand{\mM}{\symbfup{M}}
\newcommand{\mN}{\symbfup{N}}
\newcommand{\mO}{\symbfup{O}}
\newcommand{\mP}{\symbfup{P}}
\newcommand{\mQ}{\symbfup{Q}}
\newcommand{\mR}{\symbfup{R}}
\newcommand{\mS}{\symbfup{S}}
\newcommand{\mT}{\symbfup{T}}
\newcommand{\mU}{\symbfup{U}}
\newcommand{\mV}{\symbfup{V}}
\newcommand{\mW}{\symbfup{W}}
\newcommand{\mX}{\symbfup{X}}
\newcommand{\mY}{\symbfup{Y}}
\newcommand{\mZ}{\symbfup{Z}}
\newcommand{\mSigma}{\symbfup{\Sigma}}

\newcommand{\iprod}[2]{\langle #1, #2 \rangle}
\newcommand{\ind}[1]{\mathds{1}_{#1}}
