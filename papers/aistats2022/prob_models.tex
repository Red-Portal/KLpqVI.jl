
\section{Probabilistic Models Considered in~\cref{section:eval}}
\subsection{Hierarchical Logistic Regression}
The hierarchical logistic regression used in~\cref{section:logistic} is 
\begin{align*}
    \sigma_{\beta}  &\sim \mathcal{N}^+\left(0, 1.0\right) \\
    \sigma_{\alpha} &\sim \mathcal{N}^+\left(0, 1.0\right) \\
    \symbf{\beta} &\sim \mathcal{N}\left(\symbf{0}, \sigma_{\beta}^2\,\symbf{I}\right) \\
    \alpha        &\sim \mathcal{N}\left(0, \sigma_{\alpha}^2\right) \\
    p             &\sim \mathcal{N}\left(\vx_i^{\top}\symbf{\beta} + \alpha,\, \sigma_{\alpha}^2\right)\\
    y_i           &\sim \text{Bernoulli-Logit}\,(p)
\end{align*}
where \(\vx_i\) and \(y_i\) are the predictors and binary target variable of the \(i\)th datapoints.

\subsection{Gaussian Process Logistic Regression}

\subsection{Radon Hierarchical Regression}
The partially pooled linear regression model used in~\cref{section:mll} is
\begin{align*}
  \sigma_{a_1} &\sim \mathrm{Gamma}\left( \alpha = 1, \beta = 0.02 \right) \\
  \sigma_{a_2} &\sim \mathrm{Gamma}\left( \alpha = 1, \beta = 0.02 \right) \\
  \sigma_{y}  &\sim \mathrm{Gamma}\left( \alpha = 1, \beta = 0.02 \right) \\
  \mu_{a_1}    &\sim \mathcal{N}\left( 0, 1 \right) \\
  \mu_{a_2}    &\sim \mathcal{N}\left( 0, 1 \right) \\
  a_{1,\, c}     &\sim \mathcal{N}\left( \mu_{a_1}, \sigma_{a_1}^2 \right) \\
  a_{2,\, c}     &\sim \mathcal{N}\left( \mu_{a_2}, \sigma_{a_2}^2 \right) \\
  y_i         &\sim \mathcal{N}\left( a_{1,\, c_i} + a_{2,\, c_i}\,x_i,\, \sigma_y^2 \right)
\end{align*}
where \(a_{1,\,c}\) is the intercept at the county \(c\), \(a_{2,\,c}\) is the slope at the county \(c\), \(c_i\) is the county of the \(i\)th datapoint, \(x_i\) and \(y_i\) are the floor predictor of the measurement and the measured radon level of the \(i\)the datapoint, respectively.
The model pools the datapoints into their respective counties, which complicates the posterior geometry~\citep{betancourt_hierarchical_2020}.

%%% Local Variables:
%%% TeX-master: "master"
%%% End:
