\vspace{-0.05in}
\section{Related Works}\label{section:related}
\vspace{-0.05in}
\paragraph{Inclusive KL minimization}
Our method directly builds on top of MSC~\citep{NEURIPS2020_b2070693}, which is a method for minimizing the inclusive KL divergence.
Concurrently,~\citet{pmlr-v124-ou20a} proposed JSA which they for training variational autoencoders with discrete latent variables.
Unlike MSC, JSA can specifically be applied models with \textit{i.i.d.} data with minibatches.
Before the two, only a few have proposed methods that minimize the inclusive KL using SGD.
Notably,~\citet{DBLP:journals/corr/BornscheinB14} use SNIS for estimating the stochastic gradients, while~\citet{li_approximate_2017} use an MCMC kernel to refine samples from \(q_{\vlambda}(\vz)\) to better resemble samples from \(p(\vz\mid\vx)\).

\vspace{-0.1in}
\paragraph{MCMC for VI}
Not restricted to inclusive KL minimization, MCMC has been widely utilized in VI.
For example,~\citet{pmlr-v37-salimans15, pmlr-v97-ruiz19a} construct alternative divergence bounds from samples of an MCMC sampler.

\vspace{-0.1in}
\paragraph{Adaptive MCMC}
As pointed out by~\citet{pmlr-v124-ou20a}, using \(q_{\vlambda}\) within the MCMC kernel makes score climbing structurally equivalent to adaptive MCMC.
In particular,~\citet{10.1007/s11222-008-9110-y, garthwaite_adaptive_2016} discuss the use of stochastic approximation in adaptive MCMC.
Also,~\citet{andrieu_ergodicity_2006, keith_adaptive_2008, holden_adaptive_2009, giordani_adaptive_2010} specifically discuss adapting the propsosal of IMH kernels.
Most similar to score climbing VI is the work of~\citet{keith_adaptive_2008} where they propose to use \textit{cross-entropy minimization}~\citep{barbakh_cross_2009}, which is mathematically identical to inclusive VI.
More recently, several other methods that apply variational inference for adapting the MCMC kernel have been developed.
For adapting the proposals of an IMH sampler, \citet{habib2018auxiliary} minimize the exclusive KL divergence while~\cite{neklyudov_metropolishastings_2019} minimize the symmetric KL divergence.
And for HMC,~\citet{zhang_variational_2018, pmlr-v139-campbell21a} have proposed to use score matching, ELBO maximization and kernelized Stein discrepancy minimization.

%% \vspace{-0.1in}
%% \paragraph{Ergodicity and Inclusive VI}
%% Meanwhile, in the context of MCMC,~\citet{10.2307/2242610} showed that it is necessary to ensure \(\sup_{\vz} w(\vz) = M < \infty\) (finite weight condition) for an IMH kernel to be geometrically ergodic.
%% While this might seem less relevant for inclusive VI, the bound
%% %
%% \vspace{-0.02in}
%% \begin{align}
%%   \DKL{p}{q_{\vlambda}} = \int p(\vz\mid\vx) \log w(\vz)\,d\vz \leq \int p(\vz\mid\vx) \log M \, d\vz = \log M.
%% \end{align}
%% \vspace{-0.02in}
%% %
%% suggests that it is in fact a sufficient condition for the KL divergence to be finite.
%% This condition can easily be violated as shown by \citet{10.1007/s11222-008-9110-y}.
%% To ensure this does not happen,~\citet{giordani_adaptive_2010, holden_adaptive_2009} use proposal distributions of the form of \(w\,q_0(\vz) + (1-w)\,q_{\vlambda}(\vz)\) for some \(0<w<1\) for their adaptive IMH sampler.
%% Here, \(q_0\) is supposed to be a heavy tailed distribution in the spirit of defensive mixtures~\citep{hesterberg_weighted_1995}.
%% %In the benchmark problems we considered, we observed that MSC converges without such precaution.
%% A research direction in the interest of both adaptive MCMC and inclusive VI would be to investigate whether such precaution is actually necessary for convergence.
%% If that is the case, it would be beneficial to consider variational families of heavy-tailed distributions as proposed by~\citet{NEURIPS2018_25db67c5} for exclusive VI.

%Therefore, for problems where MSC is not geometricaly ergodic, inclusive VI would also fail to converge.
%% On the other hand, for problems where MSC converges without problem, defensive mixtures shouldn't be necessary.
%% for problems where \(w(\vz)\) is not bounded, virtually all inclusive VI methods, including SNIS and RWS, will fail to work, as their weights will have very high variance~\citep{mcbook}.
%The boundedness of \(w(\vz)\) is more related to model specification and the selection of the variational family \(\mathcal{Q}\).

%%% Local Variables:
%%% TeX-master: "master"
%%% End:
