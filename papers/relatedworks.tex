
\section{Related Works}
\subsection{Connections in Variational Inference}
\paragraph{Inclusive VI}
\citet{DBLP:journals/corr/BornscheinB14} propose RWS which minimizes the inclusive KL divergence by occasinally utilizing independent samples from the target posterior.
\citep{kim2021adaptive} can be regarded as a blend of MSC and RWS, where an MCMC chain replaces the SNIS estimates the sleep phase updates.

\paragraph{MCMC in VI}
While VI has been regarded as a competitor to MCMC, many VI methods have attempted to make it part of their mechanism.
Some examples include~\citep{pmlr-v97-ruiz19a}.

\subsection{Connections with Adaptive Markov-chain Monte Carlo}
\citet{10.1007/s11222-008-9110-y, garthwaite_adaptive_2016} discuss the use of stochastic approximation in adaptive MCMC.
The containment condition is known to generally hold in practice~\citep{rosenthal_optimal_, bai_containment_2011}.
Thus, recent works on adaptive MCMC have mainly focused on enforcing the diminishing adaptation condition~\citep{wang_adaptive_2013}.
In the context of MSC, the diminishing adaptation condition is statisfied by using a stepsize schedule such that \(\gamma_k \rightarrow 0\).
Thus, it should be possible to treat \(\vz^{(i)}_t\) as genuine samples from the posterior.


%%% Local Variables:
%%% TeX-master: "master"
%%% End:
