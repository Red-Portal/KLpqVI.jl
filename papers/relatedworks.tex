
\section{Related Works}
\paragraph{Inclusive VI with SGD}
Our method directly builds on top of MSC~\citep{NEURIPS2020_b2070693}, which is a method for minizing the inclusive KL divergence.
While many works minimizing the inclusive KL have emerged~\citep{DBLP:journals/corr/BornscheinB14, li_approximate_2017, 10.5555/2074022.2074067, pmlr-v124-ou20a, kim2021adaptive}, only a few have been proposed for general VI based on SGD.
Notably,~\citet{DBLP:journals/corr/BornscheinB14} use SNIS for estimating the stochastic gradients, while~\citet{li_approximate_2017} use an MCMC kernel to refine samples \(q_{\vlambda}(\vz)\) to better resemble samples from \(p(\vz\mid\vx)\).
Meanwhile, \textit{general stochastic approximation} (GSA) by~\citet[Algorithm 1]{pmlr-v124-ou20a}, which is identical to MSC, has been proposed concurrently in the context of discrete latent variables.
\citet{kim2021adaptive} recently proposed a method that essentially blends GSA/MSC with RWS.

\paragraph{Adaptive MCMC}
MSC is structurally equivalent to adaptive MCMC methods as pointed out in~\cite{pmlr-v124-ou20a}.
The only difference is that MSC focuses on \(q_{\vlambda}\), while MCMC focuses on the samples \(\vz_t\) generated during optimization.
In the MCMC front, \citet{10.1007/s11222-008-9110-y, garthwaite_adaptive_2016} discuss the use of stochastic approximation in adaptive MCMC.
Since the step-size rules used in SGD generally satisfy the \textit{diminishing adaptation condition}~\cite{10.1007/s11222-008-9110-y}, it should be possible to treat \(\vz^{(i)}_t\) as genuine samples from the posterior.
Neverthless, in this work, we focus on the VI view of MSC.

\paragraph{Adaptive IMH}
Within MCMC, our method is closely related to adaptive MCMC methods that use independent proposals~\citep{andrieu_ergodicity_2006, keith_adaptive_2008, holden_adaptive_2009, giordani_adaptive_2010}.
For instance, inclusive VI is mathematically identical to \textit{cross-entroy minimization}~\citet{barbakh_cross_2009}, which was employed by~\citet{keith_adaptive_2008} for adapting the proposal distribution.
However, our work differs with previous adaptive IMH algorithms in that we use SGD for adaptation, and the family of proposals (variational family in the context of VI) is taken to be much more general using methods such as ADVI~\citep{JMLR:v18:16-107}.

\paragraph{Ergodicity and Inclusive VI}
Meanwhile, in the context of MCMC,~\citet{10.2307/2242610} showed that it is necessary to ensure \(\sup_{\vz} w(\vz) = M < \infty\) for an IMH kernel to be geometrically ergodic.
While this might seem less relevant for inclusive VI, the bound
\begin{align}
  \DKL{p}{q_{\vlambda}} = \int p(\vz\mid\vx) \log w(\vz)\,d\vz \leq \int p(\vz\mid\vx) \log M \, d\vz = \log M.
\end{align}
suggests that it is in fact a necessary condition for the KL divergence to be finite.
This condition can easily be violated as shown by \citet{10.1007/s11222-008-9110-y}.
To ensure that the boundedness condition is satisfied,~\citet{giordani_adaptive_2010} use \(w\,q_0(\vz) + (1-w)\,q_{\vlambda}(\vz)\) for some \(0<w<1\) as the proposal distribution for their adaptive IMH algorith.
Here, \(q_0\) is supposed to be a heavy tailed distribution in the spirit of defensive mixtures~\citep{hesterberg_weighted_1995}.
However, in the benchmark problems we considered, we observed that MSC converges without such precaution.
A research direction in the interest of both adaptive MCMC and inclusive VI would be to investigate the benefit of using variational families of heavy-tailed distributions such as elliptical distribtions~\citep{NEURIPS2018_25db67c5}.

%Therefore, for problems where MSC is not geometricaly ergodic, inclusive VI would also fail to converge.
%% On the other hand, for problems where MSC converges without problem, defensive mixtures shouldn't be necessary.
%% for problems where \(w(\vz)\) is not bounded, virtually all inclusive VI methods, including SNIS and RWS, will fail to work, as their weights will have very high variance~\citep{mcbook}.
%The boundedness of \(w(\vz)\) is more related to model specification and the selection of the variational family \(\mathcal{Q}\).

%%% Local Variables:
%%% TeX-master: "master"
%%% End:
