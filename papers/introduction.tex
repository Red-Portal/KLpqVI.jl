
\section{Introduction}
Variational inference (VI,~\citealt{jordan_introduction_1999, blei_variational_2017, zhang_advances_2019}) is a method of converting Bayesian inference into an optimization problem.
Instead of working directly with our target distribution \(p\,(\vz\mid\vx)\), we find a \textit{variational approximation} \(q_{\lambda}(\vz) \in \mathcal{Q}\) that closely approximates \(p\,(\vz\mid\vx)\) according to a discrepancy measure \(D(p, q)\).
Naturally, choosing a good discrepancy measure, or objective function, is a critical part of the problem.
This fact had lead to a quest for good divergence measures~\citep{NIPS2016_7750ca35, NIPS2017_35464c84, NEURIPS2018_1cd138d0, pmlr-v97-ruiz19a}.
So far, the exclusive KL divergence \(\DKL{q_{\lambda}}{p}\) (or reverse KL divergence) has been used ``exclusively'' among various discrepancy measures.
This is partly because the exclusive KL is defined as an average over \(q_{\lambda} \in \mathcal{Q}\), which can be estimated efficiently.
In contrast, the inclusive KL is defined as
%
\begin{align}
  %% \DKL{p}{q_{\lambda}} = \int p\,(\vz\mid\vx) \log\big(\, p\,(\vz\mid\vx)/q_{\lambda}(\vz) \,\big)\,d\vz
  %% = \Esub{p(\vz\mid\vx)}{\log\big(\, p\,(\vz\mid\vx)/q_{\lambda}(\vz) \,\big) } \label{eq:klpq}
  \DKL{p}{q_{\lambda}} = \int p\,(\vz\mid\vx) \log \frac{p\,(\vz\mid\vx)\,}{\,q_{\lambda}(\vz)} \,d\vz
  = \Esub{p(\vz\mid\vx)}{\log \frac{p\,(\vz\mid\vx)\,}{\,q_{\lambda}(\vz)} } \label{eq:klpq}
\end{align}
%
where the average is taken over \(p\,(\vz\mid\vx)\). 
Interestingly, this is a chicken-and-egg problem, and minimizing~\eqref{eq:klpq} has drawn the attention of researchers because it can overcome some known limitations of the exclusive KL~\citep{minka2005divergence, mackay_local_2001}.

For performing inclusive VI,~\citet{NEURIPS2020_b2070693, pmlr-v124-ou20a} recently proposed \textit{Markovian score climbing} (MSC), which is a blend of Markov-chain Monte Carlo (MCMC) and variational inference.
In MSC, stochastic gradients of the inclusive KL are obtained by operating a Markov-chain in parallel with the VI optimizer.
An interesting property of MSC emerges when combined with specific types of MCMC kernels.
In particular, we show that \textit{independent Metropolis-Hastings} (IMH,~\citealt{robert_monte_2004}) type kernels can automatically trade off bias and variance when used for MSC.
This family of kernels includes the \textit{condition importance sampling} (CIS,~\citealt{NEURIPS2020_b2070693}) kernel originally proposed for MSC.
Surprisingly, this automatic tradeoff property is unique to IMH type kernels and does not occur in MCMC kernels with state-dependent proposals such as Hamiltonian Monte Carlo (HMC,~\citealt{duane_hybrid_1987, neal_mcmc_2011, betancourt_conceptual_2017}).

Following our analysis of the CIS kernel, we also show that its performance can degrade with the number of proposals (or computational budget) used in each Markov-chain transition.
As a simple solution to this, we propose to use parallel IMH (MSC-PIMH) chains, which reduces variance with the same amount of computation.
We evaluate the performance of MSC with PIMH against other inclusive VI~\citep{DBLP:journals/corr/BornscheinB14, NEURIPS2020_b2070693} and exclusive VI~\citep{pmlr-v33-ranganath14, JMLR:v18:16-107} methods.
Finally, some interesting connections with adaptive MCMC methods~\citep{10.1007/s11222-008-9110-y} are discussed.

\paragraph{Contribution Summary}
\begin{enumerate*}[label=\textbf{(\roman*)}]
\item We show that IMH type kernels (which include the CIS kernel originally used in MSC; \textbf{\cref{section:cis_imh}}) automatically perform bias-variance tradeoff (\textbf{\cref{section:bias_variance}}).
\item We show that increasing the computation budget of the CIS kernel may \textit{increase} its variance and propose parallel IMH (PIMH) chains as an alternative (\textbf{\cref{section:cis_bias}}).
\item We evaluate the performance of MSC with PIMH against other inclusive and exclusive VI methods (\textbf{\cref{section:eval}}).
%\item We discuss connections with adaptive IMH methods (\textbf{\cref{section:related}}).
\end{enumerate*}

%%% Local Variables:
%%% TeX-master: "master"
%%% End:
