
\section{Related Works}\label{section:related}
\paragraph{Inclusive VI with SGD}
Our method directly builds on top of MSC~\citep{NEURIPS2020_b2070693}, which is a method for minizing the inclusive KL divergence.
While many works minimizing the inclusive KL have emerged~\citep{DBLP:journals/corr/BornscheinB14, li_approximate_2017, 10.5555/2074022.2074067, pmlr-v124-ou20a, kim2021adaptive}, only a few have been proposed for general VI based on SGD.
Notably,~\citet{DBLP:journals/corr/BornscheinB14} use SNIS for estimating the stochastic gradients, while~\citet{li_approximate_2017} use an MCMC kernel to refine samples from \(q_{\vlambda}(\vz)\) to better resemble samples from \(p(\vz\mid\vx)\).
Meanwhile, a synonymous method to MSC, \textit{general stochastic approximation} (GSA) by~\citet[Algorithm 1]{pmlr-v124-ou20a} has been proposed concurrently in the context of discrete latent variables.
\citet{kim2021adaptive} recently proposed a method that essentially blends GSA/MSC with RWS.

\vspace{-0.1in}
\paragraph{Adaptive MCMC}
As pointed out by~\citet{pmlr-v124-ou20a}, MSC is structurally equivalent to adaptive MCMC methods.
%The only difference is that MSC focuses on \(q_{\vlambda}\), while MCMC focuses on the samples \(\vz_t\) generated during optimization.
Strong resemblence can be found in methods using stochastic approximation for adapting the proposal distribution used inside the MCMC kernel.
In particular,~\citet{10.1007/s11222-008-9110-y, garthwaite_adaptive_2016} discuss the use of stochastic approximation in adaptive MCMC.
%Since the step-size rules used in SGD generally satisfy the \textit{diminishing adaptation condition}~\cite{10.1007/s11222-008-9110-y}, it should be possible to treat \(\vz^{(i)}_t\) as genuine samples from the posterior.
%Neverthless, in this work, we focus on the VI view of MSC.

\vspace{-0.1in}
\paragraph{Adaptive IMH}
Among adaptive MCMC methods, those that use independent proposals~\citep{andrieu_ergodicity_2006, keith_adaptive_2008, holden_adaptive_2009, giordani_adaptive_2010} are the most related to our work.
\citet{keith_adaptive_2008} propose to use \textit{cross-entroy minimization}~\citep{barbakh_cross_2009}, which is mathematically identical to inclusive VI, for adaptation.
Our work, on the other hand, contrasts with previous adaptive IMH algorithms in that we use SGD for adapting \(q_{\vlambda}(\vz)\).
This enables VI methods such as ADVI to consider proposals that are much more complex~\citep{JMLR:v18:16-107}.

\vspace{-0.1in}
\paragraph{Ergodicity and Inclusive VI}
Meanwhile, in the context of MCMC,~\citet{10.2307/2242610} showed that it is necessary to ensure \(\sup_{\vz} w(\vz) = M < \infty\) (finite weight condition) for an IMH kernel to be geometrically ergodic.
While this might seem less relevant for inclusive VI, the bound
%
\vspace{-0.02in}
\begin{align}
  \DKL{p}{q_{\vlambda}} = \int p(\vz\mid\vx) \log w(\vz)\,d\vz \leq \int p(\vz\mid\vx) \log M \, d\vz = \log M.
\end{align}
\vspace{-0.02in}
%
suggests that it is in fact a sufficient condition for the KL divergence to be finite.
This condition can easily be violated as shown by \citet{10.1007/s11222-008-9110-y}.
To ensure this does not happen,~\citet{giordani_adaptive_2010, holden_adaptive_2009} use proposal distributions of the form of \(w\,q_0(\vz) + (1-w)\,q_{\vlambda}(\vz)\) for some \(0<w<1\) for their adaptive IMH sampler.
Here, \(q_0\) is supposed to be a heavy tailed distribution in the spirit of defensive mixtures~\citep{hesterberg_weighted_1995}.
%In the benchmark problems we considered, we observed that MSC converges without such precaution.
A research direction in the interest of both adaptive MCMC and inclusive VI would be to investigate whether such precaution is actually necessary for convergence.
If that is the case, it would be beneficial to consider variational families of heavy-tailed distributions as proposed by~\citet{NEURIPS2018_25db67c5} for exclusive VI.

%Therefore, for problems where MSC is not geometricaly ergodic, inclusive VI would also fail to converge.
%% On the other hand, for problems where MSC converges without problem, defensive mixtures shouldn't be necessary.
%% for problems where \(w(\vz)\) is not bounded, virtually all inclusive VI methods, including SNIS and RWS, will fail to work, as their weights will have very high variance~\citep{mcbook}.
%The boundedness of \(w(\vz)\) is more related to model specification and the selection of the variational family \(\mathcal{Q}\).

%%% Local Variables:
%%% TeX-master: "master"
%%% End:
