
\section{Background and Motivation}
\subsection{Inclusive Variational Inference Until Now}\label{section:ivi_previous}
A typical way to perform VI is to use stochastic gradient descent (SGD,~\citealt{robbins_stochastic_1951, bottou_online_1999}), which requires unbiased gradient estimates of the optimization target.
In the case of inclusive variational inference, this corresponds to estimating
%
\begin{align}
  \nabla_{\vlambda} \DKL{p}{q_{\lambda}}
  = \Esub{p(\vz\mid\vx)}{ - \nabla_{\vlambda} \log q_{\vlambda}(\vz) }
  = - \Esub{p(\vz\mid\vx)}{ s\,(\vz; \vlambda) }
\end{align}
where \(s\,(\vz; \vlambda) = \nabla_{\vlambda} \log q_{\vlambda}(\vz)\) is known as the \textit{score function}.
Evidently, estimating \(\nabla_{\vlambda} \DKL{p}{q_{\lambda}}\) requires integrating the score function over \(p(\vz\mid\vx)\), which is prohibitive.

\paragraph{Importance Sampling}
When it is easy to sample from the variational approximation \(q_{\lambda}(\vz)\), one can use importance sampling (IS, \citealt{robert_monte_2004, mcbook}) since 
\begin{align}
  \Esub{p(\vz\mid\vx)}{ s\,(\vz; \vlambda) }
  \propto \Esub{q_{\vlambda}}{ w\,(\vz) \, s\,(\vz; \vlambda) }
  \approx \frac{1}{N} \sum^{N}_{i=1} w\,(\vz^{(i)}) \, s\,(\vz^{(i)}; \vlambda)
\end{align}
where \(w\,(\vz) = p\,(\vz,\vx) / q_{\vlambda}(\vz)\) is known as the \textit{importance weight}, and \(\vz^{(1)},\, \ldots,\, \vz^{(N)}\) are \(N\) independent samples from \(q_{\vlambda}(\vz)\).
This scheme is equivalent to adaptive IS methods~\citep{cappe_adaptive_2008, bugallo_adaptive_2017} since the IS proposal \(q_{\vlambda}(\vz)\) is iteratively optimized based on the current samples.
Though IS is unbiased, it is highly unstable in practice.
A more stable alternative is to use the \textit{normalized weight} \(\widetilde{w}^{(i)} = \nicefrac{w\,(\vz^{(i)})}{\sum_{i=1}^N w\,(\vz^{(i)}) }\), which is known as the self-normalized IS (SNIS) approximation.
Unfortunately, SNIS still fails to converge even on moderate dimensional objectives and unlike IS, it is no longer unbiased~\citep{robert_monte_2004, mcbook}.

%
  %% \begin{minipage}[l]{0.45\linewidth}
  %%   \small
  %%   \centering
  %%   \begin{algorithm2e}[H]
  %%     \DontPrintSemicolon
  %%     \SetAlgoLined
  %%     \KwIn{MCMC kernel \(K(\vz,\cdot)\),
  %%       initial sample \(\vz_0\),
  %%       initial parameter \(\vlambda_0\),
  %%       number of iterations \(T\),
  %%       stepsize schedule \(\gamma_t\)}
  %%     \For{\textcolor{black}{\(t = 1, 2, \ldots, T\)}}{
  %%       \textit{Sample} \hspace{0.035in} \( \vz_{t} \sim K(\vz_{t-1}, \cdot) \)\;
  %%       \textit{Estimate} \( s(\vz; \vlambda) = \nabla_{\vlambda} \log q_{\vlambda}(\vz_t) \)\;
  %%       \textit{Update} \hspace{0.03in} \( \vlambda_{t} = \vlambda_{t-1} + \gamma_t\, s\,(\vz_t;\vlambda_{t-1}) \)\;
  %%     }
  %%     \caption{Markovian Score Climbing}\label{alg:msc}
  %%   \end{algorithm2e}
  %%   \vspace{-0.1in}
  %% \end{minipage}
  %% \qquad
  %% \begin{minipage}[r]{0.5\linewidth}
  %%   \vspace{-0.1in}
  %%   \begin{figure}[H]
  %%     \centering
  %%     
\begin{tikzpicture}
  \begin{axis}[
      legend style={
        draw=none,
        fill=none,
        at=({current bounding box.north-|.5,0}),
        anchor=south,
      },
      legend cell align={left},
      %xmode=log,
      xlabel={Iteration},
      %ylabel={Trace},
      tick pos=left,
      tick align={outside},
      major tick length={2pt},
      axis line style={thick},
      every tick/.style={black, thick},
      xmin=1,
      xmax=500,
      ymin=-1.5,
      ymax=8.5,
      xtick={1, 100, 200, 300, 400, 500},
      %xticklabels={\(2^{5}\), \(2^{6}\), \(2^{7}\), \(2^{8}\), \(2^{9}\), \(2^{12}\)},
      %x dir=reverse,
      %table/col sep=comma,
      height={3.5cm},
      width={6.0cm},
    ]

    %% Thick lines
    \addplot [thick, red, mark size=2pt] coordinates {
(1,3.7744966552635417)
(2,6.6637992482529524)
(3,6.259990698592114)
(4,5.997717299544704)
(5,5.798112583190721)
(6,5.536690619969111)
(7,5.384524538240413)
(8,5.275574413902902)
(9,6.243033825159706)
(10,6.045791493861715)
(11,5.905188012235524)
(12,5.7947743141303745)
(13,5.703422931858933)
(14,8.469885134555476)
(15,8.227143565336176)
(16,7.342984877784093)
(17,7.243116624848073)
(18,7.155390820622436)
(19,7.178898516479454)
(20,7.089021416378296)
(21,7.008796826321647)
(22,7.997167461547135)
(23,7.898886397906378)
(24,7.809385491279324)
(25,7.727274144084045)
(26,7.651459545353988)
(27,7.581067456656407)
(28,7.515387789323801)
(29,7.453836168874047)
(30,7.395926142060064)
(31,7.341248664581308)
(32,7.289456688876174)
(33,7.240253399475244)
(34,7.1933831054477055)
(35,7.148624100384867)
(36,7.10578300088166)
(37,7.064690210863979)
(38,7.025196253605488)
(39,6.987168779851843)
(40,6.950490108091594)
(41,6.915055187553156)
(42,6.88076989988074)
(43,6.847549634301591)
(44,6.815318085267477)
(45,6.7840062323083865)
(46,6.753551470075622)
(47,6.723896862918796)
(48,6.617974367341658)
(49,6.588663118009894)
(50,6.560133738483243)
(51,6.53232580294924)
(52,6.505185180776635)
(53,7.086144053463915)
(54,7.053330601491464)
(55,7.021590266940274)
(56,6.9908242549989055)
(57,6.96094663638384)
(58,6.9318820776879)
(59,6.9035640850973525)
(60,6.8759336203764345)
(61,7.014231450491013)
(62,6.986201373565165)
(63,6.958801249509174)
(64,6.931993237170517)
(65,6.9057425215538855)
(66,6.880017012245051)
(67,6.85478707800088)
(68,6.830025312434275)
(69,6.805706326532405)
(70,6.781806564411765)
(71,6.758304139262528)
(72,6.735178686889957)
(73,6.712411234639783)
(74,6.689984083811594)
(75,6.667880703930543)
(76,6.64608563747212)
(77,6.624584413824602)
(78,6.603363471434953)
(79,6.5824100872211995)
(80,6.561712312451603)
(81,6.5412589143914746)
(82,6.521039323104858)
(83,6.50104358287279)
(84,6.481262307754261)
(85,6.461686640871737)
(86,6.442308217051617)
(87,6.423119128492223)
(88,6.404111893168742)
(89,6.385279425716841)
(90,6.3666150105649075)
(91,6.348112277109757)
(92,6.329765176752502)
(93,6.311567961630603)
(94,6.293515164899163)
(95,6.2756015824296405)
(96,6.257822255807558)
(97,6.2401724565226555)
(98,6.22264767125555)
(99,6.205243588174375)
(100,6.187956084163267)
(101,6.170781212912098)
(102,6.153671126843844)
(103,6.136419931382466)
(104,6.119036785809355)
(105,6.064959739335891)
(106,6.046831896859059)
(107,6.028687447161313)
(108,6.010535278933221)
(109,5.992379731321853)
(110,5.974134891383233)
(111,5.955812895881109)
(112,5.937420230175251)
(113,5.9189604925332375)
(114,5.900435500996035)
(115,5.881814695282706)
(116,5.8631132600438844)
(117,5.8443975022639565)
(118,5.825669716106254)
(119,5.806931346411407)
(120,5.788137282007762)
(121,5.769290013827989)
(122,5.750391217071285)
(123,5.7313312876144265)
(124,5.712119241561277)
(125,5.692763481548353)
(126,5.6732718189835065)
(127,5.653651480729689)
(128,5.633909110123747)
(129,5.614050768747075)
(130,5.594081942797848)
(131,5.574007556005396)
(132,5.553831989643474)
(133,5.533559109250085)
(134,5.513192297069295)
(135,5.492734488920092)
(136,5.472188214098125)
(137,5.451555636963526)
(138,5.430838599007848)
(139,5.410038660381831)
(140,5.389157140071637)
(141,5.368195154112285)
(142,5.347153651409994)
(143,5.326033446902893)
(144,5.304835251919728)
(145,5.283559701699293)
(146,5.262207380111723)
(147,5.2407788416798065)
(148,5.219274631037527)
(149,5.197672437295372)
(150,5.175970520592031)
(151,5.154167161853462)
(152,5.13226067332589)
(153,5.110249408626393)
(154,5.0881688398692315)
(155,5.06601732551165)
(156,5.043793266659102)
(157,5.021495112993182)
(158,4.999121368886506)
(159,4.976670599722386)
(160,4.954141438428249)
(161,5.191006290169998)
(162,5.14400742059719)
(163,5.101685252488614)
(164,5.062686948459874)
(165,5.026166818537952)
(166,4.991556348247825)
(167,4.958452797003244)
(168,4.926559254683935)
(169,4.8956498277572855)
(170,4.865548209565154)
(171,4.836113870945296)
(172,4.807232825507383)
(173,4.778811261861212)
(174,4.75077103748358)
(175,4.723046417753763)
(176,4.69558166879905)
(177,4.668329248206368)
(178,4.641248421876633)
(179,4.614304189242829)
(180,4.587466434543097)
(181,4.560709245717947)
(182,4.534010358917163)
(183,4.507350698101204)
(184,4.480713987408027)
(185,4.4540864198622225)
(186,4.427456370308532)
(187,4.523628310287821)
(188,4.486699536043139)
(189,4.450735375209012)
(190,4.415594926155893)
(191,4.3811593359719705)
(192,4.347327786637241)
(193,4.314014344949306)
(194,4.281145460788181)
(195,4.005859145840157)
(196,3.972415578682953)
(197,3.9405614222545506)
(198,3.909921812436536)
(199,4.263244595975296)
(200,4.029416734861362)
(201,3.980400660806118)
(202,4.084531707220658)
(203,4.035785177933581)
(204,3.9912066827646253)
(205,3.9497273338394945)
(206,3.9106207007141363)
(207,3.873350165046371)
(208,4.500610942131567)
(209,4.382741418052798)
(210,4.289006051318851)
(211,4.210186311141098)
(212,4.141469355460519)
(213,4.08001262507038)
(214,4.023996091756889)
(215,3.972185399017401)
(216,4.702976810796432)
(217,3.973982463034967)
(218,3.941531309340503)
(219,3.909378112447463)
(220,3.8774927945428)
(221,3.8458481859987703)
(222,3.8144196918512683)
(223,3.6510533435461836)
(224,3.605777992580194)
(225,3.562462584768049)
(226,3.5208017039223325)
(227,3.5216778254756242)
(228,3.483596193312658)
(229,3.446435554704156)
(230,3.410077741596372)
(231,3.3744222782199045)
(232,3.339383004446229)
(233,3.3048854778457737)
(234,3.2708649526980746)
(235,3.2372647915263677)
(236,3.2040352042278633)
(237,3.113321495389751)
(238,3.081568907248121)
(239,3.049925497322219)
(240,3.018380630744046)
(241,2.986924613574507)
(242,2.955548605330029)
(243,2.9242445407350783)
(244,2.893005059782424)
(245,3.1302590888276267)
(246,3.0777786531572806)
(247,3.125531047413423)
(248,3.081057294496985)
(249,3.038765597078614)
(250,2.9982768037763634)
(251,3.048133011073017)
(252,2.9874254642455775)
(253,2.930994445143802)
(254,2.9042713653602363)
(255,2.8261745151584723)
(256,2.7559484070887637)
(257,2.691974451659706)
(258,2.63305117657447)
(259,3.115940764934308)
(260,3.0382470008399087)
(261,2.658471041797533)
(262,2.6118584341256588)
(263,2.5514243927685865)
(264,2.5104023563847413)
(265,2.470626244775206)
(266,2.4319313561721994)
(267,2.277200480132656)
(268,2.229708094428129)
(269,2.2451607839577488)
(270,2.193382485481482)
(271,2.1444656877870827)
(272,2.0979477373099265)
(273,2.5831564459708494)
(274,2.480059236323189)
(275,2.3958002969223973)
(276,2.3208245164511934)
(277,2.2530168879376036)
(278,2.1908675413037537)
(279,2.002981987163964)
(280,1.9642225275856824)
(281,2.843614938198219)
(282,3.767420747621469)
(283,3.3271713759724157)
(284,3.0275028106828152)
(285,2.2146424619016223)
(286,2.168139797246237)
(287,2.1234762854126585)
(288,1.772367259344571)
(289,1.558763984372561)
(290,4.026440608499908)
(291,1.7470822699466395)
(292,1.7022536651032665)
(293,2.0104379056405657)
(294,3.5211410798808203)
(295,1.9357814309063113)
(296,2.576924560852675)
(297,2.4338088531090003)
(298,2.3104639087600325)
(299,2.202390576643417)
(300,1.643876472336153)
(301,2.0376994688872054)
(302,1.9559370369776303)
(303,1.881551230687668)
(304,1.4646802748928107)
(305,1.4322270341313428)
(306,1.4002817575600877)
(307,1.538152513392606)
(308,2.1326566247594196)
(309,1.9946994163910778)
(310,1.6191682553938982)
(311,1.5654212559277996)
(312,1.3900217070447454)
(313,1.794292981671723)
(314,1.8296160059799795)
(315,1.5117221975527868)
(316,1.4409794973813466)
(317,1.37770959608865)
(318,1.3884759726456242)
(319,1.368417034468942)
(320,1.4609430643619852)
(321,3.1543067563725993)
(322,2.8229319236587207)
(323,2.1489985351249796)
(324,1.2000155576852638)
(325,1.823113149843068)
(326,1.718316616364271)
(327,1.7583250326946156)
(328,2.1543943460574235)
(329,1.7000739593037353)
(330,1.2583054306861852)
(331,1.8772732493472462)
(332,1.9023700355421629)
(333,1.4931731678028215)
(334,1.4177121064463907)
(335,4.35404217191541)
(336,3.724022457719723)
(337,1.6632619819292422)
(338,1.0115530158327066)
(339,1.209429606534667)
(340,1.0899258237804499)
(341,1.0411466276251813)
(342,0.9949294440980541)
(343,2.1525201638048124)
(344,1.0144944025719815)
(345,1.1478229181228012)
(346,0.8549847909862858)
(347,0.8188406867811859)
(348,1.2020894041956345)
(349,0.9133880455654195)
(350,0.8758948820368333)
(351,1.349495046291947)
(352,1.6537877982880178)
(353,1.2209582058788802)
(354,1.1297223217379055)
(355,0.641659675646016)
(356,0.6144231743221713)
(357,0.5873361016352829)
(358,1.091019124122061)
(359,0.5524267578365096)
(360,1.9145902051505095)
(361,0.6600472954294019)
(362,0.6075864448246355)
(363,0.5597175613979284)
(364,2.3032838574682764)
(365,0.4739583779233134)
(366,0.425992269109581)
(367,0.3948567041061888)
(368,0.45585814916588996)
(369,0.6636321310419044)
(370,0.8805527540878841)
(371,0.7853068984095586)
(372,0.2688720148387491)
(373,1.0394279326535099)
(374,1.078383560190698)
(375,0.7759538449525651)
(376,0.5557127047709409)
(377,1.692419039690561)
(378,0.5915741982235798)
(379,0.545540458288871)
(380,0.4665766561736018)
(381,0.48475186411906757)
(382,0.6258998333693229)
(383,1.8122109723359905)
(384,0.15025085992354215)
(385,0.18150127708681896)
(386,0.1104414684770707)
(387,0.07006249359197803)
(388,2.6360729331677035)
(389,0.024591998671976478)
(390,0.7696743073428292)
(391,2.9812211171125043)
(392,0.23352495670551687)
(393,0.22129087905599465)
(394,2.0984651920505675)
(395,0.40576650266754566)
(396,0.3171731536815612)
(397,0.2386083212652199)
(398,0.16816427327365968)
(399,0.10931911720543108)
(400,0.9878608008834779)
(401,0.12572908460932575)
(402,-0.13569440406112454)
(403,0.006907472182684948)
(404,-0.05257846578602554)
(405,-0.10576934101508861)
(406,0.8744028410352778)
(407,0.12428320652976588)
(408,-0.06869875478205256)
(409,3.1316307492025066)
(410,2.4860339962830436)
(411,1.8820243448801854)
(412,0.351132337869247)
(413,0.23047505188987638)
(414,1.556007048904318)
(415,-0.11847166018741428)
(416,-0.10122514693619467)
(417,0.8528023821003083)
(418,-0.08637339999537819)
(419,2.31384077831202)
(420,1.776107239722026)
(421,0.5750110010158973)
(422,-0.02161128423735148)
(423,-0.08292887294337903)
(424,-0.13957006772527425)
(425,-0.19200685272457463)
(426,0.227492658965748)
(427,1.2370412873647356)
(428,0.7160783807828484)
(429,0.6578035250247485)
(430,-0.2968533062634071)
(431,-0.3416178088519879)
(432,-0.38384156618059784)
(433,0.008311855481129449)
(434,0.09274084200232235)
(435,-0.03759766553210597)
(436,-0.3581244790550582)
(437,0.23508374492251694)
(438,1.392011611352374)
(439,-0.0411825335396272)
(440,-0.03952189179099408)
(441,-0.29039366375216336)
(442,-0.3855057019628827)
(443,-0.4617464608351649)
(444,2.9573101128370927)
(445,1.0018923184634239)
(446,0.09443147524716622)
(447,2.9036718293512527)
(448,2.1688564767694034)
(449,0.13822116216131408)
(450,0.028210459320987313)
(451,-0.07303973013586207)
(452,0.017034298005684567)
(453,0.19332662995036198)
(454,0.7517491833275733)
(455,3.083696637535374)
(456,2.391864618437262)
(457,0.30828008411965646)
(458,0.014317085133737306)
(459,-0.5142396370233577)
(460,-0.5850012414436001)
(461,1.000046564537687)
(462,-0.24062928304012265)
(463,-0.36662412567792835)
(464,-0.45393076756246975)
(465,0.24397633727188506)
(466,0.6190879346999232)
(467,-0.385917415695445)
(468,-0.08419553090693865)
(469,3.0227045395300403)
(470,-0.22207985715518164)
(471,-0.3433161996376626)
(472,-0.04109457469621014)
(473,-0.761729776690127)
(474,2.5999045314088245)
(475,0.6830857009678768)
(476,-0.07714505100396685)
(477,0.33030066809627456)
(478,-0.7859522790800566)
(479,0.9405290166038587)
(480,-0.4045168193026836)
(481,-0.8095658867480129)
(482,-0.67045180683307)
(483,-0.60258136347103)
(484,-0.10938010987056446)
(485,1.581815228730805)
(486,-0.5957596246520283)
(487,1.1136302615632039)
(488,0.2912497163493506)
(489,0.8898564850206743)
(490,0.00017697424080620472)
(491,-0.619213003055177)
(492,-0.6717402257958227)
(493,-0.5564940368185518)
(494,0.6456044236658014)
(495,2.4695107372333664)
(496,1.824465378233932)
(497,0.05410644315973134)
(498,-0.9500824729082895)
(499,0.051068480661035975)
(500,0.45787619518298217)
    };
    \addlegendentry{\(\DKL{p}{q_{\vlambda_t}}\)}

    %% Thick lines
    \addplot [thick, blue, mark size=2pt] coordinates {
(1,-0.7836035098126551)
(2,2.035991363823083)
(3,2.035991363823083)
(4,2.035991363823083)
(5,2.035991363823083)
(6,0.6321787246883416)
(7,0.6321787246883416)
(8,0.6321787246883416)
(9,-0.3012293748518743)
(10,-0.3012293748518743)
(11,-0.3012293748518743)
(12,-0.3012293748518743)
(13,-0.3012293748518743)
(14,-1.8777140557488892)
(15,-1.8777140557488892)
(16,0.49772396735363167)
(17,0.49772396735363167)
(18,0.49772396735363167)
(19,-1.2492169820212973)
(20,-1.2492169820212973)
(21,-1.2492169820212973)
(22,1.365452349709312)
(23,1.365452349709312)
(24,1.365452349709312)
(25,1.365452349709312)
(26,1.365452349709312)
(27,1.365452349709312)
(28,1.365452349709312)
(29,1.365452349709312)
(30,1.365452349709312)
(31,1.365452349709312)
(32,1.365452349709312)
(33,1.365452349709312)
(34,1.365452349709312)
(35,1.365452349709312)
(36,1.365452349709312)
(37,1.365452349709312)
(38,1.365452349709312)
(39,1.365452349709312)
(40,1.365452349709312)
(41,1.365452349709312)
(42,1.365452349709312)
(43,1.365452349709312)
(44,1.365452349709312)
(45,1.365452349709312)
(46,1.365452349709312)
(47,1.365452349709312)
(48,0.43069157861106633)
(49,0.43069157861106633)
(50,0.43069157861106633)
(51,0.43069157861106633)
(52,0.43069157861106633)
(53,0.9298162353147278)
(54,0.9298162353147278)
(55,0.9298162353147278)
(56,0.9298162353147278)
(57,0.9298162353147278)
(58,0.9298162353147278)
(59,0.9298162353147278)
(60,0.9298162353147278)
(61,1.3219492373797337)
(62,1.3219492373797337)
(63,1.3219492373797337)
(64,1.3219492373797337)
(65,1.3219492373797337)
(66,1.3219492373797337)
(67,1.3219492373797337)
(68,1.3219492373797337)
(69,1.3219492373797337)
(70,1.3219492373797337)
(71,1.3219492373797337)
(72,1.3219492373797337)
(73,1.3219492373797337)
(74,1.3219492373797337)
(75,1.3219492373797337)
(76,1.3219492373797337)
(77,1.3219492373797337)
(78,1.3219492373797337)
(79,1.3219492373797337)
(80,1.3219492373797337)
(81,1.3219492373797337)
(82,1.3219492373797337)
(83,1.3219492373797337)
(84,1.3219492373797337)
(85,1.3219492373797337)
(86,1.3219492373797337)
(87,1.3219492373797337)
(88,1.3219492373797337)
(89,1.3219492373797337)
(90,1.3219492373797337)
(91,1.3219492373797337)
(92,1.3219492373797337)
(93,1.3219492373797337)
(94,1.3219492373797337)
(95,1.3219492373797337)
(96,1.3219492373797337)
(97,1.3219492373797337)
(98,1.3219492373797337)
(99,1.3219492373797337)
(100,1.3219492373797337)
(101,1.3219492373797337)
(102,1.3219492373797337)
(103,1.3219492373797337)
(104,1.3219492373797337)
(105,1.0217790531112807)
(106,1.0217790531112807)
(107,1.0217790531112807)
(108,1.0217790531112807)
(109,1.0217790531112807)
(110,1.0217790531112807)
(111,1.0217790531112807)
(112,1.0217790531112807)
(113,1.0217790531112807)
(114,1.0217790531112807)
(115,1.0217790531112807)
(116,1.0217790531112807)
(117,1.0217790531112807)
(118,1.0217790531112807)
(119,1.0217790531112807)
(120,1.0217790531112807)
(121,1.0217790531112807)
(122,1.0217790531112807)
(123,1.0217790531112807)
(124,1.0217790531112807)
(125,1.0217790531112807)
(126,1.0217790531112807)
(127,1.0217790531112807)
(128,1.0217790531112807)
(129,1.0217790531112807)
(130,1.0217790531112807)
(131,1.0217790531112807)
(132,1.0217790531112807)
(133,1.0217790531112807)
(134,1.0217790531112807)
(135,1.0217790531112807)
(136,1.0217790531112807)
(137,1.0217790531112807)
(138,1.0217790531112807)
(139,1.0217790531112807)
(140,1.0217790531112807)
(141,1.0217790531112807)
(142,1.0217790531112807)
(143,1.0217790531112807)
(144,1.0217790531112807)
(145,1.0217790531112807)
(146,1.0217790531112807)
(147,1.0217790531112807)
(148,1.0217790531112807)
(149,1.0217790531112807)
(150,1.0217790531112807)
(151,1.0217790531112807)
(152,1.0217790531112807)
(153,1.0217790531112807)
(154,1.0217790531112807)
(155,1.0217790531112807)
(156,1.0217790531112807)
(157,1.0217790531112807)
(158,1.0217790531112807)
(159,1.0217790531112807)
(160,1.0217790531112807)
(161,1.2423114854786013)
(162,1.2423114854786013)
(163,1.2423114854786013)
(164,1.2423114854786013)
(165,1.2423114854786013)
(166,1.2423114854786013)
(167,1.2423114854786013)
(168,1.2423114854786013)
(169,1.2423114854786013)
(170,1.2423114854786013)
(171,1.2423114854786013)
(172,1.2423114854786013)
(173,1.2423114854786013)
(174,1.2423114854786013)
(175,1.2423114854786013)
(176,1.2423114854786013)
(177,1.2423114854786013)
(178,1.2423114854786013)
(179,1.2423114854786013)
(180,1.2423114854786013)
(181,1.2423114854786013)
(182,1.2423114854786013)
(183,1.2423114854786013)
(184,1.2423114854786013)
(185,1.2423114854786013)
(186,1.2423114854786013)
(187,1.5558869695683315)
(188,1.5558869695683315)
(189,1.5558869695683315)
(190,1.5558869695683315)
(191,1.5558869695683315)
(192,1.5558869695683315)
(193,1.5558869695683315)
(194,1.5558869695683315)
(195,0.917279229972826)
(196,0.917279229972826)
(197,0.917279229972826)
(198,0.917279229972826)
(199,0.9141680731274467)
(200,1.3929640494285869)
(201,1.3929640494285869)
(202,0.6671376490756477)
(203,0.6671376490756477)
(204,0.6671376490756477)
(205,0.6671376490756477)
(206,0.6671376490756477)
(207,0.6671376490756477)
(208,1.2992281257769123)
(209,1.2992281257769123)
(210,1.2992281257769123)
(211,1.2992281257769123)
(212,1.2992281257769123)
(213,1.2992281257769123)
(214,1.2992281257769123)
(215,1.2992281257769123)
(216,1.38085432722204)
(217,0.8512523703023112)
(218,0.8512523703023112)
(219,0.8512523703023112)
(220,0.8512523703023112)
(221,0.8512523703023112)
(222,0.8512523703023112)
(223,1.2375331120063762)
(224,1.2375331120063762)
(225,1.2375331120063762)
(226,1.2375331120063762)
(227,0.8718031765565023)
(228,0.8718031765565023)
(229,0.8718031765565023)
(230,0.8718031765565023)
(231,0.8718031765565023)
(232,0.8718031765565023)
(233,0.8718031765565023)
(234,0.8718031765565023)
(235,0.8718031765565023)
(236,0.8718031765565023)
(237,0.8598782705021417)
(238,0.8598782705021417)
(239,0.8598782705021417)
(240,0.8598782705021417)
(241,0.8598782705021417)
(242,0.8598782705021417)
(243,0.8598782705021417)
(244,0.8598782705021417)
(245,0.8602453845752773)
(246,0.8602453845752773)
(247,1.0238970449141176)
(248,1.0238970449141176)
(249,1.0238970449141176)
(250,1.0238970449141176)
(251,0.615345855333135)
(252,0.615345855333135)
(253,0.615345855333135)
(254,1.1864181415470922)
(255,1.1864181415470922)
(256,1.1864181415470922)
(257,1.1864181415470922)
(258,1.1864181415470922)
(259,1.3523556082265717)
(260,1.3523556082265717)
(261,1.0485389868084112)
(262,1.0485389868084112)
(263,1.1697282083534306)
(264,1.1697282083534306)
(265,1.1697282083534306)
(266,1.1697282083534306)
(267,1.1920431609275126)
(268,1.1920431609275126)
(269,0.8215685695767172)
(270,0.8215685695767172)
(271,0.8215685695767172)
(272,0.8215685695767172)
(273,1.0965697758755706)
(274,0.6653053464358426)
(275,0.6653053464358426)
(276,0.6653053464358426)
(277,0.6653053464358426)
(278,0.6653053464358426)
(279,0.7621589885945609)
(280,0.7621589885945609)
(281,0.8022915450627883)
(282,1.141302987012327)
(283,1.141302987012327)
(284,1.141302987012327)
(285,0.9521612407961612)
(286,0.9521612407961612)
(287,0.9521612407961612)
(288,0.9786434630302652)
(289,1.0113722094148245)
(290,1.1387616593030225)
(291,1.0194288165757037)
(292,1.0194288165757037)
(293,0.7338430708907666)
(294,0.7342102067382161)
(295,0.9688077394844294)
(296,1.1589067320445334)
(297,1.1589067320445334)
(298,1.1589067320445334)
(299,1.1589067320445334)
(300,1.1820548728034777)
(301,1.232737289725764)
(302,1.232737289725764)
(303,1.232737289725764)
(304,1.0069588977659214)
(305,1.0069588977659214)
(306,1.0069588977659214)
(307,1.204454302073188)
(308,1.4257646024460569)
(309,1.4257646024460569)
(310,1.1691234763524792)
(311,1.1691234763524792)
(312,1.131638306203807)
(313,1.1197028150359518)
(314,1.301146589385755)
(315,0.9169187915646957)
(316,0.9169187915646957)
(317,0.9169187915646957)
(318,0.9797137782091114)
(319,1.1712284645399045)
(320,0.8321721890164198)
(321,1.2955644325203428)
(322,1.3383726728876062)
(323,1.4972363015245684)
(324,1.1248668333305)
(325,0.8683065995807949)
(326,0.8683065995807949)
(327,0.7956041382138539)
(328,1.358134831688424)
(329,1.2769599375220633)
(330,1.0671409871496715)
(331,0.8609045882832959)
(332,1.4485112701843903)
(333,0.8990298532061805)
(334,0.8990298532061805)
(335,1.2058569311475982)
(336,1.2058569311475982)
(337,0.8630943572977956)
(338,1.0217255300149832)
(339,1.0914738186081758)
(340,1.2219782138022404)
(341,1.2219782138022404)
(342,1.2219782138022404)
(343,1.5100422455378946)
(344,1.2904142502102058)
(345,1.138068730326576)
(346,1.0597230834815095)
(347,1.0597230834815095)
(348,1.374819199179429)
(349,1.078128169916082)
(350,1.0439570347355858)
(351,0.9003127862878553)
(352,0.8812992363748344)
(353,1.0383411366286026)
(354,0.926266847345808)
(355,1.062114844954737)
(356,1.062114844954737)
(357,1.062114844954737)
(358,0.9097118072447123)
(359,1.1082727825327237)
(360,1.4345936394355654)
(361,1.1247649991410194)
(362,1.1247649991410194)
(363,1.1247649991410194)
(364,1.5047643543387765)
(365,1.0617187172321227)
(366,1.188441199956972)
(367,1.188441199956972)
(368,1.0754706514554258)
(369,1.0624489759982447)
(370,1.297249545904749)
(371,1.297249545904749)
(372,1.1712919685037178)
(373,0.941573439530913)
(374,1.09470888858677)
(375,1.1046447229987544)
(376,1.0153652766370542)
(377,1.2960592902295545)
(378,1.2391938711016863)
(379,1.2447345447444875)
(380,1.2447345447444875)
(381,1.1556260776690557)
(382,1.133378534506841)
(383,0.8209624283966684)
(384,1.1850256194070439)
(385,1.0991881262041623)
(386,1.2134874652646086)
(387,1.2134874652646086)
(388,1.1892064417068284)
(389,1.1481481141483818)
(390,0.9878153815716408)
(391,0.9571889328631439)
(392,1.118990178614732)
(393,1.1483815288570796)
(394,1.384153657633736)
(395,1.0269672316016467)
(396,1.0269672316016467)
(397,1.0269672316016467)
(398,1.0269672316016467)
(399,1.0003296596140638)
(400,0.8392878507619297)
(401,1.0714315731130195)
(402,1.0547350841185803)
(403,0.9534868531517007)
(404,0.9534868531517007)
(405,0.9534868531517007)
(406,1.2834966977736695)
(407,1.1205108907923005)
(408,1.1718913961062563)
(409,0.9332171906058835)
(410,0.669682281599536)
(411,0.669682281599536)
(412,1.1846753755980906)
(413,1.1846753755980906)
(414,1.3149131875354845)
(415,1.0552961635875633)
(416,0.9951612017223168)
(417,1.2230202063107545)
(418,1.0278618211217445)
(419,0.7303893805359929)
(420,0.7303893805359929)
(421,1.093809919068426)
(422,1.1507198901338678)
(423,1.1507198901338678)
(424,1.1507198901338678)
(425,1.1507198901338678)
(426,1.1159925164326387)
(427,1.3769454642059349)
(428,1.2086722511232948)
(429,1.1324152111039927)
(430,0.9808544683371063)
(431,0.9808544683371063)
(432,0.9808544683371063)
(433,0.9903237331801154)
(434,1.2663685352882672)
(435,1.2663685352882672)
(436,1.1173240325058253)
(437,0.9894117930357167)
(438,1.026738386655571)
(439,1.0786834051101255)
(440,1.1153864147929506)
(441,1.027915039496469)
(442,1.027915039496469)
(443,1.027915039496469)
(444,1.4763798467177613)
(445,1.400748062406156)
(446,1.1828165480136437)
(447,1.0536987112715641)
(448,1.0536987112715641)
(449,1.0828435546475377)
(450,1.0828435546475377)
(451,1.300246180643236)
(452,1.1095900090964819)
(453,1.1684394446821238)
(454,1.2936375115678391)
(455,0.7109622697318552)
(456,0.7109622697318552)
(457,0.9199032460664299)
(458,1.1727856906995684)
(459,0.9970587963403374)
(460,1.0206561881101022)
(461,0.7851547100892711)
(462,1.133158672543418)
(463,1.133158672543418)
(464,1.133158672543418)
(465,0.855813824225716)
(466,1.243063760003317)
(467,1.1280858390397255)
(468,0.9867408262524181)
(469,0.7076838750366214)
(470,1.1590110334393315)
(471,0.9212694260210628)
(472,0.846414419150247)
(473,1.0059388316841902)
(474,0.8728836975951859)
(475,1.0197344982111214)
(476,1.0523988021223154)
(477,0.8097930454238312)
(478,1.0204220614604624)
(479,1.2591748381594314)
(480,1.0992764592749822)
(481,0.9680888261672947)
(482,1.1055218828498181)
(483,0.8911791304244522)
(484,1.0155665745786766)
(485,1.3857626146152513)
(486,1.118732789648739)
(487,1.2849564032155365)
(488,1.11542993495538)
(489,1.0424558554548127)
(490,1.2064049893734383)
(491,0.9375283835114513)
(492,1.0229665749070236)
(493,1.0525070946623374)
(494,1.2039629009370145)
(495,0.669623254608579)
(496,0.669623254608579)
(497,1.06407757229544)
(498,1.0328950727637332)
(499,1.2503460189968274)
(500,1.311060036615137)
    };
    \addlegendentry{\(\vz_t\)}


  \end{axis}
\end{tikzpicture}

  %%     \caption{KL divergence and trace of \(\vz_t\) of MSC with a CIS kernel.
  %%       \(\vz_t\) barely moves until \(t=250\) around which \(\DKL{p}{q_{\vlambda}}\) starts to converge.}\label{fig:motivating}
  %%   \end{figure}
  %%   \vspace{-0.1in}
  %% \end{minipage}
%
\subsection{Stochastic Approximation with Markov-Chain Monte Carlo}\label{section:msc}
%
%\vspace{-0.1in}
\paragraph{MSC and JSA}
Recently,~\citeauthor{NEURIPS2020_b2070693} and~\citeauthor{pmlr-v124-ou20a} proposed two similar but independent methods for performing inclusive variational inference.
Both methods operate a Markov-chain in parallel with the VI optimization sequence, but the formulation is slightly different.
\citeauthor{NEURIPS2020_b2070693} proposed \textit{Markovian score climbing} (MSC) which operates a MCMC kernel leaving \(p\left(\vz \mid \vx \right)\) invariant.
Also, for the MCMC kernel, they propose conditional importance sampling (CIS), which is inspired by the particle MCMC method by~\cite{andrieu_particle_2010}.

On the other hand,~\cite{pmlr-v124-ou20a} propose \textit{joint stochastic approximation} (JSA) which operats a MCMC kernel leaving the likelihoods of the independent data points \(p\left(\vz_i \mid \vx_i \right)\) invariant.
For the MCMC kernel, unlike~\citeauthor{NEURIPS2020_b2070693}, they use the independence Metropolis Hastings (IMH) sampler.
Also, they operate multiple 


They showed that MSC achieves better and more robust performance compared to methods such as SNIS and expectation propagation (EP, \citealt{10.5555/2074022.2074067}).
MSC is described in~\cref{alg:msc}.
It obtains stochastic gradients from a Markov-chain \(\{\,\vz_1, \vz_2, \ldots \vz_T\,\}\) generated from a \(\pi\)-invariant transition operator \(K(\vz, \cdot)\) running in parallel with the VI optimizer (represented by the sequence \(\{\,\vlambda_1, \vlambda_2, \ldots, \vlambda_T\,\}\)).
\citeauthor{NEURIPS2020_b2070693} specifically proposed to use the \textit{conditional importance sampling} (CIS) kernel for MSC.

%If \(K(\vz, \cdot)\) is a proper Markov-chain Monte Carlo (MCMC) kernel,~\citeauthor{NEURIPS2020_b2070693} note that the gradient estimates become asymptotically unbiased with \(t \rightarrow \infty\).

%% \vspace{-0.1in}
%% \paragraph{A Motivating Mystery}
%% Even though~\citeauthor{NEURIPS2020_b2070693} suggested that \textit{any} good MCMC kernel would work, we present a counterexample showing that the CIS kernel may operate unusually when used for MSC.
%% In fact, \textit{MSC makes the most progress when the CIS-generated Markov-chain is the least effective}.
%% In~\cref{fig:motivating}, we show the trace plot of \(\vz_t\) (\textcolor{blue}{blue line}) and \(\DKL{p}{q_{\vlambda}}\) (\textcolor{red}{red line}).
%% Notice that \(\vz_t\) barely moves until iteration 250, around which \(\DKL{p}{q_{\vlambda}}\) starts to converge.
%% In a traditional accept-reject MCMC kernel view, this implies that most of the samples are rejected, making the kernel no longer statistically effective.
%% However, our analysis will show that this is actually a \textit{positive feature} of the CIS kernel, contributing to the empirical success of MSC.


%%% Local Variables:
%%% TeX-master: "master"
%%% End:
