\vspace{-0.1in}
\section{Related Works}\label{section:related}
\vspace{-0.05in}
\paragraph{Inclusive KL minimization}
Our method directly builds on top of MSC~\citep{NEURIPS2020_b2070693}, which minimizes the inclusive KL divergence.
Concurrently, in the context of variational autoencoders with discrete latent variables,~\citet{pmlr-v124-ou20a} proposed JSA.
JSA can be viewed as a variant of MSC that takes advantage of models with \textit{i.i.d.} data likelihoods.
Meanwhile,~\citet{li_approximate_2017} used a similar approach to MSC, but their method is slow to achieve stationarity.
Other than using MCMC,~\citet{DBLP:journals/corr/BornscheinB14,le_revisiting_2019} used SNIS while~\citet{pmlr-v119-wu20h} used sequential Monte Carlo (SMC) for estimating the score, but these methods are restricted to deep generative models.
On a different note,~\citet{pmlr-v161-jerfel21a} used boosting instead of SGD to minimize the inclusive KL, which results in a more flexible variational family.
%Lastly,~\citet{10.5555/2074022.2074067} introduced expectation propagation.

\vspace{-0.1in}
\paragraph{MCMC for VI, VI for MCMC}
MCMC has been widely utilized in VI, not only for inclusive KL minimization.
For example,~\citet{pmlr-v37-salimans15, pmlr-v97-ruiz19a} construct alternative divergence bounds from samples of an MCMC sampler.
More recently, several methods that apply VI to adaptating MCMC kernels have been developed.
For adapting IMH kernels, \citet{habib2018auxiliary} minimize the exclusive KL divergence while~\cite{neklyudov_metropolishastings_2019} minimize the symmetric KL divergence.
And for HMC,~\citet{zhang_variational_2018, pmlr-v139-campbell21a} have proposed to use score matching, ELBO maximization, and Stein discrepancy minimization.

\vspace{-0.1in}
\paragraph{Adaptive MCMC}
As pointed out by~\citet{pmlr-v124-ou20a}, using \(q_{\vlambda}\) within the MCMC kernel makes MCSA structurally equivalent to adaptive MCMC.
In particular,~\citet{10.1007/s11222-008-9110-y, garthwaite_adaptive_2016} discuss the use of stochastic approximation in adaptive MCMC.
Also,~\citet{andrieu_ergodicity_2006, keith_adaptive_2008, holden_adaptive_2009, giordani_adaptive_2010} specifically discuss adapting the propsosal of IMH kernels.
Most similar to score climbing VI is the work of~\citet{keith_adaptive_2008} where they propose to use cross-entropy minimization~\citep{barbakh_cross_2009}, which is mathematically identical to inclusive VI.


%%% Local Variables:
%%% TeX-master: "master"
%%% End:
