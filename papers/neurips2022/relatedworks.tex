\vspace{-1.5ex}
\section{Related Works}\label{section:related}
\vspace{-1.5ex}
\paragraph{Inclusive KL minimization}
Our MCSA framework generalizes MSC~\citep{NEURIPS2020_b2070693} and JSA~\citet{pmlr-v124-ou20a}, which are inclusive KL minimization based on SGD and Markov chains.
Similar to MCSA is the method of~\citet{li_approximate_2017}.
However, the convergence of this method is not guaranteed since it uses short Markov chains, disqualifying for MCSA.
Other methods based on biased gradients have been proposed by~\citet{DBLP:journals/corr/BornscheinB14,le_revisiting_2020}, but these are specific for deep generative models.
On a different note,~\citet{pmlr-v161-jerfel21a} use boosting instead of SGD to minimize the inclusive KL, which gradually builds a complex variational approximation from a simple variational family.

\vspace{-1.5ex}
\paragraph{Beyond the KL Divergence}
Discovering alternative divergences for VI has been an active research area.
For example, the \(\chi^2\)~\citep{NIPS2017_35464c84}, \(f\)~\citep{NEURIPS2018_1cd138d0, NEURIPS2020_c928d86f}, \(\alpha\)~\citep{NIPS2016_7750ca35, regli_alphabeta_2018, pmlr-v48-hernandez-lobatob16}, reguarlized importance ratio~\citep{bamler_perturbative_2017} divergences have been studied for VI.
However, for gradient estimation these methods involve the importance ratio \(w\left(\vz\right) = \pi\left(\vz\right)/q\left(\vz\right)\), which leads to significant variance and low signal-to-noise ratio under model misspecification \citep{bamler_perturbative_2017, geffner2021empirical, pmlr-v139-geffner21a}.
In contrast, under stationarity, the variance of pMCSA is \({\sigma^2}/{N}\) (\(\sigma\) is the variance of the score over the posterior) regardless of model misspecification.
Meanwhile,~\citet{pmlr-v37-salimans15,pmlr-v97-ruiz19a,NEURIPS2021_05f971b5,NEURIPS2021_a1a609f1} construct implicit divergences formed by MCMC.
With the exception of \citet{pmlr-v97-ruiz19a}, most of these approaches aim to maximize auxiliary representations of the classic ELBO.
Therefore, their property is likely to be similar to the exclusive KL.

\vspace{-1.5ex}
\paragraph{Adaptive MCMC and MCSA}
As pointed out by~\citet{pmlr-v124-ou20a}, using \(q\left(\cdot; \vlambda\right)\) within the MCMC kernel makes MCSA structurally equivalent to adaptive MCMC.
In particular,~\citet{10.1007/s11222-008-9110-y, garthwaite_adaptive_2016, pmlr-v151-brofos22a, gabrie_adaptive_2022} discuss the use of stochastic approximation in adaptive MCMC.
Also,~\citet{andrieu_ergodicity_2006, keith_adaptive_2008, holden_adaptive_2009, giordani_adaptive_2010, pmlr-v151-brofos22a, habib2018auxiliary, neklyudov_metropolishastings_2019} specifically discuss adapting the propsosal of IMH kernels, and some of them use KL divergence minimization.
These methods focus on showing ergodicity the samples (\(\veta_t\) in our context) not the convergence of the variational approximation \(q\left(\cdot; \vlambda\right)\).
In this work, we focused on the convergence of \(q\left(\cdot; \vlambda\right)\), which could advance the adaptive MCMC side of the story.

%And for HMC,~\citet{zhang_variational_2018, pmlr-v139-campbell21a} have proposed to use score matching, ELBO maximization, and Stein discrepancy minimization.


%%% Local Variables:
%%% TeX-master: "master"
%%% End:
