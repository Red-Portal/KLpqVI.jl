
%\vspace{-0.05in}
\section{Practical Convergence Analysis of Markov Chain Score Ascent}
\subsection{Convergence of General Markov Chain Score Ascent}\label{section:convergence}

The basic ingredients of MCGD are the target function \(f\left(\vlambda, \eta\right)\), the gradient estimator \(g\left(\vlambda, \eta\right)\), and the Markov chain kernel \(P_{\vlambda}\left(\eta, \cdot\right)\).
Obtaining MCSA through MCGD boils down to designing \(g\) and \(P_{\vlambda}\) such that \(f\left(\vlambda\right) = \DKL{\pi}{q\left(\cdot; \vlambda\right)} \).
In the following theorem, we provide practical conditions for setting \(g\) and \(P_{\vlambda}\) such that MCGD results in MCSA.

\begin{condition}{\textbf{(Markov chain kernel)}}\label{thm:kernel_conditions}
  Let \(\veta = \left[ \vz_1, \vz_2, \ldots, \vz_N \right]\).
  A Markov chain kernel \(P_{\vlambda}\left(\veta, \cdot\right)\) is \(\Pi\)-invariant and geometrically ergodic as
  \[
  \DTV{P_{\vlambda}^{n}\left(\veta, \cdot\right)}{ \Pi } \leq C \, \rho^{n}
  \]
  for some positive constant \(C\) where its invariant distribution is defined as
  \[
  \Pi\left(\veta\right) = \pi\left(\vz_1\right) \, \pi\left(\vz_2\right) \times \ldots \times \pi\left(\vz_N\right).
  \]
\end{condition}

\begin{condition}{\textbf{(Gradient estimator)}}\label{thm:gradient_estimator}
  The target function \(f\) and the gradient estimator \(g\) are of the form of
  \begin{align*}
    f\left(\vlambda, \veta\right) =  \frac{1}{N} \sum^{N}_{n=1} \log q\left(\vz_n; \vlambda\right) + \mathbb{H}\left[\,\pi\,\right], 
    \quad\text{and}\quad
    g\left(\vlambda, \veta\right) =  \frac{1}{N} \sum^{N}_{n=1} s\left(\vz_n; \vlambda\right)
  \end{align*}
  where \(\mathbb{H}\left[\,\pi\,\right]\) is the entropy of \(\pi\).
\end{condition}


\begin{theoremEnd}{proposition}\label{thm:product_kernel}
  Let \(\veta = \left( \vz^{(1)}, \vz^{(2)}, \ldots, \vz^{(N)} \right)\) and a Markov chain kernel \(P_{\vlambda}\left(\veta, \cdot\right)\) be \(\Pi\)-invariant where \(\Pi\) is defined as
  {%\small
  \[
  \Pi\left(\veta\right) = \pi\left(\vz^{(1)}\right) \, \pi\left(\vz^{(2)}\right) \times \ldots \times \pi\left(\vz^{(N)}\right).
  \]
  }
  Then, by defining the target function \(f\) and the gradient estimator \(\vg\) to be 
  {\small
  \begin{align*}
    f\left(\vlambda, \veta\right) =  \frac{1}{N} \sum^{N}_{n=1} \log q\left(\vz^{(n)}; \vlambda\right) + \mathbb{H}\left[\,\pi\,\right], 
    \quad\text{and}\quad
    \vg\left(\vlambda, \veta\right) =  \frac{1}{N} \sum^{N}_{n=1} \vs\left(\vz^{(n)}; \vlambda\right)
  \end{align*}
  }
  where \(\mathbb{H}\left[\,\pi\,\right]\) is the entropy of \(\pi\), MCGD results in inclusive KL minimization as
  {%\small
  \begin{align*}
    \Esub{\Pi}{ f\left(\vlambda, \rvveta\right) } = \DKL{\pi}{q\left(\cdot; \vlambda\right)},
    \quad\text{and}\quad
    \Esub{\Pi}{ \vg\left(\vlambda, \rvveta\right) } = \nabla_{\vlambda} \DKL{\pi}{q\left(\cdot; \vlambda\right)}.
  \end{align*}
  }
\end{theoremEnd}
\begin{proofEnd}
  For notational convenience, we define the shorthand
  \begin{align*}
    \pi\left(\vz^{(1:N)}\right) = \pi\left(\vz^{(1)}\right) \, \pi\left(\vz^{(2)}\right) \times \ldots \times \pi\left(\vz^{(N)}\right).
  \end{align*}
  Then,
  \begin{alignat}{2}
    &\Esub{\Pi}{ f\left(\vlambda, \veta\right) }
    \nonumber
    \\
    &\quad=
    \int \left( \frac{1}{N} \sum^{N}_{n=1} \log q\left(\vz^{(n)}; \vlambda\right) + \mathbb{H}\left[\,\pi\,\right]\right) \, \pi\left(\vz^{(1:N)}\right) \, d\vz^{(1:N)}
    \nonumber
    \\
    &\quad=
     \frac{1}{N} \sum^{N}_{n=1} \left\{ \int \big(\, \log q\,(\,\vz^{(n)}; \vlambda\,) + \mathbb{H}\,[\,\pi\,] \,\big) \, \pi\left(\vz^{(1:N)}\right) \, d\vz^{(1:N)} \right\}
     &&\quad{\text{\textit{Swapped integral and sum}}}
    \nonumber
    \\
    &\quad=
    \frac{1}{N} \sum^{N}_{n=1} \int \big(\, \log q\,(\,\vz^{(n)}; \vlambda\,) + \mathbb{H}\,[\,\pi\,] \,\big) \, \pi\left(\vz^{(n)}\right) \, d\vz^{(n)}
    &&\quad{\text{\textit{Marginalized \(\vz^{(m)}\) for all \(m \neq n\)}}}
    \nonumber
    \\
    &\quad=
    \frac{1}{N} \sum^{N}_{n=1} \DKL{\pi}{q\left(\cdot; \vlambda\right)}
    \nonumber
    &&\quad{\text{\textit{Definition of \(d_{\text{KL}}\)}}}
    \\
    &\quad=
    \DKL{\pi}{q\left(\cdot; \vlambda\right)}\label{eq:F_KL}
  \end{alignat}
  For \(\Esub{\Pi}{ \vg\left(\vlambda, \rvveta\right) }\), note that 
  \begin{align}
    \nabla_{\vlambda} f\left(\vlambda, \veta\right) = \frac{1}{N} \sum^{N}_{n=1} \vs\left(\vz^{(n)}; \vlambda\right) = \vg\left(\vlambda, \veta\right).\label{eq:F_grad_G}
  \end{align}
  Therefore, it suffices to show that
  \begin{align*}
    \nabla_{\vlambda} \DKL{\pi}{q\left(\cdot; \vlambda\right)}
    &=
    \nabla_{\vlambda} \Esub{\Pi}{ f\left(\vlambda, \rvveta\right) }
    &&\text{\textit{\cref{eq:F_KL}}}
    \\
    &=
    \Esub{\Pi}{ \nabla_{\vlambda}  f\left(\vlambda, \rvveta\right) }
    &&\text{\textit{{Leibniz derivative rule}}}
    \\
    &=
    \Esub{\Pi}{ \vg\left(\vlambda, \rvveta\right) }.
    &&\text{\textit{\cref{eq:F_grad_G}}}
  \end{align*}
\end{proofEnd}

%%% Local Variables:
%%% TeX-master: "master"
%%% End:


This framework includes JSA of~\citet{pmlr-v124-ou20a} while MSC of~\citet{pmlr-v124-ou20a} is a special case where \(N=1\).
We will later propose a third novel scheme that conforms to~\cref{thm:product_kernel}.

\begin{assumption}{\textbf{(Compactness)}}\label{thm:compact}
  The space of the variational parameters \(\Lambda\) is compact with a finite diameter \(R\) such that \(\norm{\vlambda - \vlambda\prime} \leq R \) for all \(\vlambda, \vlambda\prime \in \Lambda\).
\end{assumption}
This assumption is common for proving non-asymptotic convergence of convex optimization and MCGD~\citep{duchi_ergodic_2012, NEURIPS2018_1371bcce, doan_convergence_2020}.
Furthermore,~\citeauthor{NEURIPS2020_b2070693} state that the assumptions previously used for proving asymptotic convergence of MCSA~\citep{NEURIPS2020_b2070693} are difficult to check without assuming compactness.

\begin{assumption}{\textbf{(Strong log-concavity)}}\label{thm:logconcave}
  For all variational parameters \(\vlambda, \vmu \in \Lambda_{M}\), the resulting variational density is \(\mu\)-strongly log-concave as
  \[
  \log q\left(\vz; \alpha \, \vlambda + (1 - \alpha) \, \vmu\right)
  \geq 
  \alpha \, \log q\left(\vz; \vlambda\right)
  + (1 - \alpha) \log q\left(\vz; \vmu\right) - \frac{\alpha \, (1 - \alpha)}{2}\, \mu\,\norm{\vlambda - \vmu}^2_2
  \]
  where \(0 < \alpha < 1\) and some \(\mu > 0\).
\end{assumption}
%
\input{thm_strongly_log_concave}
%
\cref{thm:logconcave} includes many of the commonly used variational families such as the normal, exponential, and uniform, to name a few.
Also, combined with \cref{thm:logconcave_parameter},~\cref{thm:logconcave} results in MCSA becoming a convex optimization problem, which is surprising since exclusive KL minimization is not convex in general even with log-concave families.
Note that it is possible to establish convergence without log-concavity with the recent non-convex MCGD results of~\citet{NEURIPS2018_1371bcce, pmlr-v99-karimi19a, doan_convergence_2020}, but in this work, we focus on the stronger convex results.

\begin{assumption}{\textbf{(Bounded variance)}}\label{thm:bounded_variance}
  Let \(\vlambda \in \Lambda\) be measurable with respect to the \(\sigma\)-field \(\mathcal{F}_{t-1}\).
  The  gradient estimator \(g\) is bounded a constant \(G < \infty\) such that
  \(
  \E{ {\lVert\, g\left(\cdot, \rvveta_{t}\right) \,\rVert}^2_{*} \;\middle|\; \mathcal{F}_{t-1}} < G^2.
  \)
\end{assumption}
This assumption is similar to the bounded variance assumption commonly used in vanilla SGD.

Under the stated conditions, the non-asymptotic convergence rate of MCSA is a special case of the ergodic mirror descent algorithm of~\citet{duchi_ergodic_2012}.


\begin{theoremEnd}[all end]{lemma}\label{thm:mixing_time}
  Assuming the kernel \(P_{\vlambda}\left(\veta, \cdot\right)\) satisfies \cref{thm:ergodicity}, the Hellinger mixing time \(\tau_{\text{Hel.}}\) is bounded as
  \begin{align*}
    \tau_{\text{Hel.}}\left(K_{\vlambda}, \epsilon\right) \leq  \frac{2}{\log \rho^{-1}} \log \frac{1}{\epsilon}
  \end{align*}
\end{theoremEnd}
\begin{proofEnd}
  The following two inequalities are equivalent.
  \begin{align*}
    d_{\text{Hel.}}\left(\, P^t_{\vlambda}\left(\veta, \cdot \right), \Pi \,\right)
    &\leq \sqrt{\DTV{ P^t_{\vlambda}\left(\veta, \cdot \right)}{\Pi}}
    \leq \rho^{t / 2}
    \\
    \log d_{\text{Hel.}}\left(\, P^t_{\vlambda}\left(\veta, \cdot \right), \Pi \,\right)
    &\leq \frac{t}{2} \log \rho
  \end{align*}

  The Hellinger mixing time \(\tau_{\text{Hel.}}\left(K_{\vlambda}, \epsilon\right)\) is the smallest \(t\) that statisfies the inequality
  \begin{align*}
    d_{\text{Hel.}}\left(\, P^t_{\vlambda}\left(\veta, \cdot \right), \Pi \,\right)
    &\leq 
    \epsilon.
  \end{align*}
  Instead, we can find the \(t\prime > t\) that satisfies
  \begin{align*}
    \log d_{\text{Hel.}}\left(\, P^t_{\vlambda}\left(\veta, \cdot \right), \Pi \,\right)
    \leq 
    \frac{t\prime}{2} \log \rho
    \leq 
    \log \epsilon
 \end{align*}
  by solving the inequalities
  \begin{alignat*}{2}
    \frac{t\prime}{2} \log \rho
    &\leq 
    \log \epsilon
    \\
    t\prime 
    &\geq 
    \frac{2}{\log \rho} \log \epsilon
    &&\quad\text{\textit{Inequality flipped since \(\log \rho \leq 1\)}}
    \\
    t\prime 
    &\geq 
    \frac{2}{\log \rho^{-1}} \log \frac{1}{\epsilon}
 \end{alignat*}
\end{proofEnd}

\begin{theoremEnd}{theorem}{(\textbf{Convergence rate})}\label{thm:convergence_rate}
  Assuming~\cref{thm:kernel_conditions,thm:gradient_estimator,thm:logconcave,thm:compact,thm:bounded_variance} hold, with a diminishing stepsize of \(\alpha_t = R / \left( G \sqrt{\kappa_1 \log \kappa_2 T} \right)\), the average iterates {\small\(\overline{\vlambda}_T = \sum^{T}_{t=1} \vlambda_{t} / T\)} of Markov chain score ascent achieve a convergence rate of
  {%\small
  \begin{align*}
    \E{ \DKL{\pi}{q\,(\cdot; {\overline{\vlambda}_{T}})} - \DKL{\pi}{q\left(\cdot; {\vlambda^*}_{T}\right)}}
    =
    \mathcal{O}\left(
    \frac{
      G \, \sqrt{\log T}
    }{
      \log \rho^{-1} \, \sqrt{T}
    } \right)  
  \end{align*}
  }
\end{theoremEnd}
\begin{proofEnd}
  \citet[Corollary 3.5]{duchi_ergodic_2012} provide a non-asymptotic convergence rate for the \textit{ergodic mirror descent} algorithm which computes the parameter update as
  \begin{align}
    \vlambda_{t+1} &= \argmin_{\vlambda \in \Lambda} \left\{\,
    \iprod{\,\vg\left(\vlambda, \veta_t\right)}{\vlambda\,} 
    +
    \frac{1}{\alpha_t} D_{\psi}\left(\vlambda, \vlambda_{t}\right)
    \,\right\}\label{eq:ergodic_mirror_descent}
    \\
    \rvveta_{t+1} &\sim P_{\vlambda_{t}}\left(\veta_t, \cdot\right)
    \nonumber
  \end{align}
  where \(D_{\psi}\) is the Bregman divergence defined as
  \begin{align*}
    D_{\psi}\left(\vlambda, \vlambda\prime\right)
    \triangleq
    \psi\left(\vlambda\right)
    - \psi\left(\vlambda\prime\right)
    - \iprod{ \nabla \psi\left(\vlambda\prime\right)}{\vlambda - \vlambda\prime}
  \end{align*}
  for some convex function \(\psi\).
  Our result is based on the fact that MCGD is a special case of the ergodic mirror descent algorithm.
  Specifically, by choosing \(\psi\left(\vlambda\right) = \frac{1}{2} \norm{\vlambda}^2_2 \), we obtain
  \begin{alignat}{2}
    D_{\psi}\left(\vlambda, \vlambda\prime\right) = \frac{1}{2} \norm{\vlambda - \vlambda\prime}_2^2 \leq \frac{1}{2} \, R^2.\label{eq:bregman}
    &&\quad\text{\textit{\cref{thm:compact}}}
  \end{alignat}
  This reduces the update in \cref{eq:ergodic_mirror_descent} into projected gradient descent which is the form used for Markov chain score climbing.

  Under our assumptions, \citet[Corollary 3.5]{duchi_ergodic_2012} show that, by assuming that the Hellinger mixing time is bounded as
  \begin{align}
    \tau_{\text{Hel.}}\left(K_{\vlambda}, \epsilon\right) \leq \kappa_1 \log\left( \kappa_2 /\epsilon \right) \label{eq:hellinger_mixing}
  \end{align}
  for any \(\epsilon > 0\), and setting a decreasing stepsize \(\alpha_t = \alpha / \sqrt{t}\),
  it follows that
  \begin{align}
    &\E{ \DKL{\pi}{q\,(\cdot; {\overline{\vlambda}_{T}})} - \DKL{\pi}{q\left(\cdot; {\vlambda^*}_{T}\right)}}
    \nonumber
    \\
    &\quad\leq
    \frac{R^2}{2 \, \alpha \, \sqrt{T}}
    +
    \frac{2 \, \alpha \, G^2}{\sqrt{T}}\left( \kappa_1 \, \log \frac{\kappa_2}{\epsilon} \right)
    +
    3 \, \epsilon \, G \, R
    +
    \frac{R \, G \, \kappa_1 \, \log \frac{\kappa_2}{\epsilon}}{T}.\label{eq:original_bound}
  \end{align}

  From this, by setting the initial stepsize as \(\alpha = R / \left(G \sqrt{\kappa_1 \log \left(\kappa_2 \, T\right) }\right)\) and \(\epsilon = 1/ \sqrt{T}\),
  \begin{alignat*}{2}
    &\E{ \DKL{q\,(\cdot; {\overline{\vlambda}_{T}})}{\pi} - \DKL{q\left(\cdot; {\vlambda^*}_{T}\right)}{\pi}}
    \\
    &\leq
    \frac{R}{2 \, \alpha \, \sqrt{T}}
    +
    \frac{2 \, \alpha \, G^2}{\sqrt{T}}\left( \kappa_1 \, \log \frac{\kappa_2}{\epsilon} \right)
    +
    3 \, \epsilon \, G \, R
    +
    \frac{R \, G \, \kappa_1 \, \log \frac{\kappa_2}{\epsilon}}{T}.
    &&\quad\text{\textit{\cref{eq:original_bound}}}
    \\
    &=
    \frac{R \, G \, \sqrt{\kappa_1 \, \log \left(\kappa_2 \, T\right) } }{2 \, \sqrt{T}}
    +
    \frac{2 \, R \, G}{\sqrt{T}}
    \frac{ \kappa_1 \, \log \kappa_2 \, \sqrt{T} }{ \sqrt{\kappa_1 \log \left(\kappa_2 \, T\right)}}
    +
    \frac{3 \, G \, R}{\sqrt{T}}
    +
    \frac{R \, G \, \kappa_1 \, \log \left(\kappa_2 \, \sqrt{T}\right)}{T}
    &&\quad\text{\textit{Plugged value of \(\epsilon\) and \(\alpha\)}}
    \\
    &\leq
    \frac{R \, G \, \sqrt{\kappa_1 \, \log \left(\kappa_2 \, T\right) } }{2 \, \sqrt{T}}
    +
    \frac{2 \, R \, G}{\sqrt{T}}
    \frac{ \kappa_1 \, \log \kappa_2 \, T }{ \sqrt{\kappa_1 \log \left(\kappa_2 \, T\right)}}
    +
    \frac{3 \, G \, R}{\sqrt{T}}
    +
    \frac{R \, G \, \kappa_1 \, \log \left(\kappa_2 \, T\right)}{T}
    &&\quad\text{\textit{Applied \(\log \sqrt{T} < \log T\)}}
    \\
    &=
    \frac{R \, G \, \sqrt{\kappa_1 \, \log \left(\kappa_2 \, T\right) } }{2 \, \sqrt{T}}
    +
    \frac{2 \, R \, G}{\sqrt{T}}
    \sqrt{\kappa_1 \, \log \left( \kappa_2 \, T\right) }
    +
    \frac{3 \, G \, R}{\sqrt{T}}
    +
    \frac{R \, G \, \kappa_1 \, \log \left(\kappa_2 \, T\right)}{T}
    &&\quad\text{\textit{Solved fraction}}
    \\
    &=
    \frac{5 \, R \, G \, \sqrt{\kappa_1 \, \log \left(\kappa_2 \, T\right) } }{2 \, \sqrt{T}}
    +
    \frac{3 \, G \, R}{\sqrt{T}}
    +
    \frac{R \, G \, \kappa_1 \, \log \left(\kappa_2 \, T\right)}{T}.
    &&\quad\text{\textit{Combined fractions}}
  \end{alignat*}

  From \cref{thm:mixing_time}, we retrieve the constants of \cref{eq:hellinger_mixing} as
  \(
  \kappa_1=\frac{2}{\log \rho^{-1} },\;  \kappa_2 = 1
  \), which follows our result
  \begin{alignat*}{2}
    &\frac{
      5 \, R \, G \, \sqrt{\kappa_1 \, \log \left(\kappa_2 \, T\right) }
    }{
      2 \, \sqrt{T}
    }
    +
    \frac{3 \, G \, R}{\sqrt{T}}
    +
    \frac{R \, G \, \kappa_1 \, \log \left(\kappa_2 \, T\right)}{T}
    \\
    &\quad=
    \frac{
      5 \, R \, G \, \sqrt{ \frac{2}{\log \rho^{-1}} \, \log T }
    }{
      2 \, \sqrt{T}
    }
    +
    \frac{3 \, G \, R}{\sqrt{T}}
    +
    \frac{R \, G \, \frac{2}{\log \rho^{-1}} \, \log T}{T}
    &&\quad\text{\textit{Plugged values of \(\kappa_1\) and \(\kappa_2\)}}
    \\
    &\quad=
    \frac{
      5 \, \sqrt{2} \, R \, 
    }{
      2
    }
    \,
    \frac{
      G \, \sqrt{\log T}
    }{
      \log \rho^{-1} \, \sqrt{T} \, 
    }
    +
    3 \, R \,
    \frac{G \, R}{\sqrt{T}}
    +
    2 \, R
    \,
    \frac{G \, \log T}{ \log \rho^{-1} \, T}
    &&\quad\text{\textit{Pulled constants forward}}
  \end{alignat*}
\end{proofEnd}

%%% Local Variables:
%%% TeX-master: "master"
%%% End:


%% This result is a direct adaptation of the ergodic mirror descent algorithm by~\cite{duchi_ergodic_2012}.
%% For accelerated variants of MCGD,~\citet{doan_convergence_2020} provide non-asymptotic convergence results.
%% However, their bound for the convex case is independent of the kernel mixing rate, which leaves out the practical effects of the mixing rate.

\subsection{Comparing Markov Chain Score Ascent Methods}\label{section:comparison}
From now on, we will show that previous MCSA schemes our framework in \cref{section:general}.
comparing these algorithms require a few additional assumptions related to their specific implementation details.
We also analyze the practical performance of these algorithms using~\cref{thm:convergence_rate} and suggest a simple but effective variant of MCSA.

First, we use the following assumptions.
\begin{assumption}{(Bounded importance weight)}\label{thm:bounded_weight}
  The importance weight ratio \(w\left(\vz\right) = \pi\left(\vz\right) / q\left(\vz; \vlambda\right)\) is bounded by some finite constant as \(w^* < \infty\) for all \(\vlambda \in \Lambda\) such that \(\rho = \left(1 - 1/w^*\right) < 1\).
\end{assumption}
The fact that \(w^*\) exists for all \(\vlambda \in \Lambda\) is restrictive and appears to be unnecessary for convergence in practice.
However, this assumption is important for the theory of MCGD to work by satisfying~\cref{thm:kernel_conditions}.
Although previous works did not take specific measures to enable \cref{thm:bounded_weight}, it can be done by using a variational family with heavy tails~\citep{NEURIPS2018_25db67c5} or using a defensive mixture~\citep{hesterberg_weighted_1995, holden_adaptive_2009} 
\begin{align*}
  q_{\text{def.}}\left(\vz; \vlambda \right) = w \, q\left(\vz; \vlambda\right) + (1 - w) \, \nu\left(\vz\right)
\end{align*}
where \(0 < w < 1\) and \(\nu\left(\cdot\right)\) is a heavy tailed distribution that satisfies \(\sup_{\vz \in \mathcal{Z}} \pi\left(\vz\right) / \nu\left(\vz\right) < \infty\).
It is possible to only use \(q_{\text{def.}}\) within the Markov chain kernels, which therefore does not restrict our choice of the variational family.
%
\begin{assumption}{(Bounded variance)}\label{thm:bounded_score}
  %There exists a finite constant \(\sigma^2\) for all \(\vlambda \in \Lambda\) such that
  The score function is bounded for all \(\vz \in \mathcal{Z}\) as
  {%
  %% \begin{align*}
  %%   \Esub{\rvvz \sim q_{\text{def.}}\left(\cdot; \vlambda\right)}{ \norm{s\left(\cdot; \rvvz\right)}_{*}^2 } < \sigma^2,
  %%   \;\;
  %%   \Esub{\rvvz \sim q_{\text{def.}}\left(\cdot; \vlambda\right)}{ \norm{s\left(\cdot; \rvvz\right)}_{*}^4 } < \sigma^4, 
  %%   \;\;\text{and}\;\;
  %%   \Esub{\rvveta \sim P\left(\veta_t, \cdot\right)}{\norm{s\left(\cdot; \rvveta_n\right)}_{*}^2  \;\middle|\; \mathcal{F}_t } < \sigma^2
  %% \end{align*}
  \(
    \norm{s\left(\cdot; \vz\right)}_{*}^2 < L^2
  \)
  }%
  %% where \(\rvveta_n\) is the \(n\)th element of a Markov chain sample generated during MCSA.
\end{assumption}
%% The later is guarenteed to hold for any geometrically ergodic Markov chain kernel~\citet{10.2307/25442663}.
In our analysis, we use the constant \(L\) to bound the gradient variance.
Practically speaking, this enables us to compare the gradient variance of different MCSA designs relative to \(L\).

%% These assumptions are weaker than the bounded score assumption (\(\norm{s\left(\cdot; \vz\right)}_{*} < L < \infty\)) imposed by most recent non-asymptotic results on the general convergence of MCGD~\citep{NEURIPS2018_1371bcce, doan_convergence_2020, pmlr-v99-karimi19a, Xiong_Xu_Liang_Zhang_2021}.


First, we show that MSC~\citep{NEURIPS2020_b2070693} satisfies \cref{thm:kernel_conditions,thm:gradient_estimator}.
MSC uses the conditional importance sampling (CIS) MCMC kernel.
This kernel is identical to the iterated sequential importance resampling (i-SIR) kernel by~\citet{andrieu_uniform_2018}, by which the geometric convergence has been established.


\begin{theoremEnd}[all end]{lemma}\label{thm:product_measure_bound}
  For the probability measures \(p_1, \ldots, p_N\) and \(q_1, \ldots, q_N\) defined on a measurable space \((\mathsf{X}, \mathcal{A})\) and an arbitrary set \(A \in \mathcal{A}\),
  \begin{align*}
    &\abs{
    \int_{A^N}
    p_1\left(dx_1\right)
    p_2\left(dx_2\right)
    \times
    \ldots
    \times
    p_N\left(dx_N\right)
    -
    q_1\left(dx_1\right)
    q_2\left(dx_2\right)
    \times
    \ldots
    \times
    q_N\left(dx_N\right)
  }
    \\
  &\qquad\leq
  \sum_{n=1}^N
  \abs{
    \int_{A}
    p_n\left(dx_n\right)
    -
    q_n\left(dx_n\right)
  }
  \end{align*}
\end{theoremEnd}
\begin{proofEnd}
  By using the following shorthand notations
  \begin{alignat*}{2}
    p_{(1:N)}\left(dx_{(1:N)}\right)
    &= 
    p_1\left(dx_1\right)
    p_2\left(dx_2\right)
    \times
    \ldots
    \times
    p_N\left(dx_N\right)
    \\
    q_{(1:N)}\left(dx_{(1:N)}\right)
    &= 
    q_1\left(dx_1\right)
    q_2\left(dx_2\right)
    \times
    \ldots
    \times
    q_N\left(dx_N\right),
  \end{alignat*}
  the result follows from induction as
  \begin{alignat}{2}
    &\abs{
      \int_{A^N}
      p_{(1:N)}\left(dx_{(1:N)}\right)
      -
      q_{(1:N)}\left(dx_{(1:N)}\right)
    }
    \nonumber
    \\
    &\quad=
    \Bigg|\;
    \left( \int_{A} p_1\left(dx_1\right) - q_1\left(dx_1\right) \right) \,
    \int_{A^{N-1}} p_{(2:N)}\left(dx_{(2:N)}\right)
    \nonumber
    \\
    &\qquad\quad+
    \int_{A} q_1\left(dx_1\right) \,
    {\left(
      \int_{A^{N-1}}
      p_{(2:N)}\left(dx_{(2:N)}\right)
      -
      q_{(2:N)}\left(dx_{(2:N)}\right)
    \right)}
    \;\Bigg|
    \nonumber
    \\
    &\quad\leq
    \Bigg|
    \int_{A} p_1\left(dx_1\right) - q_1\left(dx_1\right)
    \Bigg|\;
    \int_{A^{N-1}} p_{(2:N)}\left(dx_{(2:N)}\right)
    \nonumber
    \\
    &\qquad\quad+
    \int_{A} q_1\left(dx_1\right) \,
    {
    \Bigg|\;
      \int_{A^{N-1}}
      p_{(2:N)}\left(dx_{(2:N)}\right)
      -
      q_{(2:N)}\left(dx_{(2:N)}\right)
    }
    \;\Bigg|
    &&\quad\text{\textit{Triangle inequality}}
    \nonumber
    \\
    &\quad\leq
    \Bigg|
    \int_{A} p_1\left(dx_1\right) - q_1\left(dx_1\right)
    \Bigg|\;
    \nonumber
    \\
    &\qquad\quad+
    {
    \Bigg|\;
      \int_{A^{N-1}}
      p_{(2:N)}\left(dx_{(2:N)}\right)
      -
      q_{(2:N)}\left(dx_{(2:N)}\right)
    }
    \;\Bigg|.
    &&\quad\text{\textit{Applied \(p_n\left(A\right), q_n\left(A\right) \leq 1 \)}}
    \nonumber
  \end{alignat}
\end{proofEnd}


\begin{theoremEnd}{theorem}\label{thm:msc}
  MSC~\citep{NEURIPS2020_b2070693} is obtained by defining 
  {%\small
  \begin{align*}
  P_{\lambda}^k\left(\veta, d\veta^{\prime}\right)
  = 
  K_{\lambda}^k\left(\vz, d\vz^{\prime}\right)
  \end{align*}
  }
  with  \(\veta_t = \vz_t\) where \(K_{\vlambda}\left(\vz, \cdot\right)\) is the CIS kernel with \(q_{\text{def.}}\left(\cdot; \vlambda\right)\) as its proposal distribution.
  Then, given~\cref{thm:bounded_weight,thm:bounded_score}, the mixing rate and the gradient bounds are given as
  {%\small
  \begin{align*}
    \textstyle
  \DTV{P_{\vlambda}^k\left(\veta, \cdot\right)}{\Pi} \leq  {\left(1 - \frac{N - 1}{2 w^* + N - 2}\right)}^k\quad \text{and}\quad
  {\small
  \E{ \norm{ \vg\left(\vlambda, \rvveta\right) }_{*}^2 \,\middle|\, \mathcal{F}_{t} } \leq  L^2,
  }
  \end{align*}
  }
  where \(w^* = \sup_{\vz} \pi\left(\vz\right) / q_{\text{def.}}\left(\vz;\vlambda\right)\).
\end{theoremEnd}
\begin{proofEnd}
  MSC is described in~\cref{alg:msc}. 
  At each iteration, it performs a single MCMC transition with the CIS kernel where it internally uses \(N\) proposals.

  \paragraph{Ergodicity of the Markov Chain}
  The ergodic convergence rate of \(P_{\vlambda}\) is equal to that of \(K_{\vlambda}\), the CIS kernel proposed by~\citet{NEURIPS2020_b2070693}. 
  Although not mentioned by~\citet{NEURIPS2020_b2070693}, this kernel has been previously proposed as the iterated sequential importance resampling (i-SIR) by \citet{andrieu_uniform_2018} with its corresponding geometric convergence rate as
  \begin{alignat*}{2}
    \DTV{P^{k}_{\vlambda}\left(\veta, \cdot\right)}{\Pi}
    =
    \DTV{K^{k}_{\vlambda}\left(\vz, \cdot\right)}{\pi}
    \leq
    {\left(1 - \frac{N - 1}{2 w^* + N - 2}\right)}^k.
  \end{alignat*}

  \paragraph{\textbf{Bound on the Gradient Variance}}
  The bound on the gradient variance is straightforward given \cref{thm:bounded_score}.
  For simplicity, we denote the rejection state as \(\vz^{(1)} = \vz_{t-1} \).
  Then,
  \begin{alignat}{2}
    &\E{ \norm{ \vg\left(\vlambda, \rvveta\right) }_{*}^2 \,\middle|\, \mathcal{F}_{t} }
    \nonumber
    \\
    &\;=
    \E{ \norm{ \vg\left(\vlambda, \rvveta\right) }_{*}^2 \,\middle|\, \mathcal{F}_{t}}
    \nonumber
    \\
    &\;=
    \Esub{\rvvz \sim K_{\vlambda_{t-1}}\left(\vz_{t-1}, \cdot\right)}{
      \norm{ \vs\left(\vlambda; \rvvz\right) }_{*}^2 \,\middle|\,
      \vlambda_{t-1}, \vz_{t-1}
    }
    \nonumber
    \\
    &\;=
    \int
    \sum^{N}_{n=1}
    \frac{
      w\left(\vz^{(n)}\right)
    }{
      \sum^{N}_{m=1} w\left(\vz^{(m)}\right)
    }
    \norm{ \vs\left(\cdot; \vz^{(n)}\right) }^2_{*}
    \prod^{N}_{n=2}
    q\left(d\vz^{(n)}; \vlambda_{t-1}\right)
    \nonumber
    &&\quad\text{\textit{\citet{andrieu_uniform_2018}}}
    \\
    &\;\leq
    L^2 \,
    \int
    \sum^{N}_{n=1}
    \frac{
      w\left(\vz^{(n)}\right)
    }{
      \sum^{N}_{m=1} w\left(\vz^{(m)}\right)
    }
    \prod^{N}_{n=2}
    q\left(d\vz^{(n)}; \vlambda_{t-1}\right)
    \nonumber
    &&\quad\text{\textit{\cref{thm:bounded_score}}}
    \\
    &\;=
    L^2 \,
    \int
    \prod^{N}_{n=2}
    q\left(d\vz^{(n)}; \vlambda_{t-1}\right)
    \nonumber
   &&\quad\text{\textit{The sum of weights is 1}}
    \\
    &\;=
    L^2.
    \nonumber
  \end{alignat}

  %% This form coincides with solving the expectation of the self-normalized importance sampling estimator, which is well known to be challenging~\citep{robert_monte_2004}.
  %% We instead approximate the expectation by defining the random variables
  %% \(
  %% X = \sum^{N}_{n=1} w\left(\vz_n\right) \norm{ s\left(\cdot; \vz_n\right) }^2_{*}
  %% \)
  %% and
  %% \(
  %% Y = \sum^{N}_{m=1} w\left(\vz_m\right)
  %% \),
  %% use the delta method as
  %% \begin{alignat*}{2}
  %%   \E{\frac{\rvX}{\rvY} \;\middle|\; Z} &\approx \frac{\E{\rvX \mid Z }}{\E{\rvY \mid Z}} + \mathcal{O}\left(\frac{1}{{\E{\rvX \mid Z}}^2}\right).
  %% \end{alignat*}
  %% The required expectations are obtained as
  %% \begin{alignat}{2}
  %%   &\E{\rvX \mid \lambda_{t-1}, \vz_{t-1}}
  %%   \\
  %%   &\;=
  %%   \int \sum^{N}_{n=1} w\left(\vz_n\right) \norm{ s\left(\cdot; \vz_n\right) }^2_{*}
  %%   \prod^{N}_{n=2}
  %%   q\left(d\rvvz_n\right)
  %%   \nonumber
  %%   \\
  %%   &\quad=
  %%   \sum^{N}_{n=1} \int w\left(\vz_n\right) \norm{ s\left(\cdot; \vz_n\right) }^2_{*}
  %%   \prod^{N}_{n=2}
  %%   q\left(d\rvvz_n\right)
  %%   &&\quad\text{\textit{Swapped integral and sum}}
  %%   \nonumber
  %%   \\
  %%   &\quad=
  %%   \left\{\;
  %%   \sum^{N}_{n=2} \int w\left(\vz_n\right) \norm{ s\left(\cdot; \vz_n\right) }^2_{*}
  %%   \prod^{N}_{n=2} q\left(d\rvvz_n\right)
  %%   \right\}
  %%   +
  %%   w\left(\vz_1\right) \norm{ s\left(\cdot; \vz_1\right) }^2_{*}
  %%   &&\quad\text{\textit{Pulled out rejection state}}
  %%   \nonumber
  %%   \\
  %%   &\quad=
  %%   \left\{\;
  %%   \sum^{N}_{n=2} \int \frac{\pi\left(\vz_n\right)}{q\left(\vz_n\right)} \norm{ s\left(\cdot; \vz_n\right) }^2_{*}
  %%   \prod^{N}_{n=2}
  %%   q\left(d\rvvz_n\right)
  %%   \right\}
  %%   +
  %%   w\left(\vz_1\right) \norm{ s\left(\cdot; \vz_1\right) }^2_{*}
  %%   &&\quad\text{\textit{Definition of \(w\left(\vz\right)\)}}
  %%   \nonumber
  %%   \\
  %%   &\quad=
  %%   \sum^{N}_{n=2} \int \pi\left(d\vz_n\right) \norm{ s\left(\cdot; \vz_n\right) }^2_{*}
  %%   +
  %%   w\left(\vz_1\right) \norm{ s\left(\cdot; \vz_1\right) }^2_{*}
  %%   &&\quad\text{\textit{Cancelled out \(q\left(\cdot\right)\)}}
  %%   \nonumber
  %%   \\
  %%   &\quad=
  %%   \left(N-1\right) \, \Esub{\pi}{\norm{ s\left(\cdot; \rvvz\right) }^2_{*} }
  %%   +
  %%   w\left(\vz_1\right) \norm{ s\left(\cdot; \vz_1\right) }^2_{*}
  %%   \nonumber
  %%   \\
  %%   &\quad=
  %%   \left(N-1\right) \, \Esub{\pi}{\norm{ s\left(\cdot; \rvvz\right) }^2_{*} }
  %%   +
  %%   w\left(\vz_{t-1}\right) \norm{ s\left(\cdot; \vz_{t-1}\right) }^2_{*}
  %%   \nonumber
  %% \end{alignat}
  %% and similarly,
  %% \begin{alignat}{2}
  %%   \E{\rvY \mid \lambda_{t-1}, \vz_{t-1}}
  %%   &\quad= 
  %%   \int \sum^{N}_{n=1} w\left(\vz_n\right) \prod^{N}_{n=2} q\left(d\vz_n\right) 
  %%   \nonumber
  %%   \\
  %%   &\quad= 
  %%   \left\{\; \int \sum^{N}_{n=2} w\left(\vz_n\right) \prod^{N}_{n=2} q\left(d\vz_n\right) \;\right\} + w\left(\vz_1\right)
  %%   &&\quad\text{\textit{Pulled out rejection state}}
  %%   \nonumber
  %%   \\
  %%   &\quad= 
  %%   \left\{\; \sum^{N}_{n=2} \int w\left(\vz_n\right) \prod^{N}_{n=2} q\left(d\vz_n\right) \;\right\} + w\left(\vz_1\right)
  %%   &&\quad\text{\textit{Swapped integral and sum}}
  %%   \nonumber
  %%   \\
  %%   &\quad= 
  %%   \left\{\; \sum^{N}_{n=2} \int \frac{\pi\left(\vz_n\right)}{q\left(\vz_n\right)} \prod^{N}_{n=2} q\left(d\vz_n\right) \;\right\} + w\left(\vz_1\right)
  %%   &&\quad\text{\textit{Definition of \(w\left(\vz\right)\)}}
  %%   \nonumber
  %%   \\
  %%   &\quad= 
  %%   \left\{\; \sum^{N}_{n=2} \int \pi\left(d\vz_n\right) \;\right\} + w\left(\vz_1\right)
  %%   &&\quad\text{\textit{Cancelled out \(q\left(\cdot\right)\)}}
  %%   \nonumber
  %%   \\
  %%   &\quad= 
  %%   N - 1 + w\left(\vz_{t-1}\right)
  %%   \nonumber
  %% \end{alignat}
  %% Therefore, 
  %% \begin{alignat}{2}
  %%   \E{\frac{\rvX}{\rvY} \;\middle|\; Z}
  %%   &\approx
  %%   \frac{
  %%     \left(N-1\right) \, \Esub{\pi}{\norm{ s\left(\cdot; \rvvz\right) }^2_{*} }
  %%     +
  %%     w\left(\vz_1\right) \norm{ s\left(\cdot; \vz_1\right) }^2_{*}
  %%   }{
  %%     N-1 + w\left(\vz_1\right)
  %%   }
  %%   + \mathcal{O}\left(\frac{1}{{\left(N-1\right)}^2}\right)
  %%   \nonumber
  %%   \\
  %%   &\leq
  %%   \frac{
  %%     \left(N-1\right) \, \Esub{\pi}{\norm{ s\left(\cdot; \rvvz\right) }^2_{*} }
  %%     +
  %%     w^* \norm{ s\left(\cdot; \vz_1\right) }^2_{*}
  %%   }{
  %%     N-1
  %%   }
  %%   + \mathcal{O}\left(\frac{1}{{\left(N-1\right)}^2}\right)
  %%   &&\quad\text{\textit{\(w\left(\cdot\right) \leq w^*\)}}
  %%   \nonumber
  %%   \\
  %%   &\leq
  %%   \frac{
  %%     \left(N-1\right) \, L^2
  %%     +
  %%     w^* L^2
  %%   }{
  %%     N-1
  %%   }
  %%   + \mathcal{O}\left(\frac{1}{{\left(N-1\right)}^2}\right)
  %%   \nonumber
  %%   \\
  %%   &=
  %%   L^2
  %%   \left[
  %%   1
  %%   +
  %%   \frac{
  %%     w^*
  %%   }{
  %%     N-1
  %%   }
  %%   \right]
  %%   + \mathcal{O}\left(\frac{1}{{\left(N-1\right)}^2}\right)
  %%   \nonumber
  %% \end{alignat}
\end{proofEnd}

%%% Local Variables:
%%% TeX-master: "master"
%%% End:


Now, we provide our result on JSA~\citep{pmlr-v124-ou20a}.
JSA specifically assumes that the target distribution is formed with independently, identically distributed (\textit{i.i.d}) data.
However, we interpret JSA into a more general setup that does not assume \textit{i.i.d.} data similar to MSC.
Since JSA~\citep{pmlr-v124-ou20a} uses the independent Metropolis-Hastings (IMH) kernel, we utilize the geometric convergence rate provided by~\citet[Theorem 2.1]{10.2307/2242610} and~\citet{wang_exact_2020}.
Furthermore, to establish an upper bound on the conditional variance, we use the exact multi-transition IMH kernel derived by~\cite{Smith96exacttransition} as
{%\small
  \begin{align}
  K^n_{\vlambda}\left(\vz, d\vz\prime\right) 
  = T_n\left(\, w\left(\vz\right) \vee w\left(\vz\prime\right)\,\right) \, \pi\left(\vz\prime\right) \, d\vz\prime
  + \lambda^t\left(w\left(\vz\right)\right) \, \delta_{\vz}\left(d\vz\prime\right)
  \label{eq:imh_exact_kernel}
  \end{align}
}%
where {\(w\left(\vz\right) = \pi\left(\vz\right)/q_{def.}\left(\vz; \vlambda\right)\), \(x \vee y = \max\left(x, y\right)\)},
{%\small%
  \begin{align}
    T_t\left(w\right)      = \int_w^{\infty} \frac{t}{v^2} \, \lambda^{t-1}\left(v\right)\,dv,
    \quad\text{and}\quad
    \lambda\left(w\right) = \int_{R\left(w\right)} \left( 1 - \frac{w\left(\vz\prime\right)}{w}  \right) \pi\left(d\vz\prime\right)\label{eq:T_lambda}
  \end{align}
}
for {\(R\left(w\right) = \{\, \vz\prime \mid w\,\left(\vz\prime\right) \leq w \,\}\)}.
%

\begin{theoremEnd}[all end]{lemma}\label{thm:lambda_bound}
  For \(w^* = \sup_{\vz} w\left(\vz\right) \), \(\lambda\left(\cdot\right)\) in~\cref{eq:T_lambda} is bounded as
  \[
   \max\left(1 - \frac{1}{w}, 0\right) \leq \lambda\left(w\right) \leq 1 - \frac{1}{w^*}.
  \]
\end{theoremEnd}
\begin{proofEnd}
  The proof can be found in the proof of Theorem 3 of \citet{Smith96exacttransition}.
\end{proofEnd}

\begin{theoremEnd}[all end]{lemma}\label{thm:tn_bound}
  For \(w^* = \sup_{\vz} w\left(\vz\right) \), \(T_n\left(\cdot\right)\) in~\cref{eq:T_lambda} is bounded as
  \[
  T_n\left( w \right) \leq \frac{n}{w} \, {\left(1 - \frac{1}{w^*}\right)}^{n-1}.
  \]
\end{theoremEnd}
\begin{proofEnd}
  \begin{alignat*}{2}
    T_n\left(w\right) 
    &= \int_w^{\infty} \frac{n}{v^2} \, \lambda^{n-1}\left(v\right)\,dv
    &&\quad\text{\textit{\cref{eq:T_lambda}}}
    \\
    &\leq \int_w^{\infty} \frac{n}{v^2} \, {\left(1 - \frac{1}{w^*}\right)}^{n-1}\,dv
    &&\quad\text{\textit{\cref{thm:lambda_bound}}}
    \\
    &= n \, {\left(1 - \frac{1}{w^*}\right)}^{n-1}  \int_w^{\infty} \frac{1}{v^2} \,dv
    &&\quad\text{\textit{Pulled out constant}}
    \\
    &= n \, {\left(1 - \frac{1}{w^*}\right)}^{n-1}  \left( {-\left.\frac{1}{v}\right\rvert^{\infty}_{w}} \right)
    &&\quad\text{\textit{Solved indefinite integral}}
    \\
    &= \frac{n}{w} \, {\left(1 - \frac{1}{w^*}\right)}^{n-1}.
  \end{alignat*}
  This upper bound is in general difficult to improve unless we impose stronger assumptions on \(\pi\) and \(q\).
\end{proofEnd}

\begin{theoremEnd}[all end]{lemma}\label{thm:imh_expecation}
  For a positive test function \(f : \mathcal{Z} \rightarrow \mathbb{R}^{+}\), the estimate of a \(\pi\)-invariant independent Metropolis-Hastings kernel with a proposal \(q\) is bounded as
  \begin{align*}
    \Esub{K^n\left(\vz, \cdot\right)}{ f \,\middle|\, \rvvz }
    \leq
    n \, \rho^{n-1} 
    \Esub{q}{f}
    +
    {\rho}^n \, f\left(\rvvz\right)
    %% \leq
    %% n \, \left(
    %% \Esub{q}{f}
    %% +
    %% \frac{1}{n} \, f\left(\vz\right)
    %% \right) 
  \end{align*}
  where \(w\left(\vz\right) = \pi\left(\vz\right) / q\left(\vz\right)\) and \(\rho = 1 - \nicefrac{1}{w^*}\) for \(w^* = \sup_{\vz} w\left(\vz\right) \).
\end{theoremEnd}
\begin{proofEnd}
  \begin{alignat*}{2}
    &\Esub{K^n\left(\vz, \cdot\right)}{ f \,\middle|\, \vz }
    \\
    &\quad=
    \int T_n\left(w\left(\vz\right) \vee w\left(\vz^{\prime}\right)\right) \, f\left(\vz^{\prime}\right) \, \pi\left(\vz^{\prime}\right) d\vz^{\prime}
    +
    \lambda^{n}\left(w\left(\rvvz\right)\right) \, f\left(\rvvz\right)
    &&\quad{\text{\textit{\cref{eq:imh_exact_kernel}}}}
    \\
    &\quad\leq
    \int \frac{n}{w\left(\vz\right) \vee w\left(\vz^{\prime}\right)} \, {\left(1 - \frac{1}{w^*}\right)}^{n-1} \, f\left(\vz^{\prime}\right) \, \pi\left(\vz^{\prime}\right) d\vz^{\prime}
    +
    \lambda^{n}\left(w\left(\rvvz\right)\right) \, f\left(\rvvz\right)
    &&\quad{\text{\textit{\cref{thm:tn_bound}}}}
    \\
    &\quad\leq
    \int \frac{n}{w\left(\vz^{\prime}\right)} \, {\left(1 - \frac{1}{w^*}\right)}^{n-1} \, f\left(\vz^{\prime}\right) \, \pi\left(\vz^{\prime}\right) d\vz^{\prime}
    +
    \lambda^{n}\left(w\left(\rvvz\right)\right) \, f\left(\rvvz\right)
    &&\quad{\frac{1}{w\left(\vz\right) \vee w\left(\vz^{\prime}\right)} \leq \frac{1}{w\left(\vz^{\prime}\right)}}
    \\
    &\quad=
    n \, {\left(1 - \frac{1}{w^*}\right)}^{n-1} \, 
    \int \frac{1}{w\left(\vz^{\prime}\right)} \, f\left(\vz^{\prime}\right) \, \pi\left(\vz^{\prime}\right) d\vz^{\prime}
    +
    \lambda^{n}\left(w\left(\rvvz\right)\right) \, f\left(\rvvz\right)
    &&\quad{\text{\textit{Pulled out constant}}}
    \\
    &\quad=
    n \, {\left(1 - \frac{1}{w^*}\right)}^{n-1} \, 
    \int f\left(\vz^{\prime}\right) \, q\left(\vz^{\prime}\right) d\vz^{\prime}
    +
    \lambda^{n}\left(w\left(\rvvz\right)\right) \, f\left(\rvvz\right)
    &&\quad{\text{\textit{Definition of \(w\left(\vz\right)\)}}}
    \\
    &\quad\leq
    n \, {\left(1 - \frac{1}{w^*}\right)}^{n-1} \, 
    \int f\left(\vz^{\prime}\right) \, q\left(\vz^{\prime}\right) d\vz^{\prime}
    +
    {\left(1 - \frac{1}{w^*}\right)}^{n} \, f\left(\rvvz\right)
    &&\quad{\text{\textit{\cref{thm:lambda_bound}}}}
    \\
    &\quad=
    n \, {\left(1 - \frac{1}{w^*}\right)}^{n-1} 
    \Esub{q}{f}
    +
    {\left(1 - \frac{1}{w^*}\right)}^n \, f\left(\rvvz\right).
  \end{alignat*}
\end{proofEnd}


%%% Local Variables:
%%% TeX-master: "master"
%%% End:


\begin{theoremEnd}{theorem}\label{thm:jsa}
  JSA~\citep{pmlr-v124-ou20a} is obtained by defining 
  {\small
  \begin{align*}
  P_{\vlambda}^n\left(\veta, d\veta\prime\right)
  = 
  K_{\vlambda}^{N\,\left(n-1\right) + 1}\left(\vz^{(1)}, d\vz\prime^{(1)}\right)
  \,
  K_{\vlambda}^{N\,\left(n-1\right) + 2}\left(\vz^{(2)}, d\vz\prime^{(2)}\right)
  \cdot
  \ldots 
  \cdot
  K_{\vlambda}^{N\,\left(n-1\right) + N}\left(\vz^{(N)}, d\vz\prime^{(N)}\right)
  \end{align*}
  }
  with \(\veta_t = \big[\vz_t^{(1)}, \vz_t^{(2)}, \ldots, \vz_t^{(N)}\big]\).
  Then, given~\cref{thm:bounded_weight,thm:bounded_score}, the mixing rate and the gradient variance bounds are
  {\small
  \begin{align*}
    \DTV{P_{\vlambda}^n\left(\veta, \cdot\right)}{\Pi}
    \leq
    C\left(\rho, N\right)\,{\rho}^{n\,N}
    \quad\text{and}\quad
   % 
    \E{ \norm{ \vg\left(\vlambda, \rvveta\right) }^2_{*} \,\middle|\, \mathcal{F}_{t} }
    \leq
    L^2 \,
    \left[\,
    \frac{1}{2} + \frac{3}{2}\,\frac{1}{N}
    + \mathcal{O}\left(\nicefrac{1}{w^*}\right)
    \,\right],
  \end{align*}
  }
  where \(w^* = \sup_{\vz} \pi\left(\vz\right) / q_{\text{def.}}\left(\vz;\vlambda\right)\) and \(C\) is a finite positive constant depending on both \(\rho\) and \(N\).
\end{theoremEnd}
\begin{proofEnd}

  JSA is described in~\cref{alg:jsa}. 
  At each iteration, it performs \(N\) MCMC transitions, and uses the \(N\) samples to estimate the gradient.

  \paragraph{Ergodicity of the Markov Chain}
  The state transition of the Markov chain samples \(\vz^{(1:N)}\) can be visualized as 
  {\small
  \begin{center}
  \bgroup
  \setlength{\tabcolsep}{3pt}
  \def\arraystretch{1.8}
  \begin{tabular}{c|ccccc}
   & \(\vz^{(1)}_t\) & \(\vz^{(2)}_t\) & \(\vz^{(3)}_t\) & \(\ldots\) &  \(\vz^{(N)}_t\) \\ \midrule
   \(t=1\) & \(K_{\vlambda_1}\left(\vz_0, d\vz_1^{(1)}\right)\) & \(K_{\vlambda_1}^2\left(\vz_0, d\vz_1^{(2)}\right)\) & \(K_{\vlambda_1}^3\left(\vz_0, d\vz_1^{(3)}\right)\) & \(\ldots\) & \(K_{\vlambda_1}^N\left(\vz_0, d\vz_1^{(N)}\right)\) \\
   \(t=2\) & \(K_{\vlambda_2}^{N + 1}\left(\vz_0, d\vz_2^{(1)}\right)\) & \(K_{\vlambda_2}^{N + 2}\left(\vz_0, d\vz_2^{(2)}\right)\) & \(K_{\vlambda_2}^{N + 3}\left(\vz_0, d\vz_2^{(3)}\right)\) & \(\ldots\) & \(K_{\vlambda_2}^{2\,N}\left(\vz_0, d\vz_2^{(N)}\right)\) \\
   \(\vdots\) & & & \(\vdots\) & & \\
   \(t=k\) & \(K_{\vlambda_k}^{\left(k-1\right)\,N + 1}\left(\vz_0, d\vz_k^{(1)}\right)\) & \(K_{\vlambda_k}^{\left(k-1\right)\,N + 2}\left(\vz_0, d\vz_k^{(2)}\right)\) & \(K_{\vlambda_k}^{\left(k-1\right)\,N + 3}\left(\vz_0, d\vz_k^{(3)}\right)\) & \(\ldots\) & \(K_{\vlambda_k}^{\left(k-1\right)\,N + N}\left(\vz_0, d\vz_k^{(N)}\right)\) \\
  \end{tabular}
  \egroup
  \end{center}
  }
  where \(K_{\vlambda}\left(\vz, \cdot\right)\) is an IMH kernel.
  Therefore, the \(n\)-step transition kernel for the vector of the Markov-chain samples \(\veta = \vz^{(1:N)}\) can be represented as
  \begin{align*}
  P_{\vlambda}^n\left(\veta, d\veta\prime\right)
  = 
  K_{\vlambda}^{N\,\left(n-1\right) + 1}\left(\vz_1, d\vz\prime_1\right)
  \,
  K_{\vlambda}^{N\,\left(n-1\right) + 2}\left(\vz_2, d\vz\prime_2\right)
  \cdot
  \ldots 
  \cdot
  K_{\vlambda}^{N\,\left(n-1\right) + N}\left(\vz_N, d\vz\prime_N\right).
  \end{align*}

%%   For example, let us assume that we use a batch size of 1 and that \(m=1\) is selected at \(t=1\) and \(m=2\) is selected at \(t=2\).
%%   Then, the transitions can be visualized as
%%   {\small
%%   \begin{center}
%%   \bgroup
%%   \def\arraystretch{1.8}
%%   \setlength{\tabcolsep}{5pt}
%%   \begin{tabular}{c|cccccccccccccc}
%%       t = 1 & \(\vz^{(1,1)}_1\) & \(\vz^{(1,2)}_1\) & \ldots & \(\vz^{(1,N)}_1\) & \ldots & \(\vz^{(2,1)}_1\) & \(\vz^{(2,2)}_1\) & \ldots & \(\vz^{(2,N)}_1\) & \ldots & \(\vz^{(1,N)}_1\) & \(\vz^{(2,N)}_1\) & \ldots & \(\vz^{(M,N)}_1\) \\
%%       & \(\downarrow\;K_{1,\lambda}\) & \(\downarrow\;K_{1,\lambda}^2\) & \ldots & \(\downarrow\;K_{1,\lambda}^N\)  &  &  &  & &  &  &  &  &  & \\
%%       t = 2 & \(\vz^{(1,1)}_2\) & \(\vz^{(1,2)}_2\) & \ldots & \(\vz^{(1,N)}_2\) & \ldots & \(\vz^{(2,1)}_2\) & \(\vz^{(2,2)}_2\) & \ldots & \(\vz^{(2,N)}_2\) & \ldots & \(\vz^{(1,N)}_2\) & \(\vz^{(2,N)}_2\) & \ldots & \(\vz^{(M,N)}_2\) \\
%%       & & & & & & \(\downarrow\;K_{2,\lambda}\) & \(\downarrow\;K_{2,\lambda}^2\) & \ldots & \(\downarrow\;K_{2,\lambda}^N\)  &  &  &  & &  \\
%%       t = 3 & \(\vz^{(1,1)}_3\) & \(\vz^{(1,2)}_3\) & \ldots & \(\vz^{(1,N)}_3\) & \ldots & \(\vz^{(2,1)}_2\) & \(\vz^{(2,2)}_3\) & \ldots & \(\vz^{(2,N)}_3\) & \ldots & \(\vz^{(1,N)}_3\) & \(\vz^{(2,N)}_3\) & \ldots & \(\vz^{(M,N)}_3\) \\
%%   \end{tabular}
%%   \egroup
%%   \end{center}
%%   }
%%   where \(K_{m,\vlambda}\left(\vz, \cdot\right)\) is a componentwise IMH kernel for the \(m\)th component.
%%   Conceptually, this means that, when a batch \(\vz^{(m,1:N)}\) is selected \(k\) times, it will evolve as
%%   \begin{alignat*}{2}
%%     \vz^{(m,1:N)} \sim  K_{m,\vlambda}^{\left(k-1\right)\,N + 1}\left(\vz^{(m,1)}, d\vz^{(m,1)}\right) \, K_{m,\vlambda}^{\left(k-1\right)\,N + 2}\left(\vz^{(m,2)}, d\vz\prime^{(m,2)}\right) \cdot \ldots \cdot K_{m,\vlambda}^{\left(k-1\right)\,N + N}\left(\vz^{(m,N)}, d\vz\prime^{(m,N)}\right).
%%   \end{alignat*}

%%   The random-scan kernel can be represented as
%%   \begin{alignat*}{2}
%%     P_{\vlambda}^k\left(\veta, d\veta\prime\right)
%%     =
%%     \sum_{m=1}^M
%%     r_{m} \,
%%     \left(
%%     K_{m,\vlambda}\left(\vz^{(m,1)}, d\vz\prime^{(m,1)}\right) \,
%%     K_{m,\vlambda}\left(\vz^{(m,2)}, d\vz\prime^{(m,2)}\right)
%%     \cdot
%%     \ldots
%%     \cdot
%%     K_{m,\vlambda}^N\left(\vz^{(m,N)}, d\vz\prime^{(m,N)}\right)
%%     \right)
%%   \end{alignat*}
%%   where .

  Now, the convergence in total variation \(d_{\mathrm{TV}}\left(\cdot, \cdot\right)\) can be shown to decrease geometrically as
  \begin{alignat}{2}
    &\DTV{P_{\vlambda}^{n}\left(\veta, \cdot\right)}{\Pi}
    \nonumber
    \\
    &\quad=
    \sup_{A}
    \abs{
      \Pi\left(A\right)
      -
      P^{n}\left(\veta, A\right)
    }
    &&\quad\text{\textit{Definition of \(d_{\text{TV}}\)}}
    \nonumber
    \\
    &\quad\leq
    \sup_{A}
    \Bigg|
    \int_{A}
      \pi\left(d\vz\prime^{(1)}\right) \times \ldots \times \pi\left(d\vz\prime^{(N)}\right)
    \nonumber
      \\
      &\qquad\qquad\qquad-
      K^{(n-1)\,N\,+1}_{\vlambda}\left(\vz^{(1)}, d\vz\prime^{(1)}\right) \times \ldots \times K^{n\,N}_{\vlambda}\left(\vz^{(N)}, d\vz\prime^{(N)}\right)
    \,\Bigg|
    \nonumber
    \\
    &\quad\leq
    \sup_{A}
    \sum_{n=1}^N
    \abs{
    \int_{A}
      \pi\left(d\vz^{(n)}\right) - K^{(n-1)\,N + n}_{\vlambda}\left(\vz^{(n)}, d\vz\prime^{(n)}\right) 
    }
    &&\quad\text{\textit{\cref{thm:product_measure_bound}}}
    \nonumber
    \\
    &\quad=
    \sum_{n=1}^N
    \DTV{K^{(n-1)\,N + n}_{\vlambda}\left(\vz^{(n)}, \cdot\right)}{\pi}
    &&\quad\text{\textit{Definition of \(d_{\text{TV}}\)}}
    \nonumber
    \\
    &\quad\leq
    \sum_{n=1}^N
    \rho^{(n-1)\,N + n}
    &&\quad\text{\textit{Geometric ergodicity}}
    \nonumber
    %\label{eq:used_ergodicity}
    \\
    &\quad=
    \rho^{n\,N}
    \,
    \rho^{-N}
    \,
    \frac{\rho - \rho^{N+1}}{1 - \rho}
    &&\quad\text{\textit{Solved sum}}
    \nonumber
    \\
    &\quad=
    \frac{\rho \, \left(1 - \rho^N\right)}{\rho^N \left(1 - \rho\right)}
    \,
    {\left( \rho^{N} \right)}^n.
    \nonumber
  \end{alignat}
  Although the constant depends on \(\rho\) and \(N\), the kernel \(P\) is geometrically ergodic and converges \(N\) times faster than the base kernel \(K\).

  \paragraph{\textbf{Bound on the Gradient Variance}}
  To analyze the variance of the gradient, we require a detailed expression of the \(n\)-step marginal transition kernel, which is unavailable in general for most MCMC kernels.
  Fortunately, specifically for the IMH kernel,~\citet{Smith96exacttransition} have shown that the \(n\)-step marginal IMH kernel is given as~\cref{eq:imh_exact_kernel}.
  From this, we show that
  \begin{alignat}{2}
    &\E{ \norm{ \vg\left(\vlambda, \rvveta\right) }^2_{*} \,\middle|\, \mathcal{F}_{t} }
    \nonumber
    \\
    &\quad=
    \E{ \norm{ \vg\left(\vlambda, \rvveta\right) }^2 \,\middle|\, \vz_{t-1}^{(N)},\, \vlambda_{t-1} }
    \nonumber
    \\
    &\quad=
    \E{ \norm{ \frac{1}{N}\sum^{N}_{n=1} \vs\left(\vlambda; \rvvz^{(n)}\right) }^2_{*} \,\middle|\, \vz_{t-1}^{(N)},\, \vlambda_{t-1} }
    \nonumber
    \\
    &\quad\leq
    \E{ \frac{1}{N^2} \sum^{N}_{n=1} \norm{\vs\left(\vlambda; \rvvz^{(n)}\right) }^2_{*} \,\middle|\, \vz_{t-1}^{(N)},\, \vlambda_{t-1} }
    &&\quad\text{\textit{Triangle inequality}}
    \nonumber
    \\
    &\quad=
    \frac{1}{N^2}\sum^{N}_{n=1} \Esub{\rvvz^{(n)} \sim K^n\left(\vz_{t-1}, \cdot\right)}{ \norm{\vs\left(\vlambda; \rvvz^{(n)}\right) }^2_{*} \,\middle|\,  \vz_{t-1}^{(N)},\, \vlambda_{t-1} }
    &&\quad\text{\textit{Linearity of expectation}}
    \nonumber
    \\
    &\quad\leq
    \frac{1}{N^2}\sum^{N}_{n=1}
      n \, {\left(1 - \frac{1}{w^*}\right)}^{n-1}
      \Esub{\rvvz^{(n)} \sim q_{\text{def.}}\left(\cdot; \vlambda\right)}{ \norm{\vs\left(\vlambda; \rvvz^{(n)} \right)}_{*}^2 }
      \nonumber
      \\
      &\qquad+ 
        {\left(1 - \frac{1}{w^*}\right)}^{n} \, \norm{\vs\left(\vlambda; \vz_{t-1}^{(N)} \right)}_{*}^2
    &&\quad\text{\textit{\cref{thm:imh_expecation}}}
    \nonumber
    \\
    &\quad\leq
    \frac{1}{N^2}\sum^{N}_{n=1}
      n \, {\left(1 - \frac{1}{w^*}\right)}^{n-1}  \, L^2
      +
      {\left(1 - \frac{1}{w^*}\right)}^{n} \, L^2
    &&\quad\text{\textit{\cref{thm:bounded_score}}}
    \nonumber
    \\
    &\quad=
    \frac{L^2}{N^2}\sum^{N}_{n=1}
      n \, {\left(1 - \frac{1}{w^*}\right)}^{n-1}
      +
      {\left(1 - \frac{1}{w^*}\right)}^{n} 
    \nonumber
    &&\quad\text{\textit{Moved constant forward}}
    \\
    &\quad=
    \frac{L^2}{N^2} \,
    \left[\,
      {\left(w^*\right)}^2 + w^*
      -
      {\left(1 - \frac{1}{w^*}\right)}^N
      \left(
        {\left(w^*\right)}^2 + w^* + N\,w^*
      \right)
    \,\right]
    \nonumber
    &&\quad\text{\textit{Solved sum}}
    \\
    &\quad=
    \frac{L^2}{N^2} \,
    \left[\,
    \frac{1}{2} N^2 + \frac{3}{2}\,N 
    + \mathcal{O}\left(1/w^*\right)
    \,\right]
    \nonumber
    &&\quad\text{\textit{Laurent series expansion at \(w^* \rightarrow \infty\)}}
    \\
    &\quad=
    L^2 \,
    \left[\,
    \frac{1}{2} + \frac{3}{2}\,\frac{1}{N}
    + \mathcal{O}\left(1/w^*\right)
    \,\right].
    \nonumber
  \end{alignat}
  The laurent approximation is useful for realistic values of \(w^*\) since it is bounded below exponentially by the KL divergence.
\end{proofEnd}

%%% Local Variables:
%%% TeX-master: "master"
%%% End:

%
The gradient bound suggests that, in realistic settings where \(w^*\) is large, increasing \(N\) does not improve variance.
To fix this problem, we propose a new MCSA scheme that achieves \(\mathcal{O}\left(\nicefrac{1}{N}\right)\) variance reduction.
In particular, instead of using \(N\) \textit{sequential} Markov-chain states, as we operate \(N\) parallel Markov-chains.
To obtain a similar per-SGD-iteration cost, we perform only a single Markov-chain transition for each chain.
We will later discuss the computational costs in detail.
A visual illustration can be found in~\cref{section:illustration}.


\begin{theoremEnd}{theorem}\label{thm:pmcsa}
  pMCSA, our proposed scheme, is obtained by setting
  {%\small
  \begin{align*}
    P_{\vlambda}^n\left(\veta, d\veta\prime\right)
    = 
    K_{\vlambda}^n\left(\vz^{(1)}, d\vz\prime^{(1)}\right)
    \,
    K_{\vlambda}^n\left(\vz^{(2)}, d\vz\prime^{(2)}\right)
    \cdot
    \ldots 
    \cdot
    K_{\vlambda}^n\left(\vz^{(N)}, d\vz\prime^{(N)}\right)
  \end{align*}
  }
  with \(\veta = \left[\vz^{(1)}, \vz^{(2)}, \ldots, \vz^{(N)}\right]\).
  Then, given~\cref{thm:bounded_weight,thm:bounded_score}, the mixing rate and the gradient variance bounds are
  {\small
  \begin{align*}
    \DTV{P_{\vlambda}^n\left(\veta, \cdot\right)}{\Pi}
    \leq
    C\left(N\right)\,{\rho^n}
    \quad\text{and}\quad
    \E{ \norm{ \vg\left(\vlambda, \rvveta\right) }^2_{*} \,\middle|\, \mathcal{F}_{t} }
    \leq
    L^2 \left[\; \frac{1}{N} + \frac{1}{N}\,\left(1 - \frac{1}{w^*}\right) \;\right],
  \end{align*}
  }
  where \(w^* = \sup_{\vz} \pi\left(\vz\right) / q_{\text{def.}}\left(\vz\right)\) and \(C\) is some positive constant depending on \(N\).
\end{theoremEnd}
\begin{proofEnd}

  Our proposed scheme, pMCSA, is described in~\cref{alg:jsa}. 
  At each iteration, our scheme performs a single MCMC transition for each of the \(N\) samples, or chains, to estimate the gradient.
  Similarly to JSA, we use the IMH kernel \(K_{\vlambda}\).

  \paragraph{Ergodicity of the Markov Chain}
  Since our kernel operates the same MCMC kernel \(K_{\vlambda}\) for each of the \(N\) parallel Markov chains, the \(n\)-step marginal kernel \(P_{\vlambda}\) can be represented as
  \begin{align*}
    P_{\vlambda}^n\left(\veta, d\veta\prime\right)
    = 
    K_{\vlambda}^n\left(\vz^{(1)}, d\vz\prime^{(1)}\right)
    \,
    K_{\vlambda}^n\left(\vz^{(2)}, d\vz\prime^{(2)}\right)
    \cdot
    \ldots 
    \cdot
    K_{\vlambda}^n\left(\vz^{(N)}, d\vz\prime^{(N)}\right).
  \end{align*}
  Then, the convergence in total variation \(d_{\mathrm{TV}}\left(\cdot, \cdot\right)\) can be shown to decrease geometrically as
  \begin{alignat}{2}
    &\DTV{K^{k}\left(\veta, \cdot\right)}{\Pi}
    \nonumber
    \\
    &\quad=
    \sup_{A}
    \abs{
      \Pi\left(A\right)
      -
      P^{n}\left(\veta, A\right)
    }
    &&\quad\text{\textit{Definition of \(d_{\text{TV}}\)}}
    \nonumber
    \\
    &\quad\leq
    \sup_{A}
    \big|\;
    \int_{A}
      \pi\left(d\vz\prime_1\right) \cdot \ldots \cdot \pi\left(d\vz\prime_N\right)
    \nonumber
      \\
      &\qquad\qquad\qquad-
      K^n\left(\vz_1, d\vz\prime_1\right) \cdot \ldots \cdot K^n\left(\vz_N, d\vz\prime_N\right)
    \;\big|
    \nonumber
    \\
    &\quad\leq
    \sup_{A}
    \sum_{n=1}^N
    \abs{
    \int_{A}
      \pi\left(d\vz\prime_k\right) - K^{n}\left(\vz_n, d\vz\prime_n\right) 
    }
    &&\quad\text{\textit{\cref{thm:product_measure_bound}}}
    \nonumber
    \\
    &\quad=
    \sum_{n=1}^N
    \DTV{K^n\left(\vz_n, \cdot\right)}{\pi}
    &&\quad\text{\textit{Definition of TV distance}}
    \nonumber
    \\
    &\quad\leq
    \sum_{n=1}^N
    \rho^{n}
    &&\quad\text{\textit{Geometric ergodicity}}
    \nonumber
    \\
    &\quad=
    N\,\rho^{k}
    &&\quad\text{\textit{Solved sum}}.
    \nonumber
  \end{alignat}

  \paragraph{\textbf{Bound on the Gradient Variance}}
  The bound on the gradient variance can be derived in similar manner to JSA as
  \begin{alignat}{2}
    &\E{ \norm{ \vg\left(\vlambda, \rvveta\right) }^2_{*} \,\middle|\, \mathcal{F}_{t} }
    \nonumber
    \\
    &\quad=
    \E{ \norm{ \vg\left(\vlambda, \rvveta\right) }^2 \,\middle|\, \vz_{t-1}^{(1:N)},\, \vlambda_{t-1} }
    \nonumber
    \\
    &\quad=
    \E{ \norm{ \frac{1}{N}\sum^{N}_{n=1} \vs\left(\vlambda; \rvvz_{n}\right) }^2_{*} \,\middle|\, \vz_{t-1}^{(1:N)},\, \vlambda_{t-1} }
    \nonumber
    \\
    &\quad\leq
    \E{  \frac{1}{N^2} \sum^{N}_{n=1} \norm{\vs\left(\vlambda; \rvvz_{n}\right) }^2_{*} \,\middle|\, \vz_{t-1}^{(1:N)},\, \vlambda_{t-1} }
    &&\quad\text{\textit{Triangle inequality}}
    \nonumber
    \\
    &\quad=
    \frac{1}{N^2}\sum^{N}_{n=1} \Esub{\rvvz_{n} \sim K\left(\vz_{t-1}^{(n)}, \cdot\right)}{ \norm{\vs\left(\vlambda; \rvvz_{n}\right) }^2_{*} \,\middle|\,  \vz_{t-1}^{(1:N)},\, \vlambda_{t-1} }
    &&\quad\text{\textit{Linearity of expectation}}
    \nonumber
    \\
    &\quad\leq
    \frac{1}{N^2}\sum^{N}_{n=1}
      \Esub{\rvvz_n \sim q_{\text{def.}}\left(\cdot;\vlambda\right)}{ \norm{\vs\left(\vlambda; \rvvz_n \right)}_{*}^2 }
      +
      {\left(1 - \frac{1}{w^*}\right)} \, \norm{\vs\left(\vlambda; \vz_{t-1}^{(n)} \right)}_{*}^2
    &&\quad\text{\textit{\cref{thm:imh_expecation}}}
    \nonumber
    \\
    &\quad\leq
    \frac{1}{N^2}\sum^{N}_{n=1}
        L^2 + {\left(1 - \frac{1}{w^*}\right)} \, L^2
    &&\quad\text{\textit{\cref{thm:bounded_score}}}
    \nonumber
    \\
    &\quad=
    \frac{L^2}{N^2} \sum^{N}_{n=1}
      1 + {\left(1 - \frac{1}{w^*}\right)}
    \nonumber
    &&\quad\text{\textit{Moved constant forward}}
    \\
    &\quad=
    L^2 \left[ \frac{1}{N} + \frac{1}{N}\,{\left(1 - \frac{1}{w^*}\right)} \right].
    \nonumber
    &&\quad\text{\textit{Solved sum}}
  \end{alignat}
  %Therefore, the gradient decreases at a solid \(\mathcal{O}\left(1/N\right)\) rate.
\end{proofEnd}

%%% Local Variables:
%%% TeX-master: "master"
%%% End:


\paragraph{Bias v.s. Variance}
While our proposed scheme achievs superior variance reduction, the mixing rate is worse.
In a MCMC estimation perspective, this translates into higher bias.
However, we note that
\begin{enumerate*}[label=\textbf{(\roman*)}]
  \item the constant \(C\left(\rho, N\right)\) depends on \(\rho\), 
  \item all of the ergodic convergence rate are close to 1 as \(w^* \rightarrow \infty\), and
  \item the mixing rate is a conservative global bound with respect to \(\vlambda\).
\end{enumerate*}
Therefore, in general, the superior ergodic convergence rate of JSA does not translate into faster convergence of MCSA.
In fact, as MCSA converges, \(w^*\) also decreases, dramatically improving the mixing rate.
In contrast, the relative variance does not improve too much with \(w^*\).
Therefore, reducing the variance is much more effective for accelerating convergence.
We empirically show this fact on the bias and variance in~\cref{section:simulation}.

\begin{wraptable}{r}{0.6\textwidth}
  \vspace{-0.5in}
  
% Second version of table, with booktabs.
%\begin{table}
%\centering
\caption{Computational Costs of MCSA Schemes}\label{table:cost}
\setlength{\tabcolsep}{0.5pt}
  \begin{threeparttable}
\begin{tabular}{lccccc}\toprule
& \multicolumn{3}{c}{\footnotesize Kernel Application} & \multicolumn{2}{c}{\footnotesize Gradient Estimation} \\
\cmidrule(lr){2-4}\cmidrule(lr){5-6}
  & {\footnotesize\(p\left( \vz, \vx \right)\)}
  & {\footnotesize\(q\left(\vz; \vlambda\right)\)}
  & {\footnotesize\(q\left(\vz; \vlambda\right)\)}
  & {\footnotesize\(p\left( \vz, \vx \right)\)}
  & {\footnotesize\( q\left(\vz; \vlambda\right)\)}
  \\
  & {\footnotesize\# Eval.  }
  & {\footnotesize\# Eval.  }
  & {\footnotesize\# Samples}
  & {\footnotesize\# Grad.  }
  & {\footnotesize\# Grad.  }
%
\\\midrule
%
{\footnotesize
ELBO
}
& \(0\)
& \(0\)
& \(N\)
& \(N\)
& \(N\)
\\\arrayrulecolor{black!30}\midrule
%
{\footnotesize
MSC
}
& \(N-1\)
& \(N\)
& \(N-1\)
& \(0\)
& \(1\)\tnote{1}\;\;{\footnotesize or}\;\(N\)\tnote{2}
\\
%
{\footnotesize
JSA
}
& \(N\)
& \(N+1\)
& \(N\)
& \(0\)
& \(N\)
\\
%
{\footnotesize
\textit{pMCSA}
}
& \(N\)
& \(2 \, N\)
& \(N\)
& \(0\)
& \(N\)
\\\bottomrule
\end{tabular}
  \begin{tablenotes}
    \item[*]{\footnotesize We assume that the parameters are cached as much as possible}.
    \item[1]{\footnotesize Vanilla CIS kernel}.
    \item[2]{\footnotesize Rao-Blackwellized CIS kernel}.
  \end{tablenotes}
  \end{threeparttable}
%\end{table}

  \vspace{-0.2in}
\end{wraptable}
%
\subsection{Computational Cost}
The three schemes using the CIS kernel and the IMH kernel can have different computational costs depending on the parameter \(N\).
The computational costs of each scheme are organized in~\cref{table:cost} while detailed pseudocodes of the considered schemes are provided in the \textit{supplementary material}.

\vspace{-0.05in}
\paragraph{Cost of Sampling Proposals}
For the CIS kernel used by MSC, \(N\) controls the number of internal proposals sampled from \(q_{\vlambda}(\vz)\).
For JSA and our proposed scheme, the IMH kernel only uses a single sample from \(q_{\vlambda}(\vz)\), but applies the kernel \(N\) times.
Assuming caching is done as much as possible, the parallel state estimator needs twice the density evaluations of \(q_{\vlambda}(\vz)\) compared to other methods.
However, this added cost is minimal since the overall computational cost is dominated by  \(p(\vz,\vx)\).

\vspace{-0.05in}
\paragraph{Cost of Estimating the Score}
When estimating the score, MSC computes \(\nabla_{\vlambda} \log q_{\vlambda}(\vz)\) only once, while JSA and our proposed scheme compute it \(N\) times.
However,~\cite{NEURIPS2020_b2070693} also discuss a Rao-Blackwellized version of the CIS kernel, which also computes the gradient \(N\) times.
Lastly, notice that score climbing does not need to differentiate through the likelihood, unlike ELBO maximization, making its base computational cost significantly cheaper.

%% \vspace{-0.05in}
%% \subsection{Motivation and Overview}\label{section:motivation}
%% \vspace{-0.05in}
%% \paragraph{Motivating Example}
%% According to traditional MCMC theory, multiple short Markov chains will be more biased than a single long Markov chain.
%% Therefore, it is natural to expect the parallel estimator to be more biased than the sequential estimator.
%% However, we show an example where this intuition is wrong: As shown in~\cref{fig:gaussian}, in this example, the parallel state estimator enjoys not only low variance but also low bias.
%% %We ran score climbing VI with the three different estimators and compared the bias and variance of the estimators.
%% The target distribution was a 10 dimensional multivariate Gaussian where the covariance was sampled from a Wishart distribution with \(\nu = 50\) degrees of freedom.
%% The variational family was a mean-field Gaussian.
%% The bias and variance was estimated from \(512\) independent replications.


%%% Local Variables:
%%% TeX-master: "master"
%%% End:
