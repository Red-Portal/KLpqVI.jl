
%\vspace{-0.05in}
\section{Practical Performance Analysis of Markov Chain Score Ascent}
\subsection{Non-Asymptotic Convergence of Markov Chain Score Ascent}\label{section:convergence}

\begin{table*}
\vspace{-0.1in}
\centering
\caption{Convergence Rates of MCGD Algorithms}\label{table:convergence}
\setlength{\tabcolsep}{3pt}
\begin{threeparttable}
  \begin{tabular}{lllcc}\toprule
    \multicolumn{1}{c}{\footnotesize\textbf{Algorithm}} & \multicolumn{1}{c}{\footnotesize\textbf{Stepsize Rule}} & \multicolumn{1}{c}{\footnotesize\textbf{Gradient Assumption}} & {\footnotesize\textbf{Rate}} & {\footnotesize\textbf{Reference}} \\\midrule
    \multirow{2}{*}{\small Mirror Descent\tnote{1}}
    & \multirow{2}{*}{\small\(\gamma_t = \gamma / \sqrt{t}\)}
    & \multirow{2}{*}{\small\(\E{ {\|\, \vg\left(\vlambda, \veta\right) \,\|}_*^2 \mid \mathcal{F}_t } < G^2\)}
    & \multirow{2}{*}{\small\(\mathcal{O}\left(\frac{G^2 \log T}{ \log \rho^{-1} \sqrt{T}}\right)\)}
    & {\footnotesize\citet{duchi_ergodic_2012}}
    \\
    &&&& {\footnotesize{Corollary 3.5}}
    \\\cdashlinelr{1-5}
    \multirow{2}{*}{\small SGD-Nesterov\tnote{2}}
    & {\small\(\gamma_t = 2/(t + 1)\)}
    & \multirow{2}{*}{\footnotesize\( {\|\vg\left(\vlambda, \veta\right)\|}_2 < G \)}
    & \multirow{2}{*}{\small\(\mathcal{O}\left(\frac{G^2 \log T}{ \sqrt{T}}\right)\)}
    & {\footnotesize\citet{doan_convergence_2020}}
    \\
    & {\footnotesize\(\beta_t = \frac{1}{2 \, L \sqrt{t + 1}}\)}
    &&& {\footnotesize{Theorem 2}}
    \\\cdashlinelr{1-5}
    \multirow{2}{*}{\small SGD}
    & {\footnotesize\(\gamma_t = \gamma/t\)}
    & \multirow{2}{*}{\footnotesize\( {\|\,\vg\left(\vlambda, \veta\right)\|}_* < G \left( \norm{\vlambda}_2 + 1 \right) \)}
    & \multirow{2}{*}{\small\(\mathcal{O}\left(\frac{G^2 \log T}{ T}\right)\)}
    & {\footnotesize\citet{doan_finitetime_2020}}
    \\ 
    & {\footnotesize\(\gamma = \min\{\nicefrac{1}{2\,L}, \nicefrac{2 L}{\mu}\}\)}
    &&& {\footnotesize{Theorem 1,2}}
    \\ \bottomrule
  \end{tabular}
  \begin{tablenotes}
  \item[1]{ \(\mathcal{F}_t\) is the \(\sigma\)-field formed by all the iterates \(\veta_t\), \(\vlambda_t\) up to the \(t\)th SGD iteration. }
  \item[1]{ \(\norm{\vx}_*\) is the dual norm such that \(\norm{\vx}_* = \sup_{\norm{\vz} \leq 1} \iprod{\vx}{\vz}\).}
  \item[2]{ \(\beta_t\) is the stepsize of the momentum.}
  \item[2,3]{ \(L\) is the Lipschitz smoothness constant.}
  \end{tablenotes}
\end{threeparttable}
\vspace{-0.2in}
\end{table*}


\vspace{-0.05in}
\paragraph{Markov Chain Score Ascent}
As shown in~\cref{eq:mcgd}, the basic ingredients of MCGD are the target function \(f\left(\vlambda, \eta\right)\), the gradient estimator \(\vg\left(\vlambda, \eta\right)\), and the Markov chain kernel \(P_{\vlambda}\left(\eta, \cdot\right)\).
Obtaining MCSA through MCGD boils down to designing \(\vg\) and \(P_{\vlambda}\) such that \(f\left(\vlambda\right) = \DKL{\pi}{q\left(\cdot; \vlambda\right)} \).
In the following theorem, we provide practical conditions for setting \(g\) and \(P_{\vlambda}\) such that the MCGD steps in~\cref{eq:mcgd} results in MCSA.


\begin{theoremEnd}{proposition}\label{thm:product_kernel}
  Let \(\veta = \left( \vz^{(1)}, \vz^{(2)}, \ldots, \vz^{(N)} \right)\) and a Markov chain kernel \(P_{\vlambda}\left(\veta, \cdot\right)\) be \(\Pi\)-invariant where \(\Pi\) is defined as
  {%\small
  \[
  \Pi\left(\veta\right) = \pi\left(\vz^{(1)}\right) \, \pi\left(\vz^{(2)}\right) \times \ldots \times \pi\left(\vz^{(N)}\right).
  \]
  }
  Then, by defining the target function \(f\) and the gradient estimator \(\vg\) to be 
  {\small
  \begin{align*}
    f\left(\vlambda, \veta\right) =  \frac{1}{N} \sum^{N}_{n=1} \log q\left(\vz^{(n)}; \vlambda\right) + \mathbb{H}\left[\,\pi\,\right], 
    \quad\text{and}\quad
    \vg\left(\vlambda, \veta\right) =  \frac{1}{N} \sum^{N}_{n=1} \vs\left(\vz^{(n)}; \vlambda\right)
  \end{align*}
  }
  where \(\mathbb{H}\left[\,\pi\,\right]\) is the entropy of \(\pi\), MCGD results in inclusive KL minimization as
  {%\small
  \begin{align*}
    \Esub{\Pi}{ f\left(\vlambda, \rvveta\right) } = \DKL{\pi}{q\left(\cdot; \vlambda\right)},
    \quad\text{and}\quad
    \Esub{\Pi}{ \vg\left(\vlambda, \rvveta\right) } = \nabla_{\vlambda} \DKL{\pi}{q\left(\cdot; \vlambda\right)}.
  \end{align*}
  }
\end{theoremEnd}
\begin{proofEnd}
  For notational convenience, we define the shorthand
  \begin{align*}
    \pi\left(\vz^{(1:N)}\right) = \pi\left(\vz^{(1)}\right) \, \pi\left(\vz^{(2)}\right) \times \ldots \times \pi\left(\vz^{(N)}\right).
  \end{align*}
  Then,
  \begin{alignat}{2}
    &\Esub{\Pi}{ f\left(\vlambda, \veta\right) }
    \nonumber
    \\
    &\quad=
    \int \left( \frac{1}{N} \sum^{N}_{n=1} \log q\left(\vz^{(n)}; \vlambda\right) + \mathbb{H}\left[\,\pi\,\right]\right) \, \pi\left(\vz^{(1:N)}\right) \, d\vz^{(1:N)}
    \nonumber
    \\
    &\quad=
     \frac{1}{N} \sum^{N}_{n=1} \left\{ \int \big(\, \log q\,(\,\vz^{(n)}; \vlambda\,) + \mathbb{H}\,[\,\pi\,] \,\big) \, \pi\left(\vz^{(1:N)}\right) \, d\vz^{(1:N)} \right\}
     &&\quad{\text{\textit{Swapped integral and sum}}}
    \nonumber
    \\
    &\quad=
    \frac{1}{N} \sum^{N}_{n=1} \int \big(\, \log q\,(\,\vz^{(n)}; \vlambda\,) + \mathbb{H}\,[\,\pi\,] \,\big) \, \pi\left(\vz^{(n)}\right) \, d\vz^{(n)}
    &&\quad{\text{\textit{Marginalized \(\vz^{(m)}\) for all \(m \neq n\)}}}
    \nonumber
    \\
    &\quad=
    \frac{1}{N} \sum^{N}_{n=1} \DKL{\pi}{q\left(\cdot; \vlambda\right)}
    \nonumber
    &&\quad{\text{\textit{Definition of \(d_{\text{KL}}\)}}}
    \\
    &\quad=
    \DKL{\pi}{q\left(\cdot; \vlambda\right)}\label{eq:F_KL}
  \end{alignat}
  For \(\Esub{\Pi}{ \vg\left(\vlambda, \rvveta\right) }\), note that 
  \begin{align}
    \nabla_{\vlambda} f\left(\vlambda, \veta\right) = \frac{1}{N} \sum^{N}_{n=1} \vs\left(\vz^{(n)}; \vlambda\right) = \vg\left(\vlambda, \veta\right).\label{eq:F_grad_G}
  \end{align}
  Therefore, it suffices to show that
  \begin{align*}
    \nabla_{\vlambda} \DKL{\pi}{q\left(\cdot; \vlambda\right)}
    &=
    \nabla_{\vlambda} \Esub{\Pi}{ f\left(\vlambda, \rvveta\right) }
    &&\text{\textit{\cref{eq:F_KL}}}
    \\
    &=
    \Esub{\Pi}{ \nabla_{\vlambda}  f\left(\vlambda, \rvveta\right) }
    &&\text{\textit{{Leibniz derivative rule}}}
    \\
    &=
    \Esub{\Pi}{ \vg\left(\vlambda, \rvveta\right) }.
    &&\text{\textit{\cref{eq:F_grad_G}}}
  \end{align*}
\end{proofEnd}

%%% Local Variables:
%%% TeX-master: "master"
%%% End:


This framework is general enough to include both JSA and MSC.
We will later propose a third novel scheme that conforms to~\cref{thm:product_kernel}.
Note that \(N\) here can be regarded as the computational budget of each MCGD iteration since the cost of
\begin{enumerate*}[label=\textbf{(\roman*)}]
  \item generating the Markov chain samples \(\vz^{(1)}, \ldots, \vz^{(N)}\) and
  \item computing the gradient \(\vg\)
\end{enumerate*}
will linearly increase with \(N\).

In addition, to utilize the convergence analysis results of MCGD, we require \(P\) to be geometrically ergodic.
An execption is~\citet{debavelaere_convergence_2021} where they assume \(P\) to be polynomially ergodic.
\begin{assumption}{(Markov chain kernel)}\label{thm:kernel_conditions}
\vspace{-0.05in}
  The Markov chain kernel \(P\) is geometrically ergodic as
  {%\small
  \[
  \DTV{P_{\vlambda}^{n}\left(\veta, \cdot\right)}{ \Pi } \leq C \, \rho^{n}
  \]
  }
  for some positive constant \(C\).
\vspace{-0.05in}
\end{assumption}
\vspace{-0.05in}

\vspace{-0.05in}
\paragraph{Non-Asymptotic Convergence Rates}
Based on the conclusion of \cref{thm:product_kernel,thm:kernel_conditions} and additional assumptions on the objective function such as convexity, we can apply the previous convergence results of MCGD to MCSA.
\cref{table:convergence} provides a brief list of some relevant results.
We omitted terms resulting from assumptions on the objective function, such as Lipschitz smoothness.
Instead, we focus on the terms including the gradient variance (\(G\)) and mixing rate (\(\rho\)) since they are closely related to the algorithmic design of MCSA algorithms.

\vspace{-0.05in}
\paragraph{Convergence and Mixing Rate}
\citet{duchi_ergodic_2012} was the first to provide an analysis of the general MCGD setting.
Notice that their rate is dependent on the mixing rate by the \(1 / \log \rho^{-1}\) term.
For MCSA, this is very conservative since, in challenging problems, mixing could be slow such that \(\rho \approx 1\).
Fortunately, recent results by \citet{doan_convergence_2020,doan_finitetime_2020} have shown that it is possible to obtain convergence rates independent of the mixing rate.
In particular, in the analysis of \citet{doan_finitetime_2020}, the influence of the mixing rate decreases in a rate of \(\mathcal{O}\left(\nicefrac{1}{T^2}\right)\).
The fact that the convergence rate can be independent of the mixing rate is critical.
It means trading gradient variance and mixing rate could result in profitable deals.

\vspace{-0.05in}
\paragraph{Gradient Bound Assumption}
Except for \citet{doan_finitetime_2020}, most results assume that the gradient is bounded for \(\forall\veta,\vlambda\) such that \( {\| \vg\left(\vlambda, \veta\right) \|} < G \).
Although this condition is very strong for MCSA in general, it is similar to the bounded variance assumption \(\mathbb{E}\,[\norm{\vg}^2]  < G^2\) used in vanilla SGD, which is also known to be strong since it does not hold for strongly convex objectives~\citep{pmlr-v80-nguyen18c}.
%Furthermore, strictly speaking, even the assumption of~\citet{doan_finitetime_2020} is still strong to include the most basic variational families.
%% For example for the Gaussian family, the gradient bound with respect to the covariance matrix is bounded as
%% {\small
%% \begin{align*}
%%   \norm{ \nabla_{\mSigma} \log q\left(\vz; \vlambda\right) }_F
%%   \leq
%%   \norm{
%%   \nabla_{\mSigma} {\left(\vmu - \vz\right)}^{\top} \mSigma^{-1} {\left(\vmu - \vz\right)}
%%   }_{F}
%%   =
%%   \norm{
%%    \mSigma^{-\top} \, {\left(\vmu - \vz\right) \, \left(\vmu - \vz\right)}^{\top} \, \mSigma^{-\top}
%%   }_{F}
%% \end{align*}
%% }%
%where \(\vlambda = \left(\mu, \mSigma\right)\).
%Even if we bound the norm of \(\vz\), it is apparent that the gradient grows as \(\norm{\vlambda}^2_2\) instead of \(\norm{\vlambda}_2\).
Nonetheless, assuming the existence of \(G\) can lead to useful analysis with practical benefits.
For example, it can be used to compare the performance of different algorithms as done by~\citet{pmlr-v108-geffner20a}.
In a similar spirit, we will obtain the gradient bound \(G\) of different MCSA algorithms and compare them.

%

\begin{theoremEnd}{theorem}
  The gradient upper bounds of Markov chain score ascent methods can be found as follows:
  \begin{itemize}
    \item MSC~\citep{NEURIPS2020_b2070693}:
    \item JSA~\citep{pmlr-v124-ou20a}:
    \begin{align}
      G_{\text{JSA}}^2 = 
      L^2 \left[\;
      \frac{1}{2} 
      +
      \frac{1}{2 \, N} 
      +
      \frac{1}{N^2} 
      {\left(1 - \frac{1}{w^*}\right)}
      \;\right]
    \end{align}
    \item This work:
      \begin{align}
        G^2_{\text{ours}} = L^2 \left[\; \frac{1}{N} + \frac{1}{N}\,\left(1 - \frac{1}{w^*}\right) \;\right]
      \end{align}
  \end{itemize}
\end{theoremEnd}

%%% Local Variables:
%%% TeX-master: "master"
%%% End:



%% \begin{assumption}{\textbf{(Bounded variance)}}\label{thm:bounded_variance}
%%   Let \(\vlambda \in \Lambda\) be measurable with respect to the \(\sigma\)-field \(\mathcal{F}_{t-1}\).
%%   The  gradient estimator \(g\) is bounded a constant \(G < \infty\) such that
%%   \(
%%   \E{ {\lVert\, g\left(\cdot, \rvveta_{t}\right) \,\rVert}^2_{*} \;\middle|\; \mathcal{F}_{t-1}} < G^2.
%%   \)
%% \end{assumption}

%% Under the stated conditions, the non-asymptotic convergence rate of MCSA is a special case of the ergodic mirror descent algorithm of~\citet{duchi_ergodic_2012}.

%
\begin{theoremEnd}[all end]{lemma}\label{thm:mixing_time}
  Assuming the kernel \(P_{\vlambda}\left(\veta, \cdot\right)\) satisfies \cref{thm:ergodicity}, the Hellinger mixing time \(\tau_{\text{Hel.}}\) is bounded as
  \begin{align*}
    \tau_{\text{Hel.}}\left(K_{\vlambda}, \epsilon\right) \leq  \frac{2}{\log \rho^{-1}} \log \frac{1}{\epsilon}
  \end{align*}
\end{theoremEnd}
\begin{proofEnd}
  The following two inequalities are equivalent.
  \begin{align*}
    d_{\text{Hel.}}\left(\, P^t_{\vlambda}\left(\veta, \cdot \right), \Pi \,\right)
    &\leq \sqrt{\DTV{ P^t_{\vlambda}\left(\veta, \cdot \right)}{\Pi}}
    \leq \rho^{t / 2}
    \\
    \log d_{\text{Hel.}}\left(\, P^t_{\vlambda}\left(\veta, \cdot \right), \Pi \,\right)
    &\leq \frac{t}{2} \log \rho
  \end{align*}

  The Hellinger mixing time \(\tau_{\text{Hel.}}\left(K_{\vlambda}, \epsilon\right)\) is the smallest \(t\) that statisfies the inequality
  \begin{align*}
    d_{\text{Hel.}}\left(\, P^t_{\vlambda}\left(\veta, \cdot \right), \Pi \,\right)
    &\leq 
    \epsilon.
  \end{align*}
  Instead, we can find the \(t\prime > t\) that satisfies
  \begin{align*}
    \log d_{\text{Hel.}}\left(\, P^t_{\vlambda}\left(\veta, \cdot \right), \Pi \,\right)
    \leq 
    \frac{t\prime}{2} \log \rho
    \leq 
    \log \epsilon
 \end{align*}
  by solving the inequalities
  \begin{alignat*}{2}
    \frac{t\prime}{2} \log \rho
    &\leq 
    \log \epsilon
    \\
    t\prime 
    &\geq 
    \frac{2}{\log \rho} \log \epsilon
    &&\quad\text{\textit{Inequality flipped since \(\log \rho \leq 1\)}}
    \\
    t\prime 
    &\geq 
    \frac{2}{\log \rho^{-1}} \log \frac{1}{\epsilon}
 \end{alignat*}
\end{proofEnd}

\begin{theoremEnd}{theorem}{(\textbf{Convergence rate})}\label{thm:convergence_rate}
  Assuming~\cref{thm:kernel_conditions,thm:gradient_estimator,thm:logconcave,thm:compact,thm:bounded_variance} hold, with a diminishing stepsize of \(\alpha_t = R / \left( G \sqrt{\kappa_1 \log \kappa_2 T} \right)\), the average iterates {\small\(\overline{\vlambda}_T = \sum^{T}_{t=1} \vlambda_{t} / T\)} of Markov chain score ascent achieve a convergence rate of
  {%\small
  \begin{align*}
    \E{ \DKL{\pi}{q\,(\cdot; {\overline{\vlambda}_{T}})} - \DKL{\pi}{q\left(\cdot; {\vlambda^*}_{T}\right)}}
    =
    \mathcal{O}\left(
    \frac{
      G \, \sqrt{\log T}
    }{
      \log \rho^{-1} \, \sqrt{T}
    } \right)  
  \end{align*}
  }
\end{theoremEnd}
\begin{proofEnd}
  \citet[Corollary 3.5]{duchi_ergodic_2012} provide a non-asymptotic convergence rate for the \textit{ergodic mirror descent} algorithm which computes the parameter update as
  \begin{align}
    \vlambda_{t+1} &= \argmin_{\vlambda \in \Lambda} \left\{\,
    \iprod{\,\vg\left(\vlambda, \veta_t\right)}{\vlambda\,} 
    +
    \frac{1}{\alpha_t} D_{\psi}\left(\vlambda, \vlambda_{t}\right)
    \,\right\}\label{eq:ergodic_mirror_descent}
    \\
    \rvveta_{t+1} &\sim P_{\vlambda_{t}}\left(\veta_t, \cdot\right)
    \nonumber
  \end{align}
  where \(D_{\psi}\) is the Bregman divergence defined as
  \begin{align*}
    D_{\psi}\left(\vlambda, \vlambda\prime\right)
    \triangleq
    \psi\left(\vlambda\right)
    - \psi\left(\vlambda\prime\right)
    - \iprod{ \nabla \psi\left(\vlambda\prime\right)}{\vlambda - \vlambda\prime}
  \end{align*}
  for some convex function \(\psi\).
  Our result is based on the fact that MCGD is a special case of the ergodic mirror descent algorithm.
  Specifically, by choosing \(\psi\left(\vlambda\right) = \frac{1}{2} \norm{\vlambda}^2_2 \), we obtain
  \begin{alignat}{2}
    D_{\psi}\left(\vlambda, \vlambda\prime\right) = \frac{1}{2} \norm{\vlambda - \vlambda\prime}_2^2 \leq \frac{1}{2} \, R^2.\label{eq:bregman}
    &&\quad\text{\textit{\cref{thm:compact}}}
  \end{alignat}
  This reduces the update in \cref{eq:ergodic_mirror_descent} into projected gradient descent which is the form used for Markov chain score climbing.

  Under our assumptions, \citet[Corollary 3.5]{duchi_ergodic_2012} show that, by assuming that the Hellinger mixing time is bounded as
  \begin{align}
    \tau_{\text{Hel.}}\left(K_{\vlambda}, \epsilon\right) \leq \kappa_1 \log\left( \kappa_2 /\epsilon \right) \label{eq:hellinger_mixing}
  \end{align}
  for any \(\epsilon > 0\), and setting a decreasing stepsize \(\alpha_t = \alpha / \sqrt{t}\),
  it follows that
  \begin{align}
    &\E{ \DKL{\pi}{q\,(\cdot; {\overline{\vlambda}_{T}})} - \DKL{\pi}{q\left(\cdot; {\vlambda^*}_{T}\right)}}
    \nonumber
    \\
    &\quad\leq
    \frac{R^2}{2 \, \alpha \, \sqrt{T}}
    +
    \frac{2 \, \alpha \, G^2}{\sqrt{T}}\left( \kappa_1 \, \log \frac{\kappa_2}{\epsilon} \right)
    +
    3 \, \epsilon \, G \, R
    +
    \frac{R \, G \, \kappa_1 \, \log \frac{\kappa_2}{\epsilon}}{T}.\label{eq:original_bound}
  \end{align}

  From this, by setting the initial stepsize as \(\alpha = R / \left(G \sqrt{\kappa_1 \log \left(\kappa_2 \, T\right) }\right)\) and \(\epsilon = 1/ \sqrt{T}\),
  \begin{alignat*}{2}
    &\E{ \DKL{q\,(\cdot; {\overline{\vlambda}_{T}})}{\pi} - \DKL{q\left(\cdot; {\vlambda^*}_{T}\right)}{\pi}}
    \\
    &\leq
    \frac{R}{2 \, \alpha \, \sqrt{T}}
    +
    \frac{2 \, \alpha \, G^2}{\sqrt{T}}\left( \kappa_1 \, \log \frac{\kappa_2}{\epsilon} \right)
    +
    3 \, \epsilon \, G \, R
    +
    \frac{R \, G \, \kappa_1 \, \log \frac{\kappa_2}{\epsilon}}{T}.
    &&\quad\text{\textit{\cref{eq:original_bound}}}
    \\
    &=
    \frac{R \, G \, \sqrt{\kappa_1 \, \log \left(\kappa_2 \, T\right) } }{2 \, \sqrt{T}}
    +
    \frac{2 \, R \, G}{\sqrt{T}}
    \frac{ \kappa_1 \, \log \kappa_2 \, \sqrt{T} }{ \sqrt{\kappa_1 \log \left(\kappa_2 \, T\right)}}
    +
    \frac{3 \, G \, R}{\sqrt{T}}
    +
    \frac{R \, G \, \kappa_1 \, \log \left(\kappa_2 \, \sqrt{T}\right)}{T}
    &&\quad\text{\textit{Plugged value of \(\epsilon\) and \(\alpha\)}}
    \\
    &\leq
    \frac{R \, G \, \sqrt{\kappa_1 \, \log \left(\kappa_2 \, T\right) } }{2 \, \sqrt{T}}
    +
    \frac{2 \, R \, G}{\sqrt{T}}
    \frac{ \kappa_1 \, \log \kappa_2 \, T }{ \sqrt{\kappa_1 \log \left(\kappa_2 \, T\right)}}
    +
    \frac{3 \, G \, R}{\sqrt{T}}
    +
    \frac{R \, G \, \kappa_1 \, \log \left(\kappa_2 \, T\right)}{T}
    &&\quad\text{\textit{Applied \(\log \sqrt{T} < \log T\)}}
    \\
    &=
    \frac{R \, G \, \sqrt{\kappa_1 \, \log \left(\kappa_2 \, T\right) } }{2 \, \sqrt{T}}
    +
    \frac{2 \, R \, G}{\sqrt{T}}
    \sqrt{\kappa_1 \, \log \left( \kappa_2 \, T\right) }
    +
    \frac{3 \, G \, R}{\sqrt{T}}
    +
    \frac{R \, G \, \kappa_1 \, \log \left(\kappa_2 \, T\right)}{T}
    &&\quad\text{\textit{Solved fraction}}
    \\
    &=
    \frac{5 \, R \, G \, \sqrt{\kappa_1 \, \log \left(\kappa_2 \, T\right) } }{2 \, \sqrt{T}}
    +
    \frac{3 \, G \, R}{\sqrt{T}}
    +
    \frac{R \, G \, \kappa_1 \, \log \left(\kappa_2 \, T\right)}{T}.
    &&\quad\text{\textit{Combined fractions}}
  \end{alignat*}

  From \cref{thm:mixing_time}, we retrieve the constants of \cref{eq:hellinger_mixing} as
  \(
  \kappa_1=\frac{2}{\log \rho^{-1} },\;  \kappa_2 = 1
  \), which follows our result
  \begin{alignat*}{2}
    &\frac{
      5 \, R \, G \, \sqrt{\kappa_1 \, \log \left(\kappa_2 \, T\right) }
    }{
      2 \, \sqrt{T}
    }
    +
    \frac{3 \, G \, R}{\sqrt{T}}
    +
    \frac{R \, G \, \kappa_1 \, \log \left(\kappa_2 \, T\right)}{T}
    \\
    &\quad=
    \frac{
      5 \, R \, G \, \sqrt{ \frac{2}{\log \rho^{-1}} \, \log T }
    }{
      2 \, \sqrt{T}
    }
    +
    \frac{3 \, G \, R}{\sqrt{T}}
    +
    \frac{R \, G \, \frac{2}{\log \rho^{-1}} \, \log T}{T}
    &&\quad\text{\textit{Plugged values of \(\kappa_1\) and \(\kappa_2\)}}
    \\
    &\quad=
    \frac{
      5 \, \sqrt{2} \, R \, 
    }{
      2
    }
    \,
    \frac{
      G \, \sqrt{\log T}
    }{
      \log \rho^{-1} \, \sqrt{T} \, 
    }
    +
    3 \, R \,
    \frac{G \, R}{\sqrt{T}}
    +
    2 \, R
    \,
    \frac{G \, \log T}{ \log \rho^{-1} \, T}
    &&\quad\text{\textit{Pulled constants forward}}
  \end{alignat*}
\end{proofEnd}

%%% Local Variables:
%%% TeX-master: "master"
%%% End:

%\input{thm_convergence_rate2}

%% This result is a direct adaptation of the ergodic mirror descent algorithm by~\cite{duchi_ergodic_2012}.
%% For accelerated variants of MCGD,~\citet{doan_convergence_2020} provide non-asymptotic convergence results.
%% However, their bound for the convex case is independent of the kernel mixing rate, which leaves out the practical effects of the mixing rate.

\subsection{Performance Analysis of Markov Chain Score Ascent Methods}\label{section:comparison}
We will now analyze the previously proposed MCSA algorithms: MSC and JSA.
Both methods naturally satisfy the MCSA framework defined in~\cref{thm:product_kernel}.
Furthermore, we establish the
\begin{enumerate*}[label=\textbf{(\roman*)}]
  \item geometric convergence rate of the kernel \(P\) formed by each method and
  \item the upper bound on the gradient variance.
\end{enumerate*}
To do this, we use the following assumptions.
\begin{assumption}{(Bounded importance weight)}\label{thm:bounded_weight}
  The importance weight ratio \(w\left(\vz\right) = \pi\left(\vz\right) / q\left(\vz; \vlambda\right)\) is bounded by some finite constant as \(w^* < \infty\) for all \(\vlambda \in \Lambda\) such that \(\rho = \left(1 - 1/w^*\right) < 1\).
\end{assumption}
The fact that \(w^*\) exists for all \(\vlambda \in \Lambda\) is restrictive and is not entirely relevant in practice.
However, this assumption is necessary for analyzing MCGD through~\cref{thm:kernel_conditions}.
Although previous works did not take specific measures to enable \cref{thm:bounded_weight}, it can be done by using a variational family with heavy tails~\citep{NEURIPS2018_25db67c5} or using a defensive mixture~\citep{hesterberg_weighted_1995, holden_adaptive_2009} 
\begin{align*}
  q_{\text{def.}}\left(\vz; \vlambda \right) = w \, q\left(\vz; \vlambda\right) + (1 - w) \, \nu\left(\vz\right)
\end{align*}
where \(0 < w < 1\) and \(\nu\left(\cdot\right)\) is a heavy tailed distribution that satisfies \(\sup_{\vz \in \mathcal{Z}} \pi\left(\vz\right) / \nu\left(\vz\right) < \infty\).
A typical (and practical) example is the Cauchy distribution.
Note that we only use \(q_{\text{def.}}\) within the Markov chain kernels.
Therefore \(q_{\text{def.}}\) does not restrict our choice of the variational family.

%
%% \begin{assumption}{(Location scale family)}\label{thm:location_scale}
%%   The variational family is the location-scale family with the location (\(\vm\)) and scale (\(\mC\)) parameters denoted as \(\vlambda = (\vm_{\vlambda}, \mC_{\vlambda})\) with the base density \(\phi\) such that the probability density and samples are given by
%%   \begin{align*}
%%     &\text{\textit{(sampling)}}\quad \rvvz = \mC_{\vlambda} \, \rvvu + \vm_{\vlambda}, \quad \rvvu \sim \phi\left(\cdot\right), 
%%     &\text{\textit{(density)}}\quad q\left(\vz; \vlambda\right) = \phi\left( \mC^{-1}_{\vlambda}\left( \vz - \vm_{\vlambda} \right) \right)
%%   \end{align*}
%% \end{assumption}
%% Furthermore, the squared \(L^2\)-norm of variational parameters are given as
%% \begin{align}
%%   \norm{\vlambda}_2^2 = \norm{\vm_{\vlambda}}_2^2 + \norm{\mC_{\vlambda}}_F^2 \label{eq:parameter_norm}
%% \end{align}
%% for \(\vlambda = (\vm_{\vlambda}, \mC_{\vlambda})\).

%% \begin{assumption}{(Lipschitz continuity of base distribution)}\label{thm:lipschitz}
%%   the base density is log-Lipschitz continuous as
%%   \(
%%     \abs{ \log \phi\left(\vu\right) - \log \phi\left(\vu\prime\right) } \leq L \, \norm{\vu - \vu\prime}
%%   \)
%%   for some finite constant \(L > 0\).
%% \end{assumption}
%% This is equivalent to assuming \(\norm{\nabla \log \phi\left(\vu\right) } < L\)

%% \begin{assumption}{Contrained scale}\label{thm:solution_space}
%%   The parameter space \(\Lambda_M\) is constrained such that the singular values of the scale matrix are lower bounded such that
%%   {\small
%%   \[
%%   \Lambda_M = \left\{\, (\vm, \mC) \,\middle|\, \sigma_{\text{min}}(\mC) \geq \frac{1}{\sqrt{M}} \,\right\}
%%   \]
%%   }
%%   for some finite constant \(M > 0\).
%% \end{assumption}

%% \begin{assumption}{Uniformly Lipschitz continuous score}\label{thm:solution_space}
%%   The score gradient is bivariate uniformly Lipschitz continuous as
%%   {\small
%%   \[
%%   \norm{ \nabla \log q\left(\vz; \vlambda\right) }_2 \leq L \left( \norm{\vlambda}_2 + \norm{\vz}_2 + 1 \right)
%%   \]
%%   }
%%   for some finite constant \(L > 0\).
%% \end{assumption}

%% \begin{assumption}{(Bounded variance)}\label{thm:bounded_variance}
%%   The samples \(\veta_t\) generated by the Markov chain kernel as \(\rvveta \sim P_{\vlambda}\left(\veta, \cdot\right)\) have a finite second moment for all \(t\) such as
%%   \(
%%    \mathbb{E}[\, {\| \veta_t \|}^2 \,|\, \mathcal{F}_{t-1} \,] < V,
%%   \)
%%   for some finite constant \(V > 0\).
%% \end{assumption}

\begin{assumption}{(Bounded Score)}\label{thm:bounded_score}
  The score gradient is bounded for \(\forall \vlambda \in \Lambda\) and \(\forall \vz \in \mathcal{Z}\) such that \(\norm{\vs\left(\vlambda; \vz\right)}_* \leq L \) for some finite constant \(L > 0\).
\end{assumption}
\vspace{-0.05in}
Informally, this assumption is equivalent to assuming 
%% \begin{enumerate*}[label=\textbf{(\roman*)}]
%% \item \(\Lambda\) is compact with a finite radius,
%% \item the Markov chains are variance bounding~\citep[Theorem 1]{10.2307/25442663},
%% \item and that the score gradient is continuous with respect to \(\vlambda\) and \(\vz\).
%% \end{enumerate*}
Admittedly, this assumption is strong, but it provides a straightforward way to compare the gradient variance of different MCSA implementations.

\vspace{-0.05in}
\paragraph{Markovian Score Climbing}
MSC is a simple instance of MCSA where \(\eta_t = \vz_t\) and \(P_{\vlambda_t} = K_{\vlambda_t}\) where \(K_{\vlambda_t}\) is the conditional importance sampling (CIS) kernel (originally proposed as the iterated sequential importance resampling kernel by~\citet{andrieu_uniform_2018}).
Although MSC uses only a single sample such that \(N=1\), the CIS kernel internally uses \(N\) proposals to generate a single sample.
Therefore, \(N\) in MSC has a different meaning, but it still indicates the computational budget.
See \cref{alg:msc} for a detailed pseudocode.


\begin{theoremEnd}[all end]{lemma}\label{thm:product_measure_bound}
  For the probability measures \(p_1, \ldots, p_N\) and \(q_1, \ldots, q_N\) defined on a measurable space \((\mathsf{X}, \mathcal{A})\) and an arbitrary set \(A \in \mathcal{A}\),
  \begin{align*}
    &\abs{
    \int_{A^N}
    p_1\left(dx_1\right)
    p_2\left(dx_2\right)
    \times
    \ldots
    \times
    p_N\left(dx_N\right)
    -
    q_1\left(dx_1\right)
    q_2\left(dx_2\right)
    \times
    \ldots
    \times
    q_N\left(dx_N\right)
  }
    \\
  &\qquad\leq
  \sum_{n=1}^N
  \abs{
    \int_{A}
    p_n\left(dx_n\right)
    -
    q_n\left(dx_n\right)
  }
  \end{align*}
\end{theoremEnd}
\begin{proofEnd}
  By using the following shorthand notations
  \begin{alignat*}{2}
    p_{(1:N)}\left(dx_{(1:N)}\right)
    &= 
    p_1\left(dx_1\right)
    p_2\left(dx_2\right)
    \times
    \ldots
    \times
    p_N\left(dx_N\right)
    \\
    q_{(1:N)}\left(dx_{(1:N)}\right)
    &= 
    q_1\left(dx_1\right)
    q_2\left(dx_2\right)
    \times
    \ldots
    \times
    q_N\left(dx_N\right),
  \end{alignat*}
  the result follows from induction as
  \begin{alignat}{2}
    &\abs{
      \int_{A^N}
      p_{(1:N)}\left(dx_{(1:N)}\right)
      -
      q_{(1:N)}\left(dx_{(1:N)}\right)
    }
    \nonumber
    \\
    &\quad=
    \Bigg|\;
    \left( \int_{A} p_1\left(dx_1\right) - q_1\left(dx_1\right) \right) \,
    \int_{A^{N-1}} p_{(2:N)}\left(dx_{(2:N)}\right)
    \nonumber
    \\
    &\qquad\quad+
    \int_{A} q_1\left(dx_1\right) \,
    {\left(
      \int_{A^{N-1}}
      p_{(2:N)}\left(dx_{(2:N)}\right)
      -
      q_{(2:N)}\left(dx_{(2:N)}\right)
    \right)}
    \;\Bigg|
    \nonumber
    \\
    &\quad\leq
    \Bigg|
    \int_{A} p_1\left(dx_1\right) - q_1\left(dx_1\right)
    \Bigg|\;
    \int_{A^{N-1}} p_{(2:N)}\left(dx_{(2:N)}\right)
    \nonumber
    \\
    &\qquad\quad+
    \int_{A} q_1\left(dx_1\right) \,
    {
    \Bigg|\;
      \int_{A^{N-1}}
      p_{(2:N)}\left(dx_{(2:N)}\right)
      -
      q_{(2:N)}\left(dx_{(2:N)}\right)
    }
    \;\Bigg|
    &&\quad\text{\textit{Triangle inequality}}
    \nonumber
    \\
    &\quad\leq
    \Bigg|
    \int_{A} p_1\left(dx_1\right) - q_1\left(dx_1\right)
    \Bigg|\;
    \nonumber
    \\
    &\qquad\quad+
    {
    \Bigg|\;
      \int_{A^{N-1}}
      p_{(2:N)}\left(dx_{(2:N)}\right)
      -
      q_{(2:N)}\left(dx_{(2:N)}\right)
    }
    \;\Bigg|.
    &&\quad\text{\textit{Applied \(p_n\left(A\right), q_n\left(A\right) \leq 1 \)}}
    \nonumber
  \end{alignat}
\end{proofEnd}


\begin{theoremEnd}{theorem}\label{thm:msc}
  MSC~\citep{NEURIPS2020_b2070693} is obtained by defining 
  {%\small
  \begin{align*}
  P_{\lambda}^k\left(\veta, d\veta^{\prime}\right)
  = 
  K_{\lambda}^k\left(\vz, d\vz^{\prime}\right)
  \end{align*}
  }
  with  \(\veta_t = \vz_t\) where \(K_{\vlambda}\left(\vz, \cdot\right)\) is the CIS kernel with \(q_{\text{def.}}\left(\cdot; \vlambda\right)\) as its proposal distribution.
  Then, given~\cref{thm:bounded_weight,thm:bounded_score}, the mixing rate and the gradient bounds are given as
  {%\small
  \begin{align*}
    \textstyle
  \DTV{P_{\vlambda}^k\left(\veta, \cdot\right)}{\Pi} \leq  {\left(1 - \frac{N - 1}{2 w^* + N - 2}\right)}^k\quad \text{and}\quad
  {\small
  \E{ \norm{ \vg\left(\vlambda, \rvveta\right) }_{*}^2 \,\middle|\, \mathcal{F}_{t} } \leq  L^2,
  }
  \end{align*}
  }
  where \(w^* = \sup_{\vz} \pi\left(\vz\right) / q_{\text{def.}}\left(\vz;\vlambda\right)\).
\end{theoremEnd}
\begin{proofEnd}
  MSC is described in~\cref{alg:msc}. 
  At each iteration, it performs a single MCMC transition with the CIS kernel where it internally uses \(N\) proposals.

  \paragraph{Ergodicity of the Markov Chain}
  The ergodic convergence rate of \(P_{\vlambda}\) is equal to that of \(K_{\vlambda}\), the CIS kernel proposed by~\citet{NEURIPS2020_b2070693}. 
  Although not mentioned by~\citet{NEURIPS2020_b2070693}, this kernel has been previously proposed as the iterated sequential importance resampling (i-SIR) by \citet{andrieu_uniform_2018} with its corresponding geometric convergence rate as
  \begin{alignat*}{2}
    \DTV{P^{k}_{\vlambda}\left(\veta, \cdot\right)}{\Pi}
    =
    \DTV{K^{k}_{\vlambda}\left(\vz, \cdot\right)}{\pi}
    \leq
    {\left(1 - \frac{N - 1}{2 w^* + N - 2}\right)}^k.
  \end{alignat*}

  \paragraph{\textbf{Bound on the Gradient Variance}}
  The bound on the gradient variance is straightforward given \cref{thm:bounded_score}.
  For simplicity, we denote the rejection state as \(\vz^{(1)} = \vz_{t-1} \).
  Then,
  \begin{alignat}{2}
    &\E{ \norm{ \vg\left(\vlambda, \rvveta\right) }_{*}^2 \,\middle|\, \mathcal{F}_{t} }
    \nonumber
    \\
    &\;=
    \E{ \norm{ \vg\left(\vlambda, \rvveta\right) }_{*}^2 \,\middle|\, \mathcal{F}_{t}}
    \nonumber
    \\
    &\;=
    \Esub{\rvvz \sim K_{\vlambda_{t-1}}\left(\vz_{t-1}, \cdot\right)}{
      \norm{ \vs\left(\vlambda; \rvvz\right) }_{*}^2 \,\middle|\,
      \vlambda_{t-1}, \vz_{t-1}
    }
    \nonumber
    \\
    &\;=
    \int
    \sum^{N}_{n=1}
    \frac{
      w\left(\vz^{(n)}\right)
    }{
      \sum^{N}_{m=1} w\left(\vz^{(m)}\right)
    }
    \norm{ \vs\left(\cdot; \vz^{(n)}\right) }^2_{*}
    \prod^{N}_{n=2}
    q\left(d\vz^{(n)}; \vlambda_{t-1}\right)
    \nonumber
    &&\quad\text{\textit{\citet{andrieu_uniform_2018}}}
    \\
    &\;\leq
    L^2 \,
    \int
    \sum^{N}_{n=1}
    \frac{
      w\left(\vz^{(n)}\right)
    }{
      \sum^{N}_{m=1} w\left(\vz^{(m)}\right)
    }
    \prod^{N}_{n=2}
    q\left(d\vz^{(n)}; \vlambda_{t-1}\right)
    \nonumber
    &&\quad\text{\textit{\cref{thm:bounded_score}}}
    \\
    &\;=
    L^2 \,
    \int
    \prod^{N}_{n=2}
    q\left(d\vz^{(n)}; \vlambda_{t-1}\right)
    \nonumber
   &&\quad\text{\textit{The sum of weights is 1}}
    \\
    &\;=
    L^2.
    \nonumber
  \end{alignat}

  %% This form coincides with solving the expectation of the self-normalized importance sampling estimator, which is well known to be challenging~\citep{robert_monte_2004}.
  %% We instead approximate the expectation by defining the random variables
  %% \(
  %% X = \sum^{N}_{n=1} w\left(\vz_n\right) \norm{ s\left(\cdot; \vz_n\right) }^2_{*}
  %% \)
  %% and
  %% \(
  %% Y = \sum^{N}_{m=1} w\left(\vz_m\right)
  %% \),
  %% use the delta method as
  %% \begin{alignat*}{2}
  %%   \E{\frac{\rvX}{\rvY} \;\middle|\; Z} &\approx \frac{\E{\rvX \mid Z }}{\E{\rvY \mid Z}} + \mathcal{O}\left(\frac{1}{{\E{\rvX \mid Z}}^2}\right).
  %% \end{alignat*}
  %% The required expectations are obtained as
  %% \begin{alignat}{2}
  %%   &\E{\rvX \mid \lambda_{t-1}, \vz_{t-1}}
  %%   \\
  %%   &\;=
  %%   \int \sum^{N}_{n=1} w\left(\vz_n\right) \norm{ s\left(\cdot; \vz_n\right) }^2_{*}
  %%   \prod^{N}_{n=2}
  %%   q\left(d\rvvz_n\right)
  %%   \nonumber
  %%   \\
  %%   &\quad=
  %%   \sum^{N}_{n=1} \int w\left(\vz_n\right) \norm{ s\left(\cdot; \vz_n\right) }^2_{*}
  %%   \prod^{N}_{n=2}
  %%   q\left(d\rvvz_n\right)
  %%   &&\quad\text{\textit{Swapped integral and sum}}
  %%   \nonumber
  %%   \\
  %%   &\quad=
  %%   \left\{\;
  %%   \sum^{N}_{n=2} \int w\left(\vz_n\right) \norm{ s\left(\cdot; \vz_n\right) }^2_{*}
  %%   \prod^{N}_{n=2} q\left(d\rvvz_n\right)
  %%   \right\}
  %%   +
  %%   w\left(\vz_1\right) \norm{ s\left(\cdot; \vz_1\right) }^2_{*}
  %%   &&\quad\text{\textit{Pulled out rejection state}}
  %%   \nonumber
  %%   \\
  %%   &\quad=
  %%   \left\{\;
  %%   \sum^{N}_{n=2} \int \frac{\pi\left(\vz_n\right)}{q\left(\vz_n\right)} \norm{ s\left(\cdot; \vz_n\right) }^2_{*}
  %%   \prod^{N}_{n=2}
  %%   q\left(d\rvvz_n\right)
  %%   \right\}
  %%   +
  %%   w\left(\vz_1\right) \norm{ s\left(\cdot; \vz_1\right) }^2_{*}
  %%   &&\quad\text{\textit{Definition of \(w\left(\vz\right)\)}}
  %%   \nonumber
  %%   \\
  %%   &\quad=
  %%   \sum^{N}_{n=2} \int \pi\left(d\vz_n\right) \norm{ s\left(\cdot; \vz_n\right) }^2_{*}
  %%   +
  %%   w\left(\vz_1\right) \norm{ s\left(\cdot; \vz_1\right) }^2_{*}
  %%   &&\quad\text{\textit{Cancelled out \(q\left(\cdot\right)\)}}
  %%   \nonumber
  %%   \\
  %%   &\quad=
  %%   \left(N-1\right) \, \Esub{\pi}{\norm{ s\left(\cdot; \rvvz\right) }^2_{*} }
  %%   +
  %%   w\left(\vz_1\right) \norm{ s\left(\cdot; \vz_1\right) }^2_{*}
  %%   \nonumber
  %%   \\
  %%   &\quad=
  %%   \left(N-1\right) \, \Esub{\pi}{\norm{ s\left(\cdot; \rvvz\right) }^2_{*} }
  %%   +
  %%   w\left(\vz_{t-1}\right) \norm{ s\left(\cdot; \vz_{t-1}\right) }^2_{*}
  %%   \nonumber
  %% \end{alignat}
  %% and similarly,
  %% \begin{alignat}{2}
  %%   \E{\rvY \mid \lambda_{t-1}, \vz_{t-1}}
  %%   &\quad= 
  %%   \int \sum^{N}_{n=1} w\left(\vz_n\right) \prod^{N}_{n=2} q\left(d\vz_n\right) 
  %%   \nonumber
  %%   \\
  %%   &\quad= 
  %%   \left\{\; \int \sum^{N}_{n=2} w\left(\vz_n\right) \prod^{N}_{n=2} q\left(d\vz_n\right) \;\right\} + w\left(\vz_1\right)
  %%   &&\quad\text{\textit{Pulled out rejection state}}
  %%   \nonumber
  %%   \\
  %%   &\quad= 
  %%   \left\{\; \sum^{N}_{n=2} \int w\left(\vz_n\right) \prod^{N}_{n=2} q\left(d\vz_n\right) \;\right\} + w\left(\vz_1\right)
  %%   &&\quad\text{\textit{Swapped integral and sum}}
  %%   \nonumber
  %%   \\
  %%   &\quad= 
  %%   \left\{\; \sum^{N}_{n=2} \int \frac{\pi\left(\vz_n\right)}{q\left(\vz_n\right)} \prod^{N}_{n=2} q\left(d\vz_n\right) \;\right\} + w\left(\vz_1\right)
  %%   &&\quad\text{\textit{Definition of \(w\left(\vz\right)\)}}
  %%   \nonumber
  %%   \\
  %%   &\quad= 
  %%   \left\{\; \sum^{N}_{n=2} \int \pi\left(d\vz_n\right) \;\right\} + w\left(\vz_1\right)
  %%   &&\quad\text{\textit{Cancelled out \(q\left(\cdot\right)\)}}
  %%   \nonumber
  %%   \\
  %%   &\quad= 
  %%   N - 1 + w\left(\vz_{t-1}\right)
  %%   \nonumber
  %% \end{alignat}
  %% Therefore, 
  %% \begin{alignat}{2}
  %%   \E{\frac{\rvX}{\rvY} \;\middle|\; Z}
  %%   &\approx
  %%   \frac{
  %%     \left(N-1\right) \, \Esub{\pi}{\norm{ s\left(\cdot; \rvvz\right) }^2_{*} }
  %%     +
  %%     w\left(\vz_1\right) \norm{ s\left(\cdot; \vz_1\right) }^2_{*}
  %%   }{
  %%     N-1 + w\left(\vz_1\right)
  %%   }
  %%   + \mathcal{O}\left(\frac{1}{{\left(N-1\right)}^2}\right)
  %%   \nonumber
  %%   \\
  %%   &\leq
  %%   \frac{
  %%     \left(N-1\right) \, \Esub{\pi}{\norm{ s\left(\cdot; \rvvz\right) }^2_{*} }
  %%     +
  %%     w^* \norm{ s\left(\cdot; \vz_1\right) }^2_{*}
  %%   }{
  %%     N-1
  %%   }
  %%   + \mathcal{O}\left(\frac{1}{{\left(N-1\right)}^2}\right)
  %%   &&\quad\text{\textit{\(w\left(\cdot\right) \leq w^*\)}}
  %%   \nonumber
  %%   \\
  %%   &\leq
  %%   \frac{
  %%     \left(N-1\right) \, L^2
  %%     +
  %%     w^* L^2
  %%   }{
  %%     N-1
  %%   }
  %%   + \mathcal{O}\left(\frac{1}{{\left(N-1\right)}^2}\right)
  %%   \nonumber
  %%   \\
  %%   &=
  %%   L^2
  %%   \left[
  %%   1
  %%   +
  %%   \frac{
  %%     w^*
  %%   }{
  %%     N-1
  %%   }
  %%   \right]
  %%   + \mathcal{O}\left(\frac{1}{{\left(N-1\right)}^2}\right)
  %%   \nonumber
  %% \end{alignat}
\end{proofEnd}

%%% Local Variables:
%%% TeX-master: "master"
%%% End:


\vspace{-0.05in}
\paragraph{Joint Stochastic Approximation}
JSA~\citep{pmlr-v124-ou20a} was originally proposed for models where the likelihood is factorizable for each datapoint.
In this setting, they perform minibatching by using a random-scan verion of the independent Metropolis-Hastings (IMH, \citealt{hastings_monte_1970,robert_monte_2004}) kernel.
We generalize JSA to non-factorizable likelihoods in this work by using the vanilla IMH kernel.
At each MCGD step, JSA performs multiple MCMC transitions and estimates the gradient by averaging all the intermediate samples.
See \cref{alg:jsa} for a detailed pseudocode.

\vspace{-0.05in}
\paragraph{Independent Metropolis-Hastings}
A key element of JSA~\citep{pmlr-v124-ou20a} is that it uses the IMH kernel.
Similarly to MSC, the IMH kernel uses the variational approximation \(q\left(\cdot; \vlambda_t\right)\) to generate the proposals.
To show the geometric ergodicity of the joint kernel \(P\), we utilize the geometric convergence rate of IMH kernels provided by~\citet[Theorem 2.1]{10.2307/2242610} and~\citet{wang_exact_2020}.
Furthermore, to derive an upper bound on the gradient variance, we use the exact \(n\)-step marginal IMH kernel derived by~\citet{Smith96exacttransition} as
{%\small
  \begin{align}
  K^n_{\vlambda}\left(\vz, d\vz\prime\right) 
  = T_n\left(\, w\left(\vz\right) \vee w\left(\vz\prime\right)\,\right) \, \pi\left(\vz\prime\right) \, d\vz\prime
  + \lambda^n\left(w\left(\vz\right)\right) \, \delta_{\vz}\left(d\vz\prime\right)
  \label{eq:imh_exact_kernel}
  \end{align}
}%
where {\(w\left(\vz\right) = \pi\left(\vz\right)/q_{\text{def.}}\left(\vz; \vlambda\right)\), \(x \vee y = \max\left(x, y\right)\)},
{\small
  \begin{align}
    T_n\left(w\right)      = \int_w^{\infty}
    \frac{n}{v^2}
    %\left(n / v^2\right)
    \, \lambda^{n-1}\left(v\right)\,dv,
    \quad\text{and}\quad
    \lambda\left(w\right) =
    \int_{R\left(w\right)}
    \left( 1 - \frac{w\left(\vz\prime\right)}{w}  \right)
    %\left( 1 - w\left(\vz\prime\right)/w  \right)
    \pi\left(d\vz\prime\right)\label{eq:T_lambda}
  \end{align}
}
for {\(R\left(w\right) = \{\, \vz\prime \mid w\,\left(\vz\prime\right) \leq w \,\}\)}.
%

\begin{theoremEnd}[all end]{lemma}\label{thm:lambda_bound}
  For \(w^* = \sup_{\vz} w\left(\vz\right) \), \(\lambda\left(\cdot\right)\) in~\cref{eq:T_lambda} is bounded as
  \[
   \max\left(1 - \frac{1}{w}, 0\right) \leq \lambda\left(w\right) \leq 1 - \frac{1}{w^*}.
  \]
\end{theoremEnd}
\begin{proofEnd}
  The proof can be found in the proof of Theorem 3 of \citet{Smith96exacttransition}.
\end{proofEnd}

\begin{theoremEnd}[all end]{lemma}\label{thm:tn_bound}
  For \(w^* = \sup_{\vz} w\left(\vz\right) \), \(T_n\left(\cdot\right)\) in~\cref{eq:T_lambda} is bounded as
  \[
  T_n\left( w \right) \leq \frac{n}{w} \, {\left(1 - \frac{1}{w^*}\right)}^{n-1}.
  \]
\end{theoremEnd}
\begin{proofEnd}
  \begin{alignat*}{2}
    T_n\left(w\right) 
    &= \int_w^{\infty} \frac{n}{v^2} \, \lambda^{n-1}\left(v\right)\,dv
    &&\quad\text{\textit{\cref{eq:T_lambda}}}
    \\
    &\leq \int_w^{\infty} \frac{n}{v^2} \, {\left(1 - \frac{1}{w^*}\right)}^{n-1}\,dv
    &&\quad\text{\textit{\cref{thm:lambda_bound}}}
    \\
    &= n \, {\left(1 - \frac{1}{w^*}\right)}^{n-1}  \int_w^{\infty} \frac{1}{v^2} \,dv
    &&\quad\text{\textit{Pulled out constant}}
    \\
    &= n \, {\left(1 - \frac{1}{w^*}\right)}^{n-1}  \left( {-\left.\frac{1}{v}\right\rvert^{\infty}_{w}} \right)
    &&\quad\text{\textit{Solved indefinite integral}}
    \\
    &= \frac{n}{w} \, {\left(1 - \frac{1}{w^*}\right)}^{n-1}.
  \end{alignat*}
  This upper bound is in general difficult to improve unless we impose stronger assumptions on \(\pi\) and \(q\).
\end{proofEnd}

\begin{theoremEnd}[all end]{lemma}\label{thm:imh_expecation}
  For a positive test function \(f : \mathcal{Z} \rightarrow \mathbb{R}^{+}\), the estimate of a \(\pi\)-invariant independent Metropolis-Hastings kernel with a proposal \(q\) is bounded as
  \begin{align*}
    \Esub{K^n\left(\vz, \cdot\right)}{ f \,\middle|\, \rvvz }
    \leq
    n \, \rho^{n-1} 
    \Esub{q}{f}
    +
    {\rho}^n \, f\left(\rvvz\right)
    %% \leq
    %% n \, \left(
    %% \Esub{q}{f}
    %% +
    %% \frac{1}{n} \, f\left(\vz\right)
    %% \right) 
  \end{align*}
  where \(w\left(\vz\right) = \pi\left(\vz\right) / q\left(\vz\right)\) and \(\rho = 1 - \nicefrac{1}{w^*}\) for \(w^* = \sup_{\vz} w\left(\vz\right) \).
\end{theoremEnd}
\begin{proofEnd}
  \begin{alignat*}{2}
    &\Esub{K^n\left(\vz, \cdot\right)}{ f \,\middle|\, \vz }
    \\
    &\quad=
    \int T_n\left(w\left(\vz\right) \vee w\left(\vz^{\prime}\right)\right) \, f\left(\vz^{\prime}\right) \, \pi\left(\vz^{\prime}\right) d\vz^{\prime}
    +
    \lambda^{n}\left(w\left(\rvvz\right)\right) \, f\left(\rvvz\right)
    &&\quad{\text{\textit{\cref{eq:imh_exact_kernel}}}}
    \\
    &\quad\leq
    \int \frac{n}{w\left(\vz\right) \vee w\left(\vz^{\prime}\right)} \, {\left(1 - \frac{1}{w^*}\right)}^{n-1} \, f\left(\vz^{\prime}\right) \, \pi\left(\vz^{\prime}\right) d\vz^{\prime}
    +
    \lambda^{n}\left(w\left(\rvvz\right)\right) \, f\left(\rvvz\right)
    &&\quad{\text{\textit{\cref{thm:tn_bound}}}}
    \\
    &\quad\leq
    \int \frac{n}{w\left(\vz^{\prime}\right)} \, {\left(1 - \frac{1}{w^*}\right)}^{n-1} \, f\left(\vz^{\prime}\right) \, \pi\left(\vz^{\prime}\right) d\vz^{\prime}
    +
    \lambda^{n}\left(w\left(\rvvz\right)\right) \, f\left(\rvvz\right)
    &&\quad{\frac{1}{w\left(\vz\right) \vee w\left(\vz^{\prime}\right)} \leq \frac{1}{w\left(\vz^{\prime}\right)}}
    \\
    &\quad=
    n \, {\left(1 - \frac{1}{w^*}\right)}^{n-1} \, 
    \int \frac{1}{w\left(\vz^{\prime}\right)} \, f\left(\vz^{\prime}\right) \, \pi\left(\vz^{\prime}\right) d\vz^{\prime}
    +
    \lambda^{n}\left(w\left(\rvvz\right)\right) \, f\left(\rvvz\right)
    &&\quad{\text{\textit{Pulled out constant}}}
    \\
    &\quad=
    n \, {\left(1 - \frac{1}{w^*}\right)}^{n-1} \, 
    \int f\left(\vz^{\prime}\right) \, q\left(\vz^{\prime}\right) d\vz^{\prime}
    +
    \lambda^{n}\left(w\left(\rvvz\right)\right) \, f\left(\rvvz\right)
    &&\quad{\text{\textit{Definition of \(w\left(\vz\right)\)}}}
    \\
    &\quad\leq
    n \, {\left(1 - \frac{1}{w^*}\right)}^{n-1} \, 
    \int f\left(\vz^{\prime}\right) \, q\left(\vz^{\prime}\right) d\vz^{\prime}
    +
    {\left(1 - \frac{1}{w^*}\right)}^{n} \, f\left(\rvvz\right)
    &&\quad{\text{\textit{\cref{thm:lambda_bound}}}}
    \\
    &\quad=
    n \, {\left(1 - \frac{1}{w^*}\right)}^{n-1} 
    \Esub{q}{f}
    +
    {\left(1 - \frac{1}{w^*}\right)}^n \, f\left(\rvvz\right).
  \end{alignat*}
\end{proofEnd}


%%% Local Variables:
%%% TeX-master: "master"
%%% End:


\begin{theoremEnd}{theorem}\label{thm:jsa}
  JSA~\citep{pmlr-v124-ou20a} is obtained by defining 
  {\small
  \begin{align*}
  P_{\vlambda}^n\left(\veta, d\veta\prime\right)
  = 
  K_{\vlambda}^{N\,\left(n-1\right) + 1}\left(\vz^{(1)}, d\vz\prime^{(1)}\right)
  \,
  K_{\vlambda}^{N\,\left(n-1\right) + 2}\left(\vz^{(2)}, d\vz\prime^{(2)}\right)
  \cdot
  \ldots 
  \cdot
  K_{\vlambda}^{N\,\left(n-1\right) + N}\left(\vz^{(N)}, d\vz\prime^{(N)}\right)
  \end{align*}
  }
  with \(\veta_t = \big[\vz_t^{(1)}, \vz_t^{(2)}, \ldots, \vz_t^{(N)}\big]\).
  Then, given~\cref{thm:bounded_weight,thm:bounded_score}, the mixing rate and the gradient variance bounds are
  {\small
  \begin{align*}
    \DTV{P_{\vlambda}^n\left(\veta, \cdot\right)}{\Pi}
    \leq
    C\left(\rho, N\right)\,{\rho}^{n\,N}
    \quad\text{and}\quad
   % 
    \E{ \norm{ \vg\left(\vlambda, \rvveta\right) }^2_{*} \,\middle|\, \mathcal{F}_{t} }
    \leq
    L^2 \,
    \left[\,
    \frac{1}{2} + \frac{3}{2}\,\frac{1}{N}
    + \mathcal{O}\left(\nicefrac{1}{w^*}\right)
    \,\right],
  \end{align*}
  }
  where \(w^* = \sup_{\vz} \pi\left(\vz\right) / q_{\text{def.}}\left(\vz;\vlambda\right)\) and \(C\) is a finite positive constant depending on both \(\rho\) and \(N\).
\end{theoremEnd}
\begin{proofEnd}

  JSA is described in~\cref{alg:jsa}. 
  At each iteration, it performs \(N\) MCMC transitions, and uses the \(N\) samples to estimate the gradient.

  \paragraph{Ergodicity of the Markov Chain}
  The state transition of the Markov chain samples \(\vz^{(1:N)}\) can be visualized as 
  {\small
  \begin{center}
  \bgroup
  \setlength{\tabcolsep}{3pt}
  \def\arraystretch{1.8}
  \begin{tabular}{c|ccccc}
   & \(\vz^{(1)}_t\) & \(\vz^{(2)}_t\) & \(\vz^{(3)}_t\) & \(\ldots\) &  \(\vz^{(N)}_t\) \\ \midrule
   \(t=1\) & \(K_{\vlambda_1}\left(\vz_0, d\vz_1^{(1)}\right)\) & \(K_{\vlambda_1}^2\left(\vz_0, d\vz_1^{(2)}\right)\) & \(K_{\vlambda_1}^3\left(\vz_0, d\vz_1^{(3)}\right)\) & \(\ldots\) & \(K_{\vlambda_1}^N\left(\vz_0, d\vz_1^{(N)}\right)\) \\
   \(t=2\) & \(K_{\vlambda_2}^{N + 1}\left(\vz_0, d\vz_2^{(1)}\right)\) & \(K_{\vlambda_2}^{N + 2}\left(\vz_0, d\vz_2^{(2)}\right)\) & \(K_{\vlambda_2}^{N + 3}\left(\vz_0, d\vz_2^{(3)}\right)\) & \(\ldots\) & \(K_{\vlambda_2}^{2\,N}\left(\vz_0, d\vz_2^{(N)}\right)\) \\
   \(\vdots\) & & & \(\vdots\) & & \\
   \(t=k\) & \(K_{\vlambda_k}^{\left(k-1\right)\,N + 1}\left(\vz_0, d\vz_k^{(1)}\right)\) & \(K_{\vlambda_k}^{\left(k-1\right)\,N + 2}\left(\vz_0, d\vz_k^{(2)}\right)\) & \(K_{\vlambda_k}^{\left(k-1\right)\,N + 3}\left(\vz_0, d\vz_k^{(3)}\right)\) & \(\ldots\) & \(K_{\vlambda_k}^{\left(k-1\right)\,N + N}\left(\vz_0, d\vz_k^{(N)}\right)\) \\
  \end{tabular}
  \egroup
  \end{center}
  }
  where \(K_{\vlambda}\left(\vz, \cdot\right)\) is an IMH kernel.
  Therefore, the \(n\)-step transition kernel for the vector of the Markov-chain samples \(\veta = \vz^{(1:N)}\) can be represented as
  \begin{align*}
  P_{\vlambda}^n\left(\veta, d\veta\prime\right)
  = 
  K_{\vlambda}^{N\,\left(n-1\right) + 1}\left(\vz_1, d\vz\prime_1\right)
  \,
  K_{\vlambda}^{N\,\left(n-1\right) + 2}\left(\vz_2, d\vz\prime_2\right)
  \cdot
  \ldots 
  \cdot
  K_{\vlambda}^{N\,\left(n-1\right) + N}\left(\vz_N, d\vz\prime_N\right).
  \end{align*}

%%   For example, let us assume that we use a batch size of 1 and that \(m=1\) is selected at \(t=1\) and \(m=2\) is selected at \(t=2\).
%%   Then, the transitions can be visualized as
%%   {\small
%%   \begin{center}
%%   \bgroup
%%   \def\arraystretch{1.8}
%%   \setlength{\tabcolsep}{5pt}
%%   \begin{tabular}{c|cccccccccccccc}
%%       t = 1 & \(\vz^{(1,1)}_1\) & \(\vz^{(1,2)}_1\) & \ldots & \(\vz^{(1,N)}_1\) & \ldots & \(\vz^{(2,1)}_1\) & \(\vz^{(2,2)}_1\) & \ldots & \(\vz^{(2,N)}_1\) & \ldots & \(\vz^{(1,N)}_1\) & \(\vz^{(2,N)}_1\) & \ldots & \(\vz^{(M,N)}_1\) \\
%%       & \(\downarrow\;K_{1,\lambda}\) & \(\downarrow\;K_{1,\lambda}^2\) & \ldots & \(\downarrow\;K_{1,\lambda}^N\)  &  &  &  & &  &  &  &  &  & \\
%%       t = 2 & \(\vz^{(1,1)}_2\) & \(\vz^{(1,2)}_2\) & \ldots & \(\vz^{(1,N)}_2\) & \ldots & \(\vz^{(2,1)}_2\) & \(\vz^{(2,2)}_2\) & \ldots & \(\vz^{(2,N)}_2\) & \ldots & \(\vz^{(1,N)}_2\) & \(\vz^{(2,N)}_2\) & \ldots & \(\vz^{(M,N)}_2\) \\
%%       & & & & & & \(\downarrow\;K_{2,\lambda}\) & \(\downarrow\;K_{2,\lambda}^2\) & \ldots & \(\downarrow\;K_{2,\lambda}^N\)  &  &  &  & &  \\
%%       t = 3 & \(\vz^{(1,1)}_3\) & \(\vz^{(1,2)}_3\) & \ldots & \(\vz^{(1,N)}_3\) & \ldots & \(\vz^{(2,1)}_2\) & \(\vz^{(2,2)}_3\) & \ldots & \(\vz^{(2,N)}_3\) & \ldots & \(\vz^{(1,N)}_3\) & \(\vz^{(2,N)}_3\) & \ldots & \(\vz^{(M,N)}_3\) \\
%%   \end{tabular}
%%   \egroup
%%   \end{center}
%%   }
%%   where \(K_{m,\vlambda}\left(\vz, \cdot\right)\) is a componentwise IMH kernel for the \(m\)th component.
%%   Conceptually, this means that, when a batch \(\vz^{(m,1:N)}\) is selected \(k\) times, it will evolve as
%%   \begin{alignat*}{2}
%%     \vz^{(m,1:N)} \sim  K_{m,\vlambda}^{\left(k-1\right)\,N + 1}\left(\vz^{(m,1)}, d\vz^{(m,1)}\right) \, K_{m,\vlambda}^{\left(k-1\right)\,N + 2}\left(\vz^{(m,2)}, d\vz\prime^{(m,2)}\right) \cdot \ldots \cdot K_{m,\vlambda}^{\left(k-1\right)\,N + N}\left(\vz^{(m,N)}, d\vz\prime^{(m,N)}\right).
%%   \end{alignat*}

%%   The random-scan kernel can be represented as
%%   \begin{alignat*}{2}
%%     P_{\vlambda}^k\left(\veta, d\veta\prime\right)
%%     =
%%     \sum_{m=1}^M
%%     r_{m} \,
%%     \left(
%%     K_{m,\vlambda}\left(\vz^{(m,1)}, d\vz\prime^{(m,1)}\right) \,
%%     K_{m,\vlambda}\left(\vz^{(m,2)}, d\vz\prime^{(m,2)}\right)
%%     \cdot
%%     \ldots
%%     \cdot
%%     K_{m,\vlambda}^N\left(\vz^{(m,N)}, d\vz\prime^{(m,N)}\right)
%%     \right)
%%   \end{alignat*}
%%   where .

  Now, the convergence in total variation \(d_{\mathrm{TV}}\left(\cdot, \cdot\right)\) can be shown to decrease geometrically as
  \begin{alignat}{2}
    &\DTV{P_{\vlambda}^{n}\left(\veta, \cdot\right)}{\Pi}
    \nonumber
    \\
    &\quad=
    \sup_{A}
    \abs{
      \Pi\left(A\right)
      -
      P^{n}\left(\veta, A\right)
    }
    &&\quad\text{\textit{Definition of \(d_{\text{TV}}\)}}
    \nonumber
    \\
    &\quad\leq
    \sup_{A}
    \Bigg|
    \int_{A}
      \pi\left(d\vz\prime^{(1)}\right) \times \ldots \times \pi\left(d\vz\prime^{(N)}\right)
    \nonumber
      \\
      &\qquad\qquad\qquad-
      K^{(n-1)\,N\,+1}_{\vlambda}\left(\vz^{(1)}, d\vz\prime^{(1)}\right) \times \ldots \times K^{n\,N}_{\vlambda}\left(\vz^{(N)}, d\vz\prime^{(N)}\right)
    \,\Bigg|
    \nonumber
    \\
    &\quad\leq
    \sup_{A}
    \sum_{n=1}^N
    \abs{
    \int_{A}
      \pi\left(d\vz^{(n)}\right) - K^{(n-1)\,N + n}_{\vlambda}\left(\vz^{(n)}, d\vz\prime^{(n)}\right) 
    }
    &&\quad\text{\textit{\cref{thm:product_measure_bound}}}
    \nonumber
    \\
    &\quad=
    \sum_{n=1}^N
    \DTV{K^{(n-1)\,N + n}_{\vlambda}\left(\vz^{(n)}, \cdot\right)}{\pi}
    &&\quad\text{\textit{Definition of \(d_{\text{TV}}\)}}
    \nonumber
    \\
    &\quad\leq
    \sum_{n=1}^N
    \rho^{(n-1)\,N + n}
    &&\quad\text{\textit{Geometric ergodicity}}
    \nonumber
    %\label{eq:used_ergodicity}
    \\
    &\quad=
    \rho^{n\,N}
    \,
    \rho^{-N}
    \,
    \frac{\rho - \rho^{N+1}}{1 - \rho}
    &&\quad\text{\textit{Solved sum}}
    \nonumber
    \\
    &\quad=
    \frac{\rho \, \left(1 - \rho^N\right)}{\rho^N \left(1 - \rho\right)}
    \,
    {\left( \rho^{N} \right)}^n.
    \nonumber
  \end{alignat}
  Although the constant depends on \(\rho\) and \(N\), the kernel \(P\) is geometrically ergodic and converges \(N\) times faster than the base kernel \(K\).

  \paragraph{\textbf{Bound on the Gradient Variance}}
  To analyze the variance of the gradient, we require a detailed expression of the \(n\)-step marginal transition kernel, which is unavailable in general for most MCMC kernels.
  Fortunately, specifically for the IMH kernel,~\citet{Smith96exacttransition} have shown that the \(n\)-step marginal IMH kernel is given as~\cref{eq:imh_exact_kernel}.
  From this, we show that
  \begin{alignat}{2}
    &\E{ \norm{ \vg\left(\vlambda, \rvveta\right) }^2_{*} \,\middle|\, \mathcal{F}_{t} }
    \nonumber
    \\
    &\quad=
    \E{ \norm{ \vg\left(\vlambda, \rvveta\right) }^2 \,\middle|\, \vz_{t-1}^{(N)},\, \vlambda_{t-1} }
    \nonumber
    \\
    &\quad=
    \E{ \norm{ \frac{1}{N}\sum^{N}_{n=1} \vs\left(\vlambda; \rvvz^{(n)}\right) }^2_{*} \,\middle|\, \vz_{t-1}^{(N)},\, \vlambda_{t-1} }
    \nonumber
    \\
    &\quad\leq
    \E{ \frac{1}{N^2} \sum^{N}_{n=1} \norm{\vs\left(\vlambda; \rvvz^{(n)}\right) }^2_{*} \,\middle|\, \vz_{t-1}^{(N)},\, \vlambda_{t-1} }
    &&\quad\text{\textit{Triangle inequality}}
    \nonumber
    \\
    &\quad=
    \frac{1}{N^2}\sum^{N}_{n=1} \Esub{\rvvz^{(n)} \sim K^n\left(\vz_{t-1}, \cdot\right)}{ \norm{\vs\left(\vlambda; \rvvz^{(n)}\right) }^2_{*} \,\middle|\,  \vz_{t-1}^{(N)},\, \vlambda_{t-1} }
    &&\quad\text{\textit{Linearity of expectation}}
    \nonumber
    \\
    &\quad\leq
    \frac{1}{N^2}\sum^{N}_{n=1}
      n \, {\left(1 - \frac{1}{w^*}\right)}^{n-1}
      \Esub{\rvvz^{(n)} \sim q_{\text{def.}}\left(\cdot; \vlambda\right)}{ \norm{\vs\left(\vlambda; \rvvz^{(n)} \right)}_{*}^2 }
      \nonumber
      \\
      &\qquad+ 
        {\left(1 - \frac{1}{w^*}\right)}^{n} \, \norm{\vs\left(\vlambda; \vz_{t-1}^{(N)} \right)}_{*}^2
    &&\quad\text{\textit{\cref{thm:imh_expecation}}}
    \nonumber
    \\
    &\quad\leq
    \frac{1}{N^2}\sum^{N}_{n=1}
      n \, {\left(1 - \frac{1}{w^*}\right)}^{n-1}  \, L^2
      +
      {\left(1 - \frac{1}{w^*}\right)}^{n} \, L^2
    &&\quad\text{\textit{\cref{thm:bounded_score}}}
    \nonumber
    \\
    &\quad=
    \frac{L^2}{N^2}\sum^{N}_{n=1}
      n \, {\left(1 - \frac{1}{w^*}\right)}^{n-1}
      +
      {\left(1 - \frac{1}{w^*}\right)}^{n} 
    \nonumber
    &&\quad\text{\textit{Moved constant forward}}
    \\
    &\quad=
    \frac{L^2}{N^2} \,
    \left[\,
      {\left(w^*\right)}^2 + w^*
      -
      {\left(1 - \frac{1}{w^*}\right)}^N
      \left(
        {\left(w^*\right)}^2 + w^* + N\,w^*
      \right)
    \,\right]
    \nonumber
    &&\quad\text{\textit{Solved sum}}
    \\
    &\quad=
    \frac{L^2}{N^2} \,
    \left[\,
    \frac{1}{2} N^2 + \frac{3}{2}\,N 
    + \mathcal{O}\left(1/w^*\right)
    \,\right]
    \nonumber
    &&\quad\text{\textit{Laurent series expansion at \(w^* \rightarrow \infty\)}}
    \\
    &\quad=
    L^2 \,
    \left[\,
    \frac{1}{2} + \frac{3}{2}\,\frac{1}{N}
    + \mathcal{O}\left(1/w^*\right)
    \,\right].
    \nonumber
  \end{alignat}
  The laurent approximation is useful for realistic values of \(w^*\) since it is bounded below exponentially by the KL divergence.
\end{proofEnd}

%%% Local Variables:
%%% TeX-master: "master"
%%% End:

%

\paragraph{Parallel Markov Chain Score Ascent (proposed)}
The gradient bound of JSA suggests that in challenging settings where \(w^*\) is large, increasing \(N\) does not improve variance.
To fix this limitation, we propose a new MCSA scheme we call parallel MCSA (pMCSA) that achieves \(\mathcal{O}\left(\nicefrac{1}{N}\right)\) variance reduction.
In particular, instead of using \(N\) \textit{sequential} Markov chain states, we operate \(N\) parallel Markov chains.
To obtain a similar per-SGD-iteration cost, we perform only a single Markov-chain transition for each chain.
We will later discuss the computational costs in detail.
See~\cref{alg:pmcsa} for a detailed pseudocode.


\begin{theoremEnd}{theorem}\label{thm:pmcsa}
  pMCSA, our proposed scheme, is obtained by setting
  {%\small
  \begin{align*}
    P_{\vlambda}^n\left(\veta, d\veta\prime\right)
    = 
    K_{\vlambda}^n\left(\vz^{(1)}, d\vz\prime^{(1)}\right)
    \,
    K_{\vlambda}^n\left(\vz^{(2)}, d\vz\prime^{(2)}\right)
    \cdot
    \ldots 
    \cdot
    K_{\vlambda}^n\left(\vz^{(N)}, d\vz\prime^{(N)}\right)
  \end{align*}
  }
  with \(\veta = \left[\vz^{(1)}, \vz^{(2)}, \ldots, \vz^{(N)}\right]\).
  Then, given~\cref{thm:bounded_weight,thm:bounded_score}, the mixing rate and the gradient variance bounds are
  {\small
  \begin{align*}
    \DTV{P_{\vlambda}^n\left(\veta, \cdot\right)}{\Pi}
    \leq
    C\left(N\right)\,{\rho^n}
    \quad\text{and}\quad
    \E{ \norm{ \vg\left(\vlambda, \rvveta\right) }^2_{*} \,\middle|\, \mathcal{F}_{t} }
    \leq
    L^2 \left[\; \frac{1}{N} + \frac{1}{N}\,\left(1 - \frac{1}{w^*}\right) \;\right],
  \end{align*}
  }
  where \(w^* = \sup_{\vz} \pi\left(\vz\right) / q_{\text{def.}}\left(\vz\right)\) and \(C\) is some positive constant depending on \(N\).
\end{theoremEnd}
\begin{proofEnd}

  Our proposed scheme, pMCSA, is described in~\cref{alg:jsa}. 
  At each iteration, our scheme performs a single MCMC transition for each of the \(N\) samples, or chains, to estimate the gradient.
  Similarly to JSA, we use the IMH kernel \(K_{\vlambda}\).

  \paragraph{Ergodicity of the Markov Chain}
  Since our kernel operates the same MCMC kernel \(K_{\vlambda}\) for each of the \(N\) parallel Markov chains, the \(n\)-step marginal kernel \(P_{\vlambda}\) can be represented as
  \begin{align*}
    P_{\vlambda}^n\left(\veta, d\veta\prime\right)
    = 
    K_{\vlambda}^n\left(\vz^{(1)}, d\vz\prime^{(1)}\right)
    \,
    K_{\vlambda}^n\left(\vz^{(2)}, d\vz\prime^{(2)}\right)
    \cdot
    \ldots 
    \cdot
    K_{\vlambda}^n\left(\vz^{(N)}, d\vz\prime^{(N)}\right).
  \end{align*}
  Then, the convergence in total variation \(d_{\mathrm{TV}}\left(\cdot, \cdot\right)\) can be shown to decrease geometrically as
  \begin{alignat}{2}
    &\DTV{K^{k}\left(\veta, \cdot\right)}{\Pi}
    \nonumber
    \\
    &\quad=
    \sup_{A}
    \abs{
      \Pi\left(A\right)
      -
      P^{n}\left(\veta, A\right)
    }
    &&\quad\text{\textit{Definition of \(d_{\text{TV}}\)}}
    \nonumber
    \\
    &\quad\leq
    \sup_{A}
    \big|\;
    \int_{A}
      \pi\left(d\vz\prime_1\right) \cdot \ldots \cdot \pi\left(d\vz\prime_N\right)
    \nonumber
      \\
      &\qquad\qquad\qquad-
      K^n\left(\vz_1, d\vz\prime_1\right) \cdot \ldots \cdot K^n\left(\vz_N, d\vz\prime_N\right)
    \;\big|
    \nonumber
    \\
    &\quad\leq
    \sup_{A}
    \sum_{n=1}^N
    \abs{
    \int_{A}
      \pi\left(d\vz\prime_k\right) - K^{n}\left(\vz_n, d\vz\prime_n\right) 
    }
    &&\quad\text{\textit{\cref{thm:product_measure_bound}}}
    \nonumber
    \\
    &\quad=
    \sum_{n=1}^N
    \DTV{K^n\left(\vz_n, \cdot\right)}{\pi}
    &&\quad\text{\textit{Definition of TV distance}}
    \nonumber
    \\
    &\quad\leq
    \sum_{n=1}^N
    \rho^{n}
    &&\quad\text{\textit{Geometric ergodicity}}
    \nonumber
    \\
    &\quad=
    N\,\rho^{k}
    &&\quad\text{\textit{Solved sum}}.
    \nonumber
  \end{alignat}

  \paragraph{\textbf{Bound on the Gradient Variance}}
  The bound on the gradient variance can be derived in similar manner to JSA as
  \begin{alignat}{2}
    &\E{ \norm{ \vg\left(\vlambda, \rvveta\right) }^2_{*} \,\middle|\, \mathcal{F}_{t} }
    \nonumber
    \\
    &\quad=
    \E{ \norm{ \vg\left(\vlambda, \rvveta\right) }^2 \,\middle|\, \vz_{t-1}^{(1:N)},\, \vlambda_{t-1} }
    \nonumber
    \\
    &\quad=
    \E{ \norm{ \frac{1}{N}\sum^{N}_{n=1} \vs\left(\vlambda; \rvvz_{n}\right) }^2_{*} \,\middle|\, \vz_{t-1}^{(1:N)},\, \vlambda_{t-1} }
    \nonumber
    \\
    &\quad\leq
    \E{  \frac{1}{N^2} \sum^{N}_{n=1} \norm{\vs\left(\vlambda; \rvvz_{n}\right) }^2_{*} \,\middle|\, \vz_{t-1}^{(1:N)},\, \vlambda_{t-1} }
    &&\quad\text{\textit{Triangle inequality}}
    \nonumber
    \\
    &\quad=
    \frac{1}{N^2}\sum^{N}_{n=1} \Esub{\rvvz_{n} \sim K\left(\vz_{t-1}^{(n)}, \cdot\right)}{ \norm{\vs\left(\vlambda; \rvvz_{n}\right) }^2_{*} \,\middle|\,  \vz_{t-1}^{(1:N)},\, \vlambda_{t-1} }
    &&\quad\text{\textit{Linearity of expectation}}
    \nonumber
    \\
    &\quad\leq
    \frac{1}{N^2}\sum^{N}_{n=1}
      \Esub{\rvvz_n \sim q_{\text{def.}}\left(\cdot;\vlambda\right)}{ \norm{\vs\left(\vlambda; \rvvz_n \right)}_{*}^2 }
      +
      {\left(1 - \frac{1}{w^*}\right)} \, \norm{\vs\left(\vlambda; \vz_{t-1}^{(n)} \right)}_{*}^2
    &&\quad\text{\textit{\cref{thm:imh_expecation}}}
    \nonumber
    \\
    &\quad\leq
    \frac{1}{N^2}\sum^{N}_{n=1}
        L^2 + {\left(1 - \frac{1}{w^*}\right)} \, L^2
    &&\quad\text{\textit{\cref{thm:bounded_score}}}
    \nonumber
    \\
    &\quad=
    \frac{L^2}{N^2} \sum^{N}_{n=1}
      1 + {\left(1 - \frac{1}{w^*}\right)}
    \nonumber
    &&\quad\text{\textit{Moved constant forward}}
    \\
    &\quad=
    L^2 \left[ \frac{1}{N} + \frac{1}{N}\,{\left(1 - \frac{1}{w^*}\right)} \right].
    \nonumber
    &&\quad\text{\textit{Solved sum}}
  \end{alignat}
  %Therefore, the gradient decreases at a solid \(\mathcal{O}\left(1/N\right)\) rate.
\end{proofEnd}

%%% Local Variables:
%%% TeX-master: "master"
%%% End:


%% This suggests that our proposed scheme achieves clear \(\mathcal{O}\left(1/N\right)\) variance reduction with a slower mixing rate.
%% The mixing rate independent bounds in \cref{table:convergence} suggest that this would provide faster convergence regardless.
%% Our empirical results in \cref{section:eval} show that this is indeed true in practice.

\vspace{-0.05in}
\paragraph{Theoretical Performance of MSC, JSA, and pMCSA}
By combining~\cref{thm:msc,thm:jsa,thm:pmcsa} with the convergence rates in~\cref{table:convergence}, we can compare the theoretical convergence of the considered algorithms.
For MSC, when \(w^*\) is large, the gradient variance and the mixing rate are worse than JSA and pMCSA and cannot be improved by increasing \(N\).
Therefore, we will not discuss it further.
On the other hand, JSA and pMCSA can be seen as trading-off bias (mixing rate) for variance.
However, when \(w^*\) is large, we can expect JSA to perform worse than pMCSA due to the constant \(1/2\).

\vspace{-0.05in}
\paragraph{Effect of increasing \(N\)}
If we consider the convergence rate of~\citet{duchi_ergodic_2012}, the mixing rate affects the convergence rate through the \(\log \rho^{-1}\) term.
For JSA, the combined rate appears to worsen as we increase \(N\), while for pMCSA, the combined rate stays constant with respect to \(N\).
In practice, however, we observe that increasing \(N\) accelerates convergence in general (quite dramatically for pMCSA).
Therefore, the mixing rate independent convergence rates by~\citet{doan_finitetime_2020, doan_convergence_2020} appears to better reflect practical performance.
This is especially true in MCSA since
\begin{enumerate*}[label=\textbf{(\roman*)}]
  \item the mixing rate \(\rho\) is a conservative \textit{global} bound and 
  \item the mixing rate will improve as MCSA converges.
\end{enumerate*}
Unfortunately, it is challenging to incorporate the changes in the mixing rate into the convergence rate.

%% \paragraph{Bias v.s. Variance}
%% While our proposed scheme achievs superior variance reduction, the mixing rate is worse.
%% In a MCMC estimation perspective, this translates into higher bias.
%% However, we note that
%% \begin{enumerate*}[label=\textbf{(\roman*)}]
%%   \item the constant \(C\left(\rho, N\right)\) depends on \(\rho\), 
%%   \item all of the ergodic convergence rate are close to 1 as \(w^* \rightarrow \infty\), and
%%   \item the mixing rate is a conservative global bound with respect to \(\vlambda\).
%% \end{enumerate*}
%% Therefore, in general, the superior ergodic convergence rate of JSA does not translate into faster convergence of MCSA.
%% In fact, as MCSA converges, \(w^*\) also decreases, dramatically improving the mixing rate.
%% In contrast, the relative variance does not improve too much with \(w^*\).
%% Therefore, reducing the variance is much more effective for accelerating convergence.
%% We empirically show this fact on the bias and variance in~\cref{section:simulation}.

\begin{wraptable}{r}{0.6\textwidth}
  \vspace{-0.75in}
  
% Second version of table, with booktabs.
%\begin{table}
%\centering
\caption{Computational Costs of MCSA Schemes}\label{table:cost}
\setlength{\tabcolsep}{0.5pt}
  \begin{threeparttable}
\begin{tabular}{lccccc}\toprule
& \multicolumn{3}{c}{\footnotesize Kernel Application} & \multicolumn{2}{c}{\footnotesize Gradient Estimation} \\
\cmidrule(lr){2-4}\cmidrule(lr){5-6}
  & {\footnotesize\(p\left( \vz, \vx \right)\)}
  & {\footnotesize\(q\left(\vz; \vlambda\right)\)}
  & {\footnotesize\(q\left(\vz; \vlambda\right)\)}
  & {\footnotesize\(p\left( \vz, \vx \right)\)}
  & {\footnotesize\( q\left(\vz; \vlambda\right)\)}
  \\
  & {\footnotesize\# Eval.  }
  & {\footnotesize\# Eval.  }
  & {\footnotesize\# Samples}
  & {\footnotesize\# Grad.  }
  & {\footnotesize\# Grad.  }
%
\\\midrule
%
{\footnotesize
ELBO
}
& \(0\)
& \(0\)
& \(N\)
& \(N\)
& \(N\)
\\\arrayrulecolor{black!30}\midrule
%
{\footnotesize
MSC
}
& \(N-1\)
& \(N\)
& \(N-1\)
& \(0\)
& \(1\)\tnote{1}\;\;{\footnotesize or}\;\(N\)\tnote{2}
\\
%
{\footnotesize
JSA
}
& \(N\)
& \(N+1\)
& \(N\)
& \(0\)
& \(N\)
\\
%
{\footnotesize
\textit{pMCSA}
}
& \(N\)
& \(2 \, N\)
& \(N\)
& \(0\)
& \(N\)
\\\bottomrule
\end{tabular}
  \begin{tablenotes}
    \item[*]{\footnotesize We assume that the parameters are cached as much as possible}.
    \item[1]{\footnotesize Vanilla CIS kernel}.
    \item[2]{\footnotesize Rao-Blackwellized CIS kernel}.
  \end{tablenotes}
  \end{threeparttable}
%\end{table}

  \vspace{-0.2in}
\end{wraptable}
%
\subsection{Computational Cost}
The three schemes using the CIS kernel and the IMH kernel can have different computational costs depending on the parameter \(N\).
The computational costs of each scheme are organized in~\cref{table:cost}.

\vspace{-0.05in}
\paragraph{Cost of Sampling Proposals}
For the CIS kernel used by MSC, \(N\) controls the number of internal proposals sampled from \(q\,(\vz; \vlambda)\).
For JSA and our proposed scheme, the IMH kernel only uses a single sample from \(q\,(\vz; \vlambda)\), but applies the kernel \(N\) times.
Assuming caching is done as much as possible, the parallel state estimator needs twice the density evaluations of \(q\,(\vz; \vlambda)\) compared to other methods.
However, this added cost is minimal since the overall computational cost is dominated by  \(p\,(\vz,\vx)\).

\vspace{-0.05in}
\paragraph{Cost of Estimating the Score}
When estimating the score, MSC computes \(\nabla_{\vlambda} \log q\,(\vz; \vlambda)\) only once, while JSA and our proposed scheme compute it \(N\) times.
However,~\cite{NEURIPS2020_b2070693} also discuss a Rao-Blackwellized version of the CIS kernel, which also computes the gradient \(N\) times.
Lastly, notice that MCSA methods do not need to differentiate through the likelihood, unlike ELBO maximization, making its per-iteration cost significantly cheaper.

%%% Local Variables:
%%% TeX-master: "master"
%%% End:
