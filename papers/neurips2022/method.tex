
%\vspace{-0.05in}
\section{Practical Performance Analysis of Markov Chain Score Ascent}
\subsection{Non-Asymptotic Convergence of Markov Chain Score Ascent}\label{section:convergence}

\vspace{-0.05in}
\paragraph{Markov Chain Score Ascent}
As shown in~\cref{eq:mcgd}, the basic ingredients of MCGD are the target function \(f\left(\vlambda, \eta\right)\), the gradient estimator \(\vg\left(\vlambda, \eta\right)\), and the Markov chain kernel \(P_{\vlambda}\left(\eta, \cdot\right)\).
Obtaining MCSA through MCGD boils down to designing \(\vg\) and \(P_{\vlambda}\) such that \(f\left(\vlambda\right) = \DKL{\pi}{q\left(\cdot; \vlambda\right)} \).
In the following theorem, we provide practical conditions for setting \(g\) and \(P_{\vlambda}\) such that the MCGD steps in~\cref{eq:mcgd} results in MCSA.


\begin{theoremEnd}{proposition}\label{thm:product_kernel}
  Let \cref{thm:kernel_conditions,thm:gradient_estimator} hold.
  Then,
  \begin{align*}
    \Esub{\Pi}{ f\left(\vlambda, \rvveta\right) } = \DKL{\pi}{q\left(\cdot; \vlambda\right)},
    \quad\text{and}\quad
    \Esub{\Pi}{ g\left(\vlambda, \rvveta\right) } = \nabla_{\vlambda} \DKL{\pi}{q\left(\cdot; \vlambda\right)}.
  \end{align*}
\end{theoremEnd}
\begin{proofEnd}
  For notational convenience, we define the shorthand
  \begin{align*}
    \pi\left(\vz_{(1:N)}\right) = \pi\left(\vz_1\right) \, \pi\left(\vz_2\right) \times \ldots \times \pi\left(\vz_N\right).
  \end{align*}
  Then,
  \begin{alignat}{2}
    &\Esub{\Pi}{ f\left(\vlambda, \veta\right) }
    \nonumber
    \\
    &\quad=
    \int \left( \frac{1}{N} \sum^{N}_{n=1} \log q\left(\vz_n; \vlambda\right) + \mathbb{H}\left[\,\pi\,\right]\right) \, \pi\left(\vz_{(1:N)}\right) \, d\vz_{(1:N)}
    \nonumber
    \\
    &\quad=
     \frac{1}{N} \sum^{N}_{n=1} \left\{ \int \big(\, \log q\,(\,\vz_n; \vlambda\,) + \mathbb{H}\,[\,\pi\,] \,\big) \, \pi\left(\vz_{(1:N)}\right) \, d\vz_{(1:N)} \right\}
     &&\quad{\text{\textit{Swapped integral and sum}}}
    \nonumber
    \\
    &\quad=
    \frac{1}{N} \sum^{N}_{n=1} \int \big(\, \log q\,(\,\vz_n; \vlambda\,) + \mathbb{H}\,[\,\pi\,] \,\big) \, \pi\left(\vz_n\right) \, d\vz_{n}
    &&\quad{\text{\textit{Marginalized \(\vz_m\) for all \(m \neq n\)}}}
    \nonumber
    \\
    &\quad=
    \frac{1}{N} \sum^{N}_{n=1} \DKL{\pi}{q\left(\cdot; \vlambda\right)}
    \nonumber
    &&\quad{\text{\textit{Definition of \(D_{\text{KL}}\)}}}
    \\
    &\quad=
    \DKL{\pi}{q\left(\cdot; \vlambda\right)}\label{eq:F_KL}
  \end{alignat}
  For \(\Esub{\Pi}{ g\left(\vlambda, \rvveta\right) }\), note that 
  \begin{align}
    \nabla_{\vlambda} f\left(\vlambda, \veta\right) = \frac{1}{N} \sum^{N}_{n=1} s\left(\vz_n; \vlambda\right) = g\left(\vlambda, \veta\right).\label{eq:F_grad_G}
  \end{align}
  Therefore, it suffices to show that
  \begin{align*}
    \nabla_{\vlambda} \DKL{\pi}{q\left(\cdot; \vlambda\right)}
    &=
    \nabla_{\vlambda} \Esub{\Pi}{ f\left(\vlambda, \rvveta\right) }
    &&\text{\textit{\cref{eq:F_KL}}}
    \\
    &=
    \Esub{\Pi}{ \nabla_{\vlambda}  f\left(\vlambda, \rvveta\right) }
    &&\text{\textit{{Leibniz derivative rule}}}
    \\
    &=
    \Esub{\Pi}{ g\left(\vlambda, \rvveta\right) }.
    &&\text{\textit{\cref{eq:F_grad_G}}}
  \end{align*}
\end{proofEnd}

%%% Local Variables:
%%% TeX-master: "master"
%%% End:


This framework is general enough to include both JSA and MSC.
We will later propose a third novel scheme that conforms to~\cref{thm:product_kernel}.
Note that \(N\) here can be regarded as the computational budget of each MCGD iteration since the cost of
\begin{enumerate*}[label=\textbf{(\roman*)}]
  \item generating the Markov chain samples \(\vz^{(1)}, \ldots, \vz^{(N)}\) and
  \item computing the gradient \(\vg\)
\end{enumerate*}
will linearly increase with \(N\).

In addition, to utilize the convergence analysis results of MCGD, we require \(P\) to be geometrically ergodic.
An execption is~\citet{debavelaere_convergence_2021} where they assume \(P\) to be polynomially ergodic.
\begin{assumption}{(Markov chain kernel)}\label{thm:kernel_conditions}
\vspace{-0.05in}
  The Markov chain kernel \(P\) is geometrically ergodic as
  {%\small
  \[
  \DTV{P_{\vlambda}^{n}\left(\veta, \cdot\right)}{ \Pi } \leq C \, \rho^{n}
  \]
  }
  for some positive constant \(C\).
\vspace{-0.05in}
\end{assumption}
\vspace{-0.05in}

\vspace{-0.05in}
\paragraph{Non-Asymptotic Convergence Rates}
Based on the conclusion of \cref{thm:product_kernel,thm:kernel_conditions} and additional assumptions on the objective function such as convexity, we can apply the previous convergence results of MCGD to MCSA.
\cref{table:convergence} provides a brief list of some relevant results.
We omitted terms resulting from assumptions on the objective function, such as Lipschitz smoothness.
Instead, we focus on the terms including the gradient variance (\(G\)) and mixing rate (\(\rho\)) since they are closely related to the algorithmic design of MCSA algorithms.

\vspace{-0.05in}
\paragraph{Convergence and Mixing Rate}
An analysis on the general MCGD setting was first provided by~\citet{duchi_ergodic_2012}.
Notice their rate is dependent on the mixing rate by the \(1 / \log \rho^{-1}\) term.
For MCSA, this is conservative since, on challenging problems, the mixing is slow such that \(\rho \approx 1\).
Fortunately, recent results by \citet{doan_convergence_2020,doan_finitetime_2020} have shown that it is possible to obtain rates independent of the mixing rate.
In particular, in the result of~\citet{doan_finitetime_2020}, the influence of the mixing rate decreases in a rate of \(\mathcal{O}\left(\nicefrac{1}{T^2}\right)\).
%The fact that the convergence rate can be independent of the mixing rate is critical.
This obervation is important since it means trading gradient variance and mixing rate could result in profitable deals.

\begin{table*}
\vspace{-0.2in}
\centering
\caption{Convergence Rates of MCGD Algorithms}\label{table:convergence}
\setlength{\tabcolsep}{3pt}
\begin{threeparttable}
  \begin{tabular}{lllcc}\toprule
    \multicolumn{1}{c}{\footnotesize\textbf{Algorithm}} & \multicolumn{1}{c}{\footnotesize\textbf{Stepsize Rule}} & \multicolumn{1}{c}{\footnotesize\textbf{Gradient Assumption}} & {\footnotesize\textbf{Rate}} & {\footnotesize\textbf{Reference}} \\\midrule
    \multirow{2}{*}{\small Mirror Descent\tnote{1}}
    & \multirow{2}{*}{\small\(\gamma_t = \gamma / \sqrt{t}\)}
    & \multirow{2}{*}{\small\(\E{ {\|\, \vg\left(\vlambda, \veta\right) \,\|}_*^2 \mid \mathcal{F}_t } < G^2\)}
    & \multirow{2}{*}{\small\(\mathcal{O}\left(\frac{G^2 \log T}{ \log \rho^{-1} \sqrt{T}}\right)\)}
    & {\footnotesize\citet{duchi_ergodic_2012}}
    \\
    &&&& {\footnotesize{Corollary 3.5}}
    \\\cdashlinelr{1-5}
    \multirow{2}{*}{\small SGD-Nesterov\tnote{2}}
    & {\small\(\gamma_t = 2/(t + 1)\)}
    & \multirow{2}{*}{\footnotesize\( {\|\vg\left(\vlambda, \veta\right)\|}_2 < G \)}
    & \multirow{2}{*}{\small\(\mathcal{O}\left(\frac{G^2 \log T}{ \sqrt{T}}\right)\)}
    & {\footnotesize\citet{doan_convergence_2020}}
    \\
    & {\footnotesize\(\beta_t = \frac{1}{2 \, L \sqrt{t + 1}}\)}
    &&& {\footnotesize{Theorem 2}}
    \\\cdashlinelr{1-5}
    \multirow{2}{*}{\small SGD}
    & {\footnotesize\(\gamma_t = \gamma/t\)}
    & \multirow{2}{*}{\footnotesize\( {\|\,\vg\left(\vlambda, \veta\right)\|}_* < G \left( \norm{\vlambda}_2 + 1 \right) \)}
    & \multirow{2}{*}{\small\(\mathcal{O}\left(\frac{G^2 \log T}{ T}\right)\)}
    & {\footnotesize\citet{doan_finitetime_2020}}
    \\ 
    & {\footnotesize\(\gamma = \min\{\nicefrac{1}{2\,L}, \nicefrac{2 L}{\mu}\}\)}
    &&& {\footnotesize{Theorem 1,2}}
    \\ \bottomrule
  \end{tabular}
  \begin{tablenotes}
  \item[1]{ \(\mathcal{F}_t\) is the \(\sigma\)-field formed by all the iterates \(\veta_t\), \(\vlambda_t\) up to the \(t\)th SGD iteration. }
  \item[1]{ \(\norm{\vx}_*\) is the dual norm such that \(\norm{\vx}_* = \sup_{\norm{\vz} \leq 1} \iprod{\vx}{\vz}\).}
  \item[2]{ \(\beta_t\) is the stepsize of the momentum.}
  \item[2,3]{ \(L\) is the Lipschitz smoothness constant.}
  \end{tablenotes}
\end{threeparttable}
\vspace{-0.2in}
\end{table*}

\vspace{-0.05in}
\paragraph{Gradient Bound Assumption}
Except for \citet{doan_finitetime_2020}, most results assume that the gradient is bounded for \(\forall\veta,\vlambda\) as {\footnotesize\( {\| \vg\left(\vlambda, \veta\right) \|} < G \)}.
Admidttedly, this condition is strong for MCSA, but it is similar to the bounded variance assumption {\footnotesize\(\mathbb{E}\,[\norm{\vg}^2]  < G^2\)} used in vanilla SGD, which is also known to be strong since it does not hold for strongly convex objectives~\citep{pmlr-v80-nguyen18c}.
%Furthermore, strictly speaking, even the assumption of~\citet{doan_finitetime_2020} is still strong to include the most basic variational families.
%% For example for the Gaussian family, the gradient bound with respect to the covariance matrix is bounded as
%% {\small
%% \begin{align*}
%%   \norm{ \nabla_{\mSigma} \log q\left(\vz; \vlambda\right) }_F
%%   \leq
%%   \norm{
%%   \nabla_{\mSigma} {\left(\vmu - \vz\right)}^{\top} \mSigma^{-1} {\left(\vmu - \vz\right)}
%%   }_{F}
%%   =
%%   \norm{
%%    \mSigma^{-\top} \, {\left(\vmu - \vz\right) \, \left(\vmu - \vz\right)}^{\top} \, \mSigma^{-\top}
%%   }_{F}
%% \end{align*}
%% }%
%where \(\vlambda = \left(\mu, \mSigma\right)\).
%Even if we bound the norm of \(\vz\), it is apparent that the gradient grows as \(\norm{\vlambda}^2_2\) instead of \(\norm{\vlambda}_2\).
Nonetheless, assuming the existence of \(G\) can lead to useful analysis with practical benefits.
For example, it can be used to compare the performance of different algorithms as done by~\citet{pmlr-v108-geffner20a}.
In a similar spirit, we will obtain the gradient bound \(G\) of different MCSA algorithms and compare them.

%
\begin{theoremEnd}{proposition}\label{thm:score_bound}
  The score gradient of the variational approximation \(q\left(\cdot; \vlambda\right)\) in the location scale family with a base density that is uniformly Lipschitz continuous as
  \begin{align*}
    \norm{ \nabla_{\vlambda} \log q\left(\cdot; \vlambda\right) }_2
    \leq
    D \, L \, M \left( \norm{\vz}_2 + \norm{\vlambda}_2 + 1 \right)
  \end{align*}
  where \(\vz \in \mathbb{R}^D\).
\end{theoremEnd}
\begin{proofEnd}
  Let us denote the reparameterization function 
  \begin{align}
    \vt^{-1}\left(\vz, \vlambda\right) = \mC^{-1}_{\vlambda}\left(\vz - \vm_{\vlambda}\right)\label{eq:reparam}
  \end{align}
  such that
  \begin{align*}
    q\left(\vz; \vlambda\right) = \phi\left( \vt^{-1}\left(\vz, \vlambda\right) \right).
  \end{align*}

  for \(\vlambda = (\vm_{\vlambda}, \mC_{\vlambda})\), \(\vmu = (\vm_{\vmu}, \mC_{\vmu})\)
  Using this, we analyze the gradient of the location \(\vm_{\vlambda}\) and scale \(\mC_{\vlambda}\) parameters separately.
  First, the gradient of the location parameter is bounded as
  \begin{alignat}{2}
    \norm{
      \nabla_{\vm_{\vlambda}} \log q\left(\vz; \vlambda\right)
    }_2
    &=
    \norm{
      \nabla_{\vm_{\vlambda}} \log \phi\left( \vt^{-1}\left(\vz, \vlambda\right) \right)
    }_2
    &&\quad\text{\textit{\cref{eq:reparam}}}
    \nonumber
    \\
    &=
    \norm{
    \nabla_{\vm_{\vlambda}} \vt^{-1}\left(\vz, \vlambda\right)  \nabla_{\vt^{-1}} \log \phi\left( \vt^{-1}\left(\vz, \vlambda\right) \right) 
    }_2
    &&\quad\text{\textit{Chain rule}}
    \nonumber
    \\
    &=
    \norm{
    \nabla_{\vm_{\vlambda}} \left( \mC^{-1}_{\vlambda}\left(\vz - \vm_{\vlambda}\right) \right) \nabla_{\vt^{-1}} \log \phi\left( \vt^{-1}\left(\vz, \vlambda\right) \right) 
    }_2
    &&\quad\text{\textit{\cref{eq:reparam}}}
    \nonumber
    \\
    &=
    \norm{
      -\mC^{-1}_{\vlambda}
      \nabla_{\vt^{-1}} \log \phi\left( \vt^{-1}\left(\vz, \vlambda\right) \right) 
    }_2
    &&\quad\text{\textit{Solved derivative}}
    \nonumber
    \\
    &\leq
    \norm{
      \mC^{-1}_{\vlambda}
    }_{\mathrm{op}}
    \,
    \norm{
      \nabla_{\vt^{-1}} \log \phi\left( \vt^{-1}\left(\vz, \vlambda\right) \right) 
    }_2
    \nonumber
    \\
    &\leq
    \sqrt{M} \,
    L.
    &&\quad\text{\textit{\cref{thm:lipschitz}}}
    \label{eq:location_grad}
  \end{alignat}

  Second, for the scale parameter, we bound the derivative of the \(i,j\) element of the matrix \(\mC_{\vlambda}\).
  Let us drop the subscript \(\vlambda\) for clarity.
  Then, similarly to the location parameter,
  \begin{alignat}{2}
    \abs{
      \nabla_{\mC_{i,j}} \log q\left(\vz; \vlambda\right)
    }
    &=
    \abs{
      \nabla_{\mC_{i,j}} \log \phi\left( \vt^{-1}\left(\vz, \vlambda\right) \right)
    }
    &&\quad\text{\textit{\cref{eq:reparam}}}
    \nonumber
    \\
    &=
    \abs{
      \iprod{
        \nabla_{\mC_{i,j}} \vt^{-1}\left(\vz, \vlambda\right)   
      }{
        \nabla_{\vt} \log \phi\left( \vt^{-1}\left(\vz, \vlambda\right) \right)   
      }
    }
    &&\quad\text{\textit{Chain rule}}
    \nonumber
    \\
    &=
    \abs{
      \iprod{
        \nabla_{\mC_{i,j}} \left( \mC^{-1}_{\vlambda}\left(\vz - \vm_{\vlambda}\right) \right)
      } {
        \nabla_{\vt} \log \phi\left( \vt^{-1}\left(\vz, \vlambda\right) \right)          
      }
    }
    \nonumber
    \\
    &\leq
    \norm{
      \nabla_{\mC_{i,j}} \left( \mC^{-1}_{\vlambda}\left(\vz - \vm_{\vlambda}\right) \right)
    }_2 \,
    \norm{
      \nabla_{\vt} \log \phi\left( \vt^{-1}\left(\vz, \vlambda\right) \right)          
    }_2
    &&\quad\text{\textit{Cauchy-Schwarz inequality}}
    \nonumber
    \\
    &\leq
    L\,
    \norm{
      \nabla_{\mC_{i,j}} \left( \mC^{-1}_{\vlambda}\left(\vz - \vm_{\vlambda}\right) \right)
    }_2.
    &&\quad\text{\textit{\cref{thm:lipschitz}}}
    \nonumber
    \\
    &=
    L\,
    \norm{
    \mC^{-1}_{\vlambda} \frac{\partial \mC}{\partial_{\mC_{i,j}}} \mC^{-1}_{\vlambda}  \left(\vz - \vm_{\vlambda}\right) 
    }_2
    \nonumber
    &&\quad\text{\textit{Solved gradient}}
    \\
    &\leq
    L\,
    \norm{
      \mC^{-1}_{\vlambda}
    }_{\text{op}} \,
    \norm{
      \frac{\partial \mC}{\partial_{\mC_{i,j}}}
    }_{\text{op}} \,
    \norm{
      \mC^{-1}_{\vlambda} 
    }_{\text{op}}
    \norm{
    \vz - \vm_{\vlambda}
    }_2
    \nonumber
    &&\quad\text{\textit{Property of operator norm}}
    \\
    &=
    L\,
    \norm{
      \mC^{-1}_{\vlambda}
    }_{\text{op}} \,
    \norm{
      \mC^{-1}_{\vlambda} 
    }_{\text{op}}
    \norm{
    \vz - \vm_{\vlambda}
    }_2
    &&\quad\text{\textit{Largest eigenvalue \(\sigma_{\text{max}}\left(\frac{\partial \mC}{\partial_{\mC_{i,j}}}\right) = 1\)}}
    \nonumber
    \\
    &\leq
    L \,
    M \,
    \norm{
    \vz - \vm_{\vlambda}
    }_2
    &&\quad\text{\textit{\cref{thm:solution_space}}}
    \nonumber
    \\
    &\leq
    L \,
    M \,\left(
    \norm{\vz}_2 + \norm{\vm_{\vlambda}}_2
    \right)
    &&\quad\text{\textit{Triangle inequality}}
    \nonumber
    \\
    &\leq
    L \,
    M \, \left(
    \norm{\vz}_2 + \norm{\vlambda}_2
    \right)
    &&\quad\text{\textit{\cref{eq:parameter_norm}}}
    \nonumber
  \end{alignat}

  Now, by the poperty of norms,
  \begin{alignat}{2}
    \norm{
      \nabla_{\mC} \log q\left(\vz; \vlambda\right)
    }_{F}
    &\leq
    D \,
    \sup_{i,j}
    \abs{
      \nabla_{\mC_{i,j}} \log q\left(\vz; \vlambda\right)
    }
    \nonumber
    \\
    &\leq
    D \,
    L \,
    M \, \left(
    \norm{\vz} + \norm{\vlambda}
    \right).
    \label{eq:scale_grad}
  \end{alignat}
  Then, combining \cref{eq:location_grad,eq:scale_grad}, 
  \begin{alignat}{2}
    \norm{
      \nabla_{\vlambda} \log q\left(\vz; \vlambda\right)
    }
    &\leq
    \norm{
      \nabla_{\vm_{\vlambda}} \log q\left(\vz; \vlambda\right)
    }_2
    +
    \norm{
      \nabla_{\mC_{\vlambda}} \log q\left(\vz; \vlambda\right)
    }_{F} 
    &&\quad\text{\textit{\cref{eq:parameter_norm}}}
    \nonumber
    \\
    &\leq
    D \,
    L \,
    M \, \left(
    \norm{\vz}_2 + \norm{\vlambda}_2
    \right)
    +
    \sqrt{M} \, L
    &&\quad\text{\textit{\cref{eq:location_grad,eq:scale_grad}}}
    \nonumber
    \\
    &\leq
    D \,
    L \,
    M \, \left(
    \norm{\vz}_2 + \norm{\vlambda}_2
    +
    1 
    \right).
    \nonumber
  \end{alignat}
\end{proofEnd}

%%% Local Variables:
%%% TeX-master: "master"
%%% End:



%% \begin{assumption}{\textbf{(Bounded variance)}}\label{thm:bounded_variance}
%%   Let \(\vlambda \in \Lambda\) be measurable with respect to the \(\sigma\)-field \(\mathcal{F}_{t-1}\).
%%   The  gradient estimator \(g\) is bounded a constant \(G < \infty\) such that
%%   \(
%%   \E{ {\lVert\, g\left(\cdot, \rvveta_{t}\right) \,\rVert}^2_{*} \;\middle|\; \mathcal{F}_{t-1}} < G^2.
%%   \)
%% \end{assumption}

%% Under the stated conditions, the non-asymptotic convergence rate of MCSA is a special case of the ergodic mirror descent algorithm of~\citet{duchi_ergodic_2012}.

%
\begin{theoremEnd}[all end]{lemma}\label{thm:mixing_time}
  Assuming the kernel \(P_{\vlambda}\left(\veta, \cdot\right)\) satisfies \cref{thm:ergodicity}, the Hellinger mixing time \(\tau_{\text{Hel.}}\) is bounded as
  \begin{align*}
    \tau_{\text{Hel.}}\left(K_{\vlambda}, \epsilon\right) \leq  \frac{2}{\log \rho^{-1}} \log \frac{1}{\epsilon}
  \end{align*}
\end{theoremEnd}
\begin{proofEnd}
  The following two inequalities are equivalent.
  \begin{align*}
    d_{\text{Hel.}}\left(\, P^t_{\vlambda}\left(\veta, \cdot \right), \Pi \,\right)
    &\leq \sqrt{\DTV{ P^t_{\vlambda}\left(\veta, \cdot \right)}{\Pi}}
    \leq \rho^{t / 2}
    \\
    \log d_{\text{Hel.}}\left(\, P^t_{\vlambda}\left(\veta, \cdot \right), \Pi \,\right)
    &\leq \frac{t}{2} \log \rho
  \end{align*}

  The Hellinger mixing time \(\tau_{\text{Hel.}}\left(K_{\vlambda}, \epsilon\right)\) is the smallest \(t\) that statisfies the inequality
  \begin{align*}
    d_{\text{Hel.}}\left(\, P^t_{\vlambda}\left(\veta, \cdot \right), \Pi \,\right)
    &\leq 
    \epsilon.
  \end{align*}
  Instead, we can find the \(t\prime > t\) that satisfies
  \begin{align*}
    \log d_{\text{Hel.}}\left(\, P^t_{\vlambda}\left(\veta, \cdot \right), \Pi \,\right)
    \leq 
    \frac{t\prime}{2} \log \rho
    \leq 
    \log \epsilon
 \end{align*}
  by solving the inequalities
  \begin{alignat*}{2}
    \frac{t\prime}{2} \log \rho
    &\leq 
    \log \epsilon
    \\
    t\prime 
    &\geq 
    \frac{2}{\log \rho} \log \epsilon
    &&\quad\text{\textit{Inequality flipped since \(\log \rho \leq 1\)}}
    \\
    t\prime 
    &\geq 
    \frac{2}{\log \rho^{-1}} \log \frac{1}{\epsilon}
 \end{alignat*}
\end{proofEnd}

\begin{theoremEnd}{theorem}{(\textbf{Convergence rate})}\label{thm:convergence_rate}
  Assuming~\cref{thm:kernel_conditions,thm:gradient_estimator,thm:logconcave,thm:compact,thm:bounded_variance} hold, with a diminishing stepsize of \(\alpha_t = \alpha /  \sqrt{T} \), the average iterates {\small\(\overline{\vlambda}_T = \sum^{T}_{t=1} \vlambda_{t} / T\)} of Markov chain score ascent achieve a convergence rate of
  {%\small
  \begin{align*}
    \E{ \DKL{\pi}{q\,(\cdot; {\overline{\vlambda}_{T}})} - \DKL{\pi}{q\left(\cdot; {\vlambda^*}_{T}\right)}}
    =
    \mathcal{O}\left(
    \frac{
      \alpha \, G^2 \, \sqrt{\log T}
    }{
      \log \rho^{-1} \, \sqrt{T}
    } \right)  
  \end{align*}
  }
\end{theoremEnd}
\begin{proofEnd}
  \citet[Corollary 3.5]{duchi_ergodic_2012} provide a non-asymptotic convergence rate for the \textit{ergodic mirror descent} algorithm which computes the parameter update as
  \begin{align}
    \vlambda_{t+1} &= \argmin_{\vlambda \in \Lambda} \left\{\,
    \iprod{\,\vg\left(\vlambda, \veta_t\right)}{\vlambda\,} 
    +
    \frac{1}{\alpha_t} D_{\psi}\left(\vlambda, \vlambda_{t}\right)
    \,\right\}\label{eq:ergodic_mirror_descent}
    \\
    \rvveta_{t+1} &\sim P_{\vlambda_{t}}\left(\veta_t, \cdot\right)
    \nonumber
  \end{align}
  where \(D_{\psi}\) is the Bregman divergence defined as
  \begin{align*}
    D_{\psi}\left(\vlambda, \vlambda\prime\right)
    \triangleq
    \psi\left(\vlambda\right)
    - \psi\left(\vlambda\prime\right)
    - \iprod{ \nabla \psi\left(\vlambda\prime\right)}{\vlambda - \vlambda\prime}
  \end{align*}
  for some convex function \(\psi\).
  Our result is based on the fact that MCGD is a special case of the ergodic mirror descent algorithm.
  Specifically, by choosing \(\psi\left(\vlambda\right) = \frac{1}{2} \norm{\vlambda}^2_2 \), we obtain
  \begin{alignat}{2}
    D_{\psi}\left(\vlambda, \vlambda\prime\right) = \frac{1}{2} \norm{\vlambda - \vlambda\prime}_2^2 \leq \frac{1}{2} \, R^2.\label{eq:bregman}
    &&\quad\text{\textit{\cref{thm:compact}}}
  \end{alignat}
  This reduces the update in \cref{eq:ergodic_mirror_descent} into projected gradient descent which is the form used for Markov chain score climbing.

  Under our assumptions, \citet[Corollary 3.5]{duchi_ergodic_2012} show that, by assuming that the Hellinger mixing time is bounded as
  \begin{align}
    \tau_{\text{Hel.}}\left(K_{\vlambda}, \epsilon\right) \leq \kappa_1 \log\left( \kappa_2 /\epsilon \right) \label{eq:hellinger_mixing}
  \end{align}
  for any \(\epsilon > 0\), and setting a decreasing stepsize \(\alpha_t = \alpha / \sqrt{t}\),
  it follows that
  \begin{align}
    &\E{ \DKL{\pi}{q\,(\cdot; {\overline{\vlambda}_{T}})} - \DKL{\pi}{q\left(\cdot; {\vlambda^*}_{T}\right)}}
    \nonumber
    \\
    &\;\leq
    \frac{R^2}{2 \, \alpha \, \sqrt{T}}
    +
    \frac{2 \, \alpha \, G^2}{\sqrt{T}}\left( \kappa_1 \, \log \frac{\kappa_2}{\epsilon} \right)
    +
    3 \, \epsilon \, G \, R
    +
    \frac{R \, G \, \kappa_1 \, \log \frac{\kappa_2}{\epsilon}}{T}.
    \label{eq:duchi_base}
  \end{align}

  From \cref{thm:mixing_time}, we retrieve the constants of \cref{eq:hellinger_mixing} as
  \(
  \kappa_1=\frac{2}{\log \rho^{-1} },\;  \kappa_2 = 1
  \), and set \(\epsilon = 1/\sqrt{T}\).
  Then, using
  \begin{alignat}{2}
    \kappa_1 \, \log \frac{\kappa_2}{\epsilon}
    =
    \frac{1}{\log \rho^{-1} } \, \log T,\label{eq:kappa}
  \end{alignat}
  it follows that
  \begin{alignat*}{2}
    &\E{ \DKL{\pi}{q\,(\cdot; {\overline{\vlambda}_{T}})} - \DKL{\pi}{q\left(\cdot; {\vlambda^*}_{T}\right)}}
    \\
    &\;\leq
    \frac{R^2}{2 \, \alpha \, \sqrt{T}}
    +
    \frac{2 \, \alpha \, G^2}{\sqrt{T}}\left( \kappa_1 \, \log \frac{\kappa_2}{\epsilon} \right)
    +
    3 \, \epsilon \, G \, R
    +
    \frac{R \, G \, \kappa_1 \, \log \frac{\kappa_2}{\epsilon}}{T}.
    \nonumber
    &&\quad\text{\textit{\cref{eq:duchi_base}}}
    \\
    &\;=
    \frac{R^2}{2 \, \alpha \, \sqrt{T}}
    +
    \frac{2 \, \alpha \, G^2 \, \log T }{\log \rho^{-1} \, \sqrt{T}}
    +
    3 \, \epsilon \, G \, R
    +
    \frac{R \, G \, \log T }{\log \rho^{-1} \, T}.
    \nonumber
    &&\quad\text{\textit{\cref{eq:kappa}}}
  \end{alignat*}
\end{proofEnd}

%%% Local Variables:
%%% TeX-master: "master"
%%% End:

%\input{thm_convergence_rate2}

%% This result is a direct adaptation of the ergodic mirror descent algorithm by~\cite{duchi_ergodic_2012}.
%% For accelerated variants of MCGD,~\citet{doan_convergence_2020} provide non-asymptotic convergence results.
%% However, their bound for the convex case is independent of the kernel mixing rate, which leaves out the practical effects of the mixing rate.

\vspace{-0.05in}
\subsection{Performance Analysis of Markov Chain Score Ascent Methods}\label{section:comparison}
\vspace{-0.05in}
Both MSC and JSA naturally satisfy the MCSA framework discussed in~\cref{thm:product_kernel}.
Furthermore, we establish 
\begin{enumerate*}[label=\textbf{(\roman*)}]
  \item the geometric convergence rate of their implicitly defined kernel \(P\) and
  \item the upper bound on their gradient variance.
\end{enumerate*}
To do this, we use the following assumptions.
\begin{assumption}{(Bounded importance weight)}\label{thm:bounded_weight}
  The importance weight ratio \(w\left(\vz\right) = \pi\left(\vz\right) / q\left(\vz; \vlambda\right)\) is bounded by some finite constant as \(w^* < \infty\) for all \(\vlambda \in \Lambda\) such that \(\rho = \left(1 - 1/w^*\right) < 1\).
\end{assumption}
\vspace{-0.05in}
The fact that \(w^*\) exists for all \(\vlambda \in \Lambda\) is restrictive and not relevant in practice.
It does help excaping the slow mixing region in the initial MCGD steps, but the benefit vanishes quickly as we converge.
However, ensuring boundedness is necessary for theoretically analyzing MCGD through~\cref{thm:kernel_conditions}.
Although previous MCSA methods did not take specific measures to ensure~\cref{thm:bounded_weight}, this can be done by using a variational family with heavy tails~\citep{NEURIPS2018_25db67c5} or using a defensive mixture~\citep{hesterberg_weighted_1995, holden_adaptive_2009} as
\vspace{-0.05in}
\begin{align*}
  q_{\text{def.}}\left(\vz; \vlambda \right) = w \, q\left(\vz; \vlambda\right) + (1 - w) \, \nu\left(\vz\right)
\end{align*}
where \(0 < w < 1\) and \(\nu\left(\cdot\right)\) is a heavy tailed distribution that satisfies \(\sup_{\vz \in \mathcal{Z}} \pi\left(\vz\right) / \nu\left(\vz\right) < \infty\).
Note that \(q_{\text{def.}}\) is only used in the Markov chain kernels, which does not restrict our choice of the variational family.

%
%% \begin{assumption}{(Location scale family)}\label{thm:location_scale}
%%   The variational family is the location-scale family with the location (\(\vm\)) and scale (\(\mC\)) parameters denoted as \(\vlambda = (\vm_{\vlambda}, \mC_{\vlambda})\) with the base density \(\phi\) such that the probability density and samples are given by
%%   \begin{align*}
%%     &\text{\textit{(sampling)}}\quad \rvvz = \mC_{\vlambda} \, \rvvu + \vm_{\vlambda}, \quad \rvvu \sim \phi\left(\cdot\right), 
%%     &\text{\textit{(density)}}\quad q\left(\vz; \vlambda\right) = \phi\left( \mC^{-1}_{\vlambda}\left( \vz - \vm_{\vlambda} \right) \right)
%%   \end{align*}
%% \end{assumption}
%% Furthermore, the squared \(L^2\)-norm of variational parameters are given as
%% \begin{align}
%%   \norm{\vlambda}_2^2 = \norm{\vm_{\vlambda}}_2^2 + \norm{\mC_{\vlambda}}_F^2 \label{eq:parameter_norm}
%% \end{align}
%% for \(\vlambda = (\vm_{\vlambda}, \mC_{\vlambda})\).

%% \begin{assumption}{(Lipschitz continuity of base distribution)}\label{thm:lipschitz}
%%   the base density is log-Lipschitz continuous as
%%   \(
%%     \abs{ \log \phi\left(\vu\right) - \log \phi\left(\vu\prime\right) } \leq L \, \norm{\vu - \vu\prime}
%%   \)
%%   for some finite constant \(L > 0\).
%% \end{assumption}
%% This is equivalent to assuming \(\norm{\nabla \log \phi\left(\vu\right) } < L\)

%% \begin{assumption}{Contrained scale}\label{thm:solution_space}
%%   The parameter space \(\Lambda_M\) is constrained such that the singular values of the scale matrix are lower bounded such that
%%   {\small
%%   \[
%%   \Lambda_M = \left\{\, (\vm, \mC) \,\middle|\, \sigma_{\text{min}}(\mC) \geq \frac{1}{\sqrt{M}} \,\right\}
%%   \]
%%   }
%%   for some finite constant \(M > 0\).
%% \end{assumption}

%% \begin{assumption}{Uniformly Lipschitz continuous score}\label{thm:solution_space}
%%   The score gradient is bivariate uniformly Lipschitz continuous as
%%   {\small
%%   \[
%%   \norm{ \nabla \log q\left(\vz; \vlambda\right) }_2 \leq L \left( \norm{\vlambda}_2 + \norm{\vz}_2 + 1 \right)
%%   \]
%%   }
%%   for some finite constant \(L > 0\).
%% \end{assumption}

%% \begin{assumption}{(Bounded variance)}\label{thm:bounded_variance}
%%   The samples \(\veta_t\) generated by the Markov chain kernel as \(\rvveta \sim P_{\vlambda}\left(\veta, \cdot\right)\) have a finite second moment for all \(t\) such as
%%   \(
%%    \mathbb{E}[\, {\| \veta_t \|}^2 \,|\, \mathcal{F}_{t-1} \,] < V,
%%   \)
%%   for some finite constant \(V > 0\).
%% \end{assumption}

\begin{assumption}{(Bounded Score)}\label{thm:bounded_score}
  The score gradient is bounded for \(\forall \vlambda \in \Lambda\) and \(\forall \vz \in \mathcal{Z}\) such that \(\norm{\vs\left(\vlambda; \vz\right)}_* \leq L \) for some finite constant \(L > 0\).
\end{assumption}
\vspace{-0.05in}
Informally, this assumption is equivalent to assuming 
%% \begin{enumerate*}[label=\textbf{(\roman*)}]
%% \item \(\Lambda\) is compact with a finite radius,
%% \item the Markov chains are variance bounding~\citep[Theorem 1]{10.2307/25442663},
%% \item and that the score gradient is continuous with respect to \(\vlambda\) and \(\vz\).
%% \end{enumerate*}
Admittedly, this assumption is strong, but it provides a straightforward way to compare the gradient variance of different MCSA implementations.

\vspace{-0.05in}
\paragraph{Markovian Score Climbing}
MSC is a simple instance of MCSA where \(\eta_t = \vz_t\) and \(P_{\vlambda_t} = K_{\vlambda_t}\), \(K_{\vlambda_t}\) is the conditional importance sampling (CIS) kernel (originally proposed as the iterated sequential importance resampling kernel by~\citet{andrieu_uniform_2018}).
See \cref{alg:msc} for a detailed pseudocode.
Although MSC uses only a single sample such that \(N=1\), the CIS kernel internally uses \(N\) proposals to generate a single sample.
Therefore, note \(N\) in MSC has a different meaning, but it still indicates the computational budget.


\begin{theoremEnd}[all end]{lemma}\label{thm:product_measure_bound}
  For the probability measures \(p_1, \ldots, p_N\) and \(q_1, \ldots, q_N\) defined on a measurable space \((\mathsf{X}, \mathcal{A})\) and an arbitrary set \(A \in \mathcal{A}\),
  \begin{align*}
    &\abs{
    \int_{A^N}
    p_1\left(dx_1\right)
    p_2\left(dx_2\right)
    \times
    \ldots
    \times
    p_N\left(dx_N\right)
    -
    q_1\left(dx_1\right)
    q_2\left(dx_2\right)
    \times
    \ldots
    \times
    q_N\left(dx_N\right)
  }
    \\
  &\qquad\leq
  \sum_{n=1}^N
  \abs{
    \int_{A}
    p_n\left(dx_n\right)
    -
    q_n\left(dx_n\right)
  }
  \end{align*}
\end{theoremEnd}
\begin{proofEnd}
  By using the following shorthand notations
  \begin{alignat*}{2}
    p_{(1:N)}\left(dx_{(1:N)}\right)
    &= 
    p_1\left(dx_1\right)
    p_2\left(dx_2\right)
    \times
    \ldots
    \times
    p_N\left(dx_N\right)
    \\
    q_{(1:N)}\left(dx_{(1:N)}\right)
    &= 
    q_1\left(dx_1\right)
    q_2\left(dx_2\right)
    \times
    \ldots
    \times
    q_N\left(dx_N\right),
  \end{alignat*}
  the result follows from induction as
  \begin{alignat}{2}
    &\abs{
      \int_{A^N}
      p_{(1:N)}\left(dx_{(1:N)}\right)
      -
      q_{(1:N)}\left(dx_{(1:N)}\right)
    }
    \nonumber
    \\
    &\quad=
    \Bigg|\;
    \left( \int_{A} p_1\left(dx_1\right) - q_1\left(dx_1\right) \right) \,
    \int_{A^{N-1}} p_{(2:N)}\left(dx_{(2:N)}\right)
    \nonumber
    \\
    &\qquad\quad+
    \int_{A} q_1\left(dx_1\right) \,
    {\left(
      \int_{A^{N-1}}
      p_{(2:N)}\left(dx_{(2:N)}\right)
      -
      q_{(2:N)}\left(dx_{(2:N)}\right)
    \right)}
    \;\Bigg|
    \nonumber
    \\
    &\quad\leq
    \Bigg|
    \int_{A} p_1\left(dx_1\right) - q_1\left(dx_1\right)
    \Bigg|\;
    \int_{A^{N-1}} p_{(2:N)}\left(dx_{(2:N)}\right)
    \nonumber
    \\
    &\qquad\quad+
    \int_{A} q_1\left(dx_1\right) \,
    {
    \Bigg|\;
      \int_{A^{N-1}}
      p_{(2:N)}\left(dx_{(2:N)}\right)
      -
      q_{(2:N)}\left(dx_{(2:N)}\right)
    }
    \;\Bigg|
    &&\quad\text{\textit{Triangle inequality}}
    \nonumber
    \\
    &\quad\leq
    \Bigg|
    \int_{A} p_1\left(dx_1\right) - q_1\left(dx_1\right)
    \Bigg|\;
    \nonumber
    \\
    &\qquad\quad+
    {
    \Bigg|\;
      \int_{A^{N-1}}
      p_{(2:N)}\left(dx_{(2:N)}\right)
      -
      q_{(2:N)}\left(dx_{(2:N)}\right)
    }
    \;\Bigg|.
    &&\quad\text{\textit{Applied \(p_n\left(A\right), q_n\left(A\right) \leq 1 \)}}
    \nonumber
  \end{alignat}
\end{proofEnd}


\begin{theoremEnd}{theorem}\label{thm:msc}
  MSC~\citep{NEURIPS2020_b2070693} is obtained by defining 
  {%\small
  \begin{align*}
  P_{\lambda}^n\left(\veta, d\veta\prime\right)
  = 
  K_{\lambda}^n\left(\vz, d\vz\prime\right)
  \end{align*}
  }
  with  \(\veta_t = \vz_t\) where \(K_{\vlambda}\left(\vz, \cdot\right)\) is the conditional importance sampling (CIS) kernel~\citep{NEURIPS2020_b2070693} with \(q_{def.}\left(\cdot; \vlambda\right)\) as its proposal distribution.
  Then, given~\cref{thm:bounded_weight,thm:bounded_score}, the mixing rate and the gradient bounds are given as
  {\small
  \begin{align*}
  \DTV{P_{\vlambda}^n\left(\veta, \cdot\right)}{\Pi} \leq  {\left(1 - \frac{N - 1}{2 w^* + N - 2}\right)}^n\quad \text{and}\quad
  {\small
  \E{ \norm{ \vg\left(\vlambda, \rvveta\right) }_{*}^2 \,\middle|\, \mathcal{F}_{t} } \leq  L^2,
  }
  \end{align*}
  }
  where \(w^* = \sup_{\vz} \pi\left(\vz\right) / q_{\text{def.}}\left(\vz\right)\).
\end{theoremEnd}
\begin{proofEnd}
  MSC is described in~\cref{alg:msc}. 
  At each iteration, it performs a single MCMC transition with the CIS kernel where it internally uses \(N\) proposals.

  \paragraph{Ergodicity of the Markov Chain}
  The ergodic convergence rate \(P_{\vlambda}\) is equal to that of the i-SIR kernel, whhich has been given by \citet{andrieu_uniform_2018} as
  \begin{alignat*}{2}
    \DTV{P^{n}\left(\veta, \cdot\right)}{\Pi}
    =
    \DTV{K^{n}\left(\vz, \cdot\right)}{\pi}
    \leq
    {\left(1 - \frac{N - 1}{2 w^* + N - 2}\right)}^n.
  \end{alignat*}

  \paragraph{\textbf{Bound on the Gradient Variance}}
  The bound on the gradient variance is straightforward given \cref{thm:bounded_score}.
  For simplicity, we denote the rejection state as \(\vz^{(1)} = \vz_{t-1} \).
  Then,
  \begin{alignat}{2}
    &\E{ \norm{ \vg\left(\vlambda, \rvveta\right) }_{2} \,\middle|\, \mathcal{F}_{t} }
    \nonumber
    \\
    &\;=
    \E{ \norm{ \vg\left(\vlambda, \rvveta\right) }_2 \,\middle|\, \mathcal{F}_{t}}
    \nonumber
    \\
    &\;=
    \Esub{\rvvz \sim K\left(\vz_{t-1}, \cdot\right)}{
      \norm{ \vs\left(\vlambda; \rvvz\right) }_{*} \,\middle|\,
      \vlambda_{t-1}, \vz_{t-1}
    }
    \nonumber
    \\
    &\;=
    \int
    \sum^{N}_{n=1}
    \frac{
      w\left(\vz^{(n)}\right)
    }{
      \sum^{N}_{m=1} w\left(\vz^{(n)}\right)
    }
    \norm{ \vs\left(\cdot; \vz^{(n)}\right) }^2_{*}
    \prod^{N}_{n=2}
    q\left(d\vz^{(n)}; \vlambda_{t-1}\right)
    \nonumber
    &&\quad\text{\textit{\citep{andrieu_uniform_2018}}}
    \\
    &\;\leq
    L^2 \,
    \int
    \sum^{N}_{n=1}
    \frac{
      w\left(\vz^{(n)}\right)
    }{
      \sum^{N}_{m=1} w\left(\vz^{(n)}\right)
    }
    \prod^{N}_{n=2}
    q\left(d\vz^{(n)}; \vlambda_{t-1}\right)
    \nonumber
    &&\quad\text{\textit{\cref{thm:bounded_score}}}
    \\
    &\;=
    L^2 \,
    \int
    \prod^{N}_{n=2}
    q\left(d\vz^{(n)}; \vlambda_{t-1}\right)
    \nonumber
   &&\quad\text{\textit{The sum of weights is 1}}
    \\
    &\;=
    L^2.
    \nonumber
  \end{alignat}

  %% This form coincides with solving the expectation of the self-normalized importance sampling estimator, which is well known to be challenging~\citep{robert_monte_2004}.
  %% We instead approximate the expectation by defining the random variables
  %% \(
  %% X = \sum^{N}_{n=1} w\left(\vz_n\right) \norm{ s\left(\cdot; \vz_n\right) }^2_{*}
  %% \)
  %% and
  %% \(
  %% Y = \sum^{N}_{m=1} w\left(\vz_m\right)
  %% \),
  %% use the delta method as
  %% \begin{alignat*}{2}
  %%   \E{\frac{\rvX}{\rvY} \;\middle|\; Z} &\approx \frac{\E{\rvX \mid Z }}{\E{\rvY \mid Z}} + \mathcal{O}\left(\frac{1}{{\E{\rvX \mid Z}}^2}\right).
  %% \end{alignat*}
  %% The required expectations are obtained as
  %% \begin{alignat}{2}
  %%   &\E{\rvX \mid \lambda_{t-1}, \vz_{t-1}}
  %%   \\
  %%   &\;=
  %%   \int \sum^{N}_{n=1} w\left(\vz_n\right) \norm{ s\left(\cdot; \vz_n\right) }^2_{*}
  %%   \prod^{N}_{n=2}
  %%   q\left(d\rvvz_n\right)
  %%   \nonumber
  %%   \\
  %%   &\quad=
  %%   \sum^{N}_{n=1} \int w\left(\vz_n\right) \norm{ s\left(\cdot; \vz_n\right) }^2_{*}
  %%   \prod^{N}_{n=2}
  %%   q\left(d\rvvz_n\right)
  %%   &&\quad\text{\textit{Swapped integral and sum}}
  %%   \nonumber
  %%   \\
  %%   &\quad=
  %%   \left\{\;
  %%   \sum^{N}_{n=2} \int w\left(\vz_n\right) \norm{ s\left(\cdot; \vz_n\right) }^2_{*}
  %%   \prod^{N}_{n=2} q\left(d\rvvz_n\right)
  %%   \right\}
  %%   +
  %%   w\left(\vz_1\right) \norm{ s\left(\cdot; \vz_1\right) }^2_{*}
  %%   &&\quad\text{\textit{Pulled out rejection state}}
  %%   \nonumber
  %%   \\
  %%   &\quad=
  %%   \left\{\;
  %%   \sum^{N}_{n=2} \int \frac{\pi\left(\vz_n\right)}{q\left(\vz_n\right)} \norm{ s\left(\cdot; \vz_n\right) }^2_{*}
  %%   \prod^{N}_{n=2}
  %%   q\left(d\rvvz_n\right)
  %%   \right\}
  %%   +
  %%   w\left(\vz_1\right) \norm{ s\left(\cdot; \vz_1\right) }^2_{*}
  %%   &&\quad\text{\textit{Definition of \(w\left(\vz\right)\)}}
  %%   \nonumber
  %%   \\
  %%   &\quad=
  %%   \sum^{N}_{n=2} \int \pi\left(d\vz_n\right) \norm{ s\left(\cdot; \vz_n\right) }^2_{*}
  %%   +
  %%   w\left(\vz_1\right) \norm{ s\left(\cdot; \vz_1\right) }^2_{*}
  %%   &&\quad\text{\textit{Cancelled out \(q\left(\cdot\right)\)}}
  %%   \nonumber
  %%   \\
  %%   &\quad=
  %%   \left(N-1\right) \, \Esub{\pi}{\norm{ s\left(\cdot; \rvvz\right) }^2_{*} }
  %%   +
  %%   w\left(\vz_1\right) \norm{ s\left(\cdot; \vz_1\right) }^2_{*}
  %%   \nonumber
  %%   \\
  %%   &\quad=
  %%   \left(N-1\right) \, \Esub{\pi}{\norm{ s\left(\cdot; \rvvz\right) }^2_{*} }
  %%   +
  %%   w\left(\vz_{t-1}\right) \norm{ s\left(\cdot; \vz_{t-1}\right) }^2_{*}
  %%   \nonumber
  %% \end{alignat}
  %% and similarly,
  %% \begin{alignat}{2}
  %%   \E{\rvY \mid \lambda_{t-1}, \vz_{t-1}}
  %%   &\quad= 
  %%   \int \sum^{N}_{n=1} w\left(\vz_n\right) \prod^{N}_{n=2} q\left(d\vz_n\right) 
  %%   \nonumber
  %%   \\
  %%   &\quad= 
  %%   \left\{\; \int \sum^{N}_{n=2} w\left(\vz_n\right) \prod^{N}_{n=2} q\left(d\vz_n\right) \;\right\} + w\left(\vz_1\right)
  %%   &&\quad\text{\textit{Pulled out rejection state}}
  %%   \nonumber
  %%   \\
  %%   &\quad= 
  %%   \left\{\; \sum^{N}_{n=2} \int w\left(\vz_n\right) \prod^{N}_{n=2} q\left(d\vz_n\right) \;\right\} + w\left(\vz_1\right)
  %%   &&\quad\text{\textit{Swapped integral and sum}}
  %%   \nonumber
  %%   \\
  %%   &\quad= 
  %%   \left\{\; \sum^{N}_{n=2} \int \frac{\pi\left(\vz_n\right)}{q\left(\vz_n\right)} \prod^{N}_{n=2} q\left(d\vz_n\right) \;\right\} + w\left(\vz_1\right)
  %%   &&\quad\text{\textit{Definition of \(w\left(\vz\right)\)}}
  %%   \nonumber
  %%   \\
  %%   &\quad= 
  %%   \left\{\; \sum^{N}_{n=2} \int \pi\left(d\vz_n\right) \;\right\} + w\left(\vz_1\right)
  %%   &&\quad\text{\textit{Cancelled out \(q\left(\cdot\right)\)}}
  %%   \nonumber
  %%   \\
  %%   &\quad= 
  %%   N - 1 + w\left(\vz_{t-1}\right)
  %%   \nonumber
  %% \end{alignat}
  %% Therefore, 
  %% \begin{alignat}{2}
  %%   \E{\frac{\rvX}{\rvY} \;\middle|\; Z}
  %%   &\approx
  %%   \frac{
  %%     \left(N-1\right) \, \Esub{\pi}{\norm{ s\left(\cdot; \rvvz\right) }^2_{*} }
  %%     +
  %%     w\left(\vz_1\right) \norm{ s\left(\cdot; \vz_1\right) }^2_{*}
  %%   }{
  %%     N-1 + w\left(\vz_1\right)
  %%   }
  %%   + \mathcal{O}\left(\frac{1}{{\left(N-1\right)}^2}\right)
  %%   \nonumber
  %%   \\
  %%   &\leq
  %%   \frac{
  %%     \left(N-1\right) \, \Esub{\pi}{\norm{ s\left(\cdot; \rvvz\right) }^2_{*} }
  %%     +
  %%     w^* \norm{ s\left(\cdot; \vz_1\right) }^2_{*}
  %%   }{
  %%     N-1
  %%   }
  %%   + \mathcal{O}\left(\frac{1}{{\left(N-1\right)}^2}\right)
  %%   &&\quad\text{\textit{\(w\left(\cdot\right) \leq w^*\)}}
  %%   \nonumber
  %%   \\
  %%   &\leq
  %%   \frac{
  %%     \left(N-1\right) \, L^2
  %%     +
  %%     w^* L^2
  %%   }{
  %%     N-1
  %%   }
  %%   + \mathcal{O}\left(\frac{1}{{\left(N-1\right)}^2}\right)
  %%   \nonumber
  %%   \\
  %%   &=
  %%   L^2
  %%   \left[
  %%   1
  %%   +
  %%   \frac{
  %%     w^*
  %%   }{
  %%     N-1
  %%   }
  %%   \right]
  %%   + \mathcal{O}\left(\frac{1}{{\left(N-1\right)}^2}\right)
  %%   \nonumber
  %% \end{alignat}
\end{proofEnd}

%%% Local Variables:
%%% TeX-master: "master"
%%% End:


\vspace{-0.05in}
\paragraph{Joint Stochastic Approximation}
JSA~\citep{pmlr-v124-ou20a} was originally proposed for deep generative models where the likelihood can be factorized by each datapoint.
In this setting,~\citeauthor{pmlr-v124-ou20a} perform minibatching by using a random-scan verion of the independent Metropolis-Hastings (IMH, \citealt{hastings_monte_1970,robert_monte_2004}) kernel.
We use the general verion of JSA with the vanilla IMH kernel which can be applied to non-factorizable likelihoods.
At each MCGD step, JSA performs multiple MCMC transitions and estimates the gradient by averaging all the intermediate samples.
See \cref{alg:jsa} for a detailed pseudocode.

\vspace{-0.05in}
\paragraph{Independent Metropolis-Hastings}
%A key element of JSA~\citep{pmlr-v124-ou20a} is that it uses the IMH kernel.
The IMH kernel used in JSA generate the proposals from the variational approximation \(q\left(\cdot; \vlambda_t\right)\).
To show geometric ergodicity of the implicit kernel \(P\), we utilize the geometric convergence rate of IMH kernels provided by~\citet[Theorem 2.1]{10.2307/2242610} and~\citet{wang_exact_2020}.
Furthermore, to derive an upper bound on the gradient variance, we use the exact \(n\)-step marginal IMH kernel derived by~\citet{Smith96exacttransition} as
{%\small
  \begin{align}
  K^n_{\vlambda}\left(\vz, d\vz\prime\right) 
  = T_n\left(\, w\left(\vz\right) \vee w\left(\vz\prime\right)\,\right) \, \pi\left(\vz\prime\right) \, d\vz\prime
  + \lambda^n\left(w\left(\vz\right)\right) \, \delta_{\vz}\left(d\vz\prime\right)
  \label{eq:imh_exact_kernel}
  \end{align}
}%
where {\(w\left(\vz\right) = \pi\left(\vz\right)/q_{\text{def.}}\left(\vz; \vlambda\right)\), \(x \vee y = \max\left(x, y\right)\)},
{\small
  \begin{align}
    T_n\left(w\right)      = \int_w^{\infty}
    \frac{n}{v^2}
    %\left(n / v^2\right)
    \, \lambda^{n-1}\left(v\right)\,dv,
    \quad\text{and}\quad
    \lambda\left(w\right) =
    \int_{R\left(w\right)}
    \left( 1 - \frac{w\left(\vz\prime\right)}{w}  \right)
    %\left( 1 - w\left(\vz\prime\right)/w  \right)
    \pi\left(d\vz\prime\right)\label{eq:T_lambda}
  \end{align}
}
for {\(R\left(w\right) = \{\, \vz\prime \mid w\,\left(\vz\prime\right) \leq w \,\}\)}.
%

\begin{theoremEnd}[all end]{lemma}\label{thm:lambda_bound}
  For \(w^* = \sup_{\vz} w\left(\vz\right) \), \(\lambda\left(\cdot\right)\) in~\cref{eq:T_lambda} is bounded as
  \[
   \max\left(1 - \frac{1}{w}, 0\right) \leq \lambda\left(w\right) \leq 1 - \frac{1}{w^*}.
  \]
\end{theoremEnd}
\begin{proofEnd}
  The proof can be found in the proof of Theorem 3 of \citet{Smith96exacttransition}.
\end{proofEnd}

\begin{theoremEnd}[all end]{lemma}\label{thm:tn_bound}
  For \(w^* = \sup_{\vz} w\left(\vz\right) \), \(T_t\left(\cdot\right)\) in~\cref{eq:T_lambda} is bounded as
  \[
  T_n\left( w \right) \leq \frac{n}{w} \, {\left(1 - \frac{1}{w^*}\right)}^{n-1}.
  \]
\end{theoremEnd}
\begin{proofEnd}
  \begin{alignat*}{2}
    T_n\left(w\right) 
    &= \int_w^{\infty} \frac{n}{v^2} \, \lambda^{n-1}\left(v\right)\,dv
    &&\quad\text{\textit{\cref{eq:T_lambda}}}
    \\
    &\leq \int_w^{\infty} \frac{n}{v^2} \, {\left(1 - \frac{1}{w^*}\right)}^{n-1}\,dv
    &&\quad\text{\textit{\cref{thm:lambda_bound}}}
    \\
    &= n \, {\left(1 - \frac{1}{w^*}\right)}^{n-1}  \int_w^{\infty} \frac{1}{v^2} \,dv
    &&\quad\text{\textit{Pulled out constant}}
    \\
    &= n \, {\left(1 - \frac{1}{w^*}\right)}^{n-1}  \left( {-\left.\frac{1}{v}\right\rvert^{\infty}_{w}} \right)
    &&\quad\text{\textit{Solved indefinite integral}}
    \\
    &= \frac{n}{w} \, {\left(1 - \frac{1}{w^*}\right)}^{n-1}.
  \end{alignat*}
  This upper bound is in general difficult to improve unless we impose stronger assumptions on \(\pi\) and \(q\).
\end{proofEnd}

\begin{theoremEnd}[all end]{lemma}\label{thm:imh_expecation}
  For a positive test function \(f : \mathcal{Z} \rightarrow \mathbb{R}^{+}\), the estimate of a \(\pi\)-invariant independent Metropolis-Hastings kernel with a proposal \(q\) is bounded as
  \begin{align*}
    \Esub{K^n\left(\vz, \cdot\right)}{ f \,\middle|\, \vz }
    \leq
    n \, {\left(1 - \frac{1}{w^*}\right)}^{n-1} 
    \Esub{q}{f}
    +
    {\left(1 - \frac{1}{w^*}\right)}^n \, f\left(\vz\right)
    %% \leq
    %% n \, \left(
    %% \Esub{q}{f}
    %% +
    %% \frac{1}{n} \, f\left(\vz\right)
    %% \right) 
  \end{align*}
  where \(w\left(\vz\right) = \pi\left(\vz\right) / q\left(\vz\right)\) and \(w^* = \sup_{\vz} w\left(\vz\right) \).
\end{theoremEnd}
\begin{proofEnd}
  \begin{alignat*}{2}
    &\Esub{K^n\left(\vz, \cdot\right)}{ f \,\middle|\, \vz }
    \\
    &\quad=
    \int T_n\left(w\left(\vz\right) \vee w\left(\vz\prime\right)\right) \, f\left(\vz\prime\right) \, \pi\left(\vz\prime\right) d\vz\prime
    +
    \lambda^{n}\left(w\left(\vz\right)\right) \, f\left(\vz\right)
    &&\quad{\text{\textit{\cref{eq:imh_exact_kernel}}}}
    \\
    &\quad\leq
    \int \frac{n}{w\left(\vz\right) \vee w\left(\vz\prime\right)} \, {\left(1 - \frac{1}{w^*}\right)}^{n-1} \, f\left(\vz\prime\right) \, \pi\left(\vz\prime\right) d\vz\prime
    +
    \lambda^{n}\left(w\left(\vz\right)\right) \, f\left(\vz\right)
    &&\quad{\text{\textit{\cref{thm:tn_bound}}}}
    \\
    &\quad\leq
    \int \frac{n}{w\left(\vz\prime\right)} \, {\left(1 - \frac{1}{w^*}\right)}^{n-1} \, f\left(\vz\prime\right) \, \pi\left(\vz\prime\right) d\vz\prime
    +
    \lambda^{n}\left(w\left(\vz\right)\right) \, f\left(\vz\right)
    &&\quad{\frac{1}{w\left(\vz\right) \vee w\left(\vz\prime\right)} \leq \frac{1}{w\left(\vz\prime\right)}}
    \\
    &\quad=
    n \, {\left(1 - \frac{1}{w^*}\right)}^{n-1} \, 
    \int \frac{1}{w\left(\vz\prime\right)} \, f\left(\vz\prime\right) \, \pi\left(\vz\prime\right) d\vz\prime
    +
    \lambda^{n}\left(w\left(\vz\right)\right) \, f\left(\vz\right)
    &&\quad{\text{\textit{Pulled out constant}}}
    \\
    &\quad=
    n \, {\left(1 - \frac{1}{w^*}\right)}^{n-1} \, 
    \int f\left(\vz\prime\right) \, q\left(\vz\prime\right) d\vz\prime
    +
    \lambda^{n}\left(w\left(\vz\right)\right) \, f\left(\vz\right)
    &&\quad{\text{\textit{Definition of \(w\left(\vz\right)\)}}}
    \\
    &\quad\leq
    n \, {\left(1 - \frac{1}{w^*}\right)}^{n-1} \, 
    \int f\left(\vz\prime\right) \, q\left(\vz\prime\right) d\vz\prime
    +
    {\left(1 - \frac{1}{w^*}\right)}^{n} \, f\left(\vz\right)
    &&\quad{\text{\textit{\cref{thm:lambda_bound}}}}
    \\
    &\quad=
    n \, {\left(1 - \frac{1}{w^*}\right)}^{n-1} 
    \Esub{q}{f}
    +
    {\left(1 - \frac{1}{w^*}\right)}^n \, f\left(\vz\right).
  \end{alignat*}
\end{proofEnd}


%%% Local Variables:
%%% TeX-master: "master"
%%% End:


\begin{theoremEnd}{theorem}{\textbf{(JSA,~\citealt{pmlr-v124-ou20a})}}
  JSA is obtained by defining 
  \begin{align*}
  P_{\vlambda}\left(\veta, d\veta\prime\right)
  = 
  K_{\vlambda}\left(\vz_1, d\vz\prime_1\right)
  \,
  K^2_{\vlambda}\left(\vz_2, d\vz\prime_2\right)
  \cdot
  \ldots 
  \cdot
  K^N_{\vlambda}\left(\vz_N, d\vz\prime_N\right)
  \end{align*}
  with \(\veta = \left[\vz_1, \vz_2, \ldots, \vz_N\right]\).
  Then, given~\cref{thm:bounded_weight,thm:bounded_score}, the mixing rate and the gradient variance bounds are
  \begin{align*}
    \DTV{P_{\vlambda}^k\left(\veta, \cdot\right)}{\Pi}
    \leq
    C\left(\rho, N\right)\,{\rho}^{k\,N},
    \quad\text{and}\quad
   % 
    \E{ \norm{ g\left(\cdot, \rvveta\right) }^2_{*} \,\middle|\, \mathcal{F}_{t} }
    \leq
    L^2 \,
    \left[\,
    \frac{1}{2} + \frac{3}{2}\,\frac{1}{N}
    + \mathcal{O}\left(\nicefrac{1}{w^*}\right)
    \,\right]
  \end{align*}
  for some positive constant \(C\) depending on both \(\rho\) and \(N\).
\end{theoremEnd}
\begin{proofEnd}
  \begin{alignat}{2}
    &\DTV{K^{k}\left(\veta, \cdot\right)}{\Pi}
    \nonumber
    \\
    &\quad=
    \sup_{A}
    \abs{
      \Pi\left(A\right)
      -
      P^{k}\left(\veta, A\right)
    }
    &&\quad\text{\textit{Definition of \(d_{\text{TV}}\)}}
    \nonumber
    \\
    &\quad\leq
    \sup_{A}
    \Bigg|
    \int_{A}
      \pi\left(d\vz\prime_1\right) \times \ldots \times \pi\left(d\vz\prime_N\right)
    \nonumber
      \\
      &\qquad\qquad\qquad-
      K^{(k-1)\,N\,+1}\left(\vz_1, d\vz\prime_1\right) \times \ldots \times K^{k\,N}\left(\vz_N, d\vz\prime_N\right)
    \,\Bigg|
    \nonumber
    \\
    &\quad\leq
    \sup_{A}
    \sum_{n=1}^N
    \abs{
    \int_{A}
      \pi\left(d\vz\prime_k\right) - K^{(k-1)\,N + n}\left(\vz_n, d\vz\prime_n\right) 
    }
    &&\quad\text{\textit{\cref{thm:product_measure_bound}}}
    \nonumber
    \\
    &\quad=
    \sum_{n=1}^N
    \DTV{K^{(k-1)\,N + n}\left(\vz_n, \cdot\right)}{\pi}
    &&\quad\text{\textit{Definition of \(d_{\text{TV}}\)}}
    \nonumber
    \\
    &\quad\leq
    \sum_{n=1}^N
    \rho^{(k-1)\,N + n}
    &&\quad\text{\textit{Geometric ergodicity}}
    \nonumber
    %\label{eq:used_ergodicity}
    \\
    &\quad=
    \rho^{k\,N}
    \,
    \rho^{-N}
    \,
    \frac{\rho - \rho^{N+1}}{1 - \rho}
    &&\quad\text{\textit{Solved sum}}
    \nonumber
    \\
    &\quad=
    \frac{\rho \, \left(1 - \rho^N\right)}{\rho^N \left(1 - \rho\right)}
    \,
    {\left( \rho^{N} \right)}^k
    \nonumber
  \end{alignat}

  \begin{alignat}{2}
    &\E{ \norm{ g\left(\cdot, \rvveta\right) }^2_{*} \,\middle|\, \mathcal{F}_{t} }
    \nonumber
    \\
    &\quad=
    \E{ \norm{ g\left(\cdot, \rvveta\right) }^2 \,\middle|\, \vz_{t-1},\, \vlambda_{t-1} }
    \nonumber
    \\
    &\quad=
    \E{ \norm{ \frac{1}{N}\sum^{N}_{n=1} s\left(\cdot; \rvvz_{n}\right) }^2_{*} \,\middle|\, \vz_{t-1},\, \vlambda_{t-1} }
    \nonumber
    \\
    &\quad\leq
    \E{ \frac{1}{N^2} \sum^{N}_{n=1} \norm{s\left(\cdot; \rvvz_{n}\right) }^2_{*} \,\middle|\, \vz_{t-1},\, \vlambda_{t-1} }
    &&\quad\text{\textit{Triangle inequality}}
    \nonumber
    \\
    &\quad=
    \frac{1}{N^2}\sum^{N}_{n=1} \Esub{\rvvz_{n} \sim K^n\left(\vz_{t-1}, \cdot\right)}{ \norm{s\left(\cdot; \rvvz_{n}\right) }^2_{*} \,\middle|\,  \vz_{t-1},\, \vlambda_{t-1} }
    &&\quad\text{\textit{Linearity of expectation}}
    \nonumber
    \\
    &\quad\leq
    \frac{1}{N^2}\sum^{N}_{n=1}
      n \, {\left(1 - \frac{1}{w^*}\right)}^{n-1}
      \Esub{\rvvz_n \sim q_{\text{def.}}\left(\cdot; \vlambda\right)}{ \norm{s\left(\cdot; \rvvz_n \right)}_{*}^2 }
      \nonumber
      \\
      &\qquad+ 
        {\left(1 - \frac{1}{w^*}\right)}^{n} \, \norm{s\left(\cdot; \vz_{t-1} \right)}_{*}^2
    &&\quad\text{\textit{\cref{thm:imh_expecation}}}
    \nonumber
    \\
    &\quad\leq
    \frac{1}{N^2}\sum^{N}_{n=1}
      n \, {\left(1 - \frac{1}{w^*}\right)}^{n-1}  \, L^2
      +
      {\left(1 - \frac{1}{w^*}\right)}^{n} \, L^2
    \nonumber
    \\
    &\quad=
    \frac{L^2}{N^2}\sum^{N}_{n=1}
      n \, {\left(1 - \frac{1}{w^*}\right)}^{n-1}
      +
      {\left(1 - \frac{1}{w^*}\right)}^{n} 
    \nonumber
    &&\quad\text{\textit{Moved constant forward}}
    \\
    &\quad=
    \frac{L^2}{N^2} \,
    \left[\,
      {\left(w^*\right)}^2 + w^*
      -
      {\left(1 - \frac{1}{w^*}\right)}^N
      \left(
        {\left(w^*\right)}^2 + w^* + N\,w^*
      \right)
    \,\right]
    \nonumber
    &&\quad\text{\textit{Solved sum}}
    \\
    &\quad=
    \frac{L^2}{N^2} \,
    \left[\,
    \frac{1}{2} N^2 + \frac{3}{2}\,N 
    + \mathcal{O}\left(1/w^*\right)
    \,\right]
    \nonumber
    &&\quad\text{\textit{Laurent series expansion at \(w \rightarrow \infty\)}}
    \\
    &\quad=
    L^2 \,
    \left[\,
    \frac{1}{2} + \frac{3}{2}\,\frac{1}{N}
    + \mathcal{O}\left(1/w^*\right)
    \,\right]
    \nonumber
  \end{alignat}
  The laurent approximation is useful for realistic values of \(w^*\) since it is bounded below exponentially by the KL divergence.
\end{proofEnd}

%%% Local Variables:
%%% TeX-master: "master"
%%% End:

%

\paragraph{Parallel Markov Chain Score Ascent (ours)}
For challenging problems, the mismatch between \(\pi\) and \(q_{\vlambda^*}\) can result in a large \(w^*\).
In this case, our JSA gradient bound suggests that increasing \(N\) does not improve variance, which significantly limits its applicability.
To fix this limitation, we propose a new MCSA scheme we call \textit{parallel MCSA} (pMCSA) that always achieves \(\mathcal{O}\left(\nicefrac{1}{N}\right)\) variance reduction.
Instead of using \(N\) \textit{sequential} Markov chain states, pMCSA operates \(N\) parallel Markov chains with.
At each step, pMCSA performs only a single Markov-chain transition for each chain.
This results in a per-SGD-iteration cost comparable to JSA.
We will later discuss the computational costs in detail.
See~\cref{alg:pmcsa} for a detailed pseudocode.


\begin{theoremEnd}{theorem}\label{thm:pmcsa}
  pMCSA, our proposed scheme, is obtained by setting
  {%\small
  \begin{align*}
    P_{\vlambda}^k\left(\veta, d\veta^{\prime}\right)
    = 
    K_{\vlambda}^k\left(\vz^{(1)}, d\vz^{\prime\; (1)}\right)
    \,
    K_{\vlambda}^k\left(\vz^{(2)}, d\vz^{\prime\; (2)}\right)
    \cdot
    \ldots 
    \cdot
    K_{\vlambda}^k\left(\vz^{(N)}, d\vz^{\prime\; (N)}\right)
  \end{align*}
  }
  with \(\veta = \left(\vz^{(1)}, \vz^{(2)}, \ldots, \vz^{(N)}\right)\).
  Then, given~\cref{thm:bounded_weight,thm:bounded_score}, the mixing rate and the gradient variance bounds are
  {%\small
  \begin{align*}
    \DTV{P_{\vlambda}^k\left(\veta, \cdot\right)}{\Pi}
    \leq
    C\left(N\right)\,{\rho^k}
    \quad\text{and}\quad
    \V{ \rvvg_{\mathrm{pMCSA}} \,\middle|\, \mathcal{F}_{t-1} }
    \leq
    L^2 \left[\; \frac{1}{N} + \frac{1}{N}\,\left(1 - \frac{1}{w^*}\right) \;\right],
  \end{align*}
  }
  where \(w^* = \sup_{\vz} \pi\left(\vz\right) / q_{\text{def.}}\left(\vz;\vlambda\right)\) and \(C\) is some positive constant depending on \(N\).
\end{theoremEnd}
\begin{proofEnd}

  Our proposed scheme, pMCSA, is described in~\cref{alg:jsa}. 
  At each iteration, our scheme performs a single MCMC transition for each of the \(N\) samples, or chains, to estimate the gradient.
  That is,
  \begin{align*}
    \rvvz^{(1)}_{t} &\mid \rvvz_{t-1}^{(1)},\, \rvvlambda_{t-1} \sim K_{\rvvlambda_{t-1}}\left(\rvvz_{t-1}^{(1)}, \cdot\right) \\
    \rvvz^{(2)}_{t} &\mid \rvvz_{t-1}^{(2)},\, \rvvlambda_{t-1}  \sim K_{\rvvlambda_{t-1}}\left(\rvvz_{t-1}^{(2)}, \cdot\right) \\
    &\qquad\qquad\vdots
    \\
    \rvvz^{(N)}_{t} &\mid \rvvz^{(N)}_{t-1},\, \rvvlambda_{t-1}  \sim K_{\rvvlambda_{t-1}}\left(\rvvz_{t-1}^{(N)}, \cdot\right)
    \\
    \rvvg_{\text{pMCSA}} &= -\frac{1}{N}\,\sum^{N}_{n=1} \, \vs\left(\vlambda, \rvvz^{(n)}_t\right),
  \end{align*}
  where \(K_{\rvvlambda_{t-1}}^n\) is an \(n\)-transition IMH kernel using \(q_{\text{def.}}\left(\cdot; \rvvlambda_{t-1}\right)\).

  \paragraph{Ergodicity of the Markov Chain}
  Since our kernel operates the same MCMC kernel \(K_{\vlambda}\) for each of the \(N\) parallel Markov chains, the \(n\)-step marginal kernel \(P_{\vlambda}\) can be represented as
  \begin{align*}
    P_{\vlambda}^k\left(\veta, d\veta^{\prime}\right)
    = 
    K_{\vlambda}^k\left(\vz^{(1)}, d\vz^{\prime\; (1)}\right)
    \,
    K_{\vlambda}^k\left(\vz^{(2)}, d\vz^{\prime\; (2)}\right)
    \cdot
    \ldots 
    \cdot
    K_{\vlambda}^k\left(\vz^{(N)}, d\vz^{\prime\; (N)}\right).
  \end{align*}
  Then, the convergence in total variation \(d_{\mathrm{TV}}\left(\cdot, \cdot\right)\) can be shown to decrease geometrically as
  \begin{alignat}{2}
    &\DTV{K^{k}_{\vlambda}\left(\veta, \cdot\right)}{\Pi}
    \nonumber
    \\
    &\quad=
    \sup_{A}
    \abs{
      \Pi\left(A\right)
      -
      P^{k}_{\vlambda}\left(\veta, A\right)
    }
    &&\quad\text{\textit{Definition of \(d_{\text{TV}}\)}}
    \nonumber
    \\
    &\quad\leq
    \sup_{A}
    \big|\;
    \int_{A}
      \pi\left(d\vz^{\prime}_1\right) \cdot \ldots \cdot \pi\left(d\vz^{\prime}_N\right)
    \nonumber
      \\
      &\qquad\qquad\qquad-
      K^k_{\vlambda}\left(\vz_1, d\vz^{\prime}_1\right) \cdot \ldots \cdot K^k_{\vlambda}\left(\vz_N, d\vz^{\prime}_N\right)
    \;\big|
    \nonumber
    \\
    &\quad\leq
    \sup_{A}
    \sum_{n=1}^N
    \abs{
    \int_{A}
      \pi\left(d\vz^{\prime}_k\right) - K^{k}_{\vlambda}\left(\vz_n, d\vz^{\prime}_n\right) 
    }
    &&\quad\text{\textit{\cref{thm:product_measure_bound}}}
    \nonumber
    \\
    &\quad=
    \sum_{n=1}^N
    \DTV{K^k_{\vlambda}\left(\vz_n, \cdot\right)}{\pi}
    &&\quad\text{\textit{Definition of \(d_{\text{TV}}\)}}
    \nonumber
    \\
    &\quad\leq
    \sum_{n=1}^N
    \rho^{k}
    &&\quad\text{\textit{Geometric ergodicity}}
    \nonumber
    \\
    &\quad=
    N\,\rho^{k}.
    \nonumber
  \end{alignat}

  \paragraph{\textbf{Bound on the Gradient Variance}}
  Let \(\rvvg\) be the sample mean of the score function with \(N\) samples from the exact posterior such that
  \begin{align*}
    &\rvvg = \frac{1}{N} \sum^{N}_{n=1} \vs\left(\vlambda; \rvvz^{(n)}\right),
    &\rvvz^{(n)} \sim \pi\left(\cdot\right).
  \end{align*}
  Then, the bound on the gradient variance can be derived through the decomposition
  {%\small
  \begin{alignat}{2}
    &\E{ \norm{\rvvg_{\mathrm{pMCSA}}}^2_{2} \,\middle|\, \mathcal{F}_{t-1} }
    \nonumber
    \\
    &\quad=
    \V{ \rvvg_{\mathrm{pMCSA}} \,\middle|\, \mathcal{F}_{t-1}  }
    +
    \norm{ \E{ \rvvg_{\mathrm{pMCSA}} \,\middle|\, \mathcal{F}_{t-1}  } }_2^2
    \nonumber
    \\
    &\quad=
    \V{ \rvvg_{\mathrm{pMCSA}} \,\middle|\, \mathcal{F}_{t-1}  }
    +
    \norm{
      \E{ \rvvg_{\mathrm{pMCSA}}  \;\,\middle|\, \mathcal{F}_{t-1}  } - \vmu  + \vmu
    }_2^2
    \nonumber
    \\
    &\quad=
    \V{ \rvvg_{\mathrm{pMCSA}} \,\middle|\, \mathcal{F}_{t-1}  }
    \nonumber
    \\
    &\quad\qquad+
    \norm{
      \E{ \rvvg_{\mathrm{pMCSA}}  \;\,\middle|\, \mathcal{F}_{t-1}  } - \vmu  
    }_2^2
    \nonumber
    +
    2 \, {\left(
      \E{ \rvvg_{\mathrm{pMCSA}}  \;\,\middle|\, \mathcal{F}_{t-1}  } - \vmu 
    \right)}^{\top}
    \vmu
    + 
    \norm{\vmu}^2
    &&\quad\text{\textit{Expanded quadratic}}
    \nonumber
    \\
    &\quad\leq
    \V{ \rvvg_{\mathrm{pMCSA}} \,\middle|\, \mathcal{F}_{t-1}  }
    \nonumber
    \\
    &\quad\qquad+
    \norm{
      \E{ \rvvg_{\mathrm{pMCSA}}  \;\,\middle|\, \mathcal{F}_{t-1}  } - \vmu 
    }_2^2
    +
    2 \,\norm{
      \E{ \rvvg_{\mathrm{pMCSA}}  \;\,\middle|\, \mathcal{F}_{t-1}  } - \vmu
    }_2
    \norm{\vmu}_2
    + 
    \norm{
      \vmu
    }_2^2
    \nonumber
    &&\quad\text{\textit{Cauchy-Shwarz inequality}}
    \\
    &\quad=
    \V{ \rvvg_{\mathrm{pMCSA}} \,\middle|\, \mathcal{F}_{t-1}  }
    +
    \mathsf{Bias}\left[\rvvg_{\text{pMCSA}}\right]^2
    +
    2 \,\mathsf{Bias}\left[\rvvg_{\text{pMCSA}}\right]
    \norm{\vmu}_2
    + 
    \norm{
      \vmu
    }_2^2
    \nonumber
    &&\quad\text{\textit{Definition of bias}}
    \\
    &\quad=
    \V{ \rvvg_{\mathrm{pMCSA}} \,\middle|\, \mathcal{F}_{t-1}  }
    +
    \mathsf{Bias}\left[\rvvg_{\text{pMCSA}}\right]\,
    \left(
    \mathsf{Bias}\left[\rvvg_{\text{pMCSA}}\right]
    +
    2\,\norm{\vmu}_2
    \right)
    + 
    \norm{
      \vmu
    }_2^2
    \nonumber
  \end{alignat}
  }

  The variance term is bounded as
  \begin{alignat}{2}
    &\V{ \frac{1}{N}\sum^{N}_{n=1} \vs\left(\vlambda; \rvvz^{(n)}\right) \,\middle|\, \rvvz_{t-1}^{(1:N)},\, \rvvlambda_{t-1} }
    \nonumber
    \\
    &\quad=
    \frac{1}{N^2}\sum^{N}_{n=1} \V{ \vs\left(\vlambda; \rvvz^{(n)}\right) \,\middle|\, \rvvz_{t-1}^{(1:N)},\, \rvvlambda_{t-1} }
    &&\quad\text{\textit{\(\rvvz^{(i)} \,\bot\, \rvvz^{(j)}\) for \(i \neq j\)}}
    \nonumber
    \\
    &\quad\leq
    \frac{1}{N^2}\sum^{N}_{n=1} \E{ \norm{\vs\left(\vlambda; \rvvz^{(n)}\right) }^2_2 \,\middle|\, \rvvz_{t-1}^{(1:N)},\, \rvvlambda_{t-1} }_2
    \nonumber
    \\
    &\quad=
    \frac{1}{N^2}\sum^{N}_{n=1} \Esub{\rvvz^{(n)} \sim K_{\rvvlambda_{t-1}}\left(\rvvz_{t-1}^{(n)}, \cdot\right)}{ \norm{\vs\left(\vlambda; \rvvz^{(n)}\right) }^2_2 \,\middle|\,  \rvvz_{t-1}^{(1:N)},\, \rvvlambda_{t-1} }
    \nonumber
    \\
    &\quad\leq
    \frac{1}{N^2}\sum^{N}_{n=1}
      \Esub{\rvvz^{(n)} \sim q_{\text{def.}}\left(\cdot;\rvvlambda_{t-1}\right)}{ \norm{\vs\left(\vlambda; \rvvz^{(n)} \right)}^2_2 }
      +
      {\rho} \, \norm{\vs\left(\vlambda; \rvvz_{t-1}^{(n)} \right)}^2_2
    &&\quad\text{\textit{\cref{thm:imh_expecation}}}
    \nonumber
    \\
    &\quad\leq
    \frac{1}{N^2}\sum^{N}_{n=1}
        L^2 + {\rho} \, L^2
    &&\quad\text{\textit{\cref{thm:bounded_score}}}
    \nonumber
    \\
    &\quad=
    \frac{L^2}{N^2} \sum^{N}_{n=1}
      1 + {\rho}
    \nonumber
    \\
    &\quad=
    L^2 \left[ \frac{1}{N} + \frac{1}{N}\,{\left(1 - \frac{1}{w^*}\right)} \right].
    \nonumber
  \end{alignat}
\end{proofEnd}

%%% Local Variables:
%%% TeX-master: "master"
%%% End:


%% This suggests that our proposed scheme achieves clear \(\mathcal{O}\left(1/N\right)\) variance reduction with a slower mixing rate.
%% The mixing rate independent bounds in \cref{table:convergence} suggest that this would provide faster convergence regardless.
%% Our empirical results in \cref{section:eval} show that this is indeed true in practice.

\vspace{-0.05in}
\paragraph{Theoretical Performance of MSC, JSA, and pMCSA}
By combining~\cref{thm:msc,thm:jsa,thm:pmcsa} with the convergence rates in~\cref{table:convergence}, we can compare the theoretical convergence of the considered algorithms.
For MSC, when \(w^*\) is large, the gradient variance and the mixing rate are worse than JSA and pMCSA and cannot be improved by increasing \(N\).
Therefore, we will not discuss it further.
On the other hand, JSA and pMCSA can be seen as trading-off bias (mixing rate) for variance.
However, when \(w^*\) is large, we can expect JSA to perform worse than pMCSA due to the constant \(1/2\).

\vspace{-0.05in}
\paragraph{Effect of increasing \(N\)}
If we consider the convergence rate of~\citet{duchi_ergodic_2012}, the mixing rate affects the convergence rate through the \(\log \rho^{-1}\) term.
For JSA, the combined rate appears to worsen as we increase \(N\), while for pMCSA, the combined rate stays constant with respect to \(N\).
In practice, however, we observe that increasing \(N\) accelerates convergence in general (quite dramatically for pMCSA).
Therefore, the mixing rate independent convergence rates by~\citet{doan_finitetime_2020, doan_convergence_2020} appears to better reflect practical performance.
This is especially true in MCSA since
\begin{enumerate*}[label=\textbf{(\roman*)}]
  \item the mixing rate \(\rho\) is a conservative \textit{global} bound and 
  \item the mixing rate will improve as MCSA converges.
\end{enumerate*}
Unfortunately, it is challenging to incorporate the changes in the mixing rate into the convergence rate.

%% \paragraph{Bias v.s. Variance}
%% While our proposed scheme achievs superior variance reduction, the mixing rate is worse.
%% In a MCMC estimation perspective, this translates into higher bias.
%% However, we note that
%% \begin{enumerate*}[label=\textbf{(\roman*)}]
%%   \item the constant \(C\left(\rho, N\right)\) depends on \(\rho\), 
%%   \item all of the ergodic convergence rate are close to 1 as \(w^* \rightarrow \infty\), and
%%   \item the mixing rate is a conservative global bound with respect to \(\vlambda\).
%% \end{enumerate*}
%% Therefore, in general, the superior ergodic convergence rate of JSA does not translate into faster convergence of MCSA.
%% In fact, as MCSA converges, \(w^*\) also decreases, dramatically improving the mixing rate.
%% In contrast, the relative variance does not improve too much with \(w^*\).
%% Therefore, reducing the variance is much more effective for accelerating convergence.
%% We empirically show this fact on the bias and variance in~\cref{section:simulation}.

\begin{wraptable}{r}{0.55\textwidth}
  \centering
  \vspace{-0.75in}
  
% Second version of table, with booktabs.
\begin{table*}
\centering
\caption{Computational Cost of Markov-chain Schemes}\label{table:cost}
\setlength{\tabcolsep}{2pt}
  \begin{threeparttable}
\begin{tabular}{lccccc}\toprule
& \multicolumn{3}{c}{\footnotesize Estimation} & \multicolumn{2}{c}{\footnotesize Stochastic gradient} \\
\cmidrule(lr){2-4}\cmidrule(lr){5-6}
  & \(\footnotesize p\left( \vz, \vx \right)\)
  & \(\footnotesize q_{\vlambda}(\vz)\)
  & \(\footnotesize q_{\vlambda}(\vz)\)
  & \(\footnotesize p\left( \vz, \vx \right)\)
  & \(\footnotesize q_{\vlambda}(\vz)\)
  \\
  & {\footnotesize\# Eval.  }
  & {\footnotesize\# Eval.  }
  & {\footnotesize\# Samples}
  & {\footnotesize\# Grad.  }
  & {\footnotesize\# Grad.  }
%
\\\midrule
%
{\footnotesize
ADVI
}
& \(0\)
& \(0\)
& \(N\)
& \(N\)
& \(0\)
\\\arrayrulecolor{black!30}\midrule
%
{\footnotesize
Single state estimator with CIS
}
& \(N-1\)
& \(N\)
& \(N-1\)
& \(0\)
& \(1\)\tnote{1}\;\;{\footnotesize or}\;\(N\)\tnote{2}
\\
%
{\footnotesize
Sequential state estimator with IMH
}
& \(N\)
& \(N\)
& \(N\)
& \(0\)
& \(N\)
\\
%
{\footnotesize
Parallel state estimator with IMH
}
& \(N\)
& \(N\)
& \(N\)
& \(0\)
& \(N\)
\\\bottomrule
\end{tabular}
  \begin{tablenotes}
    \item[*]{\footnotesize \(N\) is the number of samples used in each method}.
    \item[1]{\footnotesize Vanilla CIS kernel}.
    \item[2]{\footnotesize Rao-Blackwellized CIS kernel}.
  \end{tablenotes}
  \end{threeparttable}
\end{table*}

  \vspace{-0.2in}
\end{wraptable}
%
\subsection{Computational Cost}
The three schemes using the CIS kernel and the IMH kernel can have different computational costs depending on the parameter \(N\).
The computational costs of each scheme are organized in~\cref{table:cost}.

\vspace{-0.05in}
\paragraph{Cost of Sampling Proposals}
For the CIS kernel used by MSC, \(N\) controls the number of internal proposals sampled from \(q\,(\vz; \vlambda)\).
For JSA and pMCSA, the IMH kernel only uses a single sample from \(q\,(\vz; \vlambda)\), but applies the kernel \(N\) times.
Assuming caching is done as much as possible, pMCSA needs twice more evaluations of \(q\,(\vz; \vlambda)\) compared to other methods.
However, in general, this added cost should be minimal compared to the cost of evaluating \(p\,(\vz,\vx)\).

\vspace{-0.05in}
\paragraph{Cost of Estimating the Score}
When estimating the score, MSC computes \(\nabla_{\vlambda} \log q\,(\vz; \vlambda)\) only once, while JSA and our proposed scheme compute it \(N\) times.
However,~\cite{NEURIPS2020_b2070693} also discuss a Rao-Blackwellized version of the CIS kernel, which also computes the gradient \(N\) times.
Lastly, notice that MCSA methods do not need to differentiate through the likelihood, unlike ELBO maximization, making its per-iteration cost significantly cheaper.

%%% Local Variables:
%%% TeX-master: "master"
%%% End:
