
\vspace{-0.07in}
\section{Markov Chain Score Ascent}\label{section:mcsa}
\vspace{-0.08in}

First, we develop Markov chain score ascent (MCSA), a framework for inclusive KL minimization with MCGD.
This framework will establish the connection between MSC/JSA and MCGD.

\vspace{-0.08in}
\subsection{Markov Chain Score Ascent as a Special Case of Markov Chain Gradient Descent}\label{section:convergence}
\vspace{-0.07in}
As shown in~\cref{eq:mcgd}, the basic ingredients of MCGD are the target function \(f\left(\vlambda, \eta\right)\), the gradient estimator \(\vg\left(\vlambda, \eta\right)\), and the Markov chain kernel \(P_{\vlambda}\left(\eta, \cdot\right)\).
Obtaining MCSA from MCGD boils down to designing \(\vg\) and \(P_{\vlambda}\) such that \(f\left(\vlambda\right) = \DKL{\pi}{q\left(\cdot; \vlambda\right)} \).
The following proposition provides sufficient conditions on \(\vg\) and \(P_{\vlambda}\) to achieve this goal.

\vspace{0.07in}

\begin{theoremEnd}{proposition}\label{thm:product_kernel}
  Let \(\veta = \left( \vz^{(1)}, \vz^{(2)}, \ldots, \vz^{(N)} \right)\) and a Markov chain kernel \(P_{\vlambda}\left(\veta, \cdot\right)\) be \(\Pi\)-invariant where \(\Pi\) is defined as
  {%\small
  \[
  \Pi\left(\veta\right) = \pi\left(\vz^{(1)}\right) \, \pi\left(\vz^{(2)}\right) \times \ldots \times \pi\left(\vz^{(N)}\right).
  \]
  }
  Then, by defining the target function \(f\) and the gradient estimator \(\vg\) to be 
  {\small
  \begin{align*}
    f\left(\vlambda, \veta\right) =  \frac{1}{N} \sum^{N}_{n=1} \log q\left(\vz^{(n)}; \vlambda\right) + \mathbb{H}\left[\,\pi\,\right], 
    \quad\text{and}\quad
    \vg\left(\vlambda, \veta\right) =  \frac{1}{N} \sum^{N}_{n=1} \vs\left(\vz^{(n)}; \vlambda\right)
  \end{align*}
  }
  where \(\mathbb{H}\left[\,\pi\,\right]\) is the entropy of \(\pi\), MCGD results in inclusive KL minimization as
  {%\small
  \begin{align*}
    \Esub{\Pi}{ f\left(\vlambda, \rvveta\right) } = \DKL{\pi}{q\left(\cdot; \vlambda\right)},
    \quad\text{and}\quad
    \Esub{\Pi}{ \vg\left(\vlambda, \rvveta\right) } = \nabla_{\vlambda} \DKL{\pi}{q\left(\cdot; \vlambda\right)}.
  \end{align*}
  }
\end{theoremEnd}
\begin{proofEnd}
  For notational convenience, we define the shorthand
  \begin{align*}
    \pi\left(\vz^{(1:N)}\right) = \pi\left(\vz^{(1)}\right) \, \pi\left(\vz^{(2)}\right) \times \ldots \times \pi\left(\vz^{(N)}\right).
  \end{align*}
  Then,
  \begin{alignat}{2}
    &\Esub{\Pi}{ f\left(\vlambda, \veta\right) }
    \nonumber
    \\
    &\quad=
    \int \left( \frac{1}{N} \sum^{N}_{n=1} \log q\left(\vz^{(n)}; \vlambda\right) + \mathbb{H}\left[\,\pi\,\right]\right) \, \pi\left(\vz^{(1:N)}\right) \, d\vz^{(1:N)}
    \nonumber
    \\
    &\quad=
     \frac{1}{N} \sum^{N}_{n=1} \left\{ \int \big(\, \log q\,(\,\vz^{(n)}; \vlambda\,) + \mathbb{H}\,[\,\pi\,] \,\big) \, \pi\left(\vz^{(1:N)}\right) \, d\vz^{(1:N)} \right\}
     &&\quad{\text{\textit{Swapped integral and sum}}}
    \nonumber
    \\
    &\quad=
    \frac{1}{N} \sum^{N}_{n=1} \int \big(\, \log q\,(\,\vz^{(n)}; \vlambda\,) + \mathbb{H}\,[\,\pi\,] \,\big) \, \pi\left(\vz^{(n)}\right) \, d\vz^{(n)}
    &&\quad{\text{\textit{Marginalized \(\vz^{(m)}\) for all \(m \neq n\)}}}
    \nonumber
    \\
    &\quad=
    \frac{1}{N} \sum^{N}_{n=1} \DKL{\pi}{q\left(\cdot; \vlambda\right)}
    \nonumber
    &&\quad{\text{\textit{Definition of \(d_{\text{KL}}\)}}}
    \\
    &\quad=
    \DKL{\pi}{q\left(\cdot; \vlambda\right)}\label{eq:F_KL}
  \end{alignat}
  For \(\Esub{\Pi}{ \vg\left(\vlambda, \rvveta\right) }\), note that 
  \begin{align}
    \nabla_{\vlambda} f\left(\vlambda, \veta\right) = \frac{1}{N} \sum^{N}_{n=1} \vs\left(\vz^{(n)}; \vlambda\right) = \vg\left(\vlambda, \veta\right).\label{eq:F_grad_G}
  \end{align}
  Therefore, it suffices to show that
  \begin{align*}
    \nabla_{\vlambda} \DKL{\pi}{q\left(\cdot; \vlambda\right)}
    &=
    \nabla_{\vlambda} \Esub{\Pi}{ f\left(\vlambda, \rvveta\right) }
    &&\text{\textit{\cref{eq:F_KL}}}
    \\
    &=
    \Esub{\Pi}{ \nabla_{\vlambda}  f\left(\vlambda, \rvveta\right) }
    &&\text{\textit{{Leibniz derivative rule}}}
    \\
    &=
    \Esub{\Pi}{ \vg\left(\vlambda, \rvveta\right) }.
    &&\text{\textit{\cref{eq:F_grad_G}}}
  \end{align*}
\end{proofEnd}

%%% Local Variables:
%%% TeX-master: "master"
%%% End:


This simple connection between MCGD and VI paves the way toward the non-asymptotic analysis of JSA and MSC.
%We will later propose a third novel scheme that conforms to~\cref{thm:product_kernel}.
Note that \(N\) here can be regarded as the computational budget of each MCGD iteration since the cost of
\begin{enumerate*}[label=\textbf{(\roman*)}]
  \item generating the Markov chain samples \(\vz^{(1)}, \ldots, \vz^{(N)}\) and
  \item computing the gradient \(\vg\)
\end{enumerate*}
will linearly increase with \(N\).

In addition, the MCGD framework often assumes \(P\) to be geometrically ergodic.
An exception is the analysis of~\citet{debavelaere_convergence_2021} where they work with polynomially ergodic kernels.
\begin{assumption}{(Markov chain kernel)}\label{thm:kernel_conditions}
\vspace{-0.05in}
  The Markov chain kernel \(P\) is geometrically ergodic as
  {%\small
  \[
  \DTV{P_{\vlambda}^{n}\left(\veta, \cdot\right)}{ \Pi } \leq C \, \rho^{n}
  \]
  }
  for some positive constant \(C\).
\end{assumption}

\vspace{-0.07in}
\subsection{Non-Asymptotic Convergence of Markov Chain Score Ascent}%\label{}
\vspace{-0.05in}
\paragraph{Non-Asymptotic Convergence}
Through~\cref{thm:product_kernel,thm:kernel_conditions} and some technical assumptions on the objective function, we can apply the existing convergence results of MCGD to MCSA.
\cref{table:convergence} provides a list of relevant results.
Apart from properties of the objective function (such as Lipschitz smoothness), the convergence rates are stated in terms of the gradient bound \(G\), kernel mixing rate \(\rho\), and the number of MCGD iteration \(T\).
We focus on \(G\) and \(\rho\) as they are closely related to the design choices of different MCSA algorithms.

\vspace{-0.1in}
\paragraph{Convergence and the Mixing Rate \(\rho\)}
\citet{duchi_ergodic_2012} was the first to provide an analysis of the general MCGD setting.
Their convergence rate is dependent on the mixing rate through the \(1 / \log \rho^{-1}\) term.
For MCSA, this results in an overly conservative rate since, on challenging problems, mixing can be slow such that \(\rho \approx 1\).
Fortunately, \citet{doan_convergence_2020,doan_finitetime_2020} have recently shown that it is possible to obtain a rate independent of the mixing rate \(\rho\).
For example, in the result of~\citet{doan_finitetime_2020}, the influence of \(\rho\) decreases in a rate of \(\mathcal{O}\left(\nicefrac{1}{T^2}\right)\).
%The fact that the convergence rate can be independent of the mixing rate is critical.
This observation is critical since it implies that \textit{trading ``lower gradient variance'' with a ``slower mixing rate'' could be profitable}.
We exploit this observation in our novel MCSA scheme in~\cref{section:pmcsa}.

\begin{table*}
\vspace{-0.25in}
\centering
\caption{Convergence Rates of MCGD Algorithms}\label{table:convergence}
\vspace{-0.05in}
\setlength{\tabcolsep}{3pt}
\begin{threeparttable}
  \begin{tabular}{lllcc}\toprule
    \multicolumn{1}{c}{\footnotesize\textbf{Algorithm}} & \multicolumn{1}{c}{\footnotesize\textbf{Stepsize Rule}} & \multicolumn{1}{c}{\footnotesize\textbf{Gradient Assumption}} & {\footnotesize\textbf{Rate}} & {\footnotesize\textbf{Reference}} \\\midrule
    \multirow{2}{*}{\small Mirror Descent\tnote{1}}
    & \multirow{2}{*}{\small\(\gamma_t = \gamma / \sqrt{t}\)}
    & \multirow{2}{*}{\small\(\E{ {\|\, \vg\left(\vlambda, \veta\right) \,\|}_*^2 \mid \mathcal{F}_t } < G^2\)}
    & \multirow{2}{*}{\small\(\mathcal{O}\left(\frac{G^2 \log T}{ \log \rho^{-1} \sqrt{T}}\right)\)}
    & {\footnotesize\citet{duchi_ergodic_2012}}
    \\
    &&&& {\footnotesize{Corollary 3.5}}
    \\\cdashlinelr{1-5}
    \multirow{2}{*}{\small SGD-Nesterov\tnote{2}}
    & {\small\(\gamma_t = 2/(t + 1)\)}
    & \multirow{2}{*}{\footnotesize\( {\|\vg\left(\vlambda, \veta\right)\|}_2 < G \)}
    & \multirow{2}{*}{\small\(\mathcal{O}\left(\frac{G^2 \log T}{ \sqrt{T}}\right)\)}
    & {\footnotesize\citet{doan_convergence_2020}}
    \\
    & {\footnotesize\(\beta_t = \frac{1}{2 \, L \sqrt{t + 1}}\)}
    &&& {\footnotesize{Theorem 2}}
    \\\cdashlinelr{1-5}
    \multirow{2}{*}{\small SGD\tnote{3}}
    & {\footnotesize\(\gamma_t = \gamma/t\)}
    & \multirow{2}{*}{\footnotesize\( {\|\,\vg\left(\vlambda, \veta\right)\|}_* < G \left( \norm{\vlambda}_2 + 1 \right) \)}
    & \multirow{2}{*}{\small\(\mathcal{O}\left(\frac{G^2 \log T}{ T}\right)\)}
    & {\footnotesize\citet{doan_finitetime_2020}}
    \\ 
    & {\footnotesize\(\gamma = \min\{\nicefrac{1}{2\,L}, \nicefrac{2 L}{\mu}\}\)}
    &&& {\footnotesize{Theorem 1,2}}
    \\ \bottomrule
  \end{tabular}
  \begin{tablenotes}[flushleft]
  \item[]{%
    \footnotesize\textit{\textbf{Notation}}: \(^1\)\(\mathcal{F}_t\) is the \(\sigma\)-field formed by all the iterates \(\veta_t\), \(\vlambda_t\) up to the \(t\)th MCGD iteration and \(\norm{\vx}_*\) is the dual norm such that \(\norm{\vx}_* = \sup_{\norm{\vz} \leq 1} \iprod{\vx}{\vz}\).
    \(^2\)\(\beta_t\) is the stepsize of the momentum.
    \(^2\)\(^3\)\(L\) is the Lipschitz smoothness constant.
    \(^3\)\(\mu\) is the strong convexity constant.
  }
  \end{tablenotes}
\end{threeparttable}
\vspace{-0.2in}
\end{table*}

\vspace{-0.1in}
\paragraph{Gradient Bound \(G\)}
Except for \citet{doan_finitetime_2020}, most results assume that the gradient is bounded for \(\forall\veta,\vlambda\) as {\footnotesize\( {\| \vg\left(\vlambda, \veta\right) \|} < G \)}.
Admittedly, this condition is strong, but it is similar to the bounded variance assumption {\footnotesize\(\mathbb{E}\,[\norm{\vg}^2]  < G^2\)} used in vanilla SGD, which is also known to be strong as it contradicts strong convexity~\citep{pmlr-v80-nguyen18c}.
Nonetheless, assuming \(G\) can have practical benefits beyond theoretical settings.
For example,~\citet{pmlr-v108-geffner20a} use \(G\) to compare the performance different VI gradient estimators.
In a similar spirit, we will obtain the gradient bound \(G\) of different MCSA algorithms and compare their theoretical performance.

\vspace{-0.07in}
\section{Demystifying Prior Markov Chain Score Ascent Methods}\label{section:comparison}
\vspace{-0.05in}
In this section, we will show that MSC and JSA both qualify as MCSA methods.
Furthermore, we establish 
\begin{enumerate*}[label=\textbf{(\roman*)}]
  \item the mixing rate of their implicitly defined kernel \(P\) and
  \item the upper bound on their gradient variance.
\end{enumerate*}
This will provide insight into their practical non-asymptotic performance.

\vspace{-0.07in}
\subsection{Technical Assumptions}
\vspace{-0.05in}
To cast previous methods into MCSA, we need some technical assumptions.
\begin{assumption}{(Bounded importance weight)}\label{thm:bounded_weight}
  The importance weight ratio \(w\left(\vz\right) = \pi\left(\vz\right) / q\left(\vz; \vlambda\right)\) is bounded by some finite constant as \(w^* < \infty\) for all \(\vlambda \in \Lambda\) such that \(\rho = \left(1 - 1/w^*\right) < 1\).
\end{assumption}
\vspace{-0.05in}
Note that \(w^*\) is bounded below exponentially by the inclusive KL as shown in~\cref{thm:wstar}.
Therefore, \(w^*\) will be large 
\begin{enumerate*}[label=\textbf{(\roman*)}]
    \item in the initial steps of VI and
    \item under model (variational family) misspecification.
\end{enumerate*}
Furthermore, \cref{thm:bounded_weight} is necessary to ensure~\cref{thm:kernel_conditions}.
Practically, this can be done by using a variational family with heavy tails~\citep{NEURIPS2018_25db67c5} or using a defensive mixture~\citep{hesterberg_weighted_1995, holden_adaptive_2009} as
\begin{align*}
  q_{\text{def.}}\left(\vz; \vlambda \right) = w \, q\left(\vz; \vlambda\right) + (1 - w) \, \nu\left(\vz\right)
\end{align*}
where \(0 < w < 1\) and \(\nu\left(\cdot\right)\) is a heavy tailed distribution such that \(\sup_{\vz \in \mathcal{Z}} \pi\left(\vz\right) / \nu\left(\vz\right) < \infty\).
Note that \(q_{\text{def.}}\) is only used in the Markov chain kernels and \(q\left(\cdot;\vlambda^*\right)\) is still he output of the VI procedure.
While these tricks help escape slowly mixing regions, this benefit quickly vanishes as we converge.
Therefore, ensuring \cref{thm:bounded_weight} seems unnecessary in practice unless we absolutely care about ergodicity (\textit{e.g.} adaptive MCMC,~\citealt{holden_adaptive_2009, pmlr-v151-brofos22a}).

\begin{assumption}{(Bounded Score)}\label{thm:bounded_score}
  The score gradient is bounded for \(\forall \vlambda \in \Lambda\) and \(\forall \vz \in \mathcal{Z}\) such that \(\norm{\vs\left(\vlambda; \vz\right)}_* \leq L \) for some finite constant \(L > 0\).
\end{assumption}
\vspace{-0.05in}
Although this assumption is strong, it enables us to compare the gradient variance of MCSA methods.
We empirically justify the bounds obtained using~\cref{thm:bounded_score} in \cref{section:simulation}.

\vspace{-0.07in}
\subsection{Markovian Score Climbing}
\vspace{-0.07in}
MSC (\cref{alg:msc}) is a simple instance of MCSA where \(\eta_t = \vz_t\) and \(P_{\vlambda_t} = K_{\vlambda_t}\), where \(K_{\vlambda_t}\) is the conditional importance sampling (CIS) kernel (originally proposed by~\citet{andrieu_uniform_2018}) where the proposals are generated from \(q\left(\cdot; \vlambda_t\right)\).
Although MSC uses only a single sample for the Markov chain, the CIS kernel internally uses \(N\) proposals to generate a single sample.
Therefore, \(N\) in MSC has a different meaning, but it still indicates the computational budget.


\begin{theoremEnd}[all end]{lemma}\label{thm:product_measure_bound}
  For the probability measures \(p_1, \ldots, p_N\) and \(q_1, \ldots, q_N\) defined on a measurable space \((\mathsf{X}, \mathcal{A})\) and an arbitrary set \(A \in \mathcal{A}\),
  \begin{align*}
    &\abs{
    \int_{A^N}
    p_1\left(dx_1\right)
    p_2\left(dx_2\right)
    \times
    \ldots
    \times
    p_N\left(dx_N\right)
    -
    q_1\left(dx_1\right)
    q_2\left(dx_2\right)
    \times
    \ldots
    \times
    q_N\left(dx_N\right)
  }
    \\
  &\qquad\leq
  \sum_{n=1}^N
  \abs{
    \int_{A}
    p_n\left(dx_n\right)
    -
    q_n\left(dx_n\right)
  }
  \end{align*}
\end{theoremEnd}
\begin{proofEnd}
  By using the following shorthand notations
  \begin{alignat*}{2}
    p_{(1:N)}\left(dx_{(1:N)}\right)
    &= 
    p_1\left(dx_1\right)
    p_2\left(dx_2\right)
    \times
    \ldots
    \times
    p_N\left(dx_N\right)
    \\
    q_{(1:N)}\left(dx_{(1:N)}\right)
    &= 
    q_1\left(dx_1\right)
    q_2\left(dx_2\right)
    \times
    \ldots
    \times
    q_N\left(dx_N\right),
  \end{alignat*}
  the result follows from induction as
  \begin{alignat}{2}
    &\abs{
      \int_{A^N}
      p_{(1:N)}\left(dx_{(1:N)}\right)
      -
      q_{(1:N)}\left(dx_{(1:N)}\right)
    }
    \nonumber
    \\
    &\quad=
    \Bigg|\;
    \left( \int_{A} p_1\left(dx_1\right) - q_1\left(dx_1\right) \right) \,
    \int_{A^{N-1}} p_{(2:N)}\left(dx_{(2:N)}\right)
    \nonumber
    \\
    &\qquad\quad+
    \int_{A} q_1\left(dx_1\right) \,
    {\left(
      \int_{A^{N-1}}
      p_{(2:N)}\left(dx_{(2:N)}\right)
      -
      q_{(2:N)}\left(dx_{(2:N)}\right)
    \right)}
    \;\Bigg|
    \nonumber
    \\
    &\quad\leq
    \Bigg|
    \int_{A} p_1\left(dx_1\right) - q_1\left(dx_1\right)
    \Bigg|\;
    \int_{A^{N-1}} p_{(2:N)}\left(dx_{(2:N)}\right)
    \nonumber
    \\
    &\qquad\quad+
    \int_{A} q_1\left(dx_1\right) \,
    {
    \Bigg|\;
      \int_{A^{N-1}}
      p_{(2:N)}\left(dx_{(2:N)}\right)
      -
      q_{(2:N)}\left(dx_{(2:N)}\right)
    }
    \;\Bigg|
    &&\quad\text{\textit{Triangle inequality}}
    \nonumber
    \\
    &\quad\leq
    \Bigg|
    \int_{A} p_1\left(dx_1\right) - q_1\left(dx_1\right)
    \Bigg|\;
    \nonumber
    \\
    &\qquad\quad+
    {
    \Bigg|\;
      \int_{A^{N-1}}
      p_{(2:N)}\left(dx_{(2:N)}\right)
      -
      q_{(2:N)}\left(dx_{(2:N)}\right)
    }
    \;\Bigg|.
    &&\quad\text{\textit{Applied \(p_n\left(A\right), q_n\left(A\right) \leq 1 \)}}
    \nonumber
  \end{alignat}
\end{proofEnd}


\begin{theoremEnd}{theorem}\label{thm:msc}
  MSC~\citep{NEURIPS2020_b2070693} is obtained by defining 
  {%\small
  \begin{align*}
  P_{\lambda}^k\left(\veta, d\veta^{\prime}\right)
  = 
  K_{\lambda}^k\left(\vz, d\vz^{\prime}\right)
  \end{align*}
  }
  with  \(\veta_t = \vz_t\) where \(K_{\vlambda}\left(\vz, \cdot\right)\) is the CIS kernel with \(q_{\text{def.}}\left(\cdot; \vlambda\right)\) as its proposal distribution.
  Then, given~\cref{thm:bounded_weight,thm:bounded_score}, the mixing rate and the gradient bounds are given as
  {%\small
  \begin{align*}
    \textstyle
  \DTV{P_{\vlambda}^k\left(\veta, \cdot\right)}{\Pi} \leq  {\left(1 - \frac{N - 1}{2 w^* + N - 2}\right)}^k\quad \text{and}\quad
  {\small
  \E{ \norm{ \vg\left(\vlambda, \rvveta\right) }_{*}^2 \,\middle|\, \mathcal{F}_{t} } \leq  L^2,
  }
  \end{align*}
  }
  where \(w^* = \sup_{\vz} \pi\left(\vz\right) / q_{\text{def.}}\left(\vz;\vlambda\right)\).
\end{theoremEnd}
\begin{proofEnd}
  MSC is described in~\cref{alg:msc}. 
  At each iteration, it performs a single MCMC transition with the CIS kernel where it internally uses \(N\) proposals.

  \paragraph{Ergodicity of the Markov Chain}
  The ergodic convergence rate of \(P_{\vlambda}\) is equal to that of \(K_{\vlambda}\), the CIS kernel proposed by~\citet{NEURIPS2020_b2070693}. 
  Although not mentioned by~\citet{NEURIPS2020_b2070693}, this kernel has been previously proposed as the iterated sequential importance resampling (i-SIR) by \citet{andrieu_uniform_2018} with its corresponding geometric convergence rate as
  \begin{alignat*}{2}
    \DTV{P^{k}_{\vlambda}\left(\veta, \cdot\right)}{\Pi}
    =
    \DTV{K^{k}_{\vlambda}\left(\vz, \cdot\right)}{\pi}
    \leq
    {\left(1 - \frac{N - 1}{2 w^* + N - 2}\right)}^k.
  \end{alignat*}

  \paragraph{\textbf{Bound on the Gradient Variance}}
  The bound on the gradient variance is straightforward given \cref{thm:bounded_score}.
  For simplicity, we denote the rejection state as \(\vz^{(1)} = \vz_{t-1} \).
  Then,
  \begin{alignat}{2}
    &\E{ \norm{ \vg\left(\vlambda, \rvveta\right) }_{*}^2 \,\middle|\, \mathcal{F}_{t} }
    \nonumber
    \\
    &\;=
    \E{ \norm{ \vg\left(\vlambda, \rvveta\right) }_{*}^2 \,\middle|\, \mathcal{F}_{t}}
    \nonumber
    \\
    &\;=
    \Esub{\rvvz \sim K_{\vlambda_{t-1}}\left(\vz_{t-1}, \cdot\right)}{
      \norm{ \vs\left(\vlambda; \rvvz\right) }_{*}^2 \,\middle|\,
      \vlambda_{t-1}, \vz_{t-1}
    }
    \nonumber
    \\
    &\;=
    \int
    \sum^{N}_{n=1}
    \frac{
      w\left(\vz^{(n)}\right)
    }{
      \sum^{N}_{m=1} w\left(\vz^{(m)}\right)
    }
    \norm{ \vs\left(\cdot; \vz^{(n)}\right) }^2_{*}
    \prod^{N}_{n=2}
    q\left(d\vz^{(n)}; \vlambda_{t-1}\right)
    \nonumber
    &&\quad\text{\textit{\citet{andrieu_uniform_2018}}}
    \\
    &\;\leq
    L^2 \,
    \int
    \sum^{N}_{n=1}
    \frac{
      w\left(\vz^{(n)}\right)
    }{
      \sum^{N}_{m=1} w\left(\vz^{(m)}\right)
    }
    \prod^{N}_{n=2}
    q\left(d\vz^{(n)}; \vlambda_{t-1}\right)
    \nonumber
    &&\quad\text{\textit{\cref{thm:bounded_score}}}
    \\
    &\;=
    L^2 \,
    \int
    \prod^{N}_{n=2}
    q\left(d\vz^{(n)}; \vlambda_{t-1}\right)
    \nonumber
   &&\quad\text{\textit{The sum of weights is 1}}
    \\
    &\;=
    L^2.
    \nonumber
  \end{alignat}

  %% This form coincides with solving the expectation of the self-normalized importance sampling estimator, which is well known to be challenging~\citep{robert_monte_2004}.
  %% We instead approximate the expectation by defining the random variables
  %% \(
  %% X = \sum^{N}_{n=1} w\left(\vz_n\right) \norm{ s\left(\cdot; \vz_n\right) }^2_{*}
  %% \)
  %% and
  %% \(
  %% Y = \sum^{N}_{m=1} w\left(\vz_m\right)
  %% \),
  %% use the delta method as
  %% \begin{alignat*}{2}
  %%   \E{\frac{\rvX}{\rvY} \;\middle|\; Z} &\approx \frac{\E{\rvX \mid Z }}{\E{\rvY \mid Z}} + \mathcal{O}\left(\frac{1}{{\E{\rvX \mid Z}}^2}\right).
  %% \end{alignat*}
  %% The required expectations are obtained as
  %% \begin{alignat}{2}
  %%   &\E{\rvX \mid \lambda_{t-1}, \vz_{t-1}}
  %%   \\
  %%   &\;=
  %%   \int \sum^{N}_{n=1} w\left(\vz_n\right) \norm{ s\left(\cdot; \vz_n\right) }^2_{*}
  %%   \prod^{N}_{n=2}
  %%   q\left(d\rvvz_n\right)
  %%   \nonumber
  %%   \\
  %%   &\quad=
  %%   \sum^{N}_{n=1} \int w\left(\vz_n\right) \norm{ s\left(\cdot; \vz_n\right) }^2_{*}
  %%   \prod^{N}_{n=2}
  %%   q\left(d\rvvz_n\right)
  %%   &&\quad\text{\textit{Swapped integral and sum}}
  %%   \nonumber
  %%   \\
  %%   &\quad=
  %%   \left\{\;
  %%   \sum^{N}_{n=2} \int w\left(\vz_n\right) \norm{ s\left(\cdot; \vz_n\right) }^2_{*}
  %%   \prod^{N}_{n=2} q\left(d\rvvz_n\right)
  %%   \right\}
  %%   +
  %%   w\left(\vz_1\right) \norm{ s\left(\cdot; \vz_1\right) }^2_{*}
  %%   &&\quad\text{\textit{Pulled out rejection state}}
  %%   \nonumber
  %%   \\
  %%   &\quad=
  %%   \left\{\;
  %%   \sum^{N}_{n=2} \int \frac{\pi\left(\vz_n\right)}{q\left(\vz_n\right)} \norm{ s\left(\cdot; \vz_n\right) }^2_{*}
  %%   \prod^{N}_{n=2}
  %%   q\left(d\rvvz_n\right)
  %%   \right\}
  %%   +
  %%   w\left(\vz_1\right) \norm{ s\left(\cdot; \vz_1\right) }^2_{*}
  %%   &&\quad\text{\textit{Definition of \(w\left(\vz\right)\)}}
  %%   \nonumber
  %%   \\
  %%   &\quad=
  %%   \sum^{N}_{n=2} \int \pi\left(d\vz_n\right) \norm{ s\left(\cdot; \vz_n\right) }^2_{*}
  %%   +
  %%   w\left(\vz_1\right) \norm{ s\left(\cdot; \vz_1\right) }^2_{*}
  %%   &&\quad\text{\textit{Cancelled out \(q\left(\cdot\right)\)}}
  %%   \nonumber
  %%   \\
  %%   &\quad=
  %%   \left(N-1\right) \, \Esub{\pi}{\norm{ s\left(\cdot; \rvvz\right) }^2_{*} }
  %%   +
  %%   w\left(\vz_1\right) \norm{ s\left(\cdot; \vz_1\right) }^2_{*}
  %%   \nonumber
  %%   \\
  %%   &\quad=
  %%   \left(N-1\right) \, \Esub{\pi}{\norm{ s\left(\cdot; \rvvz\right) }^2_{*} }
  %%   +
  %%   w\left(\vz_{t-1}\right) \norm{ s\left(\cdot; \vz_{t-1}\right) }^2_{*}
  %%   \nonumber
  %% \end{alignat}
  %% and similarly,
  %% \begin{alignat}{2}
  %%   \E{\rvY \mid \lambda_{t-1}, \vz_{t-1}}
  %%   &\quad= 
  %%   \int \sum^{N}_{n=1} w\left(\vz_n\right) \prod^{N}_{n=2} q\left(d\vz_n\right) 
  %%   \nonumber
  %%   \\
  %%   &\quad= 
  %%   \left\{\; \int \sum^{N}_{n=2} w\left(\vz_n\right) \prod^{N}_{n=2} q\left(d\vz_n\right) \;\right\} + w\left(\vz_1\right)
  %%   &&\quad\text{\textit{Pulled out rejection state}}
  %%   \nonumber
  %%   \\
  %%   &\quad= 
  %%   \left\{\; \sum^{N}_{n=2} \int w\left(\vz_n\right) \prod^{N}_{n=2} q\left(d\vz_n\right) \;\right\} + w\left(\vz_1\right)
  %%   &&\quad\text{\textit{Swapped integral and sum}}
  %%   \nonumber
  %%   \\
  %%   &\quad= 
  %%   \left\{\; \sum^{N}_{n=2} \int \frac{\pi\left(\vz_n\right)}{q\left(\vz_n\right)} \prod^{N}_{n=2} q\left(d\vz_n\right) \;\right\} + w\left(\vz_1\right)
  %%   &&\quad\text{\textit{Definition of \(w\left(\vz\right)\)}}
  %%   \nonumber
  %%   \\
  %%   &\quad= 
  %%   \left\{\; \sum^{N}_{n=2} \int \pi\left(d\vz_n\right) \;\right\} + w\left(\vz_1\right)
  %%   &&\quad\text{\textit{Cancelled out \(q\left(\cdot\right)\)}}
  %%   \nonumber
  %%   \\
  %%   &\quad= 
  %%   N - 1 + w\left(\vz_{t-1}\right)
  %%   \nonumber
  %% \end{alignat}
  %% Therefore, 
  %% \begin{alignat}{2}
  %%   \E{\frac{\rvX}{\rvY} \;\middle|\; Z}
  %%   &\approx
  %%   \frac{
  %%     \left(N-1\right) \, \Esub{\pi}{\norm{ s\left(\cdot; \rvvz\right) }^2_{*} }
  %%     +
  %%     w\left(\vz_1\right) \norm{ s\left(\cdot; \vz_1\right) }^2_{*}
  %%   }{
  %%     N-1 + w\left(\vz_1\right)
  %%   }
  %%   + \mathcal{O}\left(\frac{1}{{\left(N-1\right)}^2}\right)
  %%   \nonumber
  %%   \\
  %%   &\leq
  %%   \frac{
  %%     \left(N-1\right) \, \Esub{\pi}{\norm{ s\left(\cdot; \rvvz\right) }^2_{*} }
  %%     +
  %%     w^* \norm{ s\left(\cdot; \vz_1\right) }^2_{*}
  %%   }{
  %%     N-1
  %%   }
  %%   + \mathcal{O}\left(\frac{1}{{\left(N-1\right)}^2}\right)
  %%   &&\quad\text{\textit{\(w\left(\cdot\right) \leq w^*\)}}
  %%   \nonumber
  %%   \\
  %%   &\leq
  %%   \frac{
  %%     \left(N-1\right) \, L^2
  %%     +
  %%     w^* L^2
  %%   }{
  %%     N-1
  %%   }
  %%   + \mathcal{O}\left(\frac{1}{{\left(N-1\right)}^2}\right)
  %%   \nonumber
  %%   \\
  %%   &=
  %%   L^2
  %%   \left[
  %%   1
  %%   +
  %%   \frac{
  %%     w^*
  %%   }{
  %%     N-1
  %%   }
  %%   \right]
  %%   + \mathcal{O}\left(\frac{1}{{\left(N-1\right)}^2}\right)
  %%   \nonumber
  %% \end{alignat}
\end{proofEnd}

%%% Local Variables:
%%% TeX-master: "master"
%%% End:


\vspace{-0.1in}
\paragraph{Discussion}
\cref{thm:msc} shows that the gradient variance of MSA is insensitive to \(N\).
Although the mixing rate does improve with \(N\), when \(w^*\) is large, this will be marginal.
Overall, \textit{the performance of MSC cannot be improved by increasing the computational budget \(N\)}.
Even under stationarity, the variance of \(\veta\) is equal to that of a \textit{single} posterior sample at best.

\vspace{-0.07in}
\subsection{Joint Stochastic Approximation}
\vspace{-0.07in}
JSA (\cref{alg:jsa}) was proposed for deep generative models where the likelihood factorizes into each datapoint, for which subsampling can be used through a random-scan version of the independent Metropolis-Hastings (IMH, \citealt{hastings_monte_1970}) kernel.
Instead, we consider the general version of JSA with the vanilla IMH kernel since it can be used with any type of likelihood.
At each MCGD step, JSA performs multiple Markov chain transitions and estimates the gradient by averaging all the intermediate states, which is closer to traditional MCMC estimation.

\vspace{-0.1in}
\paragraph{Independent Metropolis-Hastings}
%A key element of JSA~\citep{pmlr-v124-ou20a} is that it uses the IMH kernel.
Similarly to MSC, the IMH kernel in JSA generates proposals from \(q\left(\cdot; \vlambda_t\right)\).
To show the geometric ergodicity of the implicit kernel \(P\), we utilize the geometric convergence rate of IMH kernels provided by~\citet[Theorem 2.1]{10.2307/2242610} and~\citet{wang_exact_2020}.
Furthermore, to derive an upper bound on the gradient variance, we use the exact \(n\)-step marginal IMH kernel derived by~\citet{Smith96exacttransition} as
{\small
  \begin{align}
  K^n_{\vlambda}\left(\vz, d\vz\prime\right) 
  = T_n\left(\, w\left(\vz\right) \vee w\left(\vz\prime\right)\,\right) \, \pi\left(\vz\prime\right) \, d\vz\prime
  + \lambda^n\left(w\left(\vz\right)\right) \, \delta_{\vz}\left(d\vz\prime\right)
  \label{eq:imh_exact_kernel}
  \end{align}
}%
where {\(w\left(\vz\right) = \pi\left(\vz\right)/q_{\text{def.}}\left(\vz; \vlambda\right)\), \(x \vee y = \max\left(x, y\right)\)}, and for {\(R\left(w\right) = \{\, \vz\prime \mid w\,\left(\vz\prime\right) \leq w \,\}\)}, 
{\small
  \begin{align}
    T_n\left(w\right)      = \int_w^{\infty}
    \frac{n}{v^2}
    %\left(n / v^2\right)
    \, \lambda^{n-1}\left(v\right)\,dv,
    \quad\text{and}\quad
    \lambda\left(w\right) =
    \int_{R\left(w\right)}
    \left( 1 - \frac{w\left(\vz\prime\right)}{w}  \right)
    %\left( 1 - w\left(\vz\prime\right)/w  \right)
    \pi\left(d\vz\prime\right).\label{eq:T_lambda}
  \end{align}
}%

\begin{theoremEnd}[all end]{lemma}\label{thm:lambda_bound}
  For \(w^* = \sup_{\vz} w\left(\vz\right) \), \(\lambda\left(\cdot\right)\) in~\cref{eq:T_lambda} is bounded as
  \[
   \max\left(1 - \frac{1}{w}, 0\right) \leq \lambda\left(w\right) \leq 1 - \frac{1}{w^*}.
  \]
\end{theoremEnd}
\begin{proofEnd}
  The proof can be found in the proof of Theorem 3 of \citet{Smith96exacttransition}.
\end{proofEnd}

\begin{theoremEnd}[all end]{lemma}\label{thm:tn_bound}
  For \(w^* = \sup_{\vz} w\left(\vz\right) \), \(T_n\left(\cdot\right)\) in~\cref{eq:T_lambda} is bounded as
  \[
  T_n\left( w \right) \leq \frac{n}{w} \, {\left(1 - \frac{1}{w^*}\right)}^{n-1}.
  \]
\end{theoremEnd}
\begin{proofEnd}
  \begin{alignat*}{2}
    T_n\left(w\right) 
    &= \int_w^{\infty} \frac{n}{v^2} \, \lambda^{n-1}\left(v\right)\,dv
    &&\quad\text{\textit{\cref{eq:T_lambda}}}
    \\
    &\leq \int_w^{\infty} \frac{n}{v^2} \, {\left(1 - \frac{1}{w^*}\right)}^{n-1}\,dv
    &&\quad\text{\textit{\cref{thm:lambda_bound}}}
    \\
    &= n \, {\left(1 - \frac{1}{w^*}\right)}^{n-1}  \int_w^{\infty} \frac{1}{v^2} \,dv
    &&\quad\text{\textit{Pulled out constant}}
    \\
    &= n \, {\left(1 - \frac{1}{w^*}\right)}^{n-1}  \left( {-\left.\frac{1}{v}\right\rvert^{\infty}_{w}} \right)
    &&\quad\text{\textit{Solved indefinite integral}}
    \\
    &= \frac{n}{w} \, {\left(1 - \frac{1}{w^*}\right)}^{n-1}.
  \end{alignat*}
  This upper bound is in general difficult to improve unless we impose stronger assumptions on \(\pi\) and \(q\).
\end{proofEnd}

\begin{theoremEnd}[all end]{lemma}\label{thm:imh_expecation}
  For a positive test function \(f : \mathcal{Z} \rightarrow \mathbb{R}^{+}\), the estimate of a \(\pi\)-invariant independent Metropolis-Hastings kernel with a proposal \(q\) is bounded as
  \begin{align*}
    \Esub{K^n\left(\vz, \cdot\right)}{ f \,\middle|\, \rvvz }
    \leq
    n \, \rho^{n-1} 
    \Esub{q}{f}
    +
    {\rho}^n \, f\left(\rvvz\right)
    %% \leq
    %% n \, \left(
    %% \Esub{q}{f}
    %% +
    %% \frac{1}{n} \, f\left(\vz\right)
    %% \right) 
  \end{align*}
  where \(w\left(\vz\right) = \pi\left(\vz\right) / q\left(\vz\right)\) and \(\rho = 1 - \nicefrac{1}{w^*}\) for \(w^* = \sup_{\vz} w\left(\vz\right) \).
\end{theoremEnd}
\begin{proofEnd}
  \begin{alignat*}{2}
    &\Esub{K^n\left(\vz, \cdot\right)}{ f \,\middle|\, \vz }
    \\
    &\quad=
    \int T_n\left(w\left(\vz\right) \vee w\left(\vz^{\prime}\right)\right) \, f\left(\vz^{\prime}\right) \, \pi\left(\vz^{\prime}\right) d\vz^{\prime}
    +
    \lambda^{n}\left(w\left(\rvvz\right)\right) \, f\left(\rvvz\right)
    &&\quad{\text{\textit{\cref{eq:imh_exact_kernel}}}}
    \\
    &\quad\leq
    \int \frac{n}{w\left(\vz\right) \vee w\left(\vz^{\prime}\right)} \, {\left(1 - \frac{1}{w^*}\right)}^{n-1} \, f\left(\vz^{\prime}\right) \, \pi\left(\vz^{\prime}\right) d\vz^{\prime}
    +
    \lambda^{n}\left(w\left(\rvvz\right)\right) \, f\left(\rvvz\right)
    &&\quad{\text{\textit{\cref{thm:tn_bound}}}}
    \\
    &\quad\leq
    \int \frac{n}{w\left(\vz^{\prime}\right)} \, {\left(1 - \frac{1}{w^*}\right)}^{n-1} \, f\left(\vz^{\prime}\right) \, \pi\left(\vz^{\prime}\right) d\vz^{\prime}
    +
    \lambda^{n}\left(w\left(\rvvz\right)\right) \, f\left(\rvvz\right)
    &&\quad{\frac{1}{w\left(\vz\right) \vee w\left(\vz^{\prime}\right)} \leq \frac{1}{w\left(\vz^{\prime}\right)}}
    \\
    &\quad=
    n \, {\left(1 - \frac{1}{w^*}\right)}^{n-1} \, 
    \int \frac{1}{w\left(\vz^{\prime}\right)} \, f\left(\vz^{\prime}\right) \, \pi\left(\vz^{\prime}\right) d\vz^{\prime}
    +
    \lambda^{n}\left(w\left(\rvvz\right)\right) \, f\left(\rvvz\right)
    &&\quad{\text{\textit{Pulled out constant}}}
    \\
    &\quad=
    n \, {\left(1 - \frac{1}{w^*}\right)}^{n-1} \, 
    \int f\left(\vz^{\prime}\right) \, q\left(\vz^{\prime}\right) d\vz^{\prime}
    +
    \lambda^{n}\left(w\left(\rvvz\right)\right) \, f\left(\rvvz\right)
    &&\quad{\text{\textit{Definition of \(w\left(\vz\right)\)}}}
    \\
    &\quad\leq
    n \, {\left(1 - \frac{1}{w^*}\right)}^{n-1} \, 
    \int f\left(\vz^{\prime}\right) \, q\left(\vz^{\prime}\right) d\vz^{\prime}
    +
    {\left(1 - \frac{1}{w^*}\right)}^{n} \, f\left(\rvvz\right)
    &&\quad{\text{\textit{\cref{thm:lambda_bound}}}}
    \\
    &\quad=
    n \, {\left(1 - \frac{1}{w^*}\right)}^{n-1} 
    \Esub{q}{f}
    +
    {\left(1 - \frac{1}{w^*}\right)}^n \, f\left(\rvvz\right).
  \end{alignat*}
\end{proofEnd}


%%% Local Variables:
%%% TeX-master: "master"
%%% End:

\vspace{-0.05in}

\begin{theoremEnd}{theorem}\label{thm:jsa}
  JSA~\citep{pmlr-v124-ou20a} is obtained by defining 
  {\small
  \begin{align*}
  P_{\vlambda}^n\left(\veta, d\veta\prime\right)
  = 
  K_{\vlambda}^{N\,\left(n-1\right) + 1}\left(\vz^{(1)}, d\vz\prime^{(1)}\right)
  \,
  K_{\vlambda}^{N\,\left(n-1\right) + 2}\left(\vz^{(2)}, d\vz\prime^{(2)}\right)
  \cdot
  \ldots 
  \cdot
  K_{\vlambda}^{N\,\left(n-1\right) + N}\left(\vz^{(N)}, d\vz\prime^{(N)}\right)
  \end{align*}
  }
  with \(\veta_t = \big[\vz_t^{(1)}, \vz_t^{(2)}, \ldots, \vz_t^{(N)}\big]\).
  Then, given~\cref{thm:bounded_weight,thm:bounded_score}, the mixing rate and the gradient variance bounds are
  {\small
  \begin{align*}
    \DTV{P_{\vlambda}^n\left(\veta, \cdot\right)}{\Pi}
    \leq
    C\left(\rho, N\right)\,{\rho}^{n\,N}
    \quad\text{and}\quad
   % 
    \E{ \norm{ \vg\left(\vlambda, \rvveta\right) }^2_{*} \,\middle|\, \mathcal{F}_{t} }
    \leq
    L^2 \,
    \left[\,
    \frac{1}{2} + \frac{3}{2}\,\frac{1}{N}
    + \mathcal{O}\left(\nicefrac{1}{w^*}\right)
    \,\right],
  \end{align*}
  }
  where \(w^* = \sup_{\vz} \pi\left(\vz\right) / q_{\text{def.}}\left(\vz;\vlambda\right)\) and \(C\) is a finite positive constant depending on both \(\rho\) and \(N\).
\end{theoremEnd}
\begin{proofEnd}

  JSA is described in~\cref{alg:jsa}. 
  At each iteration, it performs \(N\) MCMC transitions, and uses the \(N\) samples to estimate the gradient.

  \paragraph{Ergodicity of the Markov Chain}
  The state transition of the Markov chain samples \(\vz^{(1:N)}\) can be visualized as 
  {\small
  \begin{center}
  \bgroup
  \setlength{\tabcolsep}{3pt}
  \def\arraystretch{1.8}
  \begin{tabular}{c|ccccc}
   & \(\vz^{(1)}_t\) & \(\vz^{(2)}_t\) & \(\vz^{(3)}_t\) & \(\ldots\) &  \(\vz^{(N)}_t\) \\ \midrule
   \(t=1\) & \(K_{\vlambda_1}\left(\vz_0, d\vz_1^{(1)}\right)\) & \(K_{\vlambda_1}^2\left(\vz_0, d\vz_1^{(2)}\right)\) & \(K_{\vlambda_1}^3\left(\vz_0, d\vz_1^{(3)}\right)\) & \(\ldots\) & \(K_{\vlambda_1}^N\left(\vz_0, d\vz_1^{(N)}\right)\) \\
   \(t=2\) & \(K_{\vlambda_2}^{N + 1}\left(\vz_0, d\vz_2^{(1)}\right)\) & \(K_{\vlambda_2}^{N + 2}\left(\vz_0, d\vz_2^{(2)}\right)\) & \(K_{\vlambda_2}^{N + 3}\left(\vz_0, d\vz_2^{(3)}\right)\) & \(\ldots\) & \(K_{\vlambda_2}^{2\,N}\left(\vz_0, d\vz_2^{(N)}\right)\) \\
   \(\vdots\) & & & \(\vdots\) & & \\
   \(t=k\) & \(K_{\vlambda_k}^{\left(k-1\right)\,N + 1}\left(\vz_0, d\vz_k^{(1)}\right)\) & \(K_{\vlambda_k}^{\left(k-1\right)\,N + 2}\left(\vz_0, d\vz_k^{(2)}\right)\) & \(K_{\vlambda_k}^{\left(k-1\right)\,N + 3}\left(\vz_0, d\vz_k^{(3)}\right)\) & \(\ldots\) & \(K_{\vlambda_k}^{\left(k-1\right)\,N + N}\left(\vz_0, d\vz_k^{(N)}\right)\) \\
  \end{tabular}
  \egroup
  \end{center}
  }
  where \(K_{\vlambda}\left(\vz, \cdot\right)\) is an IMH kernel.
  Therefore, the \(n\)-step transition kernel for the vector of the Markov-chain samples \(\veta = \vz^{(1:N)}\) can be represented as
  \begin{align*}
  P_{\vlambda}^n\left(\veta, d\veta\prime\right)
  = 
  K_{\vlambda}^{N\,\left(n-1\right) + 1}\left(\vz_1, d\vz\prime_1\right)
  \,
  K_{\vlambda}^{N\,\left(n-1\right) + 2}\left(\vz_2, d\vz\prime_2\right)
  \cdot
  \ldots 
  \cdot
  K_{\vlambda}^{N\,\left(n-1\right) + N}\left(\vz_N, d\vz\prime_N\right).
  \end{align*}

%%   For example, let us assume that we use a batch size of 1 and that \(m=1\) is selected at \(t=1\) and \(m=2\) is selected at \(t=2\).
%%   Then, the transitions can be visualized as
%%   {\small
%%   \begin{center}
%%   \bgroup
%%   \def\arraystretch{1.8}
%%   \setlength{\tabcolsep}{5pt}
%%   \begin{tabular}{c|cccccccccccccc}
%%       t = 1 & \(\vz^{(1,1)}_1\) & \(\vz^{(1,2)}_1\) & \ldots & \(\vz^{(1,N)}_1\) & \ldots & \(\vz^{(2,1)}_1\) & \(\vz^{(2,2)}_1\) & \ldots & \(\vz^{(2,N)}_1\) & \ldots & \(\vz^{(1,N)}_1\) & \(\vz^{(2,N)}_1\) & \ldots & \(\vz^{(M,N)}_1\) \\
%%       & \(\downarrow\;K_{1,\lambda}\) & \(\downarrow\;K_{1,\lambda}^2\) & \ldots & \(\downarrow\;K_{1,\lambda}^N\)  &  &  &  & &  &  &  &  &  & \\
%%       t = 2 & \(\vz^{(1,1)}_2\) & \(\vz^{(1,2)}_2\) & \ldots & \(\vz^{(1,N)}_2\) & \ldots & \(\vz^{(2,1)}_2\) & \(\vz^{(2,2)}_2\) & \ldots & \(\vz^{(2,N)}_2\) & \ldots & \(\vz^{(1,N)}_2\) & \(\vz^{(2,N)}_2\) & \ldots & \(\vz^{(M,N)}_2\) \\
%%       & & & & & & \(\downarrow\;K_{2,\lambda}\) & \(\downarrow\;K_{2,\lambda}^2\) & \ldots & \(\downarrow\;K_{2,\lambda}^N\)  &  &  &  & &  \\
%%       t = 3 & \(\vz^{(1,1)}_3\) & \(\vz^{(1,2)}_3\) & \ldots & \(\vz^{(1,N)}_3\) & \ldots & \(\vz^{(2,1)}_2\) & \(\vz^{(2,2)}_3\) & \ldots & \(\vz^{(2,N)}_3\) & \ldots & \(\vz^{(1,N)}_3\) & \(\vz^{(2,N)}_3\) & \ldots & \(\vz^{(M,N)}_3\) \\
%%   \end{tabular}
%%   \egroup
%%   \end{center}
%%   }
%%   where \(K_{m,\vlambda}\left(\vz, \cdot\right)\) is a componentwise IMH kernel for the \(m\)th component.
%%   Conceptually, this means that, when a batch \(\vz^{(m,1:N)}\) is selected \(k\) times, it will evolve as
%%   \begin{alignat*}{2}
%%     \vz^{(m,1:N)} \sim  K_{m,\vlambda}^{\left(k-1\right)\,N + 1}\left(\vz^{(m,1)}, d\vz^{(m,1)}\right) \, K_{m,\vlambda}^{\left(k-1\right)\,N + 2}\left(\vz^{(m,2)}, d\vz\prime^{(m,2)}\right) \cdot \ldots \cdot K_{m,\vlambda}^{\left(k-1\right)\,N + N}\left(\vz^{(m,N)}, d\vz\prime^{(m,N)}\right).
%%   \end{alignat*}

%%   The random-scan kernel can be represented as
%%   \begin{alignat*}{2}
%%     P_{\vlambda}^k\left(\veta, d\veta\prime\right)
%%     =
%%     \sum_{m=1}^M
%%     r_{m} \,
%%     \left(
%%     K_{m,\vlambda}\left(\vz^{(m,1)}, d\vz\prime^{(m,1)}\right) \,
%%     K_{m,\vlambda}\left(\vz^{(m,2)}, d\vz\prime^{(m,2)}\right)
%%     \cdot
%%     \ldots
%%     \cdot
%%     K_{m,\vlambda}^N\left(\vz^{(m,N)}, d\vz\prime^{(m,N)}\right)
%%     \right)
%%   \end{alignat*}
%%   where .

  Now, the convergence in total variation \(d_{\mathrm{TV}}\left(\cdot, \cdot\right)\) can be shown to decrease geometrically as
  \begin{alignat}{2}
    &\DTV{P_{\vlambda}^{n}\left(\veta, \cdot\right)}{\Pi}
    \nonumber
    \\
    &\quad=
    \sup_{A}
    \abs{
      \Pi\left(A\right)
      -
      P^{n}\left(\veta, A\right)
    }
    &&\quad\text{\textit{Definition of \(d_{\text{TV}}\)}}
    \nonumber
    \\
    &\quad\leq
    \sup_{A}
    \Bigg|
    \int_{A}
      \pi\left(d\vz\prime^{(1)}\right) \times \ldots \times \pi\left(d\vz\prime^{(N)}\right)
    \nonumber
      \\
      &\qquad\qquad\qquad-
      K^{(n-1)\,N\,+1}_{\vlambda}\left(\vz^{(1)}, d\vz\prime^{(1)}\right) \times \ldots \times K^{n\,N}_{\vlambda}\left(\vz^{(N)}, d\vz\prime^{(N)}\right)
    \,\Bigg|
    \nonumber
    \\
    &\quad\leq
    \sup_{A}
    \sum_{n=1}^N
    \abs{
    \int_{A}
      \pi\left(d\vz^{(n)}\right) - K^{(n-1)\,N + n}_{\vlambda}\left(\vz^{(n)}, d\vz\prime^{(n)}\right) 
    }
    &&\quad\text{\textit{\cref{thm:product_measure_bound}}}
    \nonumber
    \\
    &\quad=
    \sum_{n=1}^N
    \DTV{K^{(n-1)\,N + n}_{\vlambda}\left(\vz^{(n)}, \cdot\right)}{\pi}
    &&\quad\text{\textit{Definition of \(d_{\text{TV}}\)}}
    \nonumber
    \\
    &\quad\leq
    \sum_{n=1}^N
    \rho^{(n-1)\,N + n}
    &&\quad\text{\textit{Geometric ergodicity}}
    \nonumber
    %\label{eq:used_ergodicity}
    \\
    &\quad=
    \rho^{n\,N}
    \,
    \rho^{-N}
    \,
    \frac{\rho - \rho^{N+1}}{1 - \rho}
    &&\quad\text{\textit{Solved sum}}
    \nonumber
    \\
    &\quad=
    \frac{\rho \, \left(1 - \rho^N\right)}{\rho^N \left(1 - \rho\right)}
    \,
    {\left( \rho^{N} \right)}^n.
    \nonumber
  \end{alignat}
  Although the constant depends on \(\rho\) and \(N\), the kernel \(P\) is geometrically ergodic and converges \(N\) times faster than the base kernel \(K\).

  \paragraph{\textbf{Bound on the Gradient Variance}}
  To analyze the variance of the gradient, we require a detailed expression of the \(n\)-step marginal transition kernel, which is unavailable in general for most MCMC kernels.
  Fortunately, specifically for the IMH kernel,~\citet{Smith96exacttransition} have shown that the \(n\)-step marginal IMH kernel is given as~\cref{eq:imh_exact_kernel}.
  From this, we show that
  \begin{alignat}{2}
    &\E{ \norm{ \vg\left(\vlambda, \rvveta\right) }^2_{*} \,\middle|\, \mathcal{F}_{t} }
    \nonumber
    \\
    &\quad=
    \E{ \norm{ \vg\left(\vlambda, \rvveta\right) }^2 \,\middle|\, \vz_{t-1}^{(N)},\, \vlambda_{t-1} }
    \nonumber
    \\
    &\quad=
    \E{ \norm{ \frac{1}{N}\sum^{N}_{n=1} \vs\left(\vlambda; \rvvz^{(n)}\right) }^2_{*} \,\middle|\, \vz_{t-1}^{(N)},\, \vlambda_{t-1} }
    \nonumber
    \\
    &\quad\leq
    \E{ \frac{1}{N^2} \sum^{N}_{n=1} \norm{\vs\left(\vlambda; \rvvz^{(n)}\right) }^2_{*} \,\middle|\, \vz_{t-1}^{(N)},\, \vlambda_{t-1} }
    &&\quad\text{\textit{Triangle inequality}}
    \nonumber
    \\
    &\quad=
    \frac{1}{N^2}\sum^{N}_{n=1} \Esub{\rvvz^{(n)} \sim K^n\left(\vz_{t-1}, \cdot\right)}{ \norm{\vs\left(\vlambda; \rvvz^{(n)}\right) }^2_{*} \,\middle|\,  \vz_{t-1}^{(N)},\, \vlambda_{t-1} }
    &&\quad\text{\textit{Linearity of expectation}}
    \nonumber
    \\
    &\quad\leq
    \frac{1}{N^2}\sum^{N}_{n=1}
      n \, {\left(1 - \frac{1}{w^*}\right)}^{n-1}
      \Esub{\rvvz^{(n)} \sim q_{\text{def.}}\left(\cdot; \vlambda\right)}{ \norm{\vs\left(\vlambda; \rvvz^{(n)} \right)}_{*}^2 }
      \nonumber
      \\
      &\qquad+ 
        {\left(1 - \frac{1}{w^*}\right)}^{n} \, \norm{\vs\left(\vlambda; \vz_{t-1}^{(N)} \right)}_{*}^2
    &&\quad\text{\textit{\cref{thm:imh_expecation}}}
    \nonumber
    \\
    &\quad\leq
    \frac{1}{N^2}\sum^{N}_{n=1}
      n \, {\left(1 - \frac{1}{w^*}\right)}^{n-1}  \, L^2
      +
      {\left(1 - \frac{1}{w^*}\right)}^{n} \, L^2
    &&\quad\text{\textit{\cref{thm:bounded_score}}}
    \nonumber
    \\
    &\quad=
    \frac{L^2}{N^2}\sum^{N}_{n=1}
      n \, {\left(1 - \frac{1}{w^*}\right)}^{n-1}
      +
      {\left(1 - \frac{1}{w^*}\right)}^{n} 
    \nonumber
    &&\quad\text{\textit{Moved constant forward}}
    \\
    &\quad=
    \frac{L^2}{N^2} \,
    \left[\,
      {\left(w^*\right)}^2 + w^*
      -
      {\left(1 - \frac{1}{w^*}\right)}^N
      \left(
        {\left(w^*\right)}^2 + w^* + N\,w^*
      \right)
    \,\right]
    \nonumber
    &&\quad\text{\textit{Solved sum}}
    \\
    &\quad=
    \frac{L^2}{N^2} \,
    \left[\,
    \frac{1}{2} N^2 + \frac{3}{2}\,N 
    + \mathcal{O}\left(1/w^*\right)
    \,\right]
    \nonumber
    &&\quad\text{\textit{Laurent series expansion at \(w^* \rightarrow \infty\)}}
    \\
    &\quad=
    L^2 \,
    \left[\,
    \frac{1}{2} + \frac{3}{2}\,\frac{1}{N}
    + \mathcal{O}\left(1/w^*\right)
    \,\right].
    \nonumber
  \end{alignat}
  The laurent approximation is useful for realistic values of \(w^*\) since it is bounded below exponentially by the KL divergence.
\end{proofEnd}

%%% Local Variables:
%%% TeX-master: "master"
%%% End:

%
\vspace{-0.1in}
\paragraph{Discussion}
According to~\cref{thm:jsa}, JSA benefits from increasing \(N\) both in terms of faster mixing rate and lower gradient variance.
However, similarly to MSC, for large \(w^*\), the mixing rate improvement becomes marginal.
More importantly, \textit{in the large \(w^*\) regime, the variance reduction is limited by the constant \(1/2\) term}.
This is because a large \(w^*\) means more Metropolis-Hastings rejections, increasing the correlation between the \(N\) samples.
This inefficiency persists even under stationarity unless  \(\pi\left(\cdot\right) \in \mathcal{Q}\) and \(q\left(\cdot; \vlambda\right) = \pi\left(\cdot\right)\).

%Intuitively, the inefficiency comes from the rejections persisting even under stationarity.
%On challenging problems with a large \(w^*\), the performance of JSA will also be poor.

\vspace{-0.05in}
\section{Parallel Markov Chain Score Ascent}\label{section:pmcsa}
\vspace{-0.1in}
In this section, we turn to leverage our new understanding of MCSA methods to develop a novel scheme that overcomes their limitations.

\vspace{-0.1in}
\subsection{Parallel Markov Chain Score Ascent}
\vspace{-0.07in}
\paragraph{Motivations for a Novel Scheme}
The results of~\cref{section:comparison} suggest that the performance of both MSC and JSA are heavily affected by model specification through \(w^*\).
Furthermore, model misspecification abolishes our ability to counterbalance this inefficiency by  increasing the computational budget \(N\).
Meanwhile, in \cref{section:convergence}, we discussed that the convergence of MCSA depends more on the gradient variance than the mixing rate.
Therefore, we propose a novel scheme, \textit{parallel Markov chain score ascent} (pMCSA,~\cref{alg:pmcsa}), that embraces a slower mixing rate in order to consistently achieve a \(\mathcal{O}\left(\nicefrac{1}{N}\right)\) variance reduction.

%\jrg{can we motivate this more strongly? Specifically, can we say that, by tying MCSA theory to MCGD theory, we can observe that the mixing rate isn't as important, so we develop a new scheme to exploit this previously not understood property.} 

\vspace{-0.1in}
\paragraph{Algorithm Description}
Unlike JSA, that uses \(N\) \textit{sequential} Markov chain states, pMCSA operates \(N\) parallel Markov chains.
To maintain a similar per-iteration cost with JSA, it performs only a single Markov chain transition for each chain.
Since the chains are independent, the Metropolis-Hastings rejections do not affect the variance of pMCSA.
\vspace{0.07in}


\begin{theoremEnd}{theorem}\label{thm:pmcsa}
  pMCSA, our proposed scheme, is obtained by setting
  {%\small
  \begin{align*}
    P_{\vlambda}^n\left(\veta, d\veta\prime\right)
    = 
    K_{\vlambda}^n\left(\vz^{(1)}, d\vz\prime^{(1)}\right)
    \,
    K_{\vlambda}^n\left(\vz^{(2)}, d\vz\prime^{(2)}\right)
    \cdot
    \ldots 
    \cdot
    K_{\vlambda}^n\left(\vz^{(N)}, d\vz\prime^{(N)}\right)
  \end{align*}
  }
  with \(\veta = \left[\vz^{(1)}, \vz^{(2)}, \ldots, \vz^{(N)}\right]\).
  Then, given~\cref{thm:bounded_weight,thm:bounded_score}, the mixing rate and the gradient variance bounds are
  {\small
  \begin{align*}
    \DTV{P_{\vlambda}^n\left(\veta, \cdot\right)}{\Pi}
    \leq
    C\left(N\right)\,{\rho^n}
    \quad\text{and}\quad
    \E{ \norm{ \vg\left(\vlambda, \rvveta\right) }^2_{*} \,\middle|\, \mathcal{F}_{t} }
    \leq
    L^2 \left[\; \frac{1}{N} + \frac{1}{N}\,\left(1 - \frac{1}{w^*}\right) \;\right],
  \end{align*}
  }
  where \(w^* = \sup_{\vz} \pi\left(\vz\right) / q_{\text{def.}}\left(\vz\right)\) and \(C\) is some positive constant depending on \(N\).
\end{theoremEnd}
\begin{proofEnd}

  Our proposed scheme, pMCSA, is described in~\cref{alg:jsa}. 
  At each iteration, our scheme performs a single MCMC transition for each of the \(N\) samples, or chains, to estimate the gradient.
  Similarly to JSA, we use the IMH kernel \(K_{\vlambda}\).

  \paragraph{Ergodicity of the Markov Chain}
  Since our kernel operates the same MCMC kernel \(K_{\vlambda}\) for each of the \(N\) parallel Markov chains, the \(n\)-step marginal kernel \(P_{\vlambda}\) can be represented as
  \begin{align*}
    P_{\vlambda}^n\left(\veta, d\veta\prime\right)
    = 
    K_{\vlambda}^n\left(\vz^{(1)}, d\vz\prime^{(1)}\right)
    \,
    K_{\vlambda}^n\left(\vz^{(2)}, d\vz\prime^{(2)}\right)
    \cdot
    \ldots 
    \cdot
    K_{\vlambda}^n\left(\vz^{(N)}, d\vz\prime^{(N)}\right).
  \end{align*}
  Then, the convergence in total variation \(d_{\mathrm{TV}}\left(\cdot, \cdot\right)\) can be shown to decrease geometrically as
  \begin{alignat}{2}
    &\DTV{K^{k}\left(\veta, \cdot\right)}{\Pi}
    \nonumber
    \\
    &\quad=
    \sup_{A}
    \abs{
      \Pi\left(A\right)
      -
      P^{n}\left(\veta, A\right)
    }
    &&\quad\text{\textit{Definition of \(d_{\text{TV}}\)}}
    \nonumber
    \\
    &\quad\leq
    \sup_{A}
    \big|\;
    \int_{A}
      \pi\left(d\vz\prime_1\right) \cdot \ldots \cdot \pi\left(d\vz\prime_N\right)
    \nonumber
      \\
      &\qquad\qquad\qquad-
      K^n\left(\vz_1, d\vz\prime_1\right) \cdot \ldots \cdot K^n\left(\vz_N, d\vz\prime_N\right)
    \;\big|
    \nonumber
    \\
    &\quad\leq
    \sup_{A}
    \sum_{n=1}^N
    \abs{
    \int_{A}
      \pi\left(d\vz\prime_k\right) - K^{n}\left(\vz_n, d\vz\prime_n\right) 
    }
    &&\quad\text{\textit{\cref{thm:product_measure_bound}}}
    \nonumber
    \\
    &\quad=
    \sum_{n=1}^N
    \DTV{K^n\left(\vz_n, \cdot\right)}{\pi}
    &&\quad\text{\textit{Definition of TV distance}}
    \nonumber
    \\
    &\quad\leq
    \sum_{n=1}^N
    \rho^{n}
    &&\quad\text{\textit{Geometric ergodicity}}
    \nonumber
    \\
    &\quad=
    N\,\rho^{k}
    &&\quad\text{\textit{Solved sum}}.
    \nonumber
  \end{alignat}

  \paragraph{\textbf{Bound on the Gradient Variance}}
  The bound on the gradient variance can be derived in similar manner to JSA as
  \begin{alignat}{2}
    &\E{ \norm{ \vg\left(\vlambda, \rvveta\right) }^2_{*} \,\middle|\, \mathcal{F}_{t} }
    \nonumber
    \\
    &\quad=
    \E{ \norm{ \vg\left(\vlambda, \rvveta\right) }^2 \,\middle|\, \vz_{t-1}^{(1:N)},\, \vlambda_{t-1} }
    \nonumber
    \\
    &\quad=
    \E{ \norm{ \frac{1}{N}\sum^{N}_{n=1} \vs\left(\vlambda; \rvvz_{n}\right) }^2_{*} \,\middle|\, \vz_{t-1}^{(1:N)},\, \vlambda_{t-1} }
    \nonumber
    \\
    &\quad\leq
    \E{  \frac{1}{N^2} \sum^{N}_{n=1} \norm{\vs\left(\vlambda; \rvvz_{n}\right) }^2_{*} \,\middle|\, \vz_{t-1}^{(1:N)},\, \vlambda_{t-1} }
    &&\quad\text{\textit{Triangle inequality}}
    \nonumber
    \\
    &\quad=
    \frac{1}{N^2}\sum^{N}_{n=1} \Esub{\rvvz_{n} \sim K\left(\vz_{t-1}^{(n)}, \cdot\right)}{ \norm{\vs\left(\vlambda; \rvvz_{n}\right) }^2_{*} \,\middle|\,  \vz_{t-1}^{(1:N)},\, \vlambda_{t-1} }
    &&\quad\text{\textit{Linearity of expectation}}
    \nonumber
    \\
    &\quad\leq
    \frac{1}{N^2}\sum^{N}_{n=1}
      \Esub{\rvvz_n \sim q_{\text{def.}}\left(\cdot;\vlambda\right)}{ \norm{\vs\left(\vlambda; \rvvz_n \right)}_{*}^2 }
      +
      {\left(1 - \frac{1}{w^*}\right)} \, \norm{\vs\left(\vlambda; \vz_{t-1}^{(n)} \right)}_{*}^2
    &&\quad\text{\textit{\cref{thm:imh_expecation}}}
    \nonumber
    \\
    &\quad\leq
    \frac{1}{N^2}\sum^{N}_{n=1}
        L^2 + {\left(1 - \frac{1}{w^*}\right)} \, L^2
    &&\quad\text{\textit{\cref{thm:bounded_score}}}
    \nonumber
    \\
    &\quad=
    \frac{L^2}{N^2} \sum^{N}_{n=1}
      1 + {\left(1 - \frac{1}{w^*}\right)}
    \nonumber
    &&\quad\text{\textit{Moved constant forward}}
    \\
    &\quad=
    L^2 \left[ \frac{1}{N} + \frac{1}{N}\,{\left(1 - \frac{1}{w^*}\right)} \right].
    \nonumber
    &&\quad\text{\textit{Solved sum}}
  \end{alignat}
  %Therefore, the gradient decreases at a solid \(\mathcal{O}\left(1/N\right)\) rate.
\end{proofEnd}

%%% Local Variables:
%%% TeX-master: "master"
%%% End:


\vspace{-0.1in}
\paragraph{Discussion}
Unlike JSA and MSC, the variance reduction rate of pMCSA is independent of \(w^*\).
Therefore, it should perform significantly better on challenging practical problems.
If we consider the rate of~\citet{duchi_ergodic_2012}, the combined rate is constant with respect to \(N\) since it cancels out.
In practice, however, we observe that increasing \(N\) accelerates convergence quite dramatically.
Therefore, the mixing rate independent convergence rates by~\citet{doan_finitetime_2020, doan_convergence_2020} appears to better reflect practical performance.
This is because
\begin{enumerate*}[label=\textbf{(\roman*)}]
  \item the mixing rate \(\rho\) is a conservative \textit{global} bound and 
  \item the mixing rate will improve naturally as MCSA converges.
\end{enumerate*}

%% \paragraph{Bias v.s. Variance}
%% While our proposed scheme achievs superior variance reduction, the mixing rate is worse.
%% In a MCMC estimation perspective, this translates into higher bias.
%% However, we note that
%% \begin{enumerate*}[label=\textbf{(\roman*)}]
%%   \item the constant \(C\left(\rho, N\right)\) depends on \(\rho\), 
%%   \item all of the ergodic convergence rate are close to 1 as \(w^* \rightarrow \infty\), and
%%   \item the mixing rate is a conservative global bound with respect to \(\vlambda\).
%% \end{enumerate*}
%% Therefore, in general, the superior ergodic convergence rate of JSA does not translate into faster convergence of MCSA.
%% In fact, as MCSA converges, \(w^*\) also decreases, dramatically improving the mixing rate.
%% In contrast, the relative variance does not improve too much with \(w^*\).
%% Therefore, reducing the variance is much more effective for accelerating convergence.
%% We empirically show this fact on the bias and variance in~\cref{section:simulation}.

\begin{wraptable}{r}{0.5\textwidth}
  \centering
  \vspace{-0.5in}
  
% Second version of table, with booktabs.
%\begin{table}
%\centering
\caption{Computational Costs of MCSA Schemes}\label{table:cost}
\setlength{\tabcolsep}{0.5pt}
  \begin{threeparttable}
\begin{tabular}{lccccc}\toprule
& \multicolumn{3}{c}{\footnotesize Kernel Application} & \multicolumn{2}{c}{\footnotesize Gradient Estimation} \\
\cmidrule(lr){2-4}\cmidrule(lr){5-6}
  & {\footnotesize\(p\left( \vz, \vx \right)\)}
  & {\footnotesize\(q\left(\vz; \vlambda\right)\)}
  & {\footnotesize\(q\left(\vz; \vlambda\right)\)}
  & {\footnotesize\(p\left( \vz, \vx \right)\)}
  & {\footnotesize\( q\left(\vz; \vlambda\right)\)}
  \\
  & {\footnotesize\# Eval.  }
  & {\footnotesize\# Eval.  }
  & {\footnotesize\# Samples}
  & {\footnotesize\# Grad.  }
  & {\footnotesize\# Grad.  }
%
\\\midrule
%
{\footnotesize
ELBO
}
& \(0\)
& \(0\)
& \(N\)
& \(N\)
& \(N\)
\\\arrayrulecolor{black!30}\midrule
%
{\footnotesize
MSC
}
& \(N-1\)
& \(N\)
& \(N-1\)
& \(0\)
& \(1\)\tnote{1}\;\;{\footnotesize or}\;\(N\)\tnote{2}
\\
%
{\footnotesize
JSA
}
& \(N\)
& \(N+1\)
& \(N\)
& \(0\)
& \(N\)
\\
%
{\footnotesize
\textit{pMCSA}
}
& \(N\)
& \(2 \, N\)
& \(N\)
& \(0\)
& \(N\)
\\\bottomrule
\end{tabular}
  \begin{tablenotes}
    \item[*]{\footnotesize We assume that the parameters are cached as much as possible}.
    \item[1]{\footnotesize Vanilla CIS kernel}.
    \item[2]{\footnotesize Rao-Blackwellized CIS kernel}.
  \end{tablenotes}
  \end{threeparttable}
%\end{table}

  \vspace{-0.2in}
\end{wraptable}
%
\vspace{-0.05in}
\subsection{Computational Cost Comparison}
\vspace{-0.05in}
The three schemes using the CIS and IMH kernels have different computational costs depending on \(N\) as organized in~\cref{table:cost}.

\vspace{-0.1in}
\paragraph{Cost of Sampling Proposals}
For the CIS kernel used by MSC, \(N\) controls the number of internal proposals sampled from \(q\,(\cdot; \vlambda)\).
For JSA and pMCSA, the IMH kernel only uses a single sample from \(q\,(\cdot; \vlambda)\), but applies the kernel \(N\) times.
On the other hand, pMCSA needs twice more evaluations of \(q\,(\cdot; \vlambda)\).
However, this added cost is minimal since it is dominated by that of evaluating \(p\,(\vz,\vx)\).

\vspace{-0.1in}
\paragraph{Cost of Estimating the Score}
When estimating the score, MSC computes \(\nabla_{\vlambda} \log q\,(\vz; \vlambda)\) only once, while JSA and our proposed scheme compute it \(N\) times.
However,~\cite{NEURIPS2020_b2070693} also discuss a Rao-Blackwellized version of the CIS kernel, which also computes the score \(N\) times.
Lastly, notice that MCSA methods do not need to differentiate through the likelihood \(p\,(\vz,\vx)\), unlike ELBO maximization, making its per-iteration cost significantly cheaper.

%%% Local Variables:
%%% TeX-master: "master"
%%% End:
