
\begin{theoremEnd}{proposition}\label{thm:score_bound}
  The score gradient of the variational approximation \(q\left(\cdot; \vlambda\right)\) in the location scale family with a base density that is uniformly Lipschitz continuous as
  \begin{align*}
    \norm{ \nabla_{\vlambda} \log q\left(\cdot; \vlambda\right) }_2
    \leq
    D \, L \, M \left( \norm{\vz}_2 + \norm{\vlambda}_2 + 1 \right)
  \end{align*}
  where \(\vz \in \mathbb{R}^D\).
\end{theoremEnd}
\begin{proofEnd}
  Let us denote the reparameterization function 
  \begin{align}
    \vt^{-1}\left(\vz, \vlambda\right) = \mC^{-1}_{\vlambda}\left(\vz - \vm_{\vlambda}\right)\label{eq:reparam}
  \end{align}
  such that
  \begin{align*}
    q\left(\vz; \vlambda\right) = \phi\left( \vt^{-1}\left(\vz, \vlambda\right) \right).
  \end{align*}

  for \(\vlambda = (\vm_{\vlambda}, \mC_{\vlambda})\), \(\vmu = (\vm_{\vmu}, \mC_{\vmu})\)
  Using this, we analyze the gradient of the location \(\vm_{\vlambda}\) and scale \(\mC_{\vlambda}\) parameters separately.
  First, the gradient of the location parameter is bounded as
  \begin{alignat}{2}
    \norm{
      \nabla_{\vm_{\vlambda}} \log q\left(\vz; \vlambda\right)
    }_2
    &=
    \norm{
      \nabla_{\vm_{\vlambda}} \log \phi\left( \vt^{-1}\left(\vz, \vlambda\right) \right)
    }_2
    &&\quad\text{\textit{\cref{eq:reparam}}}
    \nonumber
    \\
    &=
    \norm{
    \nabla_{\vm_{\vlambda}} \vt^{-1}\left(\vz, \vlambda\right)  \nabla_{\vt^{-1}} \log \phi\left( \vt^{-1}\left(\vz, \vlambda\right) \right) 
    }_2
    &&\quad\text{\textit{Chain rule}}
    \nonumber
    \\
    &=
    \norm{
    \nabla_{\vm_{\vlambda}} \left( \mC^{-1}_{\vlambda}\left(\vz - \vm_{\vlambda}\right) \right) \nabla_{\vt^{-1}} \log \phi\left( \vt^{-1}\left(\vz, \vlambda\right) \right) 
    }_2
    &&\quad\text{\textit{\cref{eq:reparam}}}
    \nonumber
    \\
    &=
    \norm{
      -\mC^{-1}_{\vlambda}
      \nabla_{\vt^{-1}} \log \phi\left( \vt^{-1}\left(\vz, \vlambda\right) \right) 
    }_2
    &&\quad\text{\textit{Solved derivative}}
    \nonumber
    \\
    &\leq
    \norm{
      \mC^{-1}_{\vlambda}
    }_{\mathrm{op}}
    \,
    \norm{
      \nabla_{\vt^{-1}} \log \phi\left( \vt^{-1}\left(\vz, \vlambda\right) \right) 
    }_2
    \nonumber
    \\
    &\leq
    \sqrt{M} \,
    L.
    &&\quad\text{\textit{\cref{thm:lipschitz}}}
    \label{eq:location_grad}
  \end{alignat}

  Second, for the scale parameter, we bound the derivative of the \(i,j\) element of the matrix \(\mC_{\vlambda}\).
  Let us drop the subscript \(\vlambda\) for clarity.
  Then, similarly to the location parameter,
  \begin{alignat}{2}
    \abs{
      \nabla_{\mC_{i,j}} \log q\left(\vz; \vlambda\right)
    }
    &=
    \abs{
      \nabla_{\mC_{i,j}} \log \phi\left( \vt^{-1}\left(\vz, \vlambda\right) \right)
    }
    &&\quad\text{\textit{\cref{eq:reparam}}}
    \nonumber
    \\
    &=
    \abs{
      \iprod{
        \nabla_{\mC_{i,j}} \vt^{-1}\left(\vz, \vlambda\right)   
      }{
        \nabla_{\vt} \log \phi\left( \vt^{-1}\left(\vz, \vlambda\right) \right)   
      }
    }
    &&\quad\text{\textit{Chain rule}}
    \nonumber
    \\
    &=
    \abs{
      \iprod{
        \nabla_{\mC_{i,j}} \left( \mC^{-1}_{\vlambda}\left(\vz - \vm_{\vlambda}\right) \right)
      } {
        \nabla_{\vt} \log \phi\left( \vt^{-1}\left(\vz, \vlambda\right) \right)          
      }
    }
    \nonumber
    \\
    &\leq
    \norm{
      \nabla_{\mC_{i,j}} \left( \mC^{-1}_{\vlambda}\left(\vz - \vm_{\vlambda}\right) \right)
    }_2 \,
    \norm{
      \nabla_{\vt} \log \phi\left( \vt^{-1}\left(\vz, \vlambda\right) \right)          
    }_2
    &&\quad\text{\textit{Cauchy-Schwarz inequality}}
    \nonumber
    \\
    &\leq
    L\,
    \norm{
      \nabla_{\mC_{i,j}} \left( \mC^{-1}_{\vlambda}\left(\vz - \vm_{\vlambda}\right) \right)
    }_2.
    &&\quad\text{\textit{\cref{thm:lipschitz}}}
    \nonumber
    \\
    &=
    L\,
    \norm{
    \mC^{-1}_{\vlambda} \frac{\partial \mC}{\partial_{\mC_{i,j}}} \mC^{-1}_{\vlambda}  \left(\vz - \vm_{\vlambda}\right) 
    }_2
    \nonumber
    &&\quad\text{\textit{Solved gradient}}
    \\
    &\leq
    L\,
    \norm{
      \mC^{-1}_{\vlambda}
    }_{\text{op}} \,
    \norm{
      \frac{\partial \mC}{\partial_{\mC_{i,j}}}
    }_{\text{op}} \,
    \norm{
      \mC^{-1}_{\vlambda} 
    }_{\text{op}}
    \norm{
    \vz - \vm_{\vlambda}
    }_2
    \nonumber
    &&\quad\text{\textit{Property of operator norm}}
    \\
    &=
    L\,
    \norm{
      \mC^{-1}_{\vlambda}
    }_{\text{op}} \,
    \norm{
      \mC^{-1}_{\vlambda} 
    }_{\text{op}}
    \norm{
    \vz - \vm_{\vlambda}
    }_2
    &&\quad\text{\textit{Largest eigenvalue \(\sigma_{\text{max}}\left(\frac{\partial \mC}{\partial_{\mC_{i,j}}}\right) = 1\)}}
    \nonumber
    \\
    &\leq
    L \,
    M \,
    \norm{
    \vz - \vm_{\vlambda}
    }_2
    &&\quad\text{\textit{\cref{thm:solution_space}}}
    \nonumber
    \\
    &\leq
    L \,
    M \,\left(
    \norm{\vz}_2 + \norm{\vm_{\vlambda}}_2
    \right)
    &&\quad\text{\textit{Triangle inequality}}
    \nonumber
    \\
    &\leq
    L \,
    M \, \left(
    \norm{\vz}_2 + \norm{\vlambda}_2
    \right)
    &&\quad\text{\textit{\cref{eq:parameter_norm}}}
    \nonumber
  \end{alignat}

  Now, by the poperty of norms,
  \begin{alignat}{2}
    \norm{
      \nabla_{\mC} \log q\left(\vz; \vlambda\right)
    }_{F}
    &\leq
    D \,
    \sup_{i,j}
    \abs{
      \nabla_{\mC_{i,j}} \log q\left(\vz; \vlambda\right)
    }
    \nonumber
    \\
    &\leq
    D \,
    L \,
    M \, \left(
    \norm{\vz} + \norm{\vlambda}
    \right).
    \label{eq:scale_grad}
  \end{alignat}
  Then, combining \cref{eq:location_grad,eq:scale_grad}, 
  \begin{alignat}{2}
    \norm{
      \nabla_{\vlambda} \log q\left(\vz; \vlambda\right)
    }
    &\leq
    \norm{
      \nabla_{\vm_{\vlambda}} \log q\left(\vz; \vlambda\right)
    }_2
    +
    \norm{
      \nabla_{\mC_{\vlambda}} \log q\left(\vz; \vlambda\right)
    }_{F} 
    &&\quad\text{\textit{\cref{eq:parameter_norm}}}
    \nonumber
    \\
    &\leq
    D \,
    L \,
    M \, \left(
    \norm{\vz}_2 + \norm{\vlambda}_2
    \right)
    +
    \sqrt{M} \, L
    &&\quad\text{\textit{\cref{eq:location_grad,eq:scale_grad}}}
    \nonumber
    \\
    &\leq
    D \,
    L \,
    M \, \left(
    \norm{\vz}_2 + \norm{\vlambda}_2
    +
    1 
    \right).
    \nonumber
  \end{alignat}
\end{proofEnd}

%%% Local Variables:
%%% TeX-master: "master"
%%% End:
