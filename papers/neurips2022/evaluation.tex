6
%% \begin{figure*}
%%   \vspace{-0.3in}
%%   \centering
%%   \subfloat[\(N=2^2\)]{
%%     \includegraphics[scale=0.75]{figures/gaussian_01.pdf}
%%   }
%%   \subfloat[\(N=2^4\)]{
%%     \includegraphics[scale=0.75]{figures/gaussian_02.pdf}
%%   }
%%   \subfloat[\(N=2^6\)]{
%%     \includegraphics[scale=0.75]{figures/gaussian_03.pdf}
%%   }
%%   \caption{100-D isotropic Gaussian example with a varying computational budget \(N\).
%%     MSC-PIMH converges faster than MSC-CIS and MSC-CISRB regardless of \(N\).
%%     Also, the convergence of MSC-PIMH becomes more stable/monotonic as \(N\) increases.
%%     The solid lines and colored regions are the medians and 80\% percentiles computed from 100 repetitions.
%%   }\label{fig:gaussian}
%%   \vspace{-0.15in}
%% \end{figure*}


\vspace{-0.05in}
\section{Evaluations}\label{section:eval}

\vspace{-0.05in}
\subsection{Baselines and Implementation}
\vspace{-0.05in}
\paragraph{Implementation}
For the realistic experiments, we implemented score climbing VI on top of the Turing~\citep{ge2018t} probabilistic programming framework.
For the variational family, we use diagonal Gaussians with the support transformation of~\citet{JMLR:v18:16-107}.
We use the ADAM optimizer by~\citet{kingma_adam_2015} with a learning rate of 0.01 in all of the experiments.
The computational budget is set to \(N=10\) and \(T=10^4\) for all experiments unless specified.

We compare the following methods:
\begin{enumerate}[noitemsep]
\item[\ding{182}] \textbf{ours}: our proposed scheme.
\item[\ding{183}] \textbf{JSA}: JSA with the (batch) IMH kernel~\citep{pmlr-v124-ou20a}.
\item[\ding{184}] \textbf{MSC}: MSC~\citep{NEURIPS2020_b2070693}.
\item[\ding{185}] \textbf{MSC-RB}: MSC with Rao-Blackwellization~\citep{NEURIPS2020_b2070693}.
\item[\ding{186}] \textbf{SNIS}: adaptive IS with SNIS~\cref{section:ivi_previous}.
\item[\ding{187}] \textbf{ELBO}: evidence lower-bound maximization~\citep{pmlr-v33-ranganath14, JMLR:v18:16-107} with the path derivative estimator~\citep{NIPS2017_e91068ff}.
\end{enumerate}
For ELBO, we use only a single sample as originally described by~\citet{NIPS2017_e91068ff}.
This also ensures a fair comparison against inclusive KL minimization methods since the iteration complexity of computing the ELBO gradient can be easily a few orders of magnitude larger.
Also, we only use single-HMC in the logistic regression experiment due to its high computational demands, 

\subsection{Numerical Simulation}\label{section:simulation}
\begin{figure*}
    \vspace{-0.2in}
    \centering
    
\begin{tikzpicture}
  \begin{groupplot}[
      group style = {
        group size = 3 by 1,
        horizontal sep = 45pt
      },
      width = 8.0cm,
      height = 5.0cm
    ]
    \nextgroupplot[
      title={(a) KL\label{fig:gaussian_kl}},
      title style={at={(0.5,-0.9)}},
      ymode=log,
      tuftelike,
      xlabel={
        \footnotesize
        Iteration
      },
      ylabel={
        \footnotesize
        KL 
      },
      xtick={1,2000,4000,6000,8000,10000},
      xticklabels={1,2k,4k,6k,8k,10k},
      ytick={4,32,256,2048},
      xmin=1,
      xmax=10000,
      ymin=4,
      ymax=2048,
      scaled x ticks = false,
      %% ylabel near ticks,
      %% xlabel near ticks,
      log basis y=2,
      major tick length=1.5pt,
      every tick/.style={
        black,
        semithick,
      },
      xtick pos=bottom,
      ytick pos=left,
      xtick align=outside,
      ytick align=outside,
      % smooth,
      % enlargelimits = true,
      % ymajorgrids,
      % yminorgrids,
      % xmajorgrids,
      width =4.0cm,
      height=3.5cm,
      % legend pos=north,
      legend style = { 
        legend columns = 1,
        draw           = none,
        at={(-1.5,0)},
        anchor=south west,
        legend cell align={left},
      }
    ]
    
\addplot[semithick,densely dotted,color=blue] coordinates {
(1,10898.729414105339) +- (437.0289356133326,325.7186190273169)
(101,7186.309978993618) +- (234.1586617436824,276.1151320725239)
(201,4609.10051315413) +- (186.11375633550415,241.4752488891054)
(301,3133.292636526039) +- (138.22130319161897,99.32942158136848)
(401,2210.6147817836295) +- (112.89283850263018,59.241010511505465)
(501,1567.8911456210253) +- (59.15951637999228,66.86864054852026)
(601,1155.373774580687) +- (26.85160198462995,52.31106223710299)
(701,858.3240050898718) +- (19.320240950683115,31.721460782862323)
(801,677.6849487848408) +- (13.438468499567875,24.445489318541263)
(901,549.1403757054038) +- (7.835150065943594,22.15786003633525)
(1001,446.77745953862086) +- (8.198842751837617,17.916919134717432)
(1101,374.0730648940016) +- (7.016179311556812,11.250146765452826)
(1201,311.47674853539127) +- (3.097474115164573,12.256828364176556)
(1301,257.8218928777583) +- (3.719047126471537,9.001940897686069)
(1401,214.06105540061003) +- (3.3027492923143598,7.904836370129004)
(1501,179.32534029519815) +- (3.3479048530342084,5.406600702530568)
(1601,152.72779796139187) +- (2.5594994915703637,4.089104233639972)
(1701,131.4531530019642) +- (2.830345709101948,3.799183126506577)
(1801,114.33332010058372) +- (2.4262865953644734,3.1268167986333566)
(1901,99.34016626699399) +- (1.7555940552514357,2.790474256787334)
(2001,84.68533403262421) +- (1.5661213082310184,2.1089066284873184)
(2101,73.24845230861635) +- (1.2866295175537772,1.6983571267806354)
(2201,63.71152048086374) +- (1.1267716950489302,1.5503358749373817)
(2301,55.81102499214026) +- (1.1056291226956247,1.3614927832201218)
(2401,48.93000673422726) +- (1.0669061470313892,1.2960090521001888)
(2501,43.2284664300814) +- (0.887737288683546,0.9188842769858994)
(2601,38.15439476052582) +- (0.7881135278540654,0.9317257054767509)
(2701,33.49925530189017) +- (0.6538615823262148,0.898545997680408)
(2801,29.586642414803695) +- (0.6413247150477375,0.8368790799024559)
(2901,26.328794786837747) +- (0.5711506758413805,0.6932301129946374)
(3001,23.491623549773276) +- (0.5788946050683741,0.6549616588083822)
(3101,21.42622975218405) +- (0.5433482533701124,0.6352110585531143)
(3201,19.353753713900538) +- (0.5535081984928034,0.49019371596803296)
(3301,17.531290373742877) +- (0.5885490272406138,0.41762770026076623)
(3401,16.08583991966455) +- (0.4969764083276935,0.37313540255119904)
(3501,15.071646860979165) +- (0.2516463248783527,0.4862090879590504)
(3601,13.727829142919234) +- (0.33197482478182394,0.3377795547534568)
(3701,12.596563800749735) +- (0.3093315422807077,0.30630905562773236)
(3801,11.71114023563899) +- (0.25211039534572954,0.3044049616724447)
(3901,11.052187400217251) +- (0.2159493148848668,0.28705162358259173)
(4001,10.390775276279577) +- (0.13170385220885805,0.23437151665858913)
(4101,9.696050594212416) +- (0.16989691474490343,0.22213030818323354)
(4201,9.327706941011463) +- (0.175510461301716,0.18040984161885198)
(4301,9.384964078350954) +- (0.22729957182531457,0.17713191840249642)
(4401,9.169774699263556) +- (0.3067720476267066,0.1964495446680523)
(4501,8.557676555888904) +- (0.17041836613190497,0.15593249371766582)
(4601,8.198394611291391) +- (0.1749765607631648,0.10976446052419497)
(4701,7.948300862673214) +- (0.19150415013600686,0.15114547860461425)
(4801,7.711914847179794) +- (0.1291428148609377,0.0765276387892202)
(4901,7.4231838109740504) +- (0.08810714294461697,0.12827953307710338)
(5001,7.068528942250738) +- (0.12895617507497104,0.053862932961243004)
(5101,7.125159859981604) +- (0.18548768192218468,0.11503391761350468)
(5201,6.973463323061703) +- (0.13199781017642298,0.11637977312676728)
(5301,6.951208777861892) +- (0.14517295001272945,0.16530254702650105)
(5401,6.923405489382942) +- (0.1503046595449069,0.09626450566156652)
(5501,6.8685682253869675) +- (0.18660466618782579,0.05421333433217512)
(5601,6.787367274864009) +- (0.1099340477462043,0.16390227879880293)
(5701,6.906958791404472) +- (0.14471398057918083,0.13548379090370144)
(5801,6.907504010921642) +- (0.17731098771028364,0.15806892349370116)
(5901,6.6460165706529235) +- (0.11064306854718708,0.14421837163342488)
(6001,6.679712872019403) +- (0.13904668423591193,0.08463038833220615)
(6101,6.838450443546291) +- (0.18617281709809852,0.13406831948702624)
(6201,6.785828545357482) +- (0.11914327971901617,0.12701333610889165)
(6301,6.916053724936024) +- (0.13131914544015721,0.0960797141823857)
(6401,6.887050277683083) +- (0.08815397133992064,0.11395096502962243)
(6501,6.842090000793686) +- (0.08415560593306992,0.06596680272326783)
(6601,6.9501856107238575) +- (0.08798545099252753,0.15018307950833876)
(6701,6.624268915579724) +- (0.15218071052838944,0.052781197876734076)
(6801,6.554400527376059) +- (0.08151926558820044,0.06942679839444654)
(6901,6.933665115214021) +- (0.09858882144139614,0.11215762779178462)
(7001,6.625499810058136) +- (0.10892207524815234,0.0672828861208643)
(7101,7.034705869664673) +- (0.3706625746353307,0.23329379742791545)
(7201,6.638206383840522) +- (0.21669726941411493,0.13584250446274027)
(7301,6.798243563030785) +- (0.14590856222836557,0.16154801458120005)
(7401,6.418623647662702) +- (0.06107735081830601,0.113377616119541)
(7501,6.3112853980917) +- (0.12621628253410222,0.09115611876403928)
(7601,6.581289244084148) +- (0.08250144226648004,0.15138607268597593)
(7701,6.370857250501686) +- (0.08736645149800193,0.06273227863672037)
(7801,6.412609377092498) +- (0.059576779816202574,0.09201190405595128)
(7901,6.384909853082906) +- (0.08299248010948723,0.12141374141657746)
(8001,6.50253169814482) +- (0.08582551717187759,0.15565762689471008)
(8101,6.485233996807006) +- (0.06427660070462604,0.0936440728638317)
(8201,6.44458450006981) +- (0.04088575798084282,0.10870569919597628)
(8301,6.364987872607138) +- (0.08867939409297598,0.05829675648471344)
(8401,6.306725832993774) +- (0.14601996498230374,0.09314312619140352)
(8501,6.437801365261367) +- (0.10056409778399544,0.08244228071599746)
(8601,6.380994941815997) +- (0.08835962581269108,0.07197438517089161)
(8701,6.558718879344648) +- (0.0662230286783867,0.08180818126113731)
(8801,6.441735715276785) +- (0.04588161380454281,0.0874366896551475)
(8901,6.350631689908274) +- (0.06386092673638899,0.07028328386211413)
(9001,6.434839194220001) +- (0.06998384825624981,0.07715508522807113)
(9101,6.504361323965091) +- (0.06141597819964595,0.11861190884191863)
(9201,6.688127622605732) +- (0.19276294497540025,0.07163831814758392)
(9301,6.8111618289034705) +- (0.1766889174603845,0.11996057756391121)
(9401,6.65345764137084) +- (0.1075491197425329,0.11917452417188645)
(9501,6.596300247221759) +- (0.1663616166646964,0.08208975023630494)
(9601,6.696112095590874) +- (0.10225542024212775,0.14655054955233915)
(9701,6.609209163719768) +- (0.0784055549674143,0.10872300123625411)
(9801,6.410644089648558) +- (0.05005495076926181,0.04734487905668683)
(9901,6.371121883574855) +- (0.0974821889495594,0.10241867621338852)
};

    \addlegendentry{\scriptsize ours \(N=16\)}
    
\addplot[semithick,color=blue] coordinates {
(1,674.1728812745977)
(21,535.1461979531229)
(41,422.52121843807447)
(61,335.8809415204394)
(81,265.40977494241474)
(101,211.50951522838494)
(121,175.51155454899603)
(141,147.77588042924407)
(161,126.11473862356637)
(181,109.64533322950084)
(201,95.74491330264482)
(221,83.71530842690348)
(241,73.76017477533719)
(261,65.22032934745812)
(281,58.28695719993392)
(301,52.78253891497576)
(321,47.477558072004356)
(341,42.49064265718288)
(361,37.934085119358265)
(381,34.09730766112332)
(401,30.77578410032954)
(421,28.128426627827388)
(441,25.714434989784923)
(461,23.62909133993079)
(481,21.80452156592272)
(501,20.24510761379805)
(521,18.73541940554687)
(541,17.314643782693214)
(561,16.14496172088164)
(581,15.062314851735966)
(601,14.09934832329352)
(621,13.19693342260921)
(641,12.399893362156645)
(661,11.67087895906523)
(681,10.959192145300513)
(701,10.305054000762443)
(721,9.72091640165836)
(741,9.205142776230248)
(761,8.735985890198567)
(781,8.26723064180002)
(801,7.792088019215786)
(821,7.345214653065982)
(841,6.946948923767171)
(861,6.557256152398811)
(881,6.164578456836395)
(901,5.791586271926954)
(921,5.436336727992936)
(941,5.129567706998669)
(961,4.8656571446823)
(981,4.627524260062516)
(1001,4.4165594328512965)
(1021,4.203059066631445)
(1041,4.008074028937928)
(1061,3.821683006886011)
(1081,3.6618379471299978)
(1101,3.522658150814545)
(1121,3.396823867159628)
(1141,3.266657684291628)
(1161,3.1622344975636736)
(1181,3.0551207376066247)
(1201,2.949730261595906)
(1221,2.8514916540632385)
(1241,2.752997560702422)
(1261,2.665956304611464)
(1281,2.578671351057178)
(1301,2.496518535659496)
(1321,2.420781358014108)
(1341,2.3458015164121098)
(1361,2.2752974300992226)
(1381,2.200541941052604)
(1401,2.128603540122256)
(1421,2.0600157325730732)
(1441,2.0010610496382975)
(1461,1.9371174182502575)
(1481,1.8815064153040706)
(1501,1.8338259535905794)
(1521,1.7856940341397909)
(1541,1.7361176662756823)
(1561,1.6881040215128917)
(1581,1.6478443590021814)
(1601,1.6147611289360868)
(1621,1.58217445339058)
(1641,1.550913477596046)
(1661,1.5206568343274895)
(1681,1.4834527580506518)
(1701,1.446739015406943)
(1721,1.4219493708767004)
(1741,1.4026973686591289)
(1761,1.3816284952911215)
(1781,1.360668572640154)
(1801,1.3402888052554676)
(1821,1.3193878837117727)
(1841,1.297668854810417)
(1861,1.2789322547771378)
(1881,1.2616195514837014)
(1901,1.243053613389652)
(1921,1.2234801398825461)
(1941,1.2033730305886128)
(1961,1.1847468888338077)
(1981,1.1635053776144244)
(2001,1.1448652183964558)
(2021,1.1302608941436771)
(2041,1.1144742510988717)
(2061,1.0986364187021795)
(2081,1.0880705625961136)
(2101,1.070342122232888)
(2121,1.046224264272304)
(2141,1.0269080928370626)
(2161,1.01106930982785)
(2181,0.9967272091846098)
(2201,0.9844403614912443)
(2221,0.9730842286880441)
(2241,0.9622784058743394)
(2261,0.9525113380336814)
(2281,0.9413856651430478)
(2301,0.9285874522730234)
(2321,0.9187240809901153)
(2341,0.910267572570421)
(2361,0.9022345733708256)
(2381,0.8954940231833761)
(2401,0.8885239685153205)
(2421,0.8786514284771367)
(2441,0.8703762272592732)
(2461,0.8640129829413908)
(2481,0.8575139115346432)
(2501,0.8508374162133704)
(2521,0.8458032804542084)
(2541,0.8391412960031154)
(2561,0.8340510842041201)
(2581,0.8265613119634233)
(2601,0.81973979908759)
(2621,0.81243193740008)
(2641,0.8062333933432232)
(2661,0.8008449107159247)
(2681,0.7968794152082377)
(2701,0.7916529049232138)
(2721,0.7868574746396115)
(2741,0.7831465321576953)
(2761,0.7822518740573741)
(2781,0.7767223595491575)
(2801,0.770213098286272)
(2821,0.7647188436532883)
(2841,0.7604366539057257)
(2861,0.7543659608306369)
(2881,0.7491550717490842)
(2901,0.7454610364808221)
(2921,0.7440031842445168)
(2941,0.7398116441350646)
(2961,0.7366602411869115)
(2981,0.7345133746283836)
(3001,0.7316424397742153)
(3021,0.7283503477882716)
(3041,0.7251451197742186)
(3061,0.7217826272266551)
(3081,0.7199112130231408)
(3101,0.7202041251692572)
(3121,0.7192132744893216)
(3141,0.7173794755184448)
(3161,0.7157204371710881)
(3181,0.7154813480228633)
(3201,0.7156690063811377)
(3221,0.7161244543427628)
(3241,0.7120174792291623)
(3261,0.7054999030471045)
(3281,0.701541756444674)
(3301,0.6981761774100509)
(3321,0.6943871056823666)
(3341,0.6927591405997524)
(3361,0.6908185406079135)
(3381,0.6886892321025804)
(3401,0.6879081991433798)
(3421,0.6867834422904132)
(3441,0.6843048929856276)
(3461,0.6827472269160194)
(3481,0.6815874977568505)
(3501,0.6794226646408674)
(3521,0.6771398516164834)
(3541,0.6759229506871498)
(3561,0.6760982844007856)
(3581,0.6744615272209294)
(3601,0.6738139293709752)
(3621,0.6726704916941183)
(3641,0.6714970102567901)
(3661,0.6704017499177787)
(3681,0.6696969908763136)
(3701,0.6679715163189377)
(3721,0.6667121529949073)
(3741,0.6675365524559094)
(3761,0.6662259122776099)
(3781,0.6644538360877545)
(3801,0.6637701268545936)
(3821,0.6623402352900202)
(3841,0.6610653257680806)
(3861,0.6599400387050711)
(3881,0.6591634581713194)
(3901,0.6560826723093747)
(3921,0.6544825859109897)
(3941,0.6543290408193748)
(3961,0.6543816179723743)
(3981,0.6544339030751423)
(4001,0.6544464308146539)
(4021,0.6525572737636436)
(4041,0.6513142050806142)
(4061,0.6521637839106585)
(4081,0.6516986741599156)
(4101,0.6493677141034866)
(4121,0.6490932401485587)
(4141,0.6479452209917469)
(4161,0.6471677117564283)
(4181,0.6460674383032559)
(4201,0.6459900945580457)
(4221,0.6446042640541232)
(4241,0.6436322315402097)
(4261,0.6442592321684885)
(4281,0.6445248700741395)
(4301,0.6418593901967249)
(4321,0.6412908145163466)
(4341,0.6422757798085501)
(4361,0.6423765528540569)
(4381,0.6419261644741592)
(4401,0.6435120729356336)
(4421,0.6462505027802162)
(4441,0.6513868335928386)
(4461,0.6588815444560359)
(4481,0.6540629356962187)
(4501,0.6475956852161662)
(4521,0.6458751050188034)
(4541,0.6397198632386344)
(4561,0.6375120378519803)
(4581,0.637074299184837)
(4601,0.6348885350349767)
(4621,0.6350517885874046)
(4641,0.6383458847182067)
(4661,0.6375540586870246)
(4681,0.6342411079374544)
(4701,0.6328981874294175)
(4721,0.6334592475730465)
(4741,0.6300769424069501)
(4761,0.6305008841420866)
(4781,0.629813902448301)
(4801,0.6293194398039278)
(4821,0.6292092849092609)
(4841,0.6299491533265871)
(4861,0.6291811652372004)
(4881,0.6295567624387659)
(4901,0.6305860782026771)
(4921,0.6327378487083716)
(4941,0.6291582705980466)
(4961,0.6290452298970833)
(4981,0.6306867983814572)
};

    \addlegendentry{\scriptsize ours \(N=64\)}
    
\addplot[semithick,densely dotted,color=red] coordinates {
(1,10940.830797453684) +- (427.9042040766235,324.81472611238314)
(101,10238.809653845416) +- (602.7915222980628,357.4867822458964)
(201,8810.702465043349) +- (584.9845150477104,420.79713059476853)
(301,7566.78030552095) +- (683.1861405871987,512.6878766000364)
(401,6185.442621226261) +- (1060.7616343225855,331.181325898523)
(501,5142.716536111092) +- (894.4318305364768,339.79924715149355)
(601,4344.738299174162) +- (765.6790129597775,364.05385169119745)
(701,3653.0787265840963) +- (428.8664300825003,317.1781222628406)
(801,3085.6955098322596) +- (282.8784253614581,255.4777337910714)
(901,2718.1304891133723) +- (348.04517765921764,192.42020183917748)
(1001,2399.626867893752) +- (298.0598385569683,135.61717727454197)
(1101,2146.0782691232953) +- (151.01101169152298,148.9992814208233)
(1201,1858.4404338286233) +- (122.7063645738931,117.61628724177854)
(1301,1622.4000689036438) +- (99.8748706107749,87.74576251460871)
(1401,1445.2796767879445) +- (94.74654693726075,106.41439718914876)
(1501,1239.2669615591735) +- (138.64935528098272,42.66688332728927)
(1601,1115.6104517936217) +- (137.83277461888179,54.10149074406036)
(1701,989.9172837585946) +- (99.09789725123903,37.40687835805636)
(1801,892.5167912808408) +- (56.894014635283725,28.291033528709363)
(1901,843.6537044595959) +- (101.96299412980215,45.95081635691383)
(2001,751.5091280891376) +- (81.52860256055976,33.43349650310256)
(2101,670.0041166935973) +- (56.476225834549496,28.090972317653836)
(2201,603.6960773857293) +- (35.34449335117813,29.496316376209734)
(2301,550.9511855408629) +- (29.139575399146793,25.108602274548844)
(2401,502.37034986245646) +- (16.644477916403503,24.712351366374662)
(2501,470.60553582466) +- (14.459256868578564,29.97453544281393)
(2601,437.0634803028509) +- (18.712229089321227,24.640406565075693)
(2701,399.0791927446262) +- (14.55105349052593,17.00307389942003)
(2801,381.065565935762) +- (24.77035299505758,18.988254560644805)
(2901,338.7942983381631) +- (10.709355775350048,14.376930720191638)
(3001,315.84692577721444) +- (12.282406751483109,13.123975824946399)
(3101,295.1642931394366) +- (11.04058823285186,11.84802940037332)
(3201,281.9498074557879) +- (12.198797351720202,12.37619295529646)
(3301,265.81707362153554) +- (13.011977866818086,11.994375175249814)
(3401,243.9797001209401) +- (7.804629454559802,13.229742742892427)
(3501,231.54151731808366) +- (11.834476558929936,13.442980888608304)
(3601,214.47521763580897) +- (14.636762050940604,12.37370813759614)
(3701,194.73675661728643) +- (10.78230926590092,8.56620770951389)
(3801,177.46642962217078) +- (9.749407458690541,7.097263806764573)
(3901,159.79592853527993) +- (6.263523430421941,7.496867564418096)
(4001,149.21935355889437) +- (5.510244613218134,6.978417549107462)
(4101,139.7156318270831) +- (6.018933215097803,6.050968890778648)
(4201,131.2829168759266) +- (5.078467630361217,6.104213332291067)
(4301,122.16786568309715) +- (5.401440241208732,5.285239123485653)
(4401,114.81694048762571) +- (4.43837281464036,5.308260514251259)
(4501,107.71296270072669) +- (4.449495249381641,4.787351895114909)
(4601,100.39310260750061) +- (3.526364233926884,3.9503938404499763)
(4701,93.97355574773394) +- (2.928557681698379,3.7110096923763223)
(4801,87.99495447951978) +- (3.032596358168604,3.4784625610836883)
(4901,82.49085580590325) +- (4.1133316641780056,2.4407418566941033)
(5001,77.94390711141419) +- (3.769662241915725,2.346413587793151)
(5101,72.38536365128466) +- (3.091987510543973,2.085039404271754)
(5201,68.32106216129762) +- (3.3636078226460455,1.6888659772604626)
(5301,65.10760995910596) +- (3.2939236745197746,1.599098125654649)
(5401,62.217577490112646) +- (2.0911985305563405,1.228275745232736)
(5501,58.00902439429019) +- (2.692366316448343,1.031349208995323)
(5601,54.18538292969859) +- (2.394726360289198,1.4851228476985412)
(5701,50.53351807068629) +- (1.977486029876637,1.5500398705779048)
(5801,47.62645491804304) +- (2.03263697422134,1.378399104009354)
(5901,45.27306123910961) +- (1.732708892002897,1.3372292199999691)
(6001,42.82340492142255) +- (1.7976555638367486,1.1831668252339895)
(6101,40.731401084352996) +- (1.3142885042479762,1.6924582863047135)
(6201,38.34784593084102) +- (1.0193379249004053,1.5143238259393499)
(6301,36.233032938857946) +- (1.2555551404139038,1.5807415693423792)
(6401,34.30324216448641) +- (0.9461934847689548,1.3557133224162143)
(6501,32.35328601240836) +- (1.081368560650425,1.3725805959527264)
(6601,30.968459383255848) +- (1.0881715900257234,1.2493449883052001)
(6701,29.34008077748532) +- (1.3599075333203139,0.8287384844422299)
(6801,27.61232664067733) +- (0.9380949112732111,1.083537126134356)
(6901,26.382701355112157) +- (0.7806451178085148,1.0851819347538267)
(7001,24.702515361508997) +- (0.6935298663682694,0.8887279183928243)
(7101,23.599711069535395) +- (0.6816522922197343,0.8400958417613502)
(7201,22.272742843900623) +- (0.711939081518068,0.7241928660799566)
(7301,21.222388540738713) +- (0.862185140545801,0.6547400533016372)
(7401,19.87075274109267) +- (0.8085185835911943,0.5222110141824317)
(7501,18.645471641620603) +- (0.7551793176940222,0.4621189825000691)
(7601,17.93075741089574) +- (0.7928800303463817,0.33777975544009564)
(7701,17.17980336957291) +- (0.676822707948169,0.3926353008315928)
(7801,16.28893375958741) +- (0.6359431337528498,0.4181952412698191)
(7901,15.572333613290198) +- (0.5652038658334284,0.39438937130950613)
(8001,15.049902777848617) +- (0.4967592416544768,0.5279596532392716)
(8101,14.358358918687877) +- (0.5235411126006024,0.4110939693082205)
(8201,13.91731575555891) +- (0.4136195495430677,0.4059194659479193)
(8301,13.36051140932982) +- (0.36382727046018637,0.3534501895251605)
(8401,12.785429247886514) +- (0.3645986337612168,0.3448513611874162)
(8501,12.317361658238518) +- (0.37616293182255944,0.3839547994207937)
(8601,11.843798820571479) +- (0.45098755028783266,0.24670277248646322)
(8701,11.407029300601646) +- (0.34524231725606747,0.3371539066640956)
(8801,11.086251788789266) +- (0.22977038636020275,0.23589952991969376)
(8901,10.567936260012072) +- (0.23888482128816868,0.22699796086560653)
(9001,10.224534541643841) +- (0.2758309844231217,0.16870811818969145)
(9101,9.93168697010099) +- (0.28526784536971483,0.17422742642346556)
(9201,9.628394093372643) +- (0.17525237899793922,0.16748792522524525)
(9301,9.350539969434443) +- (0.21294072208768355,0.1705829179304228)
(9401,9.191421743705153) +- (0.22205027983382486,0.2219953733956057)
(9501,8.977788985008196) +- (0.16474745993271434,0.22606650593420774)
(9601,8.616039826491605) +- (0.12601232726964184,0.14565300701701567)
(9701,8.38382808793779) +- (0.14516882872107573,0.13805869375903157)
(9801,8.123897762960596) +- (0.18981109781053895,0.1625142551530585)
(9901,7.909849023813023) +- (0.17711732223447196,0.10380586977522377)
};

    \addlegendentry{\scriptsize JSA \(N=16\)}
    
\addplot[semithick,color=red] coordinates {
(1,10914.033864027462) +- (453.3234162596127,327.40685377418595)
(101,9466.937396573405) +- (465.90572394081937,327.37566166159195)
(201,7348.013795395318) +- (332.84487593278755,359.47612731566824)
(301,5846.440314912059) +- (255.195202545553,342.77121340865597)
(401,4737.597292704253) +- (261.701173962706,236.30626733109875)
(501,3927.4249181585005) +- (188.3834638530957,269.5339611067011)
(601,3143.784464117007) +- (124.63910230603051,238.3996754998252)
(701,2619.6557059637034) +- (144.85714028811663,175.23943509857554)
(801,2233.3184838891493) +- (108.70615698921165,143.20392386520507)
(901,1922.5847918804357) +- (67.925345083926,99.11571380548367)
(1001,1638.9087671891289) +- (79.4276244691091,86.09910112652142)
(1101,1388.508704090737) +- (54.49266896394215,60.21516474599139)
(1201,1233.1808374942202) +- (31.436584213017795,67.04766114466884)
(1301,1086.4756772060675) +- (33.84514439413101,43.178117136566016)
(1401,923.8558478496843) +- (34.65552125826571,32.329885358884894)
(1501,830.2459039371532) +- (22.365261330615454,33.49142827029357)
(1601,735.9213024699629) +- (20.18142744727095,29.06220399128256)
(1701,664.3188020730025) +- (21.35304633619444,23.12800045189522)
(1801,599.0129055182515) +- (13.98599941964585,22.02345974332343)
(1901,540.0317131615211) +- (13.646107405496082,18.220036290426947)
(2001,490.1148890239979) +- (12.008249846594595,19.47082815626743)
(2101,446.66024491238386) +- (13.248624728481843,14.671592982534037)
(2201,406.28124909676455) +- (8.332992162827622,16.36759120052153)
(2301,370.0785808920775) +- (8.524800983648447,13.68380764129654)
(2401,333.9627431381957) +- (9.72339192891701,9.66925032961251)
(2501,306.74424174572107) +- (7.008838028217895,9.684281582791527)
(2601,282.56653854673687) +- (8.565348957555557,8.835077518819503)
(2701,261.32777204267836) +- (8.978799757926197,7.006013181962288)
(2801,242.9567323410671) +- (8.911714675206497,5.960343818512712)
(2901,223.8480248622102) +- (8.638611575146257,5.247522834120048)
(3001,208.153365890082) +- (7.647527878297183,5.276235201804269)
(3101,192.55358760311248) +- (6.041679312753388,5.222793433452665)
(3201,177.27791454220363) +- (5.918763550879447,4.152296246103617)
(3301,161.59632120250626) +- (3.996288845346214,4.650440469485943)
(3401,148.80452863624367) +- (4.179166338867617,4.191550492932748)
(3501,136.24045360623427) +- (3.731995691931786,3.0222423647005883)
(3601,126.2836724310977) +- (3.545803710091036,2.30858656314669)
(3701,119.01404992378676) +- (3.2457099350730374,2.322016398713316)
(3801,111.14715815631189) +- (3.116239418019191,2.2864800629731405)
(3901,103.61438842541762) +- (2.261054823633472,2.611246147268915)
(4001,96.71786395303579) +- (2.190507944522423,2.3729113113281386)
(4101,90.0882508644442) +- (2.0939435389203993,2.2659928823150466)
(4201,84.03658491090428) +- (1.844540678540156,1.8632068698653512)
(4301,78.76740219522549) +- (2.0147936922158323,1.73778654174113)
(4401,74.01986367813004) +- (1.4103413026127924,1.5042154441789393)
(4501,69.79044164763776) +- (1.2716482088530938,1.7541546296858002)
(4601,65.38070506369999) +- (1.3199501500062638,1.6901264839073349)
(4701,60.961291630070605) +- (1.1449534975129012,1.5203475390894496)
(4801,57.0780792077679) +- (0.9994282768777296,1.6739041450800656)
(4901,52.96075443053581) +- (1.523308934613297,1.4577589835530986)
(5001,49.50192702776863) +- (1.537199919521285,1.5270035716243413)
(5101,46.482095740381624) +- (1.595180824099124,1.2666277736116314)
(5201,43.452086043562375) +- (0.900928532057037,1.3892082565095478)
(5301,40.66583235063095) +- (0.7857204252336842,1.2641730861060196)
(5401,37.787021303154184) +- (0.780700963265474,1.2550184714086967)
(5501,35.49346510339906) +- (0.6370924464332148,1.0398118000486107)
(5601,33.87009724979451) +- (0.7023434087623315,0.9269466191364089)
(5701,31.7545835031433) +- (0.6981550221748094,0.6674956715389193)
(5801,29.711179477703297) +- (0.7015849413244055,0.7625816472460158)
(5901,27.91486101731695) +- (0.4924213443232084,0.644976743386156)
(6001,26.50565926201241) +- (0.4272079838057934,0.7591084153726335)
(6101,24.94948997836535) +- (0.6089811295701537,0.8077256256182679)
(6201,23.69038454691036) +- (0.44022555851334033,0.8029911408150703)
(6301,22.068374438646043) +- (0.446552948082509,0.6946180969074476)
(6401,20.75923596542438) +- (0.47600937761418294,0.5829074086360997)
(6501,19.614388071681887) +- (0.462393547379925,0.5004618457216026)
(6601,18.543924091722378) +- (0.40979920291357175,0.4299416516770336)
(6701,17.728711228929065) +- (0.41096123769404613,0.4162799632402958)
(6801,16.822548153587952) +- (0.41003497621751706,0.40158113155438)
(6901,16.07463408313152) +- (0.3523195726477546,0.38919064803163117)
(7001,15.399120640420634) +- (0.27238287922206617,0.26920381222639733)
(7101,14.704770947040608) +- (0.2478065933785416,0.3468276279181186)
(7201,14.152220110245624) +- (0.15354422303690285,0.35322446923020046)
(7301,13.458104008961012) +- (0.19345419685429732,0.3236194628145377)
(7401,12.940747653116347) +- (0.12589117947269202,0.40003047858994734)
(7501,12.363943934557202) +- (0.16502439251509138,0.3767887072495899)
(7601,11.718332357523328) +- (0.1911062840761435,0.3003245060821289)
(7701,11.215777408077216) +- (0.1456699409789799,0.2964124125175722)
(7801,10.78808589043869) +- (0.13338636389270775,0.2659955609521347)
(7901,10.294060741770862) +- (0.07842106697521878,0.260850968656527)
(8001,9.890320427889534) +- (0.08108055880663834,0.23980805674045946)
(8101,9.523215689711112) +- (0.08590317692240568,0.22761843938997295)
(8201,9.176090260968712) +- (0.07536487749589504,0.22261979868248716)
(8301,8.89810815031779) +- (0.07069976063882777,0.1851486393002606)
(8401,8.55887059139863) +- (0.08253552617095927,0.14659451029659287)
(8501,8.401041739901828) +- (0.0885823973763511,0.1502955115027813)
(8601,8.256779010715238) +- (0.10009285928690126,0.13227870124699947)
(8701,8.001807188205586) +- (0.14414841325199745,0.10646100854480522)
(8801,7.743391799470537) +- (0.09114786501766403,0.10718843968575431)
(8901,7.561365947852449) +- (0.06409396465564932,0.11853304102510798)
(9001,7.380465810603883) +- (0.07366993832945479,0.0757890095614302)
(9101,7.180648022319023) +- (0.08279335151069223,0.069968912855769)
(9201,7.037989843961848) +- (0.05578335593353945,0.0697872295198767)
(9301,6.904767679725564) +- (0.06876416825096232,0.07480462736509796)
(9401,6.786131906691608) +- (0.04953013953511132,0.07369881730580374)
(9501,6.763390803680063) +- (0.06650341673538751,0.06955662769614879)
(9601,6.613767380440846) +- (0.05649993236521045,0.07331418905651788)
(9701,6.547966373102065) +- (0.0697114070385556,0.0727322771340626)
(9801,6.583398759012587) +- (0.05932959996859477,0.0745477267043011)
(9901,6.532915300704445) +- (0.07591803952696985,0.05923134302040456)
};

    \addlegendentry{\scriptsize JSA \(N=64\)}
    
\addplot[semithick,densely dotted,color=teal] coordinates {
(1,10940.830797453684) +- (431.447875401278,317.5197611211279)
(101,10635.648169053737) +- (894.5563608445809,684.6939159919657)
(201,9236.027205041184) +- (1195.5754747653245,593.7068370061079)
(301,8029.053784745232) +- (666.4882514049568,630.530437541729)
(401,6923.066005048523) +- (966.6016839919976,523.1782531928357)
(501,6340.678512242639) +- (1392.2889383586926,871.6069306744976)
(601,5494.09484129117) +- (987.3124668859418,1178.460050597635)
(701,4506.171509068896) +- (1857.3658392663392,570.6524070936234)
(801,3829.932773218734) +- (3947.2756013501084,305.40409954507004)
(901,3335.25176022727) +- (6450.341339944932,294.5975214766818)
(1001,2805.92191637044) +- (9872.138351149693,240.57873803384837)
(1101,2515.8525772177977) +- (20593.84859671632,243.7961561061611)
(1201,2310.184997000281) +- (20039.421212131776,282.3535360479086)
(1301,2023.4359879052067) +- (57358.10133606436,214.16472124906568)
(1401,1804.3006580417737) +- (55133.539997256405,152.08104650096857)
(1501,1613.7665244227583) +- (61736.34116791306,161.74653624448615)
(1601,1445.1591314738882) +- (95399.85550939426,180.31197802807037)
(1701,1288.6059358941147) +- (174018.61237567538,184.1421528246451)
(1801,1130.5264963240215) +- (273626.3399933798,91.6049239472818)
(1901,983.4763244736419) +- (173089.2960843529,32.74630358347258)
(2001,881.4914163324141) +- (204119.24186730885,53.43038411582813)
(2101,803.6805052744944) +- (291624.1321996033,55.29788400622647)
(2201,729.741438413505) +- (321180.7933238846,42.52445094242182)
(2301,658.1625319078678) +- (347567.5011522378,38.28555902347091)
(2401,611.9348944236024) +- (319057.58271067694,33.429649183126)
(2501,555.3459718645624) +- (388255.2467289627,28.92406734962958)
(2601,510.341759280513) +- (685316.6522945645,22.52211055404615)
(2701,464.39660746935635) +- (923850.3271406787,18.45142556194412)
(2801,429.81557576155893) +- (1.6421303835324084e6,20.507598144655162)
(2901,402.5636887128188) +- (1.8436627533539266e6,17.79856178078046)
(3001,377.09881587741677) +- (1.2746077054141322e6,14.829615955097267)
(3101,351.37717291977503) +- (1.5727487351170087e6,17.99092433073423)
(3201,322.4567346789085) +- (1.5593587916946735e6,17.582643030832173)
(3301,295.05891041369296) +- (2.4094640613353844e6,11.104096248719713)
(3401,279.2255214944727) +- (2.0218141346090676e6,15.593545973422863)
(3501,256.6781010189596) +- (2.246411539728873e6,9.988087521315805)
(3601,236.13406941804277) +- (1.276045374009151e6,9.13246600698767)
(3701,221.0663345551979) +- (1.7051558154400203e6,11.007997660612432)
(3801,207.63760616661858) +- (1.9875234160788215e6,11.29875108680784)
(3901,192.96750311991914) +- (1.2583317192540795e6,10.980197092807032)
(4001,177.42777782952908) +- (1.7335522164102476e6,7.729986531833731)
(4101,165.28527526531792) +- (1.5006037909765814e6,8.006129785766262)
(4201,154.80035043053616) +- (2.459180614780179e6,7.067016284354196)
(4301,145.4413591241343) +- (2.9938036249286006e6,7.68698716996721)
(4401,134.31356683352644) +- (4.373793819805813e6,5.289005072748836)
(4501,125.84087386800138) +- (4.920717068093384e6,5.428050918068678)
(4601,118.42966512136243) +- (2.832437134699934e6,5.035412950661112)
(4701,110.23621924147479) +- (2.547977786784059e6,5.537482156972857)
(4801,104.56712855199852) +- (2.291972766608969e6,5.991334998212608)
(4901,98.34829954242639) +- (1.5373461944296092e6,5.435135327057537)
(5001,93.33161092695106) +- (1.905588841118095e6,4.90908500861525)
(5101,87.28696593610807) +- (1.8152111956497193e6,3.890107667192339)
(5201,82.69852198733977) +- (4.135297535748225e6,2.967128797322104)
(5301,78.88110870795659) +- (1.3051583172110727e6,3.5774786973268675)
(5401,73.65087238043904) +- (1.0688622836481805e6,3.010934683693435)
(5501,69.29311424401152) +- (894710.2583989671,2.3786454687968046)
(5601,64.85647246419106) +- (1.1623517946927766e6,2.4278393738453232)
(5701,60.56442445230283) +- (1.5921662600303008e6,1.7756918239028252)
(5801,57.58104709775388) +- (1.9779535290856129e6,2.0741478810825384)
(5901,54.23948801560894) +- (1.6157906898991577e6,2.004646127951574)
(6001,50.69954547319414) +- (485617.937934962,1.4155072179528716)
(6101,47.98413353841717) +- (851778.7335733256,1.403544018001611)
(6201,45.23530602160555) +- (512918.53987228393,0.7258646286685746)
(6301,43.27955858702718) +- (848523.8666546276,1.23147620263709)
(6401,40.83431097643199) +- (1.2649297709426112e6,1.0602979918708968)
(6501,38.59438232221885) +- (1.2781994300150324e6,1.2075216672033022)
(6601,36.38283349370645) +- (1.316109678576139e6,0.921181171409124)
(6701,34.55290669687497) +- (1.0899065124374733e6,0.8391050155326241)
(6801,32.81149432414456) +- (792770.1423200464,0.72812170986073)
(6901,31.239239323790294) +- (1.1618519523646594e6,0.7401107992483524)
(7001,30.09532293231864) +- (1.7775378495388813e6,1.107725634587819)
(7101,28.015867044182627) +- (1.3097612565868516e6,0.8306959578636892)
(7201,26.520347130479948) +- (1.1942293100285747e6,0.6291664252159173)
(7301,25.22766938669457) +- (1.0185475598512485e6,0.6638450920676462)
(7401,23.966804572332265) +- (628100.7672353467,0.5708027518527743)
(7501,23.067357451758575) +- (175152.08309542842,0.6428641837784852)
(7601,21.68673532377359) +- (284652.3553844076,0.7330410697198033)
(7701,20.70699227097934) +- (265891.38882847846,0.7309735567666635)
(7801,19.715684940536654) +- (303992.44639573706,0.7752193751641592)
(7901,18.640551208682595) +- (453394.17669071216,0.7215521762366954)
(8001,17.932083512805473) +- (333848.7362006052,0.7734635072862766)
(8101,17.338301399373236) +- (218458.82249151758,0.7347612325833772)
(8201,16.712197573867797) +- (161231.25728952273,0.6103831333113909)
(8301,16.187166306482357) +- (211057.59343944155,0.806338431168987)
(8401,15.56254361276059) +- (200484.1562360491,0.39283306544942853)
(8501,14.992238250374204) +- (115860.77376787747,0.4243467985255851)
(8601,14.379976779488704) +- (65668.75096420594,0.39683919262814804)
(8701,13.940619210991326) +- (24492.61212178859,0.39651137294717564)
(8801,13.354979363505947) +- (14035.71005440327,0.31864822895886746)
(8901,12.876874398122688) +- (14980.381865820713,0.42227964183601685)
(9001,12.425018336139482) +- (6133.95866685882,0.39374278861188117)
(9101,11.994401744133757) +- (3927.1631091371264,0.16950958072218292)
(9201,11.575926186281702) +- (772.8574091912972,0.24341018702252626)
(9301,11.27764780732133) +- (300.665934857616,0.24133422664021253)
(9401,11.01835442909117) +- (72.9485656705874,0.1939162825661196)
(9501,10.677295802261554) +- (43.03622684567229,0.2475064990215703)
(9601,10.418088595653213) +- (11.861119845518964,0.28466568073231713)
(9701,10.063712250056462) +- (6.0268118010733005,0.3296120813423009)
(9801,9.905338035387171) +- (5.039275258911438,0.3056049890481596)
(9901,9.674152673708464) +- (4.472132320415598,0.24429022300965642)
};

    \addlegendentry{\scriptsize MSC \(N=16\)}
    
\addplot[semithick,color=teal] coordinates {
(1,692.3784015520071)
(21,608.8038306412288)
(41,543.7246314577845)
(61,492.5350008544163)
(81,456.3563977806764)
(101,414.14362846049875)
(121,377.73643928723646)
(141,348.40843114351236)
(161,315.1446404926167)
(181,292.33677692032086)
(201,272.1191915865955)
(221,252.94246661959707)
(241,234.42269334169842)
(261,224.07062954784192)
(281,209.23830910779128)
(301,195.39084705769025)
(321,181.015174621581)
(341,170.15180332184204)
(361,161.63631140138898)
(381,151.56009838869764)
(401,143.6470925681004)
(421,132.40292635768094)
(441,120.9772077770162)
(461,111.91718117306488)
(481,106.06853032061353)
(501,100.67139733183905)
(521,94.78151524270393)
(541,90.3229620799295)
(561,84.01833010982291)
(581,79.14840171036586)
(601,75.73382017396179)
(621,70.65151104238353)
(641,66.74934593970889)
(661,63.27607253882502)
(681,58.83888159979258)
(701,55.039643754367404)
(721,52.87254909424795)
(741,50.610637880462214)
(761,48.23039591447281)
(781,46.063581580756704)
(801,44.09818616143666)
(821,41.1565417146809)
(841,38.85918246579784)
(861,37.357567752218124)
(881,35.76726684005296)
(901,34.656305928396094)
(921,33.73407535882824)
(941,32.48435598116013)
(961,30.463742634291748)
(981,28.65471050836016)
(1001,27.015734007328682)
(1021,25.90994756381228)
(1041,25.27725870783898)
(1061,24.540812986634506)
(1081,23.83824894569835)
(1101,23.33999966071915)
(1121,22.742330664323894)
(1141,22.06855708181262)
(1161,21.403568012827453)
(1181,20.600532566467244)
(1201,20.163447317490164)
(1221,19.73328280311621)
(1241,19.091359592438348)
(1261,18.54930270317503)
(1281,17.91849832274954)
(1301,17.337476255988175)
(1321,16.88590875688871)
(1341,16.51182176041283)
(1361,16.019784086551397)
(1381,15.447867210444985)
(1401,15.040618295306365)
(1421,14.602327638268275)
(1441,14.225851735223356)
(1461,13.718731694846579)
(1481,13.079733419002862)
(1501,12.593235048465056)
(1521,12.251031012190897)
(1541,11.94102806413309)
(1561,11.552647809931294)
(1581,11.174677768236137)
(1601,10.87546146266805)
(1621,10.622903595998013)
(1641,10.273412304665635)
(1661,9.841668953452416)
(1681,9.439649306127905)
(1701,9.198496715391785)
(1721,8.936468332529952)
(1741,8.767059821527816)
(1761,8.616309246199897)
(1781,8.457934702281335)
(1801,8.257684425168861)
(1821,8.094208751908543)
(1841,7.932379266638484)
(1861,7.648017867639183)
(1881,7.406285092219161)
(1901,7.235287525444181)
(1921,7.084755319630563)
(1941,6.89170879928228)
(1961,6.723907787951347)
(1981,6.532230989798379)
(2001,6.370113092249375)
(2021,6.270716735464374)
(2041,6.13259463475914)
(2061,6.032378007932314)
(2081,5.956104127551316)
(2101,5.876389628842713)
(2121,5.797253061419908)
(2141,5.706135158614417)
(2161,5.570527343235684)
(2181,5.4564707955437495)
(2201,5.255539653520047)
(2221,5.133836034478538)
(2241,5.064668201604455)
(2261,4.989278921945733)
(2281,4.880509714807695)
(2301,4.764773663472513)
(2321,4.689543865266595)
(2341,4.607904478805182)
(2361,4.488322201094549)
(2381,4.38895005863148)
(2401,4.307444009571128)
(2421,4.226441799618551)
(2441,4.138323763007775)
(2461,4.074858054876914)
(2481,4.003449490231758)
(2501,3.9414928475983797)
(2521,3.8925369588474963)
(2541,3.833851363920581)
(2561,3.7411295488382494)
(2581,3.6638009136538523)
(2601,3.607233408611184)
(2621,3.550022910467198)
(2641,3.469832738091933)
(2661,3.4189836239821974)
(2681,3.3752687568982176)
(2701,3.3266698974542734)
(2721,3.2322852720089745)
(2741,3.174339585669355)
(2761,3.1194518726378764)
(2781,3.0633489563343317)
(2801,3.017455267861238)
(2821,2.9686143178954496)
(2841,2.90834809790908)
(2861,2.8627066254436144)
(2881,2.8195571498739014)
(2901,2.769569553089706)
(2921,2.7178545921922304)
(2941,2.63812960729105)
(2961,2.5930800521080153)
(2981,2.5550324378319713)
(3001,2.518102994228199)
(3021,2.4468299337062134)
(3041,2.383792748334902)
(3061,2.347293282597052)
(3081,2.309161317719882)
(3101,2.279874917502767)
(3121,2.247005972139289)
(3141,2.219294327668489)
(3161,2.196248976656639)
(3181,2.1776192800576903)
(3201,2.1477986020756985)
(3221,2.1181338158820893)
(3241,2.08755283375726)
(3261,2.0542654867598205)
(3281,2.0275138657871263)
(3301,2.001579781109214)
(3321,1.9727012357269318)
(3341,1.9433525824764144)
(3361,1.9097059050840788)
(3381,1.8770572082065609)
(3401,1.8534441930609824)
(3421,1.8316233408157467)
(3441,1.8043295835254716)
(3461,1.7797712534187489)
(3481,1.7495966296824723)
(3501,1.7252718475700988)
(3521,1.7104884048388656)
(3541,1.6959543628616078)
(3561,1.6819324048162134)
(3581,1.659162775948154)
(3601,1.6354927886583335)
(3621,1.621548661130853)
(3641,1.603323976561815)
(3661,1.5856340224978174)
(3681,1.5706938520545528)
(3701,1.5278266483922125)
(3721,1.4901828387849296)
(3741,1.4705200302587698)
(3761,1.4574995281977126)
(3781,1.4399307887041557)
(3801,1.419978320922458)
(3821,1.4057454732291306)
(3841,1.3888003291255386)
(3861,1.3761219584484656)
(3881,1.3618374380141571)
(3901,1.3447863129000244)
(3921,1.3303796734357511)
(3941,1.3092007941953454)
(3961,1.2795259743664322)
(3981,1.2600244401354608)
(4001,1.2475688483095158)
(4021,1.216436001244421)
(4041,1.2059769079340896)
(4061,1.200726740025686)
(4081,1.196458670225435)
(4101,1.1853267042025397)
(4121,1.177289075887166)
(4141,1.1656628365005923)
(4161,1.1551532800325395)
(4181,1.1466390320624604)
(4201,1.1371544756137308)
(4221,1.1236550997019457)
(4241,1.1120707152272455)
(4261,1.0987609344957965)
(4281,1.0908939760799727)
(4301,1.0822239212954679)
(4321,1.0731688649365168)
(4341,1.0645220824816597)
(4361,1.056281725016438)
(4381,1.0404296395219295)
(4401,1.022941445808681)
(4421,1.0113855546196118)
(4441,1.0037021259595875)
(4461,0.9930010086659586)
(4481,0.9864499849859265)
(4501,0.9805152582767207)
(4521,0.9700443689157325)
(4541,0.9566725674751126)
(4561,0.9480822505009863)
(4581,0.9366944701003925)
(4601,0.9274614583216317)
(4621,0.9238069632110995)
(4641,0.9188757421614931)
(4661,0.9137997219588424)
(4681,0.9081721199308381)
(4701,0.8988583951310818)
(4721,0.8924551491186088)
(4741,0.8873906300133221)
(4761,0.8776822246197169)
(4781,0.8692972656121665)
(4801,0.8514846217857129)
(4821,0.8403020491135562)
(4841,0.832972264489946)
(4861,0.8251656974460055)
(4881,0.818494068751734)
(4901,0.8155741039768231)
(4921,0.8060875708861155)
(4941,0.7933887760598877)
(4961,0.7852220994000707)
(4981,0.7785785433867306)
};

    \addlegendentry{\scriptsize MSC \(N=64\)}

%%     \nextgroupplot[
%%       title={(b) Bias\label{fig:gaussian_bias}},
%%       title style={at={(0.5,-0.9)}},
%%       ymode=log,
%%       tuftelike,
%%       xlabel={
%%         \footnotesize
%%         Iteration
%%       },
%%       ylabel={
%%         \footnotesize
%%         Score Bias
%%       },
%%       xtick={1,2000,4000,6000,8000,10000},
%%       xticklabels={1,2k,4k,6k,8k,10k},
%%       ytick={0.01,1,100},
%%       %% ylabel near ticks,
%%       %% xlabel near ticks,
%%       log basis y=10,
%%       minor tick length=1.5pt,
%%       major tick length=2.5pt,
%%       every tick/.style={
%%         black,
%%         semithick,
%%       },
%%       xmin  =1,
%%       xmax  =5000,
%%       ymin  =0.01,
%%       ymax  =100,
%%       % smooth,
%%       % enlargelimits = true,
%%       % ymajorgrids,
%%       % yminorgrids,
%%       % xmajorgrids,
%%       width =4.0cm,
%%       height=3.5cm,
%%       % legend pos=north,
%%       legend style = { 
%%         legend columns = 2,
%%         draw           = none,
%%         at={(0,1)},
%%         anchor=south west,
%%         legend cell align={left},
%%     }]
%%     
\addplot[semithick,densely dotted,color=blue] coordinates {
(1,504.09160445656363)
(21,427.23437415528974)
(41,337.2467646281292)
(61,254.00477821992297)
(81,195.45500279007655)
(101,155.35983884280122)
(121,140.35769459498383)
(141,130.9930805245736)
(161,108.30402175764755)
(181,84.17409329900498)
(201,68.20214868283983)
(221,56.27290621309985)
(241,45.84792366741953)
(261,40.25780928486482)
(281,36.590009435439384)
(301,32.693193682033076)
(321,27.97581714424948)
(341,22.511157492128238)
(361,18.10086410861547)
(381,15.625791379332169)
(401,14.918076223025022)
(421,13.932023666486986)
(441,12.839035852192858)
(461,11.75644323707622)
(481,10.555409273978249)
(501,9.366563194085021)
(521,8.79358179864761)
(541,8.413523847150831)
(561,7.992263103222412)
(581,7.482695483255794)
(601,6.99506861028807)
(621,6.347601393404788)
(641,5.637542672161834)
(661,5.139064823075893)
(681,4.673581251941731)
(701,4.2399678623220565)
(721,3.6930760460074983)
(741,3.3172896076625222)
(761,3.055146058647883)
(781,2.8604511921186146)
(801,2.7200783634898364)
(821,2.543938509754454)
(841,2.3959581246119255)
(861,2.2714971857467305)
(881,2.1304847591950704)
(901,2.0164497064411258)
(921,1.9413833777703167)
(941,1.8301471500223228)
(961,1.7469765496673533)
(981,1.684858618144965)
(1001,1.6304435704428468)
(1021,1.5826146922195627)
(1041,1.505457013063275)
(1061,1.4445226915586722)
(1081,1.3828433082025133)
(1101,1.3250024874771436)
(1121,1.275951800371494)
(1141,1.2101906009260932)
(1161,1.1561641672130123)
(1181,1.106907861824832)
(1201,1.0583021469045666)
(1221,1.0110455667322973)
(1241,0.9799540810736357)
(1261,0.9296533033960644)
(1281,0.8764515108446661)
(1301,0.8392594955312943)
(1321,0.7979837477885268)
(1341,0.7583078790158375)
(1361,0.7218926926776201)
(1381,0.6836863544767918)
(1401,0.6421302501331759)
(1421,0.6106906315575829)
(1441,0.5905188308709315)
(1461,0.5703377708590642)
(1481,0.5497356197664369)
(1501,0.5251221250675029)
(1521,0.5000880572635257)
(1541,0.47620086407113876)
(1561,0.45373121842862046)
(1581,0.43459628651005633)
(1601,0.4180827611039226)
(1621,0.39841451120023397)
(1641,0.37827778252999644)
(1661,0.35808951448216253)
(1681,0.34150028718583575)
(1701,0.3288291400736564)
(1721,0.31980631598927795)
(1741,0.3054012845609321)
(1761,0.29484974023061783)
(1781,0.28719803606501426)
(1801,0.27496240827433205)
(1821,0.2659654258768127)
(1841,0.25674413709292837)
(1861,0.24918313032779563)
(1881,0.24127049223806626)
(1901,0.23220012667529333)
(1921,0.22424015745161177)
(1941,0.21463917394329368)
(1961,0.20639768423823868)
(1981,0.19703289193357176)
(2001,0.19260859968549676)
(2021,0.18511096482061548)
(2041,0.1813734273420164)
(2061,0.17657414568443985)
(2081,0.17037676236350113)
(2101,0.1607588852374664)
(2121,0.15483493402098186)
(2141,0.14967256793767353)
(2161,0.14572385523841133)
(2181,0.14233451757295662)
(2201,0.13968759980991433)
(2221,0.13663423868716254)
(2241,0.13364502549634236)
(2261,0.1280190526125908)
(2281,0.12404374794894418)
(2301,0.12019341105934823)
(2321,0.11519754410872951)
(2341,0.11327541064854348)
(2361,0.1111048717180297)
(2381,0.10670024102538311)
(2401,0.10533144003773828)
(2421,0.10522801522363961)
(2441,0.10357840217998578)
(2461,0.10137966583577755)
(2481,0.09832328869039138)
(2501,0.09590589691236498)
(2521,0.09305157045013725)
(2541,0.08969372080175496)
(2561,0.08650987198381892)
(2581,0.08434528384539279)
(2601,0.08003017500867586)
(2621,0.07784157758682993)
(2641,0.07508364008269489)
(2661,0.0731469258549528)
(2681,0.07153081472151421)
(2701,0.06936669310154271)
(2721,0.06668587164479364)
(2741,0.063646590929139)
(2761,0.06033653436162468)
(2781,0.05728109059552163)
(2801,0.05437942010585356)
(2821,0.05256175495713135)
(2841,0.05020304395218364)
(2861,0.048061674074761966)
(2881,0.046901613014397735)
(2901,0.04501109405995157)
(2921,0.0427526980572274)
(2941,0.041358006878338675)
(2961,0.03985104510088342)
(2981,0.03948082103101008)
(3001,0.037271998904816264)
(3021,0.03666919691774838)
(3041,0.03483524042485241)
(3061,0.032773476455260714)
(3081,0.03159484565016279)
(3101,0.030849922323554886)
(3121,0.03028519034937411)
(3141,0.029114011392599434)
(3161,0.028013106244652614)
(3181,0.027259145212949004)
(3201,0.02635291696435911)
(3221,0.02567420578380978)
(3241,0.024172266322657447)
(3261,0.023293236452233034)
(3281,0.02122236622374271)
(3301,0.02025284412118169)
(3321,0.019812089414892754)
(3341,0.02004041838455889)
(3361,0.019713080497064064)
(3381,0.018746141616771354)
(3401,0.017602501334190435)
(3421,0.01714041182851393)
(3441,0.017159602795727875)
(3461,0.017583456676805196)
(3481,0.017120342892973747)
(3501,0.01587526860533739)
(3521,0.01588174172250198)
(3541,0.015413757808433572)
(3561,0.01801406349081351)
(3581,0.016912266856637746)
(3601,0.016243346732948735)
(3621,0.015953901993913625)
(3641,0.016589316260346456)
(3661,0.016332125153530858)
(3681,0.015845807818958717)
(3701,0.014816693234251251)
(3721,0.016160355245350533)
(3741,0.017435821512553865)
(3761,0.01645987157535248)
(3781,0.016627582553149174)
(3801,0.015302738076309063)
(3821,0.015056810352158917)
(3841,0.014594434995426515)
(3861,0.014177024044511214)
(3881,0.013983856199475824)
(3901,0.01419066883275611)
(3921,0.015223291604225388)
(3941,0.016246070167240396)
(3961,0.015741866552898783)
(3981,0.016862273358153237)
(4001,0.015654318811891065)
(4021,0.01595948588288025)
(4041,0.016685934707500814)
(4061,0.015212635941344236)
(4081,0.014237779234790677)
(4101,0.014513923180172514)
(4121,0.014471922018180067)
(4141,0.016400474491185427)
(4161,0.01644777735170859)
(4181,0.01741638113442453)
(4201,0.016070022552519023)
(4221,0.015505390860517887)
(4241,0.016213408881269632)
(4261,0.01695131261814367)
(4281,0.016007153889682928)
(4301,0.017696164659588415)
(4321,0.016374132833719825)
(4341,0.015832776924951802)
(4361,0.016744783893715906)
(4381,0.018059637370970508)
(4401,0.016618767683666005)
(4421,0.015900660916973133)
(4441,0.015918611998426247)
(4461,0.015023295278013143)
(4481,0.015352096643158245)
(4501,0.015767566443845377)
(4521,0.01469983434903242)
(4541,0.015073498446346062)
(4561,0.015177688821819635)
(4581,0.0144645646646953)
(4601,0.015372434627625248)
(4621,0.013917357698092093)
(4641,0.014148326514991122)
(4661,0.013729198803284912)
(4681,0.013644412887977797)
(4701,0.01441549268192992)
(4721,0.013908917819356305)
(4741,0.013765606497711002)
(4761,0.014220198995144885)
(4781,0.014624824771112654)
(4801,0.013859709119361331)
(4821,0.013582861699564353)
(4841,0.014180881601899512)
(4861,0.01473775653498571)
(4881,0.013664264278394246)
(4901,0.014076304634765875)
(4921,0.013603776088397608)
(4941,0.013988446066432746)
(4961,0.014350769194544866)
(4981,0.014593612087474027)
};

%%     
\addplot[semithick,color=blue] coordinates {
(1,403.1593339785967)
(21,309.98524176426565)
(41,237.2267055205514)
(61,178.9414249158197)
(81,129.64982333704475)
(101,91.78051674955663)
(121,72.53620454956302)
(141,57.89693336789156)
(161,46.9843396495354)
(181,40.040711241715826)
(201,33.82607576207008)
(221,28.342704896920925)
(241,23.98777027255283)
(261,20.006189961390398)
(281,17.120187787320845)
(301,15.419275366609735)
(321,13.446820261094741)
(341,11.427782388338334)
(361,9.662464550429451)
(381,8.374881000758503)
(401,7.412922652250016)
(421,6.583356720430059)
(441,5.743025871613684)
(461,5.083670777458663)
(481,4.537197244744219)
(501,4.132969515003278)
(521,3.781893993819044)
(541,3.417172208807819)
(561,3.1500992912166383)
(581,2.8621328089012374)
(601,2.6142378643798057)
(621,2.386965807991726)
(641,2.201617648517461)
(661,2.0349655050976305)
(681,1.8693935446513557)
(701,1.7333076807298233)
(721,1.6193281612603654)
(741,1.5138248733718969)
(761,1.421972608711971)
(781,1.3164450727590271)
(801,1.2133704262758402)
(821,1.12856937004317)
(841,1.0574229312978993)
(861,0.9828886473102403)
(881,0.9104560279735112)
(901,0.8395903068380884)
(921,0.7767189890735177)
(941,0.7234651544751859)
(961,0.6801762986469718)
(981,0.6422445731229581)
(1001,0.6088264778112557)
(1021,0.5742161463870105)
(1041,0.5425110578603631)
(1061,0.5083525986019497)
(1081,0.4801557900705362)
(1101,0.45805031736643603)
(1121,0.4355737056159082)
(1141,0.4105421948515384)
(1161,0.3915935150786809)
(1181,0.37371506505560786)
(1201,0.35718964376656037)
(1221,0.3423182731551012)
(1241,0.32665114681960417)
(1261,0.31188690056943663)
(1281,0.29804342080086504)
(1301,0.28371302704785584)
(1321,0.2700597859038727)
(1341,0.25897318668437896)
(1361,0.24883490715382556)
(1381,0.23664469889476591)
(1401,0.22634084297879384)
(1421,0.21437401770587644)
(1441,0.20388941251713427)
(1461,0.19385771150180742)
(1481,0.1839390896617052)
(1501,0.1767692863208991)
(1521,0.16987476884305602)
(1541,0.1619113683336879)
(1561,0.15598866082575058)
(1581,0.14901836728127565)
(1601,0.1426301955674663)
(1621,0.13802291333822794)
(1641,0.13218940304744992)
(1661,0.12841794553329378)
(1681,0.12274709500459224)
(1701,0.11708792851743337)
(1721,0.11229840907604419)
(1741,0.10827094067459613)
(1761,0.10456787792763794)
(1781,0.09961995927169458)
(1801,0.09741355487541345)
(1821,0.09525434724596865)
(1841,0.09218981041141065)
(1861,0.08940227589630757)
(1881,0.08536942613486467)
(1901,0.08280269097366243)
(1921,0.07964112776947474)
(1941,0.07674801013273931)
(1961,0.07294496847381862)
(1981,0.07075523971326766)
(2001,0.06820975024606056)
(2021,0.06656536093149887)
(2041,0.0638114578358342)
(2061,0.06229304611635269)
(2081,0.05953752708215683)
(2101,0.05752916235027937)
(2121,0.05553343531019762)
(2141,0.052141712174151936)
(2161,0.05071517742967212)
(2181,0.04889304538104698)
(2201,0.047420136043623565)
(2221,0.04509664302488561)
(2241,0.04326854284183556)
(2261,0.04425036118162337)
(2281,0.043081957003957325)
(2301,0.04136701310771658)
(2321,0.0403252913924393)
(2341,0.03829877833614911)
(2361,0.03734877764184469)
(2381,0.036186559441979904)
(2401,0.03511862186178085)
(2421,0.03438986483606331)
(2441,0.033359145602215035)
(2461,0.03392325122099937)
(2481,0.0332735330757472)
(2501,0.03156964060015272)
(2521,0.03043929456929834)
(2541,0.02956081698667037)
(2561,0.02944985532873813)
(2581,0.02903398745223354)
(2601,0.02809903479024469)
(2621,0.027434686394138322)
(2641,0.0270081801712895)
(2661,0.02601665740598908)
(2681,0.025209008710248115)
(2701,0.025402763905536616)
(2721,0.025459614467402327)
(2741,0.025656029859470048)
(2761,0.02550829399352108)
(2781,0.024692305245312303)
(2801,0.024212900200415007)
(2821,0.02371948393391148)
(2841,0.023310469795775196)
(2861,0.02330274379248838)
(2881,0.022823360645750357)
(2901,0.023219721990023732)
(2921,0.02191003295392714)
(2941,0.022373860923162434)
(2961,0.021427251254230856)
(2981,0.020170549555505048)
(3001,0.020244076074724365)
(3021,0.02042763090029297)
(3041,0.02128403861054612)
(3061,0.021905338862036012)
(3081,0.02266176544754251)
(3101,0.022217531643816787)
(3121,0.022202691888310137)
(3141,0.022294205610866466)
(3161,0.023746183635599646)
(3181,0.023019177621727838)
(3201,0.023277884643385212)
(3221,0.022999213050339078)
(3241,0.022404069313926585)
(3261,0.021828839642499546)
(3281,0.022090643936050678)
(3301,0.022625979970458725)
(3321,0.021882241910271362)
(3341,0.02030834427474647)
(3361,0.021627048381688413)
(3381,0.02088878165816195)
(3401,0.021046862637196406)
(3421,0.01996946718149523)
(3441,0.02001887125933722)
(3461,0.01948669303822455)
(3481,0.01953266779517255)
(3501,0.018763299580645414)
(3521,0.01882289761695858)
(3541,0.01873876353019098)
(3561,0.018522368825324896)
(3581,0.020332769621257633)
(3601,0.019876289564601614)
(3621,0.019472689359859296)
(3641,0.019257583277678698)
(3661,0.020597977688299903)
(3681,0.020972513941959897)
(3701,0.019494287134665344)
(3721,0.01910525063558052)
(3741,0.019168468472421048)
(3761,0.019498002456139127)
(3781,0.02006242957350639)
(3801,0.01935097340987867)
(3821,0.01976197557181775)
(3841,0.02027813110178579)
(3861,0.020144776693950726)
(3881,0.018867274984560405)
(3901,0.01946421443566405)
(3921,0.019075176442777588)
(3941,0.01858524163549374)
(3961,0.019318565401662623)
(3981,0.019014766272818395)
(4001,0.02005726924806016)
(4021,0.01927106019643885)
(4041,0.019140381070856017)
(4061,0.019868075295866986)
(4081,0.020753561872573593)
(4101,0.02026407492403084)
(4121,0.019640880656216213)
(4141,0.018602049116366363)
(4161,0.017890119789185553)
(4181,0.01908758714289069)
(4201,0.019567437297579704)
(4221,0.01992050134232569)
(4241,0.01931868867857274)
(4261,0.0198507571563684)
(4281,0.019278927050901232)
(4301,0.018763511176028484)
(4321,0.018973252215794857)
(4341,0.018513986214244903)
(4361,0.01878954843153778)
(4381,0.01862493945219386)
(4401,0.017683474120755067)
(4421,0.017551760572913026)
(4441,0.016091048445019553)
(4461,0.015146750681132632)
(4481,0.015767395545280918)
(4501,0.0169393595246673)
(4521,0.016076575424868615)
(4541,0.017122877695683495)
(4561,0.018937027588048213)
(4581,0.01884839660179415)
(4601,0.017904474494245856)
(4621,0.017635290541989233)
(4641,0.017283252059444164)
(4661,0.01671290564820424)
(4681,0.01806435375294779)
(4701,0.018099993950648013)
(4721,0.018704232035058804)
(4741,0.01789818898345755)
(4761,0.017180713723960496)
(4781,0.017298032859063554)
(4801,0.01863924383090621)
(4821,0.01859449513183987)
(4841,0.01890779386083484)
(4861,0.018766300706100945)
(4881,0.01801465776721756)
(4901,0.018674018224675435)
(4921,0.018601461800768698)
(4941,0.017226430223607498)
(4961,0.0171076737710643)
(4981,0.0169493124800418)
};

%%     
\addplot[semithick,densely dotted,color=red] coordinates {
(1,264.43832604537226)
(21,264.4844631596923)
(41,276.72868520573695)
(61,222.34333302899168)
(81,173.85509514734545)
(101,146.23721087912367)
(121,131.33786230928538)
(141,125.15265761117942)
(161,115.81628625772808)
(181,107.38580206057647)
(201,102.69237577650408)
(221,95.47802527363525)
(241,91.51343184250359)
(261,97.58447307690666)
(281,79.84655273693002)
(301,65.23034464464111)
(321,59.32336676529326)
(341,58.19851811207399)
(361,50.45118046043224)
(381,38.6098826831277)
(401,37.15076364016588)
(421,35.19457336032286)
(441,32.686518120377855)
(461,31.36809778226128)
(481,31.108495649392907)
(501,29.082274191764846)
(521,27.83426245043475)
(541,25.582604582873348)
(561,21.87006564310765)
(581,19.899323649217664)
(601,18.461688472645935)
(621,14.407103081630149)
(641,13.311320090603836)
(661,12.750781761163525)
(681,11.1548518991534)
(701,10.009429070401488)
(721,9.975133844334968)
(741,9.89696074288613)
(761,9.82048723954595)
(781,9.835895846064565)
(801,9.652605523039238)
(821,9.034805671705909)
(841,8.498799826917772)
(861,8.377235604803788)
(881,8.13992559950643)
(901,7.653798344002586)
(921,7.237923743811134)
(941,6.574628201718868)
(961,6.285288130434494)
(981,6.126443414820998)
(1001,6.07987628437865)
(1021,5.28584480023812)
(1041,4.499616451517825)
(1061,4.167102648995721)
(1081,3.8585763644256543)
(1101,3.764400236047805)
(1121,3.7230456017514926)
(1141,3.6538924339787426)
(1161,3.5257590439615023)
(1181,3.395881632936545)
(1201,3.3024389774486114)
(1221,2.729268759780223)
(1241,2.4480836284015384)
(1261,2.3566020938476058)
(1281,2.2820951619252345)
(1301,2.225596355458892)
(1321,2.1562182324040884)
(1341,2.0834638925675764)
(1361,2.011089351281796)
(1381,1.8976981308042036)
(1401,1.7659687249739682)
(1421,1.56520562498506)
(1441,1.454043911399093)
(1461,1.3902981443068816)
(1481,1.339734612457312)
(1501,1.2986871842698702)
(1521,1.2413361099441942)
(1541,1.2184937233947215)
(1561,1.1606870179529736)
(1581,1.1301312869840814)
(1601,1.0971749086603224)
(1621,1.0687882392404053)
(1641,1.0535064605705524)
(1661,1.0097529689085907)
(1681,0.9938056595277094)
(1701,0.9794990461496823)
(1721,0.9400778954203473)
(1741,0.8601680236565648)
(1761,0.8295046847209454)
(1781,0.761300077407763)
(1801,0.742829833981347)
(1821,0.7211017090668925)
(1841,0.6933370692467763)
(1861,0.6698096762816831)
(1881,0.6434897954499861)
(1901,0.6358509350313185)
(1921,0.614815158764657)
(1941,0.5151219840480447)
(1961,0.4669295220120989)
(1981,0.4501443894634647)
(2001,0.4417981723343264)
(2021,0.4144016137658372)
(2041,0.4073131996773057)
(2061,0.3865392853321793)
(2081,0.37660067095908045)
(2101,0.35965763142991664)
(2121,0.3700150712793865)
(2141,0.37063080631703926)
(2161,0.3666813309581267)
(2181,0.35256049230282366)
(2201,0.3346253417709887)
(2221,0.31501156691397436)
(2241,0.29058658132155685)
(2261,0.28631777547111265)
(2281,0.29322445047858114)
(2301,0.2898267248255313)
(2321,0.28323358677993554)
(2341,0.2725194191642046)
(2361,0.2609122882105981)
(2381,0.25761614518723525)
(2401,0.2532455789958843)
(2421,0.25934847638968644)
(2441,0.24318012158614666)
(2461,0.21967538178695586)
(2481,0.20838964806291324)
(2501,0.2026436144483339)
(2521,0.19494488759988554)
(2541,0.19347560311658255)
(2561,0.18899812410651476)
(2581,0.19031855187066118)
(2601,0.18181583220641712)
(2621,0.17160788076681718)
(2641,0.16592216168957896)
(2661,0.15516774034866432)
(2681,0.15269667943028176)
(2701,0.14772858646541304)
(2721,0.14666148636548698)
(2741,0.14020417493457493)
(2761,0.13987892001588417)
(2781,0.14015570024241797)
(2801,0.13520669746178393)
(2821,0.13070487363147681)
(2841,0.12083710955262551)
(2861,0.12290061618199252)
(2881,0.11883975994062027)
(2901,0.11459096471708269)
(2921,0.11095047574696615)
(2941,0.11355532601647877)
(2961,0.10486731269722391)
(2981,0.09487741699004021)
(3001,0.08063861675232697)
(3021,0.08860842719478582)
(3041,0.09377730576861598)
(3061,0.09563278286459503)
(3081,0.09454439515175583)
(3101,0.09132933186780706)
(3121,0.08177187640392154)
(3141,0.08340535914754263)
(3161,0.07204775494947321)
(3181,0.07022028196949161)
(3201,0.07911673158803453)
(3221,0.07462786027870932)
(3241,0.06842897856868446)
(3261,0.07015043707388212)
(3281,0.06306221350439437)
(3301,0.059373554991299135)
(3321,0.057732187324516446)
(3341,0.05694645681521505)
(3361,0.06125296685852606)
(3381,0.05947902524493733)
(3401,0.05814385596080715)
(3421,0.05996403821012515)
(3441,0.05416566097225005)
(3461,0.04942294059038174)
(3481,0.047509785144896524)
(3501,0.044952013575107724)
(3521,0.04330871842751925)
(3541,0.040652157853408276)
(3561,0.039844130007752)
(3581,0.039902162521852615)
(3601,0.03592214086496691)
(3621,0.03949854046476306)
(3641,0.03851439724618905)
(3661,0.029876887161474852)
(3681,0.03481243761544229)
(3701,0.03366638005899991)
(3721,0.030629749529298034)
(3741,0.03306354355441897)
(3761,0.035144615073827615)
(3781,0.036819707857621536)
(3801,0.03140108457478717)
(3821,0.023689859031136642)
(3841,0.02568924368071939)
(3861,0.027535014001878175)
(3881,0.02697643358284809)
(3901,0.02477854516147676)
(3921,0.02223289102365858)
(3941,0.024157747015836506)
(3961,0.02682084371617341)
(3981,0.022522649944792632)
(4001,0.023996067933975136)
(4021,0.021211419004469406)
(4041,0.020569607639894676)
(4061,0.01958678263830925)
(4081,0.02692702567654586)
(4101,0.021131310384124878)
(4121,0.024466083478974096)
(4141,0.02383956547145982)
(4161,0.02730964341864367)
(4181,0.021786242209058296)
(4201,0.02115401769497282)
(4221,0.02412239632913778)
(4241,0.0201767810112241)
(4261,0.02027661065218508)
(4281,0.023250011108585204)
(4301,0.020981696461615313)
(4321,0.019285183665599524)
(4341,0.023687748593661743)
(4361,0.02177469087229392)
(4381,0.020550017085540905)
(4401,0.024333838809870074)
(4421,0.01969018358902884)
(4441,0.015852086512477357)
(4461,0.020907003281810787)
(4481,0.019277184811289233)
(4501,0.01772815533863252)
(4521,0.017154662814609783)
(4541,0.01693042344034773)
(4561,0.020611082009001913)
(4581,0.02052167555735324)
(4601,0.01688999679919331)
(4621,0.017032516676165576)
(4641,0.0170454321071557)
(4661,0.015224406596108301)
(4681,0.015917014601272074)
(4701,0.01611728556312021)
(4721,0.01522102782014286)
(4741,0.016339106950445912)
(4761,0.020532787702670544)
(4781,0.01675113693192935)
(4801,0.016767486940151087)
(4821,0.016686803340714994)
(4841,0.014898004377502595)
(4861,0.01621393154686864)
(4881,0.017591818023961488)
(4901,0.019556281312522277)
(4921,0.015778990929924956)
(4941,0.01824091545199683)
(4961,0.01756647762275955)
(4981,0.016224433830432896)
};

%%     
\addplot[semithick,color=red] coordinates {
(1,270.375527651362)
(21,208.21271869509215)
(41,158.91123063319344)
(61,131.68485389624146)
(81,119.0651068287381)
(101,105.45692114607428)
(121,96.10687919049471)
(141,84.24919246149969)
(161,76.80034836178669)
(181,73.71060086183614)
(201,56.75976583457246)
(221,51.29247986548725)
(241,46.14495639624985)
(261,41.67288658561438)
(281,36.53747736172078)
(301,34.16240623295442)
(321,28.136459169138007)
(341,25.70385197812485)
(361,23.278454790155198)
(381,21.159448537605027)
(401,20.09909126042063)
(421,19.55482330651891)
(441,17.862850871478816)
(461,17.05101807099854)
(481,15.842993020545773)
(501,12.356394924010386)
(521,11.256982038840677)
(541,10.786461262313761)
(561,10.313930993655251)
(581,9.744826824656691)
(601,8.930563155128718)
(621,8.095501022966003)
(641,7.6041450407535045)
(661,7.2171832541342695)
(681,6.866745647480146)
(701,6.507870725731923)
(721,6.06586346460326)
(741,5.652562407928098)
(761,4.971252107904693)
(781,4.5241299404949284)
(801,4.337882982498736)
(821,4.167246937855213)
(841,4.061946421682138)
(861,3.6212896342587806)
(881,3.190148951167666)
(901,2.821408273319257)
(921,2.6489248214899344)
(941,2.5891406081481922)
(961,2.4271649168665403)
(981,2.349665670137547)
(1001,2.2571967831620814)
(1021,2.1662471806263417)
(1041,2.071042762336692)
(1061,1.9726531506887617)
(1081,1.8528756480521396)
(1101,1.7717461064817615)
(1121,1.7066502854851568)
(1141,1.6688090463376648)
(1161,1.6366019460193704)
(1181,1.5495188786843144)
(1201,1.5758300599394124)
(1221,1.5337354826850103)
(1241,1.4889178353693133)
(1261,1.4347928668225338)
(1281,1.4117601579348933)
(1301,1.3870324357624462)
(1321,1.3686420136150452)
(1341,1.3687951821010822)
(1361,1.3111176577911947)
(1381,1.2935955248575928)
(1401,1.2786135113921977)
(1421,1.2152967844236924)
(1441,1.1389697494900803)
(1461,1.1464192449119501)
(1481,1.0784610315098462)
(1501,1.0322027233471882)
(1521,0.9488641415444814)
(1541,0.9038013959406088)
(1561,0.9287031664714922)
(1581,0.8755965667015287)
(1601,0.8342772053380668)
(1621,0.8172043713189219)
(1641,0.7773573667967006)
(1661,0.6988460036426272)
(1681,0.6662939016676748)
(1701,0.6240889483732499)
(1721,0.6144007133905243)
(1741,0.5940313479087689)
(1761,0.5709137126693624)
(1781,0.5295336982865766)
(1801,0.4982823852107371)
(1821,0.497577215148274)
(1841,0.4715895427505335)
(1861,0.46416875575425537)
(1881,0.4245949272660405)
(1901,0.3989772201020418)
(1921,0.3804412663394925)
(1941,0.35919798104518175)
(1961,0.3495074072688029)
(1981,0.3313066224732202)
(2001,0.31516632757821333)
(2021,0.31989027378250734)
(2041,0.3160878136303775)
(2061,0.2944668994338691)
(2081,0.289749583768314)
(2101,0.2855941382964129)
(2121,0.2654660710874225)
(2141,0.26891950517556334)
(2161,0.24924899729055844)
(2181,0.23166385775406495)
(2201,0.21937401518787297)
(2221,0.1988024183464876)
(2241,0.171958269063756)
(2261,0.1608289707731907)
(2281,0.16506348738005747)
(2301,0.1795912457780548)
(2321,0.17920344974011593)
(2341,0.17940189102283868)
(2361,0.17332362631398507)
(2381,0.1548900098426681)
(2401,0.149220831864217)
(2421,0.14757826655057454)
(2441,0.13339703735181693)
(2461,0.12238033307001348)
(2481,0.12387238797048411)
(2501,0.11006554201671245)
(2521,0.12134464317612836)
(2541,0.10126613470662107)
(2561,0.10482438794607535)
(2581,0.10336517633823226)
(2601,0.089015005213495)
(2621,0.09407576181879329)
(2641,0.09722034461942769)
(2661,0.10898477490176564)
(2681,0.1063667091762684)
(2701,0.11325915167776104)
(2721,0.10138197601111451)
(2741,0.10139094571782528)
(2761,0.09900576582912621)
(2781,0.0990784587663597)
(2801,0.10647865818861785)
(2821,0.08473051103315642)
(2841,0.06249651514574432)
(2861,0.0714517161184591)
(2881,0.07074163505180865)
(2901,0.07503062292458838)
(2921,0.07291646789961725)
(2941,0.06083091722447153)
(2961,0.06478854355841199)
(2981,0.06764496623181876)
(3001,0.07177922789241889)
(3021,0.060049692214310835)
(3041,0.06109187676007404)
(3061,0.056577334891131516)
(3081,0.04902919202397348)
(3101,0.04171898584322959)
(3121,0.045227193903684924)
(3141,0.04154716792553983)
(3161,0.04157305540054393)
(3181,0.030326575188141677)
(3201,0.0377150434498628)
(3221,0.03451038021127942)
(3241,0.034620108607505834)
(3261,0.04487581923186106)
(3281,0.03451533188127008)
(3301,0.030630227869820177)
(3321,0.03079161126421551)
(3341,0.0376052610078459)
(3361,0.05276396659869137)
(3381,0.049469865318446306)
(3401,0.041797300857723756)
(3421,0.046906632421167146)
(3441,0.04109861434825275)
(3461,0.036327448175980055)
(3481,0.03248428771651981)
(3501,0.023079013510492286)
(3521,0.019432423383435694)
(3541,0.022323199099222188)
(3561,0.02202548886173436)
(3581,0.02347057464764667)
(3601,0.021657356795090166)
(3621,0.01992560398691065)
(3641,0.022872554684912286)
(3661,0.0191323003920314)
(3681,0.029495222146574872)
(3701,0.026722607532556477)
(3721,0.020385530144004266)
(3741,0.018607604096732913)
(3761,0.022544515657184568)
(3781,0.02218726577870849)
(3801,0.019830123385808092)
(3821,0.01866028846727243)
(3841,0.01680463648296804)
(3861,0.01790661080341583)
(3881,0.018342615198585967)
(3901,0.019065589742327284)
(3921,0.019625662313248426)
(3941,0.017471433036635203)
(3961,0.0184616076961774)
(3981,0.01790208818563549)
(4001,0.01704717528854734)
(4021,0.01656319571445476)
(4041,0.020416218206967165)
(4061,0.022600707858525443)
(4081,0.022165997917646195)
(4101,0.018807436122243254)
(4121,0.021007900407736937)
(4141,0.015415579739910156)
(4161,0.015850418635125815)
(4181,0.01622416709243304)
(4201,0.022836703426125305)
(4221,0.018833255862573568)
(4241,0.0166148860180467)
(4261,0.01630571045510527)
(4281,0.019632749851836707)
(4301,0.017725909120832727)
(4321,0.015345885616752196)
(4341,0.020997057181632854)
(4361,0.015722772009217935)
(4381,0.015068426809441458)
(4401,0.016302446678048473)
(4421,0.01612062416463523)
(4441,0.015160807031032912)
(4461,0.017568862681208337)
(4481,0.016447686431186232)
(4501,0.01682571824959116)
(4521,0.015605498458610655)
(4541,0.014881050617723261)
(4561,0.015004342743212945)
(4581,0.016487973488932942)
(4601,0.016408755892320877)
(4621,0.015218987747043798)
(4641,0.015937488042908083)
(4661,0.016846204849722657)
(4681,0.015167081490590154)
(4701,0.016055281604957842)
(4721,0.015474701041608857)
(4741,0.015332844593384111)
(4761,0.014960173722981335)
(4781,0.014162173803047115)
(4801,0.01512934556311907)
(4821,0.01615113165316791)
(4841,0.015048957808576971)
(4861,0.014818927959437116)
(4881,0.015840684922416973)
(4901,0.015370074121163117)
(4921,0.016275413859443266)
(4941,0.015615466514329735)
(4961,0.015265581163774668)
(4981,0.0170181229335687)
};

%%     
\addplot[semithick,densely dotted,color=teal] coordinates {
(1,272.50026853330485)
(21,257.24421378658565)
(41,232.1912681580558)
(61,257.20552166822495)
(81,234.84739040269142)
(101,160.64743145544782)
(121,119.07742317339645)
(141,103.70290187612797)
(161,99.89252361322244)
(181,93.04871854027212)
(201,85.69069120971461)
(221,70.36585998844764)
(241,60.97212603410277)
(261,45.33002558861442)
(281,39.5608895073895)
(301,36.456097763512865)
(321,28.740933356026723)
(341,25.52901363611602)
(361,23.20350903638571)
(381,21.720407649619688)
(401,18.779885128599172)
(421,17.421576435825536)
(441,16.1074763580674)
(461,15.515290196170927)
(481,15.07134730482327)
(501,13.18311779981887)
(521,12.209838378822967)
(541,10.841587155228076)
(561,10.594182840844793)
(581,10.234371088024059)
(601,9.317030240428812)
(621,8.61517723811464)
(641,7.684528171537224)
(661,7.2335695785986065)
(681,6.529331754476952)
(701,5.994576815941841)
(721,5.796545142602509)
(741,5.724610078136828)
(761,5.202912889904439)
(781,4.502994442418131)
(801,4.082379398992365)
(821,3.7897032125835146)
(841,3.672661558613606)
(861,3.598013516687936)
(881,3.5065554235391154)
(901,3.429592628596078)
(921,3.318034719944905)
(941,3.2782723849545037)
(961,3.263329615297124)
(981,3.211646418938383)
(1001,3.2820424382682827)
(1021,2.698049288164986)
(1041,2.4336480036463213)
(1061,2.320826796106181)
(1081,2.181413996449601)
(1101,2.071182758207722)
(1121,1.9976010007841591)
(1141,1.910399796010578)
(1161,1.7980486644451186)
(1181,1.7599992524699097)
(1201,1.7017163168613276)
(1221,1.610905563076833)
(1241,1.525962416259149)
(1261,1.4613151578361527)
(1281,1.4314613934452156)
(1301,1.4082011608645204)
(1321,1.4057227345976617)
(1341,1.4026845782322097)
(1361,1.3843476457452615)
(1381,1.3581401263894912)
(1401,1.294317148755814)
(1421,1.271393884484822)
(1441,1.2807044332876896)
(1461,1.2557527400806001)
(1481,1.1679376675179483)
(1501,1.1648406419182724)
(1521,1.1481286554634789)
(1541,1.1683963043647285)
(1561,1.1782140525178306)
(1581,1.1483990684627132)
(1601,1.1086648877459175)
(1621,1.0343191968237648)
(1641,1.0270163902092404)
(1661,1.0145975667699156)
(1681,0.9946707703757138)
(1701,0.9693038478400382)
(1721,0.9595853487477191)
(1741,0.9425597751630845)
(1761,0.9033469100301288)
(1781,0.880020554090476)
(1801,0.8332170869264199)
(1821,0.8104120256881824)
(1841,0.7989833958086618)
(1861,0.7311166186621492)
(1881,0.7073039826298748)
(1901,0.6557725151995247)
(1921,0.624393087515341)
(1941,0.5910451186366608)
(1961,0.5804673223647014)
(1981,0.5702575765067652)
(2001,0.5462225673586819)
(2021,0.5365996390005503)
(2041,0.5224787235433391)
(2061,0.5143060521171267)
(2081,0.4754292541005507)
(2101,0.47808522964669486)
(2121,0.46869692150169173)
(2141,0.46017940478225533)
(2161,0.4223456385532547)
(2181,0.45173475699648125)
(2201,0.412717425732446)
(2221,0.4142280341185035)
(2241,0.3779518303398344)
(2261,0.38091391755552984)
(2281,0.3419977550875489)
(2301,0.3436858911697851)
(2321,0.3231562806288018)
(2341,0.32789173358026585)
(2361,0.3385872313427737)
(2381,0.3258286779721253)
(2401,0.303221932165286)
(2421,0.28087948181531464)
(2441,0.27440782080138193)
(2461,0.2707842061018173)
(2481,0.26123673174487466)
(2501,0.24671918366532447)
(2521,0.2435353726756341)
(2541,0.22448818581763524)
(2561,0.22424095194098717)
(2581,0.22493424581292382)
(2601,0.22352868186769492)
(2621,0.2068850283288644)
(2641,0.20919043772033613)
(2661,0.19314073950005334)
(2681,0.18679970681584634)
(2701,0.18343534260030286)
(2721,0.1767979905037872)
(2741,0.17300296224519207)
(2761,0.15296189403951954)
(2781,0.1545374055364307)
(2801,0.1436572655064877)
(2821,0.13919012426199903)
(2841,0.1363036656355456)
(2861,0.13970348694123494)
(2881,0.13978628426563497)
(2901,0.1248411790822404)
(2921,0.11743783659662821)
(2941,0.12028747299695139)
(2961,0.10903096058986471)
(2981,0.09172736633587089)
(3001,0.09518326000299959)
(3021,0.10147260941254166)
(3041,0.09934654076677102)
(3061,0.10932798548696848)
(3081,0.09511466789592377)
(3101,0.09858980550511987)
(3121,0.08024070148404219)
(3141,0.07677281218459388)
(3161,0.08992487405738542)
(3181,0.08035810283077297)
(3201,0.09502199925172583)
(3221,0.07021331185131796)
(3241,0.07831788367080642)
(3261,0.07297081334165316)
(3281,0.07543253861593664)
(3301,0.05957311178213551)
(3321,0.05856383420411622)
(3341,0.055072419943621706)
(3361,0.06410398693437755)
(3381,0.04973856904324476)
(3401,0.05813952209865209)
(3421,0.04881355375944365)
(3441,0.038575985656456674)
(3461,0.03390140303523509)
(3481,0.030211493606370435)
(3501,0.04317394578170752)
(3521,0.037644337885660434)
(3541,0.04468167718373228)
(3561,0.03282540868826195)
(3581,0.04631225553095447)
(3601,0.036706802406011804)
(3621,0.036729284273118085)
(3641,0.03774402355098867)
(3661,0.038424317876262576)
(3681,0.03909779729346942)
(3701,0.034228985356610724)
(3721,0.03262699448298999)
(3741,0.02953483835415635)
(3761,0.028702323376135545)
(3781,0.03696887176630771)
(3801,0.028259861097982328)
(3821,0.03172020144671912)
(3841,0.020481357294441154)
(3861,0.028875954060957745)
(3881,0.029714965561185386)
(3901,0.02543204267917285)
(3921,0.027546163725394153)
(3941,0.03762704235941588)
(3961,0.02590270936618704)
(3981,0.036226503049035796)
(4001,0.03041140391027331)
(4021,0.02487913377829854)
(4041,0.02913799618337066)
(4061,0.03161711218332705)
(4081,0.028627555167297827)
(4101,0.02209815439476108)
(4121,0.02182112621087552)
(4141,0.01942310872590165)
(4161,0.023165105507790884)
(4181,0.03101720378926636)
(4201,0.03522455849309336)
(4221,0.030524583606293045)
(4241,0.017533739762673875)
(4261,0.028184647833084116)
(4281,0.02178389446381966)
(4301,0.025616990622175113)
(4321,0.025529097974448998)
(4341,0.02096903842331533)
(4361,0.02516442529833718)
(4381,0.020332696244584978)
(4401,0.02199852299751682)
(4421,0.020101839773072932)
(4441,0.028162150781799507)
(4461,0.02177036877205399)
(4481,0.021631547336251417)
(4501,0.021256074657903915)
(4521,0.020129079664383508)
(4541,0.021957097150374073)
(4561,0.026255970472763485)
(4581,0.020342782984680452)
(4601,0.02084084794453203)
(4621,0.034549132577917474)
(4641,0.019944612033386745)
(4661,0.023227327742974792)
(4681,0.01720042195067037)
(4701,0.024257436567725145)
(4721,0.01849069158705728)
(4741,0.02017099844250375)
(4761,0.026015711735466955)
(4781,0.021647400159506093)
(4801,0.022898300885266044)
(4821,0.023018557519732953)
(4841,0.019655540605690115)
(4861,0.02142809069562427)
(4881,0.01885964656462128)
(4901,0.020116749535717203)
(4921,0.0186786532117593)
(4941,0.01759302195224053)
(4961,0.037777555480214324)
(4981,0.032441901793699365)
};

%%     
\addplot[semithick,color=teal] coordinates {
(1,269.47403144394667)
(21,237.77333180546077)
(41,200.10735318989362)
(61,170.26814355010362)
(81,152.66380516255492)
(101,131.30284673931686)
(121,120.01085405188984)
(141,105.44546594798362)
(161,89.76374321655427)
(181,82.50306434987726)
(201,74.47537519600806)
(221,72.75709320447554)
(241,66.97113787627175)
(261,65.53386478666154)
(281,67.41009780272134)
(301,67.81471360427277)
(321,62.68746969789218)
(341,58.773669802554615)
(361,54.82028986306424)
(381,51.32684723612252)
(401,48.16183732774241)
(421,41.316692580463744)
(441,35.65470007992988)
(461,30.498364431370863)
(481,28.14211181557085)
(501,25.214073031038723)
(521,22.318068586945586)
(541,21.25072919147248)
(561,18.827188724720777)
(581,17.177687768049793)
(601,16.340857474030102)
(621,14.653889827563097)
(641,13.442104878819066)
(661,12.383668229969942)
(681,10.900424038939843)
(701,9.602916833330625)
(721,8.931176172128618)
(741,8.278488755765022)
(761,7.483780239438582)
(781,6.863067660174499)
(801,6.411251376622635)
(821,5.891970510022492)
(841,5.774008148152664)
(861,5.84149983130962)
(881,5.5803145318263265)
(901,5.643137494130715)
(921,5.567268573802394)
(941,5.240183670558019)
(961,4.048477994978398)
(981,3.3509272310829483)
(1001,2.9874873248423155)
(1021,2.8229283933994944)
(1041,2.7839890880796423)
(1061,2.7037864768753344)
(1081,2.621410686737932)
(1101,2.5869312682662997)
(1121,2.536496179873763)
(1141,2.5010659225720886)
(1161,2.4105103334222155)
(1181,2.3488123011964683)
(1201,2.3020170686006463)
(1221,2.2193847553629142)
(1241,2.132136419556179)
(1261,2.0501885253926195)
(1281,1.9652220128104885)
(1301,1.8980776716117598)
(1321,1.7874245490099652)
(1341,1.7398033264769468)
(1361,1.681062228818227)
(1381,1.5329267237861437)
(1401,1.4745348894100878)
(1421,1.3894131834883776)
(1441,1.3260167623259032)
(1461,1.264882911342659)
(1481,1.2459510516071832)
(1501,1.1896133112479241)
(1521,1.1139273038329134)
(1541,1.100579731596715)
(1561,1.0772824856173655)
(1581,0.9757351339645373)
(1601,0.9476029159998723)
(1621,0.9179162834130181)
(1641,0.8818482446161056)
(1661,0.7474383999949963)
(1681,0.7552309586257925)
(1701,0.7322603352600785)
(1721,0.7053722128213554)
(1741,0.69150661010632)
(1761,0.6383292248116876)
(1781,0.6435331309453183)
(1801,0.639918615020158)
(1821,0.587477423129644)
(1841,0.5722561668646362)
(1861,0.5700324878394412)
(1881,0.5693158099828817)
(1901,0.565401636963206)
(1921,0.5427887020780897)
(1941,0.5012759339548726)
(1961,0.4806670680966498)
(1981,0.4579642392345885)
(2001,0.44730816594181977)
(2021,0.44136421767582756)
(2041,0.4145243749095051)
(2061,0.41597045113709047)
(2081,0.40241427682471154)
(2101,0.38354767963133746)
(2121,0.371772478234105)
(2141,0.33272799343859394)
(2161,0.3362285578298473)
(2181,0.32835935758102464)
(2201,0.3203594827873543)
(2221,0.3199450604679686)
(2241,0.29835308876591926)
(2261,0.30216081280230284)
(2281,0.28165520720573045)
(2301,0.25515504392316474)
(2321,0.24171943843541077)
(2341,0.2159809657271426)
(2361,0.2157195365443065)
(2381,0.23985101433980968)
(2401,0.22626738162110846)
(2421,0.2211875783457403)
(2441,0.217794511069894)
(2461,0.2148294064251354)
(2481,0.20464565517928196)
(2501,0.23058062203365481)
(2521,0.2021411146850136)
(2541,0.20043731174066948)
(2561,0.2114805501197092)
(2581,0.18293776185670357)
(2601,0.1781539564304397)
(2621,0.19440420045237905)
(2641,0.18514717315790588)
(2661,0.1520151522527829)
(2681,0.13947977849594292)
(2701,0.1347487502209913)
(2721,0.14020002188260572)
(2741,0.138466686445161)
(2761,0.13012796683388458)
(2781,0.1060356012440127)
(2801,0.09959104572349314)
(2821,0.11893212582197188)
(2841,0.09447281309986685)
(2861,0.10194913706797497)
(2881,0.10181207443282382)
(2901,0.10029775474541194)
(2921,0.1190754675296482)
(2941,0.08523448716899756)
(2961,0.0795540549248184)
(2981,0.07691950719415935)
(3001,0.07528669935974218)
(3021,0.06583784944177766)
(3041,0.08553972176793462)
(3061,0.05740618283812361)
(3081,0.06341445946102967)
(3101,0.05978089287545824)
(3121,0.06226166081947261)
(3141,0.058250061298879816)
(3161,0.07786457312262202)
(3181,0.063244300087968)
(3201,0.051363247489625154)
(3221,0.04628451575104476)
(3241,0.04851383921298896)
(3261,0.04468313432527958)
(3281,0.053877759051219686)
(3301,0.05274698213081337)
(3321,0.04379380344948183)
(3341,0.043145444715865414)
(3361,0.05070046822136305)
(3381,0.0470080125323807)
(3401,0.048420157601257285)
(3421,0.05161807229947681)
(3441,0.054306112235181725)
(3461,0.04738573281379588)
(3481,0.042773857723363465)
(3501,0.03891021444774582)
(3521,0.03530433970254908)
(3541,0.03467621198690925)
(3561,0.0351927789219189)
(3581,0.0395340063255194)
(3601,0.041797372924089596)
(3621,0.03832154343094568)
(3641,0.04296397397701454)
(3661,0.049555986905642196)
(3681,0.04057661722371152)
(3701,0.04344872110375584)
(3721,0.03642236622361892)
(3741,0.03726474234924157)
(3761,0.02972782957329793)
(3781,0.028324019488280765)
(3801,0.03718101063032847)
(3821,0.048319645072967536)
(3841,0.034769329989178074)
(3861,0.03272924119332086)
(3881,0.031838381899704164)
(3901,0.032189299805261175)
(3921,0.027789442913410362)
(3941,0.025396448065395)
(3961,0.02794758353068988)
(3981,0.03435722855488059)
(4001,0.026702857020771936)
(4021,0.0299706567067073)
(4041,0.024427407559572817)
(4061,0.025758418613057193)
(4081,0.03132270229021703)
(4101,0.024006615162902534)
(4121,0.02767628052828543)
(4141,0.02676852026769108)
(4161,0.029581896747450098)
(4181,0.02698834336122529)
(4201,0.030133541491819595)
(4221,0.026680396999713987)
(4241,0.029195082465398547)
(4261,0.025717673262896614)
(4281,0.031679590057795394)
(4301,0.023419021092845195)
(4321,0.030580230807913153)
(4341,0.030269243695890666)
(4361,0.02577191588175059)
(4381,0.022594221394825587)
(4401,0.03511028764641823)
(4421,0.02800696882652282)
(4441,0.028426959210501746)
(4461,0.020559620408940757)
(4481,0.02629028401190507)
(4501,0.030837741324555173)
(4521,0.02716992476813252)
(4541,0.02392809283331543)
(4561,0.035971150684276317)
(4581,0.027237151444256907)
(4601,0.022890319472072665)
(4621,0.021443051091877052)
(4641,0.02796168090318886)
(4661,0.029205235248599092)
(4681,0.02789381838289304)
(4701,0.024112735662669544)
(4721,0.023618552905187495)
(4741,0.025548103327738863)
(4761,0.02125052324965309)
(4781,0.023702154897030384)
(4801,0.03260611606127227)
(4821,0.026704536953835175)
(4841,0.026677211410026803)
(4861,0.024172559738640235)
(4881,0.0238343421595707)
(4901,0.025012455984432996)
(4921,0.02578651287249399)
(4941,0.026366839945875086)
(4961,0.022927498788724115)
(4981,0.02776857189924986)
};


    \nextgroupplot[
      title={(c) Variance\label{fig:gaussian_var}},
      title style={at={(0.5,-0.9)}},
      ymode=log,
      tuftelike,
      xlabel={
        \footnotesize
        Iteration
      },
      ylabel={
        \footnotesize
        Score Variance
      },
      xtick={1,2000,4000,6000,8000,10000},
      xticklabels={1,2k,4k,6k,8k,10k},
      ytick={1/32,1/2,8,128},
      xmin=1,
      xmax=10000,
      ymin=1/32,
      ymax=128,
      %% ylabel near ticks,
      %% xlabel near ticks,
      log basis y=2,
      major tick length=1.5pt,
      every tick/.style={
        black,
      },
      xtick pos=bottom,
      ytick pos=left,
      xtick align=outside,
      ytick align=outside,
      scaled x ticks = false,
      % smooth,
      % enlargelimits = true,
      % ymajorgrids,
      % yminorgrids,
      % xmajorgrids,
      width =4cm,
      height=3.5cm,
      % legend pos=north,
      legend style = { 
        legend columns = 2,
        draw           = none,
        at={(0,1)},
        anchor=south west,
        legend cell align={left},
    }]
    
\addplot[semithick,densely dotted,color=blue] coordinates {
(1,1.001530037033943)
(21,1.1207211903672014)
(41,0.9414399465257735)
(61,0.8187960952037245)
(81,0.6975037932361757)
(101,0.6125175330322539)
(121,0.5613793084921977)
(141,0.5342338557673609)
(161,0.4941009226702229)
(181,0.4580592762754424)
(201,0.40218986663140593)
(221,0.35427579754717564)
(241,0.32675534114137583)
(261,0.3082697566571132)
(281,0.2959080141625255)
(301,0.2768067132330792)
(321,0.26342233658077174)
(341,0.2414976159474806)
(361,0.22483454550492096)
(381,0.2105479551976065)
(401,0.19385477224561917)
(421,0.1799801518190565)
(441,0.1754488012146334)
(461,0.1751065160427277)
(481,0.167745732794019)
(501,0.15742517359828692)
(521,0.152537319800606)
(541,0.14553947112544313)
(561,0.14102781498860154)
(581,0.13208640513100353)
(601,0.1265484792262023)
(621,0.12268629261890235)
(641,0.12158045795413758)
(661,0.11698369117595432)
(681,0.11176676011747189)
(701,0.10822384719120937)
(721,0.10523692601527795)
(741,0.10394502292838674)
(761,0.0987442214436401)
(781,0.09788257985566569)
(801,0.09723187652262534)
(821,0.09494247295052766)
(841,0.09073963976829534)
(861,0.0866905464505721)
(881,0.08498610425924777)
(901,0.08049316057925526)
(921,0.07952294608390394)
(941,0.07811580970654305)
(961,0.07572340664867519)
(981,0.07579011675074421)
(1001,0.0731447339210019)
(1021,0.06960309079914498)
(1041,0.06695326260572045)
(1061,0.06650914502847496)
(1081,0.06502009928908394)
(1101,0.06274864363165099)
(1121,0.061386919590103284)
(1141,0.06117291781027529)
(1161,0.06015424382458367)
(1181,0.057937405779061965)
(1201,0.057134008770645325)
(1221,0.055689570496924384)
(1241,0.0531181954000071)
(1261,0.05213890361955744)
(1281,0.04979796715540416)
(1301,0.05019490734280035)
(1321,0.04948829889017911)
(1341,0.04883513876593201)
(1361,0.04799698275082562)
(1381,0.04945257137441509)
(1401,0.04925032252012443)
(1421,0.04861430710469385)
(1441,0.04732413678654614)
(1461,0.04496050511588607)
(1481,0.04388391742878024)
(1501,0.04206381356851095)
(1521,0.04098747937997616)
(1541,0.03929304138975849)
(1561,0.03929974770118367)
(1581,0.03924161167517007)
(1601,0.039753263502112925)
(1621,0.040059460300875226)
(1641,0.03907779146989625)
(1661,0.038570513119277394)
(1681,0.03794154826920954)
(1701,0.0383674602833567)
(1721,0.038744958765425885)
(1741,0.038405439917768916)
(1761,0.038987157998576064)
(1781,0.03743461111815099)
(1801,0.03766587297670994)
(1821,0.036263821012360586)
(1841,0.036165964423804146)
(1861,0.035496362276025276)
(1881,0.034650718328138694)
(1901,0.035059383787762105)
(1921,0.03495247349811903)
(1941,0.034625208234355234)
(1961,0.033478139117495746)
(1981,0.033620765135450474)
(2001,0.03277698894792824)
(2021,0.03315701690807574)
(2041,0.03224085184668851)
(2061,0.03256000508986994)
(2081,0.03216354538438422)
(2101,0.031998063564374554)
(2121,0.03147344257168905)
(2141,0.03186424873283929)
(2161,0.03183265961815996)
(2181,0.031479633488166694)
(2201,0.03198072584422724)
(2221,0.03216832454604678)
(2241,0.031907181623691966)
(2261,0.031780165900269365)
(2281,0.03256973274990268)
(2301,0.033517912035540415)
(2321,0.032887336301165734)
(2341,0.0327277097416412)
(2361,0.032196115833539235)
(2381,0.03127209904055693)
(2401,0.03033362583555683)
(2421,0.030599160596854483)
(2441,0.030782416267873258)
(2461,0.029843939031431312)
(2481,0.02973904623594479)
(2501,0.02910121368338554)
(2521,0.028892522984961096)
(2541,0.02929155205171867)
(2561,0.029367045030380175)
(2581,0.028331952247859137)
(2601,0.027455578531470716)
(2621,0.026718405643516945)
(2641,0.026154455853357298)
(2661,0.025469099863615688)
(2681,0.025584451098163003)
(2701,0.025062990899408447)
(2721,0.025343446529802832)
(2741,0.025620991472434733)
(2761,0.02512916288795379)
(2781,0.02513063875836458)
(2801,0.025445207428829622)
(2821,0.025593920507961915)
(2841,0.025453708703654918)
(2861,0.025832211418526797)
(2881,0.02449773334698461)
(2901,0.024826933953479242)
(2921,0.02531633351700697)
(2941,0.025529317137642973)
(2961,0.02655795681164748)
(2981,0.026131402887364125)
(3001,0.026520170049264435)
(3021,0.026753036734580824)
(3041,0.027902773754540977)
(3061,0.0283144632547732)
(3081,0.028192140494610746)
(3101,0.029023297572537278)
(3121,0.028529704954018773)
(3141,0.028905409851148492)
(3161,0.028812097393363212)
(3181,0.029251928819265284)
(3201,0.028365250457814677)
(3221,0.0278640423280036)
(3241,0.02736430034321468)
(3261,0.026851739223538264)
(3281,0.02790749705725952)
(3301,0.028360968761607906)
(3321,0.02816283837971707)
(3341,0.027678189328879776)
(3361,0.027200060761245183)
(3381,0.028127430105468615)
(3401,0.02719972304263416)
(3421,0.02659460365486903)
(3441,0.026298676054348935)
(3461,0.026890608137656976)
(3481,0.027026341262639298)
(3501,0.02720757522333057)
(3521,0.026019769209132956)
(3541,0.026004978342402767)
(3561,0.025880637607496156)
(3581,0.02495604669427514)
(3601,0.02423509742070313)
(3621,0.024629740803206436)
(3641,0.024226458372034736)
(3661,0.02433589190894254)
(3681,0.024820560025504198)
(3701,0.025704389367169858)
(3721,0.025247128631695776)
(3741,0.026120387434377758)
(3761,0.025853952735005886)
(3781,0.0259544834676461)
(3801,0.026095245848789264)
(3821,0.024120316929659574)
(3841,0.023777821074913344)
(3861,0.023594710503475613)
(3881,0.023239858661752856)
(3901,0.023573439618954986)
(3921,0.022290754172362544)
(3941,0.022479807624432906)
(3961,0.0227055842199184)
(3981,0.022754743713140783)
(4001,0.022823856539181404)
(4021,0.0239027249951904)
(4041,0.022981653786288605)
(4061,0.02368799791331417)
(4081,0.02463926016118704)
(4101,0.024533815096702563)
(4121,0.02520459783243099)
(4141,0.025344719404967023)
(4161,0.025418792907556935)
(4181,0.02587994011319856)
(4201,0.02720193574600949)
(4221,0.026412312198632996)
(4241,0.025658792588593125)
(4261,0.025947967959938346)
(4281,0.0250062371512737)
(4301,0.024931716759799345)
(4321,0.025093180034707226)
(4341,0.024263012038631238)
(4361,0.02383764925738161)
(4381,0.025665009893817623)
(4401,0.025089433207446145)
(4421,0.02511722510934492)
(4441,0.025056105356374182)
(4461,0.024806116383287898)
(4481,0.023614516502331893)
(4501,0.023523573931440056)
(4521,0.02211069587404338)
(4541,0.021428483050237222)
(4561,0.022060590685443075)
(4581,0.022964814846107136)
(4601,0.023288144542796606)
(4621,0.02163592588652529)
(4641,0.02239635896701111)
(4661,0.02225645567453477)
(4681,0.022522956045943204)
(4701,0.023500183504392327)
(4721,0.023560303526561733)
(4741,0.022684575557829764)
(4761,0.023204699969313163)
(4781,0.022969283853957776)
(4801,0.022847013467416404)
(4821,0.023182777114032455)
(4841,0.023350551067633393)
(4861,0.02361191077246145)
(4881,0.02353960633543811)
(4901,0.023476156222247052)
(4921,0.02224372179892928)
(4941,0.02113870250210838)
(4961,0.021302603822427768)
(4981,0.021137429521833628)
};

    
\addplot[semithick,color=blue] coordinates {
(1,4.960395448004819) +- (0.3058252219669315,0.20038451324108397)
(101,1.496524671221587) +- (0.051343259887521864,0.040214619691495646)
(201,0.8890451801738517) +- (0.007516203791536391,0.013805058727327624)
(301,0.5999113548234933) +- (0.00688063452609855,0.010634644807491878)
(401,0.4241259946157857) +- (0.003959847621774537,0.00840413737100909)
(501,0.32779551427887976) +- (0.0039930626099000666,0.0050187704736536776)
(601,0.25766716150358443) +- (0.001411346159844229,0.002869323048285799)
(701,0.20789830848270044) +- (0.001395331477930195,0.001399872585185119)
(801,0.17423609131060122) +- (0.0014925398701457204,0.0016724514917716427)
(901,0.1511771656923512) +- (0.0012957546445838686,0.0018964970080919685)
(1001,0.13766102335336033) +- (0.0013199619015540898,0.0017544702438092308)
(1101,0.12869134463202284) +- (0.0019601489016905504,0.0018502630896373162)
(1201,0.11304973777551702) +- (0.001300269504231677,0.002023586325926835)
(1301,0.10648369704433933) +- (0.002566650606705287,0.0009882485659260531)
(1401,0.10086065531493475) +- (0.002737782276282491,0.0032409793431760064)
(1501,0.08703635059556618) +- (0.0020699580792978706,0.0013636453101397161)
(1601,0.0822465294957454) +- (0.0011592854683766113,0.0017012196187817935)
(1701,0.08198058157188852) +- (0.0022312765961043507,0.003289942029683468)
(1801,0.0691361319318263) +- (0.001000235382741868,0.001615393446872171)
(1901,0.0700356005311743) +- (0.002002519540291542,0.0016824366111174022)
(2001,0.08117481580655617) +- (0.012207199636349697,0.006381486705433273)
(2101,0.06820250067956801) +- (0.002119054628163211,0.001729112157227236)
(2201,0.060997792717823976) +- (0.0011384175954090095,0.0017820145213641447)
(2301,0.05977857228226345) +- (0.0021721048710332744,0.0018489186363435592)
(2401,0.06120077384475489) +- (0.00338276262275658,0.0021989891024158797)
(2501,0.06031406277120738) +- (0.0032713018851195513,0.002232291178467828)
(2601,0.06078558829792699) +- (0.003506570867106651,0.002296714659640299)
(2701,0.06146200295032346) +- (0.0035875682116561503,0.0016280348324109045)
(2801,0.07158026635380532) +- (0.006300866516844986,0.004696347533381315)
(2901,0.06736328692762275) +- (0.004440591386622755,0.0047513507522510545)
(3001,0.05925377245913033) +- (0.0010887009105125176,0.0030505902312994854)
(3101,0.05442521954786697) +- (0.002037216645997153,0.0020410005736371875)
(3201,0.052810211099440486) +- (0.00409048738533993,0.0024470492332035673)
(3301,0.05672983163929709) +- (0.004625557013216833,0.002840384063395811)
(3401,0.05025808275610269) +- (0.0027043499824308675,0.0018186576324347725)
(3501,0.05565944149018129) +- (0.002275030273540271,0.0027681430730125967)
(3601,0.050283856409761574) +- (0.0014761883297498349,0.002141172646791971)
(3701,0.049142639429177584) +- (0.0013405078505633883,0.002686824009133719)
(3801,0.05339742776271287) +- (0.0028196819011342064,0.001986344310945659)
(3901,0.05368757665032843) +- (0.002500650331810815,0.002556686504145214)
(4001,0.053924194295088976) +- (0.003199373643619191,0.001938130713434752)
(4101,0.05333572474599143) +- (0.0017598129654775307,0.003135369694757986)
(4201,0.045895842698456424) +- (0.0013012382382765059,0.0013318558415395462)
(4301,0.04320913646795124) +- (0.0009415946757540539,0.0012077334953971217)
(4401,0.047388818188503025) +- (0.0013857050404395138,0.0012795894204382999)
(4501,0.05104433362096468) +- (0.003226033275565303,0.0015636943660121763)
(4601,0.05422013211381702) +- (0.0035154009957386392,0.0012957566428389855)
(4701,0.05990576660566081) +- (0.004574245572279104,0.0023394621486326136)
(4801,0.06042068886151952) +- (0.0031804605472749714,0.001526009436641905)
(4901,0.053216959102642694) +- (0.002529333782465143,0.0016769202231477001)
(5001,0.05358398351123153) +- (0.0020417294503450173,0.003421518242192889)
(5101,0.05423033812419959) +- (0.002867524772188844,0.002150304384822159)
(5201,0.05524211456903462) +- (0.001681465931127682,0.003568549181331447)
(5301,0.05206257927326488) +- (0.0027078059597514595,0.0028710110823998783)
(5401,0.05077907263350179) +- (0.0039006208348413215,0.0014107341723799266)
(5501,0.05270686633538678) +- (0.004808605062051469,0.0026190659354287382)
(5601,0.05666298762610977) +- (0.007014454891448113,0.003082927356459479)
(5701,0.06336707741128156) +- (0.00494440792439671,0.0032807095480279144)
(5801,0.0604558136481894) +- (0.004517117814250057,0.002975582195283967)
(5901,0.05852766931311591) +- (0.00553740349715845,0.00306070712041398)
(6001,0.05086633845375985) +- (0.0018327789653125731,0.0010112474083818213)
(6101,0.06358405482223209) +- (0.0048674219968463245,0.003842102812011104)
(6201,0.05718371388799874) +- (0.0025082897165274107,0.0014429621079549426)
(6301,0.05978849079319632) +- (0.0017627984198810842,0.004211922267785037)
(6401,0.059412339074389725) +- (0.004372332325094722,0.0021325169195779772)
(6501,0.05584822077798325) +- (0.004030337329756724,0.0033251220932175238)
(6601,0.06888095215857234) +- (0.009809508955433657,0.005671248546316621)
(6701,0.0523930788876264) +- (0.0021347033741802857,0.000822790723432755)
(6801,0.0521262180980399) +- (0.0014341964545820929,0.0024755240088551664)
(6901,0.04834224565874534) +- (0.0019087765409950483,0.001603250813422566)
(7001,0.051418723910619205) +- (0.0018659372519446507,0.0028863059766992896)
(7101,0.059230033558784784) +- (0.005553616829671398,0.004437818175720726)
(7201,0.059916411455182936) +- (0.002337171044355643,0.006259422091863934)
(7301,0.06101323324049834) +- (0.005168057668548685,0.005530915842424253)
(7401,0.06511567670323051) +- (0.008667071649005362,0.003477120299808102)
(7501,0.06280776040708319) +- (0.007419417252791299,0.0025259428498254632)
(7601,0.0563853756994387) +- (0.004409781147377272,0.0030421807089483985)
(7701,0.05680810693239284) +- (0.002617438343912759,0.004441020793316368)
(7801,0.05486135579161031) +- (0.0011692194808637646,0.0023827903327711675)
(7901,0.051779138951162415) +- (0.0016486850573055947,0.00308412585950308)
(8001,0.05421953048095897) +- (0.0031472153315863644,0.0040671679909140696)
(8101,0.061884285713320414) +- (0.0024987960712203328,0.0041069285186927165)
(8201,0.05412360048339486) +- (0.004141384489880391,0.0013537803913320975)
(8301,0.0529021601442974) +- (0.0024740229219373036,0.002830852753745998)
(8401,0.05033846792047582) +- (0.0032162493706653153,0.0013589562249220963)
(8501,0.05458655128656782) +- (0.003057926251692711,0.003096317653048665)
(8601,0.05603578360414841) +- (0.0037918575931293677,0.002907059156323992)
(8701,0.05719467839773222) +- (0.004686107546302991,0.0028113518277405464)
(8801,0.05925444583061978) +- (0.004544401552649895,0.004314179155616819)
(8901,0.05354696621161151) +- (0.003635044097089507,0.0019919813621867566)
(9001,0.051301680554318524) +- (0.002892479761535352,0.0019912005262725588)
(9101,0.05520462638964045) +- (0.0041983430379243306,0.0026975356141174245)
(9201,0.055191470204662155) +- (0.0028197161666876636,0.0033201163618623714)
(9301,0.04660639787167769) +- (0.002107986244865516,0.0013852668347122676)
(9401,0.05317849417244537) +- (0.004537461812149125,0.004362301040895057)
(9501,0.051090026050126955) +- (0.003412258025506737,0.0017884054344267514)
(9601,0.05419225834737711) +- (0.0023647770736957424,0.0030205043869010947)
(9701,0.05065585575425509) +- (0.005612548205809566,0.0015284546080117575)
(9801,0.04994708071371137) +- (0.005150137930959718,0.0025286233503446776)
(9901,0.05938049537169453) +- (0.0025270501220140476,0.007103716776779664)
};

    
\addplot[semithick,densely dotted,color=red] coordinates {
(1,31.088286714483992)
(21,33.69892582531429)
(41,30.682227472251455)
(61,25.982898836323855)
(81,21.36417395529389)
(101,19.814290720643772)
(121,19.12484219761013)
(141,20.504322523397505)
(161,18.939278184335958)
(181,16.525098731536566)
(201,15.193149991075792)
(221,14.059145828186658)
(241,14.032100307360404)
(261,13.211249702978774)
(281,12.592774332945257)
(301,12.078863255219922)
(321,11.45300633499202)
(341,10.616988616830874)
(361,10.058434729090422)
(381,9.950737510943696)
(401,9.147583176833077)
(421,9.398895962481454)
(441,9.455592138016184)
(461,8.925308616765005)
(481,8.376228242718726)
(501,8.308845653174926)
(521,8.195873114823385)
(541,8.131257821143024)
(561,8.076880668294248)
(581,7.66204623293805)
(601,7.516914863845722)
(621,6.661576683795553)
(641,6.901317537744207)
(661,6.811067889539091)
(681,6.328964997586157)
(701,6.231993425944662)
(721,5.89933745856497)
(741,5.988691386251917)
(761,5.848639821328403)
(781,5.3753918541434444)
(801,5.175692210779026)
(821,4.884657413206133)
(841,4.953310904941048)
(861,4.88997313155066)
(881,5.036710194038016)
(901,4.786653623512553)
(921,4.440308925636659)
(941,4.478292293347167)
(961,4.210384922538437)
(981,3.986526768629914)
(1001,3.775057962103677)
(1021,3.514149498353234)
(1041,3.413917457191004)
(1061,3.4548342593745227)
(1081,3.4776777885907566)
(1101,3.3652660021467593)
(1121,3.328992162267305)
(1141,3.3042291137491975)
(1161,3.304398762099975)
(1181,3.3899430832963797)
(1201,3.262138049005295)
(1221,3.068779030200355)
(1241,2.9642678126330733)
(1261,2.9426181916857197)
(1281,2.892914920744841)
(1301,2.9181931733638047)
(1321,2.918682160648117)
(1341,2.8008613289826245)
(1361,2.793200184946291)
(1381,2.719470448905453)
(1401,2.609231697973496)
(1421,2.5265225437085554)
(1441,2.4848065148331693)
(1461,2.3963484926758114)
(1481,2.362017760893405)
(1501,2.394058470549893)
(1521,2.143091486915565)
(1541,1.9953666944856125)
(1561,1.9715978494968687)
(1581,1.9327393198929028)
(1601,1.929053722032102)
(1621,1.8541754549571903)
(1641,1.9200403256752425)
(1661,1.8583768964322753)
(1681,1.8418867104414232)
(1701,1.7232409414875618)
(1721,1.5948217124737027)
(1741,1.5210162621327947)
(1761,1.5204597776228828)
(1781,1.5001670723163365)
(1801,1.5425918716296585)
(1821,1.5183620887532463)
(1841,1.492486054157296)
(1861,1.5274795002956423)
(1881,1.5007781012165853)
(1901,1.4662050036473673)
(1921,1.5544416999322488)
(1941,1.4372842874475389)
(1961,1.389041601876578)
(1981,1.3267355506725533)
(2001,1.2993035694997328)
(2021,1.381250653891693)
(2041,1.365374202394692)
(2061,1.3751717396054641)
(2081,1.333566690165843)
(2101,1.321243538222196)
(2121,1.3286823500252771)
(2141,1.2852870673325003)
(2161,1.2875006829457456)
(2181,1.2570193683324344)
(2201,1.2379206000529608)
(2221,1.1545605704802469)
(2241,1.1547838038281442)
(2261,1.0777812370946387)
(2281,1.0594495977414689)
(2301,1.0521569927665562)
(2321,1.0553548755676079)
(2341,1.0498975635935541)
(2361,1.0680677639497884)
(2381,1.024209251994865)
(2401,0.980515792143713)
(2421,0.9349869621303278)
(2441,0.9955205683001722)
(2461,1.0568488029646024)
(2481,1.0403928748173736)
(2501,1.0189394250033144)
(2521,0.9893029519920824)
(2541,0.9821353426900181)
(2561,0.9329449098658947)
(2581,0.9232843065717213)
(2601,0.9183763581148401)
(2621,0.9065377614151247)
(2641,0.8885106948665088)
(2661,0.8793511314367101)
(2681,0.8957407177328731)
(2701,0.8420077903296671)
(2721,0.7787977580977761)
(2741,0.8290002900688507)
(2761,0.844881642298288)
(2781,0.7706037753176401)
(2801,0.7885371644318866)
(2821,0.8124212897315849)
(2841,0.8020003729735355)
(2861,0.74464895454654)
(2881,0.7311711160347253)
(2901,0.6913817534447231)
(2921,0.6748745215410862)
(2941,0.656001574825916)
(2961,0.6607693158075727)
(2981,0.6599573909312825)
(3001,0.6864620928981616)
(3021,0.6842681120442623)
(3041,0.6623126777617263)
(3061,0.6257259153950713)
(3081,0.6461710168945423)
(3101,0.5872988742835556)
(3121,0.6249616362924616)
(3141,0.5883070716119903)
(3161,0.632628936488179)
(3181,0.5743550675511294)
(3201,0.5263948223377664)
(3221,0.5379156535729921)
(3241,0.5033054042747688)
(3261,0.47707894136610185)
(3281,0.4641089773798328)
(3301,0.4667835582684815)
(3321,0.4552907416754994)
(3341,0.45668275307884953)
(3361,0.43263364120633724)
(3381,0.42206297528096887)
(3401,0.3982910115777276)
(3421,0.41204388579085854)
(3441,0.41231450995853025)
(3461,0.41317338006753834)
(3481,0.37758866377298606)
(3501,0.38791107880092857)
(3521,0.37658677629069415)
(3541,0.4103065956753332)
(3561,0.38503005537402424)
(3581,0.3675910335552316)
(3601,0.38169504332610377)
(3621,0.37811716560250774)
(3641,0.366483583163976)
(3661,0.3413338769945491)
(3681,0.3116582389886489)
(3701,0.3278525824935091)
(3721,0.3389672901298552)
(3741,0.3287959987453503)
(3761,0.29668217242049066)
(3781,0.29609010868816266)
(3801,0.2937677418939969)
(3821,0.31119856813106617)
(3841,0.31515545601806083)
(3861,0.29322133331082206)
(3881,0.305543070229181)
(3901,0.28365396801862136)
(3921,0.28909166677383585)
(3941,0.2977312705398557)
(3961,0.29881806302188385)
(3981,0.32396007362186835)
(4001,0.3281428004951464)
(4021,0.31497692531955046)
(4041,0.29109696274879393)
(4061,0.2779053730121146)
(4081,0.2625559792888939)
(4101,0.27142579534915623)
(4121,0.2965494892429732)
(4141,0.28268894518178955)
(4161,0.2878001076910501)
(4181,0.2650055546376292)
(4201,0.2585751917602026)
(4221,0.26582805399710024)
(4241,0.25982253581341386)
(4261,0.2593970821128862)
(4281,0.26921406325899344)
(4301,0.24499134421028498)
(4321,0.2396391791216851)
(4341,0.21491307776570273)
(4361,0.20428095198431204)
(4381,0.20246670077233583)
(4401,0.21373286416184825)
(4421,0.21719383346857918)
(4441,0.21526928404841322)
(4461,0.2024604105458751)
(4481,0.21120742771333859)
(4501,0.21882477503937814)
(4521,0.19371034274624538)
(4541,0.18964020988086125)
(4561,0.19535616747065124)
(4581,0.1736869786476835)
(4601,0.17767455725361408)
(4621,0.18042176516072322)
(4641,0.19673726402761682)
(4661,0.1826092018851139)
(4681,0.17798534787164802)
(4701,0.18020120749268886)
(4721,0.1867846222245605)
(4741,0.18886186163968277)
(4761,0.1889159122424162)
(4781,0.21645948895019432)
(4801,0.21850860566118385)
(4821,0.18914442431040945)
(4841,0.19558222884971982)
(4861,0.1792175674084826)
(4881,0.16939379452442618)
(4901,0.16940660829728396)
(4921,0.19077089195122776)
(4941,0.15648377980860986)
(4961,0.1532459992548103)
(4981,0.15097343149483117)
};

    
\addplot[semithick,color=red] coordinates {
(1,311.44467411232415) +- (14.22575769876363,9.500195505283159)
(101,247.12282219186864) +- (26.32461230844325,11.719518974031928)
(201,122.26184449786908) +- (24.635348851808715,18.866870701425825)
(301,74.58350695310395) +- (8.569286807247508,1.9419715382533553)
(401,63.06191587258753) +- (2.063175267945489,2.0426722915888504)
(501,57.18157152938241) +- (1.2069418774554848,1.8596821116594384)
(601,51.78954451122523) +- (1.4866090866801684,1.9038506199785488)
(701,47.27799862690904) +- (1.2439222959986793,2.083630542971278)
(801,40.648821973217345) +- (1.4044834598265155,0.9289231154480291)
(901,39.00443844795582) +- (1.1922197256747324,1.3214592637061955)
(1001,37.930510280948376) +- (1.8403278984418847,1.729817703437618)
(1101,32.11991865679038) +- (2.141341613987663,0.7059851421276839)
(1201,33.08202828506357) +- (5.149793779968519,2.9070931811141385)
(1301,33.055834083097935) +- (18.271283467238383,2.3820245594332157)
(1401,23.901081915249115) +- (1.8369608190317237,0.8257193109904257)
(1501,23.048390449366558) +- (1.2246103843128928,0.3748752053815032)
(1601,21.040584707292318) +- (0.4250178188543394,0.5249791577853031)
(1701,20.308138400677727) +- (0.504360222297418,0.5611786268778296)
(1801,17.653175420477602) +- (0.44741575981103665,0.2908524229933569)
(1901,16.716073399932426) +- (0.4109624795118201,0.4954016747253931)
(2001,15.728745023883947) +- (0.3084109107494637,0.39491169289943606)
(2101,15.627840204030985) +- (0.3879853055328546,0.311000794087926)
(2201,14.012980692596816) +- (0.15639980308100299,0.12431737637379925)
(2301,12.913010822733149) +- (0.24299003695743693,0.14417162918596915)
(2401,12.822337484622862) +- (0.2489923745430449,0.21664682179421035)
(2501,11.750518030520784) +- (0.33606280403842703,0.374674238409332)
(2601,11.465270377008316) +- (0.31581590218343614,0.34029896765551904)
(2701,11.128666655815957) +- (0.22818531592969293,0.15141189805980204)
(2801,11.173749710830549) +- (0.41178350509627926,0.2625392795489354)
(2901,10.776867119067138) +- (0.2688553589175626,0.4263851299407495)
(3001,10.294864615996552) +- (0.937968212960758,0.3295311384584938)
(3101,9.440353090063297) +- (0.842749113397911,0.4194853636840268)
(3201,9.963546660767843) +- (0.6130893428051536,0.7067544946039988)
(3301,9.150654572920423) +- (0.6707026736491137,0.396227218305663)
(3401,8.596406357530702) +- (0.47524567197608647,0.2645645147558948)
(3501,8.215455325990323) +- (0.27247147451222276,0.17772955149037806)
(3601,7.481823184461792) +- (0.10810405631997266,0.1371367941400603)
(3701,7.900271826328851) +- (0.4576343852293636,0.18708042867429064)
(3801,7.753456235997314) +- (0.28818257896607946,0.2444311766063363)
(3901,6.863197418210552) +- (0.13223414008736079,0.13384485712823313)
(4001,6.780316383408551) +- (0.17514621018773724,0.201190094565594)
(4101,6.651687447001566) +- (0.38505157035153736,0.233285222125601)
(4201,6.386430963349216) +- (0.18420776804971695,0.208814112242802)
(4301,6.228951376944869) +- (0.20495332207784678,0.15775632536317197)
(4401,5.946791635076613) +- (0.1034936946671614,0.19814611798878357)
(4501,5.862861473633922) +- (0.17078480993232326,0.15324950206201304)
(4601,5.540282347714446) +- (0.17623304611904267,0.1566833402653982)
(4701,5.632680584897399) +- (0.1887984323773093,0.15822018711090458)
(4801,6.191914256205784) +- (0.5969873069492158,0.3712096629576189)
(4901,5.9302184227511) +- (0.775926772372828,0.32843993083711354)
(5001,6.993827865648708) +- (1.864781268146377,0.8608288291708304)
(5101,8.048659715808865) +- (3.753393090868604,1.577813258516711)
(5201,5.260241946995323) +- (0.48068347779100407,0.27977494939480696)
(5301,5.128953815364708) +- (0.43213814147729224,0.25215543009371544)
(5401,4.717902131141496) +- (0.23155102316941534,0.17169764413036237)
(5501,5.188965782193387) +- (0.3144769459429071,0.20772939905722598)
(5601,9.877671404000754) +- (3.4698702169555062,2.3529057181582287)
(5701,8.340379876569061) +- (2.5416248153184178,1.1850484871577542)
(5801,6.549353716456398) +- (2.1607514544257747,0.9793610743129628)
(5901,5.012840379361881) +- (2.286432679849182,0.48513216121195857)
(6001,5.165669204734887) +- (0.8714074145099175,0.5057135650358058)
(6101,4.927437137580307) +- (0.3697322888440091,0.4585265944020991)
(6201,5.7664562466316145) +- (0.4963482287781975,0.7612801822261073)
(6301,4.376874193907467) +- (0.4883851728328441,0.33755583591976457)
(6401,3.9122354377501254) +- (0.3685014550284742,0.19127363930410546)
(6501,3.7350064819347386) +- (0.21457437974162108,0.16222338112070922)
(6601,3.7870022061813984) +- (0.1371148884718827,0.1929322731987635)
(6701,4.025190013661731) +- (0.4257431927102031,0.30920999711966823)
(6801,3.5160637994364947) +- (0.5252320998922135,0.18362050531383334)
(6901,3.376219297014659) +- (0.12230871469763027,0.15550415118678096)
(7001,3.6485805160835256) +- (0.19331506350574124,0.20597292913200782)
(7101,3.8319368192276406) +- (0.7142625898354651,0.3067363106429579)
(7201,4.257524035154319) +- (0.40024849488610315,0.5245791666503932)
(7301,3.76508167782314) +- (0.41445931977818784,0.21061644596796159)
(7401,3.781255686650671) +- (0.4897556516809467,0.4762652413947852)
(7501,4.116236659627191) +- (0.6318202651238352,0.6412281763729077)
(7601,3.315331694723361) +- (0.2527546919835628,0.1865919534801268)
(7701,3.0987189128294323) +- (0.23614576449966496,0.14581350814404015)
(7801,3.094398565938348) +- (0.25873675342961944,0.15350519325037926)
(7901,2.7669191250968694) +- (0.12459161506910865,0.06195581493875446)
(8001,2.8061641944811173) +- (0.13313699926910916,0.07389351583481396)
(8101,2.962884727862276) +- (0.1580564384007257,0.10355496734527758)
(8201,2.6676019902717227) +- (0.10308874456288475,0.12180383289862151)
(8301,2.6985065287190553) +- (0.07888283239805194,0.1509708017838154)
(8401,2.8227128622100883) +- (0.058467938132381825,0.2203567521644323)
(8501,2.981643280878556) +- (0.10125505290985792,0.1580788793216108)
(8601,3.3330497532964527) +- (0.16224771658467718,0.15973187708708458)
(8701,3.849780627713952) +- (0.8058000674040002,0.6066620318006928)
(8801,3.905859291456181) +- (0.4011237098767739,0.3176863206667324)
(8901,2.777191436805137) +- (0.16568731090436195,0.20195823824782888)
(9001,2.7109370727110194) +- (0.2811463691203535,0.10787916207570047)
(9101,2.694644394420485) +- (0.3697172001569067,0.13997393263796054)
(9201,2.764863558485737) +- (0.32285875419453003,0.14462112679009786)
(9301,2.7738269664674284) +- (0.361104733840079,0.13566100558095284)
(9401,2.7472206756904525) +- (0.21088399676107272,0.12362981871954126)
(9501,2.323868322675515) +- (0.19487223800195208,0.03286378225560593)
(9601,2.2924288580828804) +- (0.21411141294795843,0.08069608346329193)
(9701,2.2341850650177584) +- (0.0870668365883911,0.05323737295675812)
(9801,2.4412001307760027) +- (0.11011761273857479,0.0749042278671781)
(9901,2.463742840480561) +- (0.1451233501677094,0.0881872512902957)
};

    
\addplot[semithick,densely dotted,color=teal] coordinates {
(1,47.94734672337795)
(21,45.655418301301296)
(41,40.085154436074234)
(61,38.29442125406664)
(81,33.812947850049646)
(101,29.828975801753835)
(121,27.679171075804827)
(141,25.334971204971524)
(161,24.09332698641787)
(181,22.120555193983066)
(201,21.619478409417752)
(221,21.115294486426492)
(241,20.46854776128864)
(261,18.841676306442)
(281,17.24004726056078)
(301,17.510559868676445)
(321,16.24410522351129)
(341,15.199114765847138)
(361,14.90177307782086)
(381,14.601272173849825)
(401,13.573676572974696)
(421,12.940779212141527)
(441,12.423134206998252)
(461,12.359201580834057)
(481,11.23460953277508)
(501,11.187709440177386)
(521,11.228488933853638)
(541,10.49000728600774)
(561,9.895624429620424)
(581,9.29210280761132)
(601,9.089790536464301)
(621,8.78793985839566)
(641,8.267267095942408)
(661,7.877836181697429)
(681,7.6244981231635585)
(701,7.296336052687531)
(721,7.2336922151189205)
(741,7.054405449901629)
(761,7.172181831952374)
(781,6.893710884169552)
(801,6.549759247043665)
(821,6.772732220109604)
(841,6.490150054661282)
(861,6.291668093645052)
(881,6.004300482359082)
(901,6.088226204563213)
(921,6.0005085871300325)
(941,5.8971846320371215)
(961,5.847317119723832)
(981,5.555524955263517)
(1001,5.650618316881836)
(1021,5.154383980695526)
(1041,5.050959644976555)
(1061,4.790212202732733)
(1081,4.6699289701508135)
(1101,4.609425451368707)
(1121,4.510871943180277)
(1141,4.605904575995978)
(1161,4.430327378908105)
(1181,4.374057501535321)
(1201,4.181401581036216)
(1221,4.135990665794827)
(1241,3.910269648066594)
(1261,3.879981770856549)
(1281,3.776601111321199)
(1301,3.692583359763065)
(1321,3.6592178515410705)
(1341,3.390834711769706)
(1361,3.263184216500231)
(1381,3.390716762579512)
(1401,3.2756736826414974)
(1421,3.129365150612629)
(1441,3.030341314409866)
(1461,3.028715196960535)
(1481,3.058501685108258)
(1501,3.0441814900107094)
(1521,2.88947409446214)
(1541,2.723396138969949)
(1561,2.7676996213876888)
(1581,2.5993597293082615)
(1601,2.587468798635425)
(1621,2.652827112733426)
(1641,2.474510285171753)
(1661,2.389293101816797)
(1681,2.389747323956101)
(1701,2.3519861209186717)
(1721,2.270680668365581)
(1741,2.2567735338313217)
(1761,2.256606332059255)
(1781,2.1494189964765615)
(1801,2.1845683283071553)
(1821,2.206843494417012)
(1841,2.2080402983102894)
(1861,2.0920468335488067)
(1881,2.0251134619790374)
(1901,1.9803450757081476)
(1921,1.9334215458289545)
(1941,1.9133728429456658)
(1961,1.8812464517181193)
(1981,1.861289603483769)
(2001,1.767098710363947)
(2021,1.7645391222558624)
(2041,1.7069699549365362)
(2061,1.7056585129932542)
(2081,1.7050954583849696)
(2101,1.7168652842104675)
(2121,1.6620267003579547)
(2141,1.6038961568897985)
(2161,1.539337262857154)
(2181,1.5297701313534748)
(2201,1.5599353834368195)
(2221,1.5126930804360224)
(2241,1.4661205728447078)
(2261,1.4311487911891476)
(2281,1.4324394376125371)
(2301,1.40546464518652)
(2321,1.380676733970808)
(2341,1.2691331833237873)
(2361,1.282698196995288)
(2381,1.2857874984753952)
(2401,1.2686766489078893)
(2421,1.2386231717723464)
(2441,1.2439159104304642)
(2461,1.22722639906339)
(2481,1.1830300124509627)
(2501,1.1756453646382943)
(2521,1.2073701538275332)
(2541,1.1648940728561117)
(2561,1.15316099140591)
(2581,1.1567113905406412)
(2601,1.1683847726612402)
(2621,1.1243678443452665)
(2641,1.1230066613180296)
(2661,1.1053106941074367)
(2681,1.1063658507210037)
(2701,1.1171640323477376)
(2721,1.114857768479768)
(2741,1.0669316636509327)
(2761,1.0368665888748396)
(2781,1.056722180433498)
(2801,1.0277047479925105)
(2821,1.0037960107480544)
(2841,0.9944754332507735)
(2861,0.9715035140709145)
(2881,0.9540730350274631)
(2901,0.9481363395173746)
(2921,0.9053677925080584)
(2941,0.8972565694555928)
(2961,0.9016215324113815)
(2981,0.9021702644116981)
(3001,0.9291369594917572)
(3021,0.9041096033195855)
(3041,0.8544668214470206)
(3061,0.8613125159777042)
(3081,0.8425559441344419)
(3101,0.8698447465310775)
(3121,0.8823536285476273)
(3141,0.8784523374241345)
(3161,0.8231216411814867)
(3181,0.8419768518358013)
(3201,0.8072208586998674)
(3221,0.7962153793818268)
(3241,0.8266577942718151)
(3261,0.7741760807566169)
(3281,0.7767663186573659)
(3301,0.7783110314278068)
(3321,0.7290568225354204)
(3341,0.7488600906937398)
(3361,0.7062433975466255)
(3381,0.7526052504438443)
(3401,0.7137747982726272)
(3421,0.7112402358860153)
(3441,0.7181649389988078)
(3461,0.6983703289196769)
(3481,0.7276862751586309)
(3501,0.7462424949798497)
(3521,0.6903134495125118)
(3541,0.6935026270106093)
(3561,0.6632244598047986)
(3581,0.6315508827509215)
(3601,0.6259409680842122)
(3621,0.6456555803239599)
(3641,0.6357447960807759)
(3661,0.6493963430782429)
(3681,0.5947082924588953)
(3701,0.5747874740984966)
(3721,0.5759297291286867)
(3741,0.6186540081558234)
(3761,0.5997725386459088)
(3781,0.5703450949409208)
(3801,0.5663630446613772)
(3821,0.5759103588687341)
(3841,0.5812300254456924)
(3861,0.5422005906725855)
(3881,0.5447388317014572)
(3901,0.5511734700474719)
(3921,0.5529585548097828)
(3941,0.5938921592700316)
(3961,0.5692583400117384)
(3981,0.5652816180484133)
(4001,0.5883977927391882)
(4021,0.565807132908158)
(4041,0.5769533367680767)
(4061,0.5807619617785327)
(4081,0.5674762713675511)
(4101,0.5318276465233589)
(4121,0.5356766872037781)
(4141,0.5133410521054941)
(4161,0.5209599072074453)
(4181,0.519009890143012)
(4201,0.5134628580697237)
(4221,0.5205274494076092)
(4241,0.47095959188373204)
(4261,0.492700015277475)
(4281,0.48991041266901264)
(4301,0.511040025030789)
(4321,0.5119306629533086)
(4341,0.4851340399172207)
(4361,0.5204549987926345)
(4381,0.5197891280647655)
(4401,0.5068421775716391)
(4421,0.49736760400082014)
(4441,0.5231528377390929)
(4461,0.5336020619650366)
(4481,0.4762705349104698)
(4501,0.4725858688171623)
(4521,0.48841157321279266)
(4541,0.47529373437247424)
(4561,0.46494336826621085)
(4581,0.46940424776979267)
(4601,0.48088056052638994)
(4621,0.4874864134647723)
(4641,0.4846292443342603)
(4661,0.4791010251789604)
(4681,0.4347556965620038)
(4701,0.4689140583294162)
(4721,0.427521030533451)
(4741,0.4377314286502805)
(4761,0.45379087072483165)
(4781,0.4600359009532714)
(4801,0.4555454287844322)
(4821,0.45917613512667765)
(4841,0.46009653242888726)
(4861,0.46141852110884307)
(4881,0.45944734354885)
(4901,0.470694258786734)
(4921,0.41235237058967433)
(4941,0.4387775210563918)
(4961,0.4953545382166348)
(4981,0.42127191396535313)
};

    
\addplot[semithick,color=teal] coordinates {
(1,338.388883998231) +- (17.363190161960176,16.198755942979176)
(101,311.4911816014901) +- (46.1181238515029,22.452409460314186)
(201,236.29319703657683) +- (81.92585396071496,19.71763104657333)
(301,189.54526267520345) +- (52.271748233182336,21.118073427584335)
(401,149.53043925651832) +- (43.199279053331196,3.469566812766061)
(501,132.7532935990534) +- (10.352623913359537,3.259914408994433)
(601,114.61245746834878) +- (5.033889124120378,2.663367739819293)
(701,100.97181824600575) +- (3.807348967835992,2.173274843748331)
(801,90.87500283145926) +- (3.4896383182612567,3.4289691417052808)
(901,79.99909520729693) +- (4.3761998281360945,1.3656656134233742)
(1001,73.59838780116675) +- (4.161065455924373,4.074945566953872)
(1101,66.44784129666027) +- (1.7092338419707147,2.8491986654596815)
(1201,60.0776437890236) +- (2.0015098977734525,2.1373999674756163)
(1301,54.08529792441261) +- (2.1519420869908643,1.618667300199732)
(1401,49.60360599572461) +- (1.8061675924988876,1.3876131481100344)
(1501,45.80524968116224) +- (3.1555748388783016,1.3041032790160898)
(1601,41.73498616759228) +- (1.283462421855596,1.1547753885136487)
(1701,39.17443158269805) +- (0.9464110881040142,1.0093645950625358)
(1801,35.42580745757584) +- (1.0005976548744187,0.4024123002465245)
(1901,33.67186997611522) +- (0.934343625996469,1.0078371600881866)
(2001,30.876526970864333) +- (0.6136405270131036,0.9343236588326285)
(2101,28.949402094620154) +- (1.3698135621268293,0.8527756353597589)
(2201,27.313194043465586) +- (0.9603518099903106,1.3793415547311731)
(2301,25.50068436480133) +- (1.8915805453383712,0.9329104637826724)
(2401,24.136989387076774) +- (0.8062251155171083,0.7730569327990935)
(2501,22.331326836973155) +- (0.7380014831406143,0.6556783025900792)
(2601,20.645561226256294) +- (0.4359069999133425,0.3533957996541126)
(2701,19.402857602647025) +- (0.4357780399851734,0.376638911857075)
(2801,19.180010002397278) +- (0.6082006422213162,0.7299271677056502)
(2901,18.25623041965276) +- (0.8142569720111617,0.4629782812287786)
(3001,17.66416834026345) +- (1.1164370671690271,0.49048564189076416)
(3101,16.602198611787834) +- (0.9217812063002526,0.5175418690689533)
(3201,15.92424650476983) +- (0.6883250595819383,0.582274753456737)
(3301,15.423393459135117) +- (0.5868417059335407,0.29145512485750835)
(3401,14.397006227770387) +- (0.7058112479953582,0.4808432648265857)
(3501,13.6806534852637) +- (0.4224032619457052,0.5434541504370038)
(3601,13.122334460684392) +- (0.3991160449672364,0.5345174973696416)
(3701,12.497064616742936) +- (0.4945938984560083,0.21803120169701984)
(3801,12.598328287951418) +- (0.5475603684751196,0.4386826711198797)
(3901,11.712832768407647) +- (0.31710783839639767,0.3074555745968972)
(4001,11.435721930448079) +- (0.2605274170479852,0.40757654075502003)
(4101,10.500797081590186) +- (0.6753948046977936,0.13652261337107063)
(4201,10.40453177835483) +- (0.9736171704622247,0.4712182332785435)
(4301,10.161437277824339) +- (0.6925197042245941,0.38806654443157207)
(4401,9.971368797962118) +- (0.43807702726132725,0.36619245950672585)
(4501,9.43970954678019) +- (0.4665108002507967,0.5207941908659688)
(4601,9.32035729462736) +- (1.0386091135874072,0.44318278646854026)
(4701,8.833275647162605) +- (0.5278962738282491,0.2897788200788476)
(4801,8.506266117843658) +- (0.6119088881160408,0.6465992556672004)
(4901,8.31997318929789) +- (0.43915763735419233,0.4457227215763524)
(5001,8.526025896645844) +- (1.117754789424417,0.8016966546346431)
(5101,7.979517112154765) +- (0.33830652734896205,0.4344978256332839)
(5201,7.565991911551139) +- (0.5280094309463017,0.20330171508352457)
(5301,7.112691400619083) +- (0.35464150760950375,0.6202346170354511)
(5401,6.994686956277901) +- (0.2986421977410858,0.63276866949146)
(5501,7.108801998362345) +- (0.35500263633912965,0.46362164183461907)
(5601,6.883420168748723) +- (0.919782919328834,0.502636086920198)
(5701,6.7176517392686605) +- (1.8859810121045903,0.5769591272027954)
(5801,6.183288264383866) +- (2.1132046739277,0.41819610754618797)
(5901,5.943349266966575) +- (0.2466277065877911,0.12570853816411098)
(6001,5.936677851946257) +- (0.6497133571578289,0.12317329145108857)
(6101,6.397781696187274) +- (0.4629132003621397,0.7423619842873297)
(6201,5.804629780022313) +- (0.506600055356925,0.361539132755583)
(6301,5.984951406411382) +- (0.5959229669956772,0.2987566185804438)
(6401,5.927058958029626) +- (0.45823586168318187,0.342194906440338)
(6501,5.564559085332419) +- (0.4088805206885402,0.30051981017078333)
(6601,5.658577371164724) +- (0.4088459224098999,0.5673867075033066)
(6701,5.083609791608584) +- (0.48127721553246605,0.29977178087677725)
(6801,5.1976700979097155) +- (0.6033070360380552,0.28693717023122645)
(6901,5.425715258586312) +- (1.1092949718393248,0.7067488121393799)
(7001,5.272886219818678) +- (2.4875533410546122,0.5288567713289449)
(7101,5.460930046016404) +- (1.5203683818601021,0.6761450393680208)
(7201,5.45067707867126) +- (1.5201700020749342,0.6449575690762561)
(7301,4.928043181346803) +- (0.3024856524849149,0.36891699656294197)
(7401,4.905288485056693) +- (0.4807471066982858,0.43280644861777073)
(7501,4.575863008391998) +- (0.988646584545962,0.2571246002493659)
(7601,4.79152502497811) +- (1.162051688656483,0.33432914704034733)
(7701,4.96053386047807) +- (0.6634850384922659,0.43021642948249816)
(7801,4.628061274455964) +- (0.6818417386295694,0.4518945518962285)
(7901,4.598711294313977) +- (2.163596054398358,0.5512375363172355)
(8001,4.512034132157259) +- (1.7618335144965442,0.5242991274850652)
(8101,4.515428530271235) +- (3.9358063445181717,0.23120700531642768)
(8201,4.534382133076367) +- (3.8615309545822782,0.465183238005622)
(8301,4.6129158410578075) +- (1.8182959404997696,0.5488975542670502)
(8401,5.0247399621337845) +- (1.6154736924790933,0.5666318997292485)
(8501,4.8251982423988125) +- (0.40135970086789996,0.6756119394546074)
(8601,3.9234921131551177) +- (0.3724020958739729,0.2825040839321393)
(8701,3.8468629758584827) +- (0.23785749248413168,0.17577331941573338)
(8801,3.664277168439034) +- (0.3160594844403839,0.3078599629571874)
(8901,3.6820346848254992) +- (0.28441633454855975,0.28099870404978233)
(9001,3.7182387454732075) +- (0.28202795307925976,0.19699715181929411)
(9101,3.623921400386746) +- (0.36396007882425074,0.11132174886022961)
(9201,3.635001162717237) +- (0.29709859751126233,0.17374540449515585)
(9301,3.7335998784381044) +- (0.46976462073613856,0.47572476839060984)
(9401,3.4048802257078123) +- (0.2872059111215788,0.2623481509039851)
(9501,3.454598079218932) +- (0.22195312795612798,0.2485516226546829)
(9601,3.6993333802280657) +- (0.495103830293798,0.39494483162713356)
(9701,3.742755948885031) +- (0.3868891400108172,0.5288212579729463)
(9801,3.6704547283746978) +- (0.31303281597320387,0.22547928762861336)
(9901,3.642932418176316) +- (0.27644406446280945,0.32005347566860065)
};

    %% \addplot[] coordinates {
    %%    (0,    0.2662314842632471)
    %%    (5001, 0.2662314842632471)
    %% };
    %% \addplot[] coordinates {
    %%    (0,    0.016639467766452944)
    %%    (5001, 0.016639467766452944)
    %% };
    \addplot[dotted] coordinates {
      (0,     0.29042793082785834)
      (10000, 0.29042793082785834)
    } node[pos=.9,above,yshift=-0.9mm] {\scriptsize{\textcolor{blue}{0.29}}};
    \addplot[dotted] coordinates {
      (0,     0.0569959104101957)
      (10000, 0.0569959104101957)
    } node[pos=.9,above,yshift=-0.9mm] {\scriptsize{\textcolor{blue}{0.06}}};
  \end{groupplot}
\end{tikzpicture}

    \vspace{-0.2in}
    \caption{Visualization of the bias and variance during score climbing.
      We can see that the parallel state estimator achieves both the lowest bias and the lowest variance.
      See the main text for details about the experimental setup.
    }\label{fig:gaussian}
\end{figure*}

\paragraph{Simulation Setup}
First, we visualize the variance and bias convergence of the considered schemes on a toy problem.
We run MCSA methods against a 10 dimensional correlated multivariate Gaussian where the covariance was sampled from a Wishart distribution with \(\nu = 50\) degrees of freedom.
The variational family was a mean-field Gaussian.
The bias and variance was estimated from \(512\) independent replications.

\paragraph{Simulation Result}
The results are shown in~\cref{fig:gaussian}.
We can see that our proposed scheme converges the fastest.
This is due to the superior gradient variance.
Surprisingly, this also leads to lower bias that JSA which operates longer per-iteration Markov chains.

%% \subsection{Hierarchical Logistic Regression}\label{section:logistic}
%% \vspace{-0.05in}
%% \paragraph{Experimental Setup}
%% We now perform logistic regression with the \texttt{Pima Indians} diabetes (\(\vz \in \mathbb{R}^{11}\),~\citealt{smith_using_1988}), \texttt{German credit} (\(\vz \in \mathbb{R}^{27}\)), and \texttt{heart disease} (\(\vz \in \mathbb{R}^{16}\),~\citealt{detrano_international_1989}) datasets obtained from the UCI repository~\citep{Dua:2019}.
%% 10\% of the data points were randomly selected in each of the 100 repetitions as test data.


\begin{table*}
  \centering
  \caption{Classification Accuracy and Log Predictive Density on Logistic Gaussian Process Problems}\label{table:logistic}
  \setlength{\tabcolsep}{3pt}
  \begin{threeparttable}
  \begin{tabular}{lcccccc}
    \toprule
     & \multicolumn{2}{c}{\textbf{\texttt{sonar}}} & \multicolumn{2}{c}{\textbf{\texttt{ionosphere}}} & \multicolumn{2}{c}{\textbf{\texttt{breast}}} \\
    \cmidrule(lr){2-3}\cmidrule(lr){4-5}\cmidrule(lr){6-7}
    & Test Accuracy & Test LPD
    & Test Accuracy & Test LPD
    & Test Accuracy & Test LPD \\\midrule
    ELBO & \textbf{0.86 {\scriptsize(0.84, 0.88)}} & -0.42 {\scriptsize(-0.43, -0.40)} & 0.89 {\scriptsize(0.88, 0.90)} & -0.34 {\scriptsize(-0.36, -0.33)} & 0.93 {\scriptsize(0.92, 0.94)} & \textbf{-0.16 {\scriptsize(-0.17, -0.14)}}  \\\arrayrulecolor{black!30}\midrule
    par.-IMH & 0.85 {\scriptsize(0.83, 0.87)} & \textbf{-0.37 {\scriptsize(-0.39, -0.35)}} & \textbf{0.92 {\scriptsize(0.91, 0.93)}} & \textbf{-0.30 {\scriptsize(-0.32, -0.28)}} & 0.93 {\scriptsize(0.92, 0.94)} & -0.18 {\scriptsize(-0.19, -0.17)} \\
    seq.-IMH & 0.85 {\scriptsize(0.83, 0.87)} & -0.39 {\scriptsize(-0.41, -0.38)} & 0.91 {\scriptsize(0.89, 0.92)} & -0.33 {\scriptsize(-0.34, -0.31)} & 0.93 {\scriptsize(0.92, 0.94)} & -0.21 {\scriptsize(-0.22, -0.20)} \\
    single-CIS & 0.85 {\scriptsize(0.83, 0.87)} & -0.40 {\scriptsize(-0.41, -0.38)} & 0.90 {\scriptsize(0.89, 0.92)} & -0.33 {\scriptsize(-0.35, -0.32)} & 0.93 {\scriptsize(0.92, 0.94)} & -0.21 {\scriptsize(-0.22, -0.20)} \\
    single-CISRB & 0.85 {\scriptsize(0.82, 0.87)} & -0.39 {\scriptsize(-0.40, -0.37)} & 0.90 {\scriptsize(0.89, 0.92)} & -0.32 {\scriptsize(-0.33, -0.30)} & 0.93 {\scriptsize(0.92, 0.94)} & -0.20 {\scriptsize(-0.21, -0.19)} \\
    SNIS & 0.85 {\scriptsize(0.83, 0.88)} & -0.39 {\scriptsize(-0.41, -0.37)} & 0.91 {\scriptsize(0.89, 0.92)} & -0.31 {\scriptsize(-0.33, -0.30)} & 0.93 {\scriptsize(0.92, 0.94)} & -0.19 {\scriptsize(-0.20, -0.18)} \\\bottomrule
  \end{tabular}
  \begin{tablenotes}
    \item[*]{\footnotesize LPD denotes the average log predictive density.}
    \item[*]{\footnotesize The numbers in the parentheses denote the 80\% bootstrap confidence intervals computed from 30 repetitions.}
  \end{tablenotes}
  \end{threeparttable}
\end{table*}

%%% Local Variables:
%%% TeX-master: "master"
%%% End:

%% %

%% \vspace{-0.1in}
%% \paragraph{Probabilistic Model}
%% Instead of the usual single-level probit/logistic regression models, we choose a more complex hierarchical logistic regression model shown in~\cref{section:linear_logistic}
%% %% %
%% %% \begin{align*}
%% %% \sigma_{\beta}, \sigma_{\alpha} &\sim \mathcal{N}^{+}(0, 1.0) \\
%% %% \symbf{\beta} &\sim \mathcal{N}(\symbf{0},\, \sigma_{\beta}^2 \mI) \\
%% %% p &\sim \mathcal{N}(\vx_i^{\top}\symbf{\beta} + \alpha,\, \sigma_{\alpha}^2) \\
%% %% y_i &\sim \text{Bernoulli-Logit}\,(p)
%% %% \end{align*}
%% %% %
%% %% where \(\mathcal{N}^+(\mu, \sigma)\) is a positive constrained normal distribution with mean \(\mu\) and standard deviation \(\sigma\), \(\vx_i\) and \(y_i\) are the feature vector and target variable of the \(i\)th datapoint.
%% %% The extra degrees of freedom \(\sigma_{\beta}\) and \(\sigma_{\alpha}\) make this model relatively more challenging.

%% %
%% \begin{wrapfigure}{r}{0.6\textwidth}
%%   \vspace{-0.3in}
%% %\begin{figure}[h]
%%   %\vspace{-0.1in}
%%   %\centering
%%   \subfloat[Test Accuracy]{
%%     \includegraphics[scale=0.8]{figures/german_02.pdf}\label{fig:german_acc}
%%     \vspace{-0.1in}
%%   } 
%%   %% \subfloat[\texttt{heart}]{
%%   %%   \includegraphics[scale=0.75]{figures/heart_02.pdf}
%%   %% }
%%   %% \subfloat[\texttt{german}]{
%%   %%   \includegraphics[scale=0.75]{figures/german_02.pdf}
%%   %% }
%%   \subfloat[Test LPD]{
%%     \includegraphics[scale=0.8]{figures/german_03.pdf}\label{fig:german_lpd}
%%     \vspace{-0.05in}
%%   }
%%     \vspace{-0.05in}
%%   \caption{Test accuracy and log predictive density on the \texttt{german} dataset.
%%     The solid lines and colored regions are the mean and 80\% bootstrap confidence interval computed from 100 repetitions.
%%   }\label{fig:logistic}
%%   \vspace{-0.1in}
%% %\end{figure}
%% \end{wrapfigure}
%% %
%% \vspace{-0.05in}
%% \paragraph{Results}
%% The test accuracy and test log predictive density (Test LPD) results are shown in~\cref{table:logistic}.
%% Our proposed parallel state estimator (par.-IMH) achieves the best accuracy and predictive density results.
%% Despite having access to high-quality HMC samples, single-HMC shows poor performance.
%% This supports our analysis that par.-IMH with \(N \geq 2\) superior variance reduction to the single state estimator.
%% Also, seq.-IMH showed poor performance overall due to the correlated samples.
%% Among the two CIS kernel-based methods, single-CISRB performs only marginally better than single-CIS.

%%   \vspace{-0.1in}
%% \paragraph{Inclusive KL v.s. Exclusive KL}
%% While both ELBO and par.-IMH showed similar numerical performance, they chose different optimization paths in the parameter space.
%% This is shown in~\cref{fig:logistic}.
%% While the test accuracy suggests that ELBO converges quickly around \(t=2000\) (\cref{fig:german_acc}), in terms of uncertainty estimate, it takes much longer to converge (\cref{fig:german_lpd}).
%% This shows that inclusive KL minimization chooses a path that has better density coverage as expected.

  \vspace{-0.05in}
\subsection{Gaussian Process Classification}\label{section:bgp}
  \vspace{-0.05in}
\paragraph{Experimental Setup}
For a more challenging problem, we perform classification with latent Gaussian processes~\citep{NIPS2014_8c6744c9}.
The simplified probabilistic model is shown in~\cref{section:gp_logistic} and uses the Mat\'ern 5/2 covariance kernel with automatic relevance determination~\citep{neal_bayesian_1996}.
For the datasets, we use the \texttt{sonar} (\(\vz \in \mathbb{R}^{249}\),~\citealt{gorman_analysis_1988}), \texttt{ionosphere} (\(\vz \in \mathbb{R}^{351}\),~\citealt{Sigillito1989ClassificationOR}), and \texttt{breast} (\(\vz \in \mathbb{R}^{544}\),~\citealt{wolberg_multisurface_1990}) datasets.
For \texttt{breast}, we preprocessed the input features with z-standardization.
10\% of the data points were randomly selected in each of the 100 repetitions as test data.
For this experiment, the iteration complexity of ELBO is almost two orders of magnitude larger than all inclusive KL minimization methods.

%
\begin{wrapfigure}{r}{0.45\textwidth}
  \vspace{-0.4in}
  \centering
     \includegraphics[scale=0.9]{figures/ionosphere_01.pdf}
  %% \subfloat[\texttt{german}]{
  %%   \includegraphics[scale=0.75]{figures/breast_01.pdf}
  %% } \\
     \vspace{-0.1in}
  \caption{Test log predictive density on the \texttt{ionosphere} dataset.
    The solid lines and colored regions are the medians and 80\% percentiles computed from 100 repetitions.
  }\label{fig:gp}
  \vspace{-0.2in}
\end{wrapfigure}
%
\vspace{-0.05in}
\paragraph{Result}
The results are shown in~\cref{table:gp}.
Again, among inclusive KL minimization, our method achieved the best results.
Compared to ELBO, its accuracy was lower on \texttt{breast}, but the uncertainty estimates were much better.
This is better-shown in~\cref{fig:gp}, where ELBO quickly converges to a point with poor uncertainty calibration.
%Meanwhile, on \texttt{breast}, ELBO gives better uncertainty estimates than inclusive KL minimization methods.
%This happens when the modal estimate (preferred by the exclusive KL) gives good accuracy and uncertainty estimates.

  \vspace{-0.05in}
\subsection{Marginal Likelihood Estimation}\label{section:mll}
  \vspace{-0.05in}
\paragraph{Experimental Setup}
Lastly, we now estimate the marginal log-likelihood of a hierarchical regression model with partial pooling (\texttt{radon}, \(\vz \in \mathbb{R}^{175}\),~\citealt{gelman_data_2007}) for modeling radon levels in U.S homes.
\texttt{radon} contains multiple posterior degeneracies from the hierarchy.
We estimated the reference marginal likelihood using \textit{thermodynamic integration} (TI,~\citealt{gelman_simulating_1998, neal_annealed_2001, lartillot_computing_2006}) with HMC implemented by Stan~\citep{carpenter_stan_2017, betancourt_conceptual_2017}.
%
\begin{wrapfigure}{r}{0.45\textwidth}
  \vspace{-0.25in}
  \centering
  \begin{minipage}[b]{0.25\linewidth}
    \centering
    \includegraphics[scale=0.8]{figures/radon_03.pdf}
  \end{minipage}
  \begin{minipage}[b]{0.7\linewidth}
    \centering
    \includegraphics[scale=0.8]{figures/radon_02.pdf}
    %\subcaption{\texttt{radon}}
  \end{minipage}
    \vspace{-0.1in}
  %% \begin{minipage}[b]{0.35\linewidth}
  %%   \centering
  %%   \includegraphics[scale=0.7]{figures/sv_02.pdf}
  %%   \subcaption{\texttt{stock}}\label{fig:sv}
  %% \end{minipage}
  \caption{Marginal log-likelihood estimates on the \texttt{radon} dataset.
    The solid lines and colored regions are the medians and 80\% percentiles computed from 100 repetitions.
  }\label{fig:marginal_likelihood}
  \vspace{-0.2in}
\end{wrapfigure}
%
  \vspace{-0.2in}
\paragraph{Results}
The results are shown in~\cref{fig:marginal_likelihood}.
par.-IMH converges quickly and provides the most accurate estimate.
By contrast, other estimators converge much more slowly.
SNIS and ELBO, on the other hand, overestimate \(\log Z\), which can be attributed to the mode-seeking behavior of ELBO and the small sample bias of SNIS.


%%% Local Variables:
%%% TeX-master: "master"
%%% End:
