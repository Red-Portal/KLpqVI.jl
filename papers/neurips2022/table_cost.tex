
% Second version of table, with booktabs.
%\begin{table}
%\centering
\caption{Computational Costs of MCSA Schemes}\label{table:cost}
  \vspace{-0.05in}
\setlength{\tabcolsep}{2pt}
  \begin{threeparttable}
\begin{tabular}{lccccc}\toprule
& \multicolumn{3}{c}{\footnotesize Posterior Sampling} & \multicolumn{2}{c}{\footnotesize Stochastic gradient} \\
\cmidrule(lr){2-4}\cmidrule(lr){5-6}
  & \(\footnotesize p\left( \vz, \vx \right)\)
  & \(\footnotesize q_{\vlambda}(\vz)\)
  & \(\footnotesize q_{\vlambda}(\vz)\)
  & \(\footnotesize p\left( \vz, \vx \right)\)
  & \(\footnotesize q_{\vlambda}(\vz)\)
  \\
  & {\footnotesize\# Eval.  }
  & {\footnotesize\# Eval.  }
  & {\footnotesize\# Samples}
  & {\footnotesize\# Grad.  }
  & {\footnotesize\# Grad.  }
%
\\\midrule
%
{\footnotesize
ELBO
}
& \(0\)
& \(0\)
& \(N\)
& \(N\)
& \(N\)
\\\arrayrulecolor{black!30}\midrule
%
{\footnotesize
MSC
}
& \(N-1\)
& \(N\)
& \(N-1\)
& \(0\)
& \(1\)\tnote{1}\;\;{\footnotesize or}\;\(N\)\tnote{2}
\\
%
{\footnotesize
JSA
}
& \(N\)
& \(N+1\)
& \(N\)
& \(0\)
& \(N\)
\\
%
{\footnotesize
\textit{proposed}
}
& \(N\)
& \(2 \, N\)
& \(N\)
& \(0\)
& \(N\)
\\\bottomrule
\end{tabular}
  \begin{tablenotes}
    \item[*]{\footnotesize We assume that the parameters are cached as much as possible}.
    \item[*]{\footnotesize \(N\) is the number of samples used in each method}.
    \item[1]{\footnotesize Vanilla CIS kernel}.
    \item[2]{\footnotesize Rao-Blackwellized CIS kernel}.
  \end{tablenotes}
  \end{threeparttable}
%\end{table}
