
\begin{theoremEnd}{theorem}\label{thm:jsa}
  JSA~\citep{pmlr-v124-ou20a} is obtained by defining 
  {\small
  \begin{align*}
  P_{\vlambda}^n\left(\veta, d\veta^{\prime}\right)
  = 
  K_{\vlambda}^{N\,\left(n-1\right) + 1}\left(\vz^{(1)}, d\vz^{\prime\;(1)}\right)
  \,
  K_{\vlambda}^{N\,\left(n-1\right) + 2}\left(\vz^{(2)}, d\vz^{\prime\;(2)}\right)
  \cdot
  \ldots 
  \cdot
  K_{\vlambda}^{N\,\left(n-1\right) + N}\left(\vz^{(N)}, d\vz^{\prime\;(N)}\right)
  \end{align*}
  }
  with \(\veta_t = \left(\vz_t^{(1)}, \vz_t^{(2)}, \ldots, \vz_t^{(N)}\right)\).
  Then, given~\cref{thm:bounded_weight,thm:bounded_score}, the mixing rate and the gradient variance bounds are
  {\small
  \begin{align*}
    \DTV{P_{\vlambda}^n\left(\veta, \cdot\right)}{\Pi}
    \leq
    C\left(\rho, N\right)\,{\rho}^{n\,N}
    \quad\text{and}\quad
   % 
    \E{ \norm{ \vg\left(\vlambda, \rvveta\right) }^2_{*} \,\middle|\, \mathcal{F}_{t} }
    \leq
    L^2 \,
    \left[\,
    \frac{1}{2} + \frac{3}{2}\,\frac{1}{N}
    + \mathcal{O}\left(\nicefrac{1}{w^*}\right)
    \,\right],
  \end{align*}
  }%
  where \(w^* = \sup_{\vz} \pi\left(\vz\right) / q_{\text{def.}}\left(\vz;\vlambda\right)\) and \(C\left(\rho, N\right) > 0\) is a finite constant depending on \(\rho\) and \(N\).
\end{theoremEnd}
\begin{proofEnd}

  JSA is described in~\cref{alg:jsa}. 
  At each iteration, it performs \(N\) MCMC transitions, and uses the \(N\) samples to estimate the gradient.

  \paragraph{Ergodicity of the Markov Chain}
  The state transitions of the Markov chain samples \(\vz^{(1:N)}\) are visualized as 
  {\small
  \begin{center}
  \bgroup
  \setlength{\tabcolsep}{3pt}
  \def\arraystretch{1.8}
  \begin{tabular}{c|ccccc}
   & \(\vz^{(1)}_t\) & \(\vz^{(2)}_t\) & \(\vz^{(3)}_t\) & \(\ldots\) &  \(\vz^{(N)}_t\) \\ \midrule
   \(t=1\) & \(K_{\vlambda_1}\left(\vz_0, d\vz_1^{(1)}\right)\) & \(K_{\vlambda_1}^2\left(\vz_0, d\vz_1^{(2)}\right)\) & \(K_{\vlambda_1}^3\left(\vz_0, d\vz_1^{(3)}\right)\) & \(\ldots\) & \(K_{\vlambda_1}^N\left(\vz_0, d\vz_1^{(N)}\right)\) \\
   \(t=2\) & \(K_{\vlambda_2}^{N + 1}\left(\vz_0, d\vz_2^{(1)}\right)\) & \(K_{\vlambda_2}^{N + 2}\left(\vz_0, d\vz_2^{(2)}\right)\) & \(K_{\vlambda_2}^{N + 3}\left(\vz_0, d\vz_2^{(3)}\right)\) & \(\ldots\) & \(K_{\vlambda_2}^{2\,N}\left(\vz_0, d\vz_2^{(N)}\right)\) \\
   \(\vdots\) & & & \(\vdots\) & & \\
   \(t=k\) & \(K_{\vlambda_k}^{\left(k-1\right)\,N + 1}\left(\vz_0, d\vz_k^{(1)}\right)\) & \(K_{\vlambda_k}^{\left(k-1\right)\,N + 2}\left(\vz_0, d\vz_k^{(2)}\right)\) & \(K_{\vlambda_k}^{\left(k-1\right)\,N + 3}\left(\vz_0, d\vz_k^{(3)}\right)\) & \(\ldots\) & \(K_{\vlambda_k}^{\left(k-1\right)\,N + N}\left(\vz_0, d\vz_k^{(N)}\right)\) \\
  \end{tabular}
  \egroup
  \end{center}
  }
  where \(K_{\vlambda}\left(\vz, \cdot\right)\) is an IMH kernel.
  Therefore, the \(n\)-step transition kernel for the vector of the Markov-chain samples \(\veta = \vz^{(1:N)}\) is represented as
  \begin{align*}
  P_{\vlambda}^n\left(\veta, d\veta^{\prime}\right)
  = 
  K_{\vlambda}^{N\,\left(n-1\right) + 1}\left(\vz_1, d\vz^{\prime}_1\right)
  \,
  K_{\vlambda}^{N\,\left(n-1\right) + 2}\left(\vz_2, d\vz^{\prime}_2\right)
  \cdot
  \ldots 
  \cdot
  K_{\vlambda}^{N\,\left(n-1\right) + N}\left(\vz_N, d\vz^{\prime}_N\right).
  \end{align*}

%%   For example, let us assume that we use a batch size of 1 and that \(m=1\) is selected at \(t=1\) and \(m=2\) is selected at \(t=2\).
%%   Then, the transitions can be visualized as
%%   {\small
%%   \begin{center}
%%   \bgroup
%%   \def\arraystretch{1.8}
%%   \setlength{\tabcolsep}{5pt}
%%   \begin{tabular}{c|cccccccccccccc}
%%       t = 1 & \(\vz^{(1,1)}_1\) & \(\vz^{(1,2)}_1\) & \ldots & \(\vz^{(1,N)}_1\) & \ldots & \(\vz^{(2,1)}_1\) & \(\vz^{(2,2)}_1\) & \ldots & \(\vz^{(2,N)}_1\) & \ldots & \(\vz^{(1,N)}_1\) & \(\vz^{(2,N)}_1\) & \ldots & \(\vz^{(M,N)}_1\) \\
%%       & \(\downarrow\;K_{1,\lambda}\) & \(\downarrow\;K_{1,\lambda}^2\) & \ldots & \(\downarrow\;K_{1,\lambda}^N\)  &  &  &  & &  &  &  &  &  & \\
%%       t = 2 & \(\vz^{(1,1)}_2\) & \(\vz^{(1,2)}_2\) & \ldots & \(\vz^{(1,N)}_2\) & \ldots & \(\vz^{(2,1)}_2\) & \(\vz^{(2,2)}_2\) & \ldots & \(\vz^{(2,N)}_2\) & \ldots & \(\vz^{(1,N)}_2\) & \(\vz^{(2,N)}_2\) & \ldots & \(\vz^{(M,N)}_2\) \\
%%       & & & & & & \(\downarrow\;K_{2,\lambda}\) & \(\downarrow\;K_{2,\lambda}^2\) & \ldots & \(\downarrow\;K_{2,\lambda}^N\)  &  &  &  & &  \\
%%       t = 3 & \(\vz^{(1,1)}_3\) & \(\vz^{(1,2)}_3\) & \ldots & \(\vz^{(1,N)}_3\) & \ldots & \(\vz^{(2,1)}_2\) & \(\vz^{(2,2)}_3\) & \ldots & \(\vz^{(2,N)}_3\) & \ldots & \(\vz^{(1,N)}_3\) & \(\vz^{(2,N)}_3\) & \ldots & \(\vz^{(M,N)}_3\) \\
%%   \end{tabular}
%%   \egroup
%%   \end{center}
%%   }
%%   where \(K_{m,\vlambda}\left(\vz, \cdot\right)\) is a componentwise IMH kernel for the \(m\)th component.
%%   Conceptually, this means that, when a batch \(\vz^{(m,1:N)}\) is selected \(k\) times, it will evolve as
%%   \begin{alignat*}{2}
%%     \vz^{(m,1:N)} \sim  K_{m,\vlambda}^{\left(k-1\right)\,N + 1}\left(\vz^{(m,1)}, d\vz^{(m,1)}\right) \, K_{m,\vlambda}^{\left(k-1\right)\,N + 2}\left(\vz^{(m,2)}, d\vz\prime^{(m,2)}\right) \cdot \ldots \cdot K_{m,\vlambda}^{\left(k-1\right)\,N + N}\left(\vz^{(m,N)}, d\vz\prime^{(m,N)}\right).
%%   \end{alignat*}

%%   The random-scan kernel can be represented as
%%   \begin{alignat*}{2}
%%     P_{\vlambda}^k\left(\veta, d\veta\prime\right)
%%     =
%%     \sum_{m=1}^M
%%     r_{m} \,
%%     \left(
%%     K_{m,\vlambda}\left(\vz^{(m,1)}, d\vz\prime^{(m,1)}\right) \,
%%     K_{m,\vlambda}\left(\vz^{(m,2)}, d\vz\prime^{(m,2)}\right)
%%     \cdot
%%     \ldots
%%     \cdot
%%     K_{m,\vlambda}^N\left(\vz^{(m,N)}, d\vz\prime^{(m,N)}\right)
%%     \right)
%%   \end{alignat*}
%%   where .

  Now, the convergence in total variation \(d_{\mathrm{TV}}\left(\cdot, \cdot\right)\) can be shown to decrease geometrically as
  \begin{alignat}{2}
    &\DTV{P_{\vlambda}^{n}\left(\veta, \cdot\right)}{\Pi}
    \nonumber
    \\
    &\quad=
    \sup_{A}
    \abs{
      \Pi\left(A\right)
      -
      P^{n}\left(\veta, A\right)
    }
    &&\quad\text{\textit{Definition of \(d_{\text{TV}}\)}}
    \nonumber
    \\
    &\quad\leq
    \sup_{A}
    \Bigg|
    \int_{A}
      \pi\left(d\vz^{\prime\;(1)}\right) \times \ldots \times \pi\left(d\vz^{\prime\;(N)}\right)
    \nonumber
      \\
      &\qquad\qquad\qquad-
      K^{(n-1)\,N\,+1}_{\vlambda}\left(\vz^{(1)}, d\vz^{\prime\;(1)}\right) \times \ldots \times K^{n\,N}_{\vlambda}\left(\vz^{(N)}, d\vz^{\prime\;(N)}\right)
    \,\Bigg|
    \nonumber
    \\
    &\quad\leq
    \sup_{A}
    \sum_{n=1}^N
    \abs{
    \int_{A}
      \pi\left(d\vz^{(n)}\right) - K^{(n-1)\,N + n}_{\vlambda}\left(\vz^{(n)}, d\vz^{\prime\;(n)}\right) 
    }
    &&\quad\text{\textit{\cref{thm:product_measure_bound}}}
    \nonumber
    \\
    &\quad=
    \sum_{n=1}^N
    \DTV{K^{(n-1)\,N + n}_{\vlambda}\left(\vz^{(n)}, \cdot\right)}{\pi}
    &&\quad\text{\textit{Definition of \(d_{\text{TV}}\)}}
    \nonumber
    \\
    &\quad\leq
    \sum_{n=1}^N
    \rho^{(n-1)\,N + n}
    &&\quad\text{\textit{Geometric ergodicity}}
    \nonumber
    %\label{eq:used_ergodicity}
    \\
    &\quad=
    \rho^{n\,N}
    \,
    \rho^{-N}
    \,
    \frac{\rho - \rho^{N+1}}{1 - \rho}
    &&\quad\text{\textit{Solved sum}}
    \nonumber
    \\
    &\quad=
    \frac{\rho \, \left(1 - \rho^N\right)}{\rho^N \left(1 - \rho\right)}
    \,
    {\left( \rho^{N} \right)}^n.
    \nonumber
  \end{alignat}
  Although the constant depends on \(\rho\) and \(N\), the kernel \(P\) is geometrically ergodic and converges \(N\) times faster than the base kernel \(K\).

  \paragraph{\textbf{Bound on the Gradient Variance}}
  To analyze the variance of the gradient, we require detailed information about the \(n\)-step marginal transition kernel, which is unavailable in general for most MCMC kernels.
  Fortunately, specifically for the IMH kernel,~\citet{Smith96exacttransition} have shown that the \(n\)-step marginal IMH kernel is given as~\cref{eq:imh_exact_kernel}.
  From this, we show that
  \begin{alignat}{2}
    &\E{ \norm{ \vg\left(\vlambda, \rvveta\right) }^2_{*} \,\middle|\, \mathcal{F}_{t} }
    \nonumber
    \\
    &\quad=
    \E{ \norm{ \vg\left(\vlambda, \rvveta\right) }^2_{*} \,\middle|\, \vz_{t-1}^{(N)},\, \vlambda_{t-1} }
    \nonumber
    \\
    &\quad=
    \E{ \norm{ \frac{1}{N}\sum^{N}_{n=1} \vs\left(\vlambda; \rvvz^{(n)}\right) }^2_{*} \,\middle|\, \vz_{t-1}^{(N)},\, \vlambda_{t-1} }
    \nonumber
    \\
    &\quad\leq
    \E{ \frac{1}{N^2} \sum^{N}_{n=1} \norm{\vs\left(\vlambda; \rvvz^{(n)}\right) }^2_{*} \,\middle|\, \vz_{t-1}^{(N)},\, \vlambda_{t-1} }
    &&\quad\text{\textit{Triangle inequality}}
    \nonumber
    \\
    &\quad=
    \frac{1}{N^2}\sum^{N}_{n=1} \Esub{\rvvz^{(n)} \sim K^n\left(\vz_{t-1}, \cdot\right)}{ \norm{\vs\left(\vlambda; \rvvz^{(n)}\right) }^2_{*} \,\middle|\,  \vz_{t-1}^{(N)},\, \vlambda_{t-1} }
    &&\quad\text{\textit{Linearity of expectation}}
    \nonumber
    \\
    &\quad\leq
    \frac{1}{N^2}\sum^{N}_{n=1}
      n \, {\left(1 - \frac{1}{w^*}\right)}^{n-1}
      \Esub{\rvvz^{(n)} \sim q_{\text{def.}}\left(\cdot; \vlambda\right)}{ \norm{\vs\left(\vlambda; \rvvz^{(n)} \right)}_{*}^2 }
      \nonumber
      \\
      &\qquad+ 
        {\left(1 - \frac{1}{w^*}\right)}^{n} \, \norm{\vs\left(\vlambda; \vz_{t-1}^{(N)} \right)}_{*}^2
    &&\quad\text{\textit{\cref{thm:imh_expecation}}}
    \nonumber
    \\
    &\quad\leq
    \frac{1}{N^2}\sum^{N}_{n=1}
      n \, {\left(1 - \frac{1}{w^*}\right)}^{n-1}  \, L^2
      +
      {\left(1 - \frac{1}{w^*}\right)}^{n} \, L^2
    &&\quad\text{\textit{\cref{thm:bounded_score}}}
    \nonumber
    \\
    &\quad=
    \frac{L^2}{N^2}\sum^{N}_{n=1}
      n \, {\left(1 - \frac{1}{w^*}\right)}^{n-1}
      +
      {\left(1 - \frac{1}{w^*}\right)}^{n} 
    \nonumber
    &&\quad\text{\textit{Moved constant forward}}
    \\
    &\quad=
    \frac{L^2}{N^2} \,
    \left[\,
      {\left(w^*\right)}^2 + w^*
      -
      {\left(1 - \frac{1}{w^*}\right)}^N
      \left(
        {\left(w^*\right)}^2 + w^* + N\,w^*
      \right)
    \,\right]
    \nonumber
    &&\quad\text{\textit{Solved sum}}
    \\
    &\quad=
    \frac{L^2}{N^2} \,
    \left[\,
    \frac{1}{2} N^2 + \frac{3}{2}\,N 
    + \mathcal{O}\left(1/w^*\right)
    \,\right]
    \nonumber
    &&\quad\text{\textit{Laurent series expansion at \(w^* \rightarrow \infty\)}}
    \\
    &\quad=
    L^2 \,
    \left[\,
    \frac{1}{2} + \frac{3}{2}\,\frac{1}{N}
    + \mathcal{O}\left(1/w^*\right)
    \,\right].
    \nonumber
  \end{alignat}
  The Laurent approximation becomes exact as \(w^* \rightarrow \infty\), which is useful accurate as a consequence of \cref{thm:wstar}.
\end{proofEnd}

%%% Local Variables:
%%% TeX-master: "master"
%%% End:
