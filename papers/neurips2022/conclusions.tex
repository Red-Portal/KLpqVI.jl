
\vspace{-0.15in}
\section{Discussions}\label{section:discussion}
\vspace{-0.1in}
This paper presented a new theoretical framework for analyzing MCSA inclusive KL divergence minimization methods.
Furthermore, we proposed pMCSA, a new MCSA methods that enjoys substantially low variance.
We evaluated our theoretical analysis and the utility of pMCSA on general Bayesian inference problems.

\vspace{-0.1in}
\paragraph{Limitations}
Our work has two main limitations.
Firstly, our work aims to theoretically understand existing MCSA methods.
Therefore, our work inherits the current limitations of MCSA methods, such as the difficulty of minibatching for models with non-factorizable likelihoods~\citep{NEURIPS2020_b2070693}.
Secondly, our theoretical analysis in~\cref{section:comparison} requires~\cref{thm:bounded_score}, which does not hold in general.
However,~\cref{thm:bounded_score} is needed to fulfill the assumptions needed by MCGD methods.
Therefore, an important future step would be to relax the assumptions needed by MCGD.

\vspace{-0.1in}
\paragraph{Inclusive KL}
In \cref{section:eval}, we showed that minimizing the inclusive KL divergence can be competitive against exclusive KL divergence minimization on several problems.
This is not in line with the conclusions of~\citet{dhaka_challenges_2021} that the inclusive KL does not work on high-dimensional problems (\citeauthor{dhaka_challenges_2021} consider few hundreds of dimensions).
However, we note that dimensionality becomes a problem only in the presence of strong correlations.
Then, the practical utility of inclusive KL minimization methods depend on how correlated posteriors are in practice.
To conclude, our results motivate the development of better inference algorithms for alternative divergence measures, including the inclusive KL. 

%An interesting direction for future research would be to invstigate whether such result 
%This is against the conclusions of~\citet{dhaka_challenges_2021} that exclusive VI should outperform inclusive VI in high-dimensions. 

%%% Local Variables:
%%% TeX-master: "master"
%%% End:
