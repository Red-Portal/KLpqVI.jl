
\begin{theoremEnd}[all end]{lemma}\label{thm:iw_variance}
  Let the importance weight be defined as \(w\left(\vz\right) = \pi\left(\vz\right)/q\left(\vz\right)\).
  The variance of the importance weights is related to the \(\chi^2\) divergence as
  \begin{align*}
    \mathbb{V}_{q}{w\left(\rvvz\right)}
    =
    \DChi{\pi}{q}.
  \end{align*}
\end{theoremEnd}
\begin{proofEnd}
  {
  \begin{alignat}{2}
    \mathbb{V}_{q}{w\left(\rvvz\right)}
    =
    \Esub{q}{
      {\left(
        w\left(\rvvz\right)
      -
      \Esub{q}{
        w\left(\rvvz\right)
      }
      \right)}^2
    }
    =
    \Esub{q}{
      {\left(
      w\left(\rvvz\right)
      -
      1
      \right)}^2
    }
    =
    \int {\left( \frac{\pi\left(\vz\right)}{q\left(\vz\right)}  - 1 \right)}^2 q\left(d\vz\right)
    =
    \DChi{\pi}{q}
    \nonumber
  \end{alignat}
  }
\end{proofEnd}

\begin{theoremEnd}{theorem}{\citep{cardoso_brsnis_2022}}\label{thm:mscrb}
  The gradient variance of MSC-RB is bounded as
  \begin{align*}
  {\textstyle\small
    \V{ \rvvg_{t,\mathrm{MSC\text{-}RB}} \,\middle|\, \mathcal{F}_{t-1} } \leq
    4\,L^2 \, \left[
    \frac{1}{N-1}\,\DChi{\pi}{q\left(\cdot; \vlambda_{t-1}\right)}
    +
    \frac{1}{N} {\left(1 + w^*\right)}^2
    +
    \mathcal{O}\left({\left(N-1\right)}^{-2}\right)
    \right],
  }
  \end{align*}
  where \(w^* = \sup_{\vz} \pi\left(\vz\right) / q_{\text{def.}}\left(\vz;\vlambda\right)\).
\end{theoremEnd}
\begin{proofEnd}
  Rao-Blackwellization of the CIS kernel is to reuse the importance weights \(w\left(\vz\right) = \pi\left(\vz\right) / q_{\text{def.}}\left(\vz\right)\) internally used by the kernel when forming the estimator.
  That is, the gradient is estimated as
  \begin{align*}
    &\rvvz^{(n)}_t \mid \rvvlambda_{t-1} \sim q\left(\cdot; \rvvlambda_{t-1}\right) \\
    &\rvvg_{t,\text{MSC-RB}} = -\sum^{N}_{n=2} \frac{w\left(\rvvz^{(n)}_t\right)}{\sum^{N}_{m=2}w\left(\rvvz^{(n)}_t\right) + w\left(\rvvz_{t-1}\right)} \vs\left(\rvvz^{(n)}_t\right) + \frac{w\left(\rvvz_{t-1}\right)}{\sum^{N}_{m=2}w\left(\rvvz^{(n)}_t\right) + w\left(\rvvz_{t-1}\right)} \vs\left(\rvvz_{t-1}\right).
  \end{align*}

  By \cref{thm:second_moment_bound}, the second moment of the gradient is bounded as
  \begin{alignat*}{2}
    \E{ \norm{ \rvvg_{\text{MSC}} }_2^2 \,\middle|\, \mathcal{F}_{t-1} } 
    =
    \mathsf{MSE}\left[ \rvvg_{t, \text{MSC}} \,\middle|\, \mathcal{F}_{t-1} \right]
    +
    2 \, {\mathsf{Bias}\left[ \rvvg_{t, \text{MSC}} \,\middle|\, \mathcal{F}_{t-1} \right]}^{\top} \norm{\vmu}
    + \norm{\mu}^2
  \end{alignat*}
  
  \citet[Theorem 3]{cardoso_brsnis_2022} show that the mean-squared error of this estimator, which they call bias reduced self-normalized importance sampling, is bounded as
  \begin{align*}
    \mathsf{MSE}\left[ \rvvg_{\text{MSC-RB}} \,\middle|\, \mathcal{F}_{t-1} \right] \leq
    4\,L^2 \, \Big[\,&
     \left( 1 + \epsilon^2 \right)\, \frac{1}{N-1}\,\Vsub{\rvvz \sim q_{\text{def.}}\left(\cdot; \rvvlambda_{t-1}\right)}{w\left(\rvvz\right) \mid \rvvlambda_{t-1}} 
     \\
     &+
     \left(1 + \epsilon^{-2} \right)\, \frac{1}{N^2} {\left(1 + w^*\right)}^2 
    \,\Big],
  \end{align*}
  for some arbitrary constant \(\epsilon^2\).
  The first term is identical to the variance of an \(N-1\)-sample self-normalized importance sampling estimator~\citep{10.1214/17-STS611}, while the second term is the added variance due to ``rejections.''

  Since the variance of the importance weights is well known to be related to the \(\chi^2\) divergence,
  \begin{alignat}{2}
    \mathsf{MSE}\left[ \rvvg_{t,\text{MSC-RB}} \,\middle|\, \mathcal{F}_{t-1} \right] \leq
    4\,L^2 \, \Big[\,&
     \left( 1 + \epsilon^2 \right)\, \frac{1}{N-1}\,\DChi{\pi}{q\left(\cdot; \rvvlambda_{t-1}\right)}
    \nonumber
     \\
     &+
     \left(1 + \epsilon^{-2} \right)\, \frac{1}{N^2} {\left(1 + w^*\right)}^2 
    \,\Big].
    &&\quad\text{\textit{\cref{thm:iw_variance}}}
    \nonumber
  \end{alignat}

  For \(\epsilon^2\), \citeauthor{cardoso_brsnis_2022} choose \(\epsilon^2 = {\left(N-1\right)}^{-1/2}\), which results in their stated bound. 
  For the alternative choice of \(\epsilon^2 = {\left(N-1\right)}^{-1}\), it is possible to make the smaller terms decrease in a faster rate, and show that both \({\left(1 + w^*\right)}^2\) and \(\DChi{\pi}{q\left(\cdot; \vlambda_{t-1}\right)}\) similarly contribute to the variance such that
  \begin{align*}
    \mathsf{MSE}\left[ \rvvg_{\text{MSC-RB}} \,\middle|\, \mathcal{F}_{t-1} \right] \leq
    4\,L^2 \, \Big[\,
      \frac{1}{N-1}\,\DChi{\pi}{q\left(\cdot; \vlambda_{t-1}\right)}
      +
      \frac{1}{N} {\left(1 + w^*\right)}^2
      +
      \mathcal{O}\left({\left(N-1\right)}^{-2}\right)
    \,\Big].
  \end{align*}
  But now, the \({\left(1+w^*\right)}^2\) term decreases in a slower rate.
\end{proofEnd}

%%% Local Variables:
%%% TeX-master: "master"
%%% End:
