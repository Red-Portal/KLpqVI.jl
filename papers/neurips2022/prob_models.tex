
\section{Probabilistic Models Used in the Experiments}
%% \subsection{Hierarchical Logistic Regression}\label{section:linear_logistic}
%% The hierarchical logistic regression used in~\cref{section:logistic} is 
%% \begin{align*}
%%     \sigma_{\beta}  &\sim \mathcal{N}^+\left(0, 1.0\right) \\
%%     \sigma_{\alpha} &\sim \mathcal{N}^+\left(0, 1.0\right) \\
%%     \symbf{\beta} &\sim \mathcal{N}\left(\symbf{0}, \sigma_{\beta}^2\,\symbf{I}\right) \\
%%     \alpha        &\sim \mathcal{N}\left(0, \sigma_{\alpha}^2\right) \\
%%     p             &\sim \mathcal{N}\left(\vx_i^{\top}\symbf{\beta} + \alpha,\, \sigma_{\alpha}^2\right)\\
%%     y_i           &\sim \text{Bernoulli-Logit}\,(p)
%% \end{align*}
%% where \(\vx_i\) and \(y_i\) are the predictors and binary target variable of the \(i\)th datapoints.

\subsection{Bayesian Neural Network Regression}\label{section:gp_logistic}
The latent Gaussian process model used in~\cref{section:bgp} is 
\begin{align*}
   \log \alpha &\sim \mathcal{N}(0, 1) \\
   \log \sigma &\sim \mathcal{N}(0, 1) \\
   \log \ell_i &\sim \mathcal{N}(0, 1) \\
   f &\sim \mathcal{GP}\left(\mathbf{0}, \mSigma_{\alpha^2, \sigma^2, \mathbf{\ell}} + \delta\,\mI \right) \\
   y_i &\sim \text{Bernoulli-Logit}\left(  f\left( \vx_i \right) \right).
\end{align*}
The covariance \(\mSigma\) is computed using a kernel \(k\left(\cdot, \cdot\right)\) such that \({[\mSigma]}_{i,j} = k\left( \vx_{i}, \vx_{j} \right) \) where \(\vx_i\) and \(\vx_j\) are data points in the dataset.
For the kernel, we use the Matern 5/2 kernel with automatic relevance determination~\citep{neal_bayesian_1996} defined as
\begin{align*}
  &k\left(\vx, \vx' ;\; \alpha^2, \sigma^2, \mathbf{\ell} \right) =
  \alpha \left( 1 + \sqrt{5} r + \frac{5}{3} r^2 \right) \exp\left( - \sqrt{5} r \right)  \quad
  \text{where}\;\; r = \sum^{D}_{i=1} \frac{ {\left(\vx_i - \vx'_i\right)}^2 }{\ell^2_i}
\end{align*}
where \(D\) is the number of dimensions.
The jitter term \(\delta\) is used for numerical stability.
We set a small value of \(\delta = 1\times10^{-6}\).

\subsection{Gaussian Process Logistic Regression}\label{section:gp_logistic}
The latent Gaussian process model used in~\cref{section:bgp} is 
\begin{align*}
   \log \alpha &\sim \mathcal{N}(0, 1) \\
   \log \sigma &\sim \mathcal{N}(0, 1) \\
   \log \ell_i &\sim \mathcal{N}(0, 1) \\
   f &\sim \mathcal{GP}\left(\mathbf{0}, \mSigma_{\alpha^2, \sigma^2, \mathbf{\ell}} + \delta\,\mI \right) \\
   y_i &\sim \text{Bernoulli-Logit}\left(  f\left( \vx_i \right) \right).
\end{align*}
The covariance \(\mSigma\) is computed using a kernel \(k\left(\cdot, \cdot\right)\) such that \({[\mSigma]}_{i,j} = k\left( \vx_{i}, \vx_{j} \right) \) where \(\vx_i\) and \(\vx_j\) are data points in the dataset.
For the kernel, we use the Matern 5/2 kernel with automatic relevance determination~\citep{neal_bayesian_1996} defined as
\begin{align*}
  &k\left(\vx, \vx' ;\; \alpha^2, \sigma^2, \mathbf{\ell} \right) =
  \alpha \left( 1 + \sqrt{5} r + \frac{5}{3} r^2 \right) \exp\left( - \sqrt{5} r \right)  \quad
  \text{where}\;\; r = \sum^{D}_{i=1} \frac{ {\left(\vx_i - \vx'_i\right)}^2 }{\ell^2_i}
\end{align*}
where \(D\) is the number of dimensions.
The jitter term \(\delta\) is used for numerical stability.
We set a small value of \(\delta = 1\times10^{-6}\).

%% \subsection{Radon Hierarchical Regression}
%% The partially pooled linear regression model used in~\cref{section:mll} is
%% \begin{align*}
%%   \sigma_{a_1} &\sim \mathrm{Gamma}\left( \alpha = 1, \beta = 0.02 \right) \\
%%   \sigma_{a_2} &\sim \mathrm{Gamma}\left( \alpha = 1, \beta = 0.02 \right) \\
%%   \sigma_{y}  &\sim \mathrm{Gamma}\left( \alpha = 1, \beta = 0.02 \right) \\
%%   \mu_{a_1}    &\sim \mathcal{N}\left( 0, 1 \right) \\
%%   \mu_{a_2}    &\sim \mathcal{N}\left( 0, 1 \right) \\
%%   a_{1,\, c}     &\sim \mathcal{N}\left( \mu_{a_1}, \sigma_{a_1}^2 \right) \\
%%   a_{2,\, c}     &\sim \mathcal{N}\left( \mu_{a_2}, \sigma_{a_2}^2 \right) \\
%%   y_i         &\sim \mathcal{N}\left( a_{1,\, c_i} + a_{2,\, c_i}\,x_i,\, \sigma_y^2 \right)
%% \end{align*}
%% where \(a_{1,\,c}\) is the intercept at the county \(c\), \(a_{2,\,c}\) is the slope at the county \(c\), \(c_i\) is the county of the \(i\)th datapoint, \(x_i\) and \(y_i\) are the floor predictor of the measurement and the measured radon level of the \(i\)the datapoint, respectively.
%% The model pools the datapoints into their respective counties, which complicates the posterior geometry~\citep{betancourt_hierarchical_2020}.

%%% Local Variables:
%%% TeX-master: "master"
%%% End:
