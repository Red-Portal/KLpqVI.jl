
Minimizing the inclusive Kullback-Leibler (KL) divergence with stochastic gradient descent (SGD) is challenging since its gradient is defined as an integral over the posterior.
Recently, multiple methods have been proposed to run SGD with \textit{biased} gradient estimates obtained from a Markov chain.
This paper provides the first non-asymptotic convergence analysis of these methods by establishing their mixing rate and gradient variance.
To do this, we demonstrate that these methods--which we collectively refer to as Markov chain score ascent (MCSA) methods--can be cast as special cases of the Markov chain gradient descent framework.
Furthermore, by leveraging this new understanding, we develop a novel MCSA scheme, \textit{parallel} MCSA (pMCSA), that achieves a tighter bound on the gradient variance.
We demonstrate that this improved theoretical result translates to superior empirical performance.

% These methods, we call Markov chain score ascent (MCSA), obtain noisy estimates of the gradient by running Markov chains in conjunction with SGD iterations.
% This paper shows that these MCSA methods are a special case of the Markov chain gradient descent (MCGD) framework.
% Based on the non-asymptotic convergence results of MCGD, we analyze the practical performance of previously developed MCSA algorithms.
% Furthermore, we propose a novel MCSA scheme, parallel MCSA (pMCSA), that achieves a tighter bound on the gradient variance.
% We evaluate pMCSA against previously developed MCSA methods on general Bayesian inference benchmarks and demonstrate its performance.

%Our experiments show that, when using our proposed scheme, inclusive KL divergence minimization is competitive against evidence lower bound minimization.
%Our results motivate the use of the inclusive KL divergence for VI.

%%% Local Variables:
%%% TeX-master: "master"
%%% End:
